% Auteur : Samuele Giraudo
% Création : 26-01-2016
% Modifications : 26-01-2016

\documentclass[12pt]{article}

\usepackage[utf8]{inputenc}
\usepackage[french]{babel}
\usepackage{amsmath,amsthm,amsfonts,amssymb}
\usepackage{lmodern}
\usepackage[top=2.4cm,bottom=2.4cm,left=2cm,right=2cm]{geometry}
\usepackage{hyperref}
\usepackage{multicol}
\usepackage{enumerate}
\usepackage{listings}
\usepackage[dvipsnames]{xcolor}
\usepackage{tikz}

%\author{Samuele Giraudo}
\date{}
\title{{\bf Perfectionnement à la programmation en {\sf C}} \\
    Fiche de TP 5 \\
    {\small L2 Informatique 2015-2016} \\
    {\it \small Compter les mots}}

\theoremstyle{definition}
\newtheorem{Exercice}{Exercice}

% Configuration de listings.
\lstset{
    language=c,
    basicstyle=\ttfamily\footnotesize,
    identifierstyle=\color{Mahogany},
    keywordstyle=\color{NavyBlue},
    stringstyle=\color{Emerald},
    commentstyle=\it\color{Gray},
    columns=flexible,
    tabsize=4,
    extendedchars=true,
    showspaces=false,
    numbers=left,
    numberstyle=\tiny,
    breaklines=true,
    breakautoindent=true,
    captionpos=b,
    showstringspaces=true
}

\begin{document} 

\maketitle

Ce TP se déroule en une seule séance et est à faire par binômes.
Le travail réalisé doit être envoyé au plus tard exactement une
semaine après le début de la séance de TP. Il sera disposé sur
la plate-forme prévue à cet effet et constitué des programmes
répondant aux questions et des éventuels fichiers annexes qui
peuvent être demandés.
\bigskip
\bigskip

L'objectif de ce TP est d'écrire un utilitaire opérant sur des fichiers
texte. Il utilisera des notions de manipulation de fichiers texte et de 
chaînes de caractères. Il est voué à renforcer la maîtrise de 
l'écriture de projets et de découpage en modules. La notion de test est 
introduite dans ce sujet.
\bigskip

Contrairement aux sujets de TP précédents, nous donnons ici une 
spécification du projet à lire et à mener à bien. Les définitions de types et de 
fonctions sont laissées libres de choix.
\bigskip

Voici maintenant la spécification du projet.
\bigskip

\begin{center} *** \\ * \end{center}

Il est toujours utile, lors de l'écriture d'un texte, d'éviter d'employer
trop souvent des mêmes mots. Détecter l'usage trop fréquent d'un même 
terme est donc utile. C'est pour cette raison qu'il nous est demandé de 
concevoir un utilitaire {\tt clm} (\og {\em compter les mots} \fg)
paramétré par le nom d'un fichier texte (qui ne contient que des 
caractères ASCII). L'utilitaire compte les occurrences de chaque mot
du fichier et produit en réponse un fichier qui contient chaque
mot du fichier texte de départ et son nombre d'occurrences.
\medskip

Plus précisément, est considéré comme un mot toute suite de caractères
contigüs délimitée à gauche et à droite par des espaces, des fins de lignes,
des tabulations ou des début ou fin de fichier. Toute suite de caractères
contigüs qui contient un chiffre ou tout autre caractère non alphabétique 
est ignorée. Dans le comptage des occurrences d'un mot, on ne fait pas de
différence entre majuscules et minuscules. 
\medskip

La commande {\tt clm X}, où {\tt X} est le nom d'un fichier texte, créé
un fichier texte {\tt X.clm} où chaque ligne contient un mot 
en minuscules suivi d'une espace suivie du nombre (en base dix) 
d'occurrences du mot en question dans {\tt X}. Chaque mot y est 
mentionné suivant l'ordre d'appartition de sa première
occurrence dans le texte. Par exemple, à partir du fichier texte {\tt texte} de contenu
\begin{verbatim}
        TROIS Un un un1un deux DeUx 
        deux deuxdeux un quatre4
\end{verbatim}
la commande {\tt clm texte} créé le fichier {\tt texte.clm} qui contient
\begin{verbatim}
        trois 1
        un 3
        deux 3
        deuxdeux 1
\end{verbatim}
\medskip

Il figurera une option {\tt -a} faisant en sorte de trier
les mots du fichier de sortie dans l'ordre lexicographique. Par exemple,
la commande {\tt clm -a texte} créé le fichier {\tt texte.clm} de contenu
\begin{verbatim}
        deux 3
        deuxdeux 1
        trois 1
        un 3
\end{verbatim}
Il figurera aussi une option {\tt -n} faisant en sorte de 
trier les mots de fichier de sortie selon leur nombre d'occurrences de 
manière décroissante. Les mots ayant un même nombre d'occurrences apparaissent
dans l'ordre lexicographique. Par exemple, la commande {\tt clm -n texte} créé 
le fichier {\tt texte.clm} de contenu
\begin{verbatim}
        deux 3
        un 3
        deuxdeux 1
        trois 1
\end{verbatim}

\begin{center} *** \\ * \end{center}
\bigskip

{\bf Conseils~:}
\begin{itemize}
    \item pour chaque fonction programmée, s'interroger sur les arguments
    qui peuvent poser des problèmes. Capturer ces cas à l'aide de
    pré-assertions~;
    \item utiliser au maximum les fonctions déjà programmées dans les
    nouvelles à écrire. Il faut éviter au maximum la duplication de code~;
    \item commenter le code en évitant absolument les commentaires inutiles.
    Il faut commenter chaque fonction écrite, juste avant son prototype.
    Il faut y mentionner les trois points suivants~: (1) une description
    du rôle de la fonction, (2) une description de ses paramètres et (3)
    une description de ce qu'elle renvoie~;
    \item préférer la concision au maximum. Un code concis et lisible
    possède une grande valeur.
\end{itemize}
\bigskip
\bigskip

%%%%%%%%%%%%%%%%%%%%%%%%%%%%%%%%%%%%%%%%%%%%%%%%%%%%%%%%%%%%%%%%%%%%%%%%
%%%%%%%%%%%%%%%%%%%%%%%%%%%%%%%%%%%%%%%%%%%%%%%%%%%%%%%%%%%%%%%%%%%%%%%%
%%%%%%%%%%%%%%%%%%%%%%%%%%%%%%%%%%%%%%%%%%%%%%%%%%%%%%%%%%%%%%%%%%%%%%%%
\begin{Exercice} {\bf (Conception du projet)}\smallskip

L'objectif de cet exercice est de concevoir une architecture viable pour
le projet présenté.
\begin{enumerate}
    \item Lire attentivement l'intégralité du sujet avant de commencer
    à répondre aux questions. Il est important de bien analyser la 
    spécification du projet et d'en distiller le plan du travail à 
    accomplir.
    \smallskip
    
    \item Une fois ceci fait, proposer un découpage en modules cohérent 
    du projet. Pour chaque module proposé, décrire les types qu'il apporte 
    ainsi que ses objectifs principaux.
    \smallskip
    
    \item Au fil de l'écriture du projet, il est possible de se rendre 
    compte que le 
    découpage initialement prévu n'est pas complet ou adapté. Si c'est 
    le cas, mentionner l'historique de ses modifications.
    \smallskip
    
    \item Maintenir un {\tt Makefile} pour compiler le projet.
\end{enumerate}
\end{Exercice}
\bigskip

%%%%%%%%%%%%%%%%%%%%%%%%%%%%%%%%%%%%%%%%%%%%%%%%%%%%%%%%%%%%%%%%%%%%%%%%
%%%%%%%%%%%%%%%%%%%%%%%%%%%%%%%%%%%%%%%%%%%%%%%%%%%%%%%%%%%%%%%%%%%%%%%%
%%%%%%%%%%%%%%%%%%%%%%%%%%%%%%%%%%%%%%%%%%%%%%%%%%%%%%%%%%%%%%%%%%%%%%%%
\begin{Exercice} {\bf (Tests)}\smallskip
\label{exo:test}

Les tests sont des éléments de première importance dans le processus 
d'écriture d'un projet. Ils sont également primordiaux pour la maintenance
d'un projet. Ils servent à s'assurer que chaque partie du 
programme fonctione et sont utiles pour capturer des erreurs de 
programmation à leur source.
\smallskip

\begin{enumerate}
    \item Avant de commencer l'écriture du projet, créer un module 
    {\tt Test}. Ce module va inclure tous les autres modules (exception
    faite pour le module principal) et va servir à tester les fonctions
    du projet. Il contient pour le moment une fonction {\tt test} de 
    corps vide, sans paramètre et à type de retour {\tt int}.
    \smallskip
    
    \item Au cours de l'écriture du projet, pour chaque fonction {\tt fct}
    nouvellement écrite, ajouter dans le module {\tt Test} une fonction
    {\tt test\_fct} qui réalise des tests cohérents de la fonction {\tt fct}.
    Les tests sont construits en appelant {\tt fct} avec des 
    arguments pour lesquels la réponse est connue. Le test consiste à 
    vérifier si la fonction répond correctement. La fonction {\tt test\_fct}
    renvoie {\tt 1} si tous les tests qu'elle exécute se sont bien 
    déroulés et {\tt 0} sinon. Pour chaque test mal déroulé, la
    fonction affiche sur la sortie standard des informations pour identifier
    le test problématique.
    \smallskip
    
    \item La fonction {\tt test} du module {\tt Test} est sa seule 
    fonction visible depuis l'extérieur. Elle appelle chaque fonction 
    de test et vérifie que toutes se déroulent correctement. Elle renvoie
    ainsi {\tt 1} si toutes les fonctions de test renvoient {\tt 1} et
    {\tt 0} sinon.
    \smallskip
    
    \item Le programme dispose d'une option {\tt --test} qui exécute
    la fonction {\tt test} du module {\tt Test}. Il est important, lors 
    de la phase de programmation du projet ou encore lors d'une de ses 
    mises-à-jour, de lancer ces tests pour vérifier que tout fonctionne 
    comme attendu.
\end{enumerate}
\end{Exercice}
\bigskip

{\bf Remarque importante 1~:} {\it l'exercice~\ref{exo:test} compte beaucoup
dans l'évaluation de ce TP. Il est important de proposer pour chaque 
fonction un jeu de tests cohérent et complet.}
\bigskip

{\bf Remarque importante 2~:} {\it il n'est pas question dans ce TP 
d'utiliser les structures de données et les algorithmes les plus efficaces
qui répondent au problème (il est en effet possible de proposer des 
solutions très efficaces en étant astucieux). L'objectif principal est 
de concevoir un projet sans faille dans son architecture globale.}

\end{document}
