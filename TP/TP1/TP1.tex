% Auteur : Samuele Giraudo
% Création : 11-12-15
% Modifications : 11-12-15, déc. 2015

\documentclass[12pt]{article}

\usepackage[utf8]{inputenc}
\usepackage[french]{babel}
\usepackage{amsmath,amsthm,amsfonts,amssymb}
\usepackage{lmodern}
\usepackage[top=2.4cm,bottom=2.4cm,left=2cm,right=2cm]{geometry}
\usepackage{hyperref}
\usepackage{multicol}
\usepackage{enumitem}
\usepackage{listings}
\usepackage[dvipsnames]{xcolor}
\usepackage{tikz}

%\author{Samuele Giraudo}
\date{}
\title{{\bf Perfectionnement à la programmation en {\sf C}} \\
    Fiche de TP 1 \\
    {\small L2 Informatique 2015-2016} \\
    {\it \small Partir sur de bonnes bases}}

\theoremstyle{definition}
\newtheorem{Exercice}{Exercice}

% Configuration de listings.
\lstset{
    language=c,
    basicstyle=\ttfamily\footnotesize,
    identifierstyle=\color{Mahogany},
    keywordstyle=\color{NavyBlue},
    stringstyle=\color{Emerald},
    commentstyle=\it\color{Gray},
    columns=flexible,
    tabsize=4,
    extendedchars=true,
    showspaces=false,
    numbers=left,
    numberstyle=\tiny,
    breaklines=true,
    breakautoindent=true,
    captionpos=b,
    showstringspaces=true
}

\begin{document}

\maketitle

Ce TP se déroule en une seule séance et est à faire par binômes.
Le travail réalisé doit être envoyé au plus tard exactement une
semaine après le début de la séance de TP. Il sera disposé sur
la plate-forme prévue à cet effet et constitué des programmes
répondant aux questions et des éventuels fichiers annexes qui
peuvent être demandés.
\bigskip
\bigskip

Dans ce TP, nous révisons l'utilisation de la bibliothèque graphique
{\sf MLV}, déjà approchée au cours des semestres précédents. Nous posons 
aussi quelques bonnes habitudes à prendre en programmation pour le reste 
de ces TP.
\bigskip
\bigskip

{\bf Conseils~:}
\begin{itemize}
    \item pour chaque fonction programmée, s'interroger sur les arguments
    qui peuvent poser des problèmes. Capturer ces cas à l'aide de
    pré-assertions~;
    \item utiliser au maximum les fonctions déjà programmées dans les
    nouvelles à écrire. Il faut éviter au maximum la duplication de code~;
    \item il ne suffit pas de programmer uniquement les fonctions qui sont
    demandées. Pour mener le sujet à bien, des fonctions supplémentaires
    et non mentionnées explicitement dans le sujet seront utiles. Soigner
    leur nom~;
    \item dès qu'une fonction est écrite, la tester de manière à
    recouvrir tous les cas significatifs possibles et conserver les tests
    (le but étant de les relancer pour vérifier l'intégrité du
    programme lors de modifications futures)~;
    \item commenter le code en évitant absolument les commentaires inutiles.
    Il faut commenter chaque fonction écrite, juste avant son prototype.
    Il faut y mentionner les trois points suivants~: (1) une description
    du rôle de la fonction, (2) une description de ses paramètres et (3)
    une description de ce qu'elle renvoie~;
    \item préférer la concision au maximum. Un code concis et lisible
    possède une grande valeur.
\end{itemize}
\bigskip
\bigskip

%%%%%%%%%%%%%%%%%%%%%%%%%%%%%%%%%%%%%%%%%%%%%%%%%%%%%%%%%%%%%%%%%%%%%%%%
%%%%%%%%%%%%%%%%%%%%%%%%%%%%%%%%%%%%%%%%%%%%%%%%%%%%%%%%%%%%%%%%%%%%%%%%
%%%%%%%%%%%%%%%%%%%%%%%%%%%%%%%%%%%%%%%%%%%%%%%%%%%%%%%%%%%%%%%%%%%%%%%%
\begin{Exercice} {\bf (La documentation {\tt man})}\smallskip

La {\em section 3} du manuel de l'utilisateur de {\sf Linux} recense
la documentation de la plupart des fonctions proposées par la bibliothèque
standard. Il faut donc la consulter systématiquement pour connaître
précisément le rôle d'une fonction. Par exemple, pour afficher
la documentation de la fonction {\tt putchar}, on saisit la commande
\begin{center}
    {\tt man 3 putchar}
\end{center}
sur le terminal. Beaucoup d'informations sont alors affichées,
comprenant la description des paramètres de la fonction et de sa
valeur de retour.

\begin{enumerate}
    \item Rechercher la documentation de la fonction {\tt strcoll} et
    expliquer, en trois lignes maximum, ce qu'elle fait.
    \smallskip

    \item Écrire un programme qui illustre de manière exhaustive
    le rôle de {\tt strcoll}.
    \smallskip

    \item Rechercher la documentation de la fonction {\tt strchr} et
    expliquer, en deux lignes maximum, ce qu'elle fait.
    \smallskip

    \item Écrire un programme qui illustre de manière exhaustive
    le rôle de {\tt strchr}.
    \smallskip

    \item Rechercher la documentation de la fonction {\tt atoi} et
    expliquer, en deux lignes maximum, ce qu'elle fait.
    \smallskip

    \item Écrire un programme qui illustre de manière exhaustive
    le rôle de {\tt atoi}.
\end{enumerate}
\end{Exercice}
\bigskip

%%%%%%%%%%%%%%%%%%%%%%%%%%%%%%%%%%%%%%%%%%%%%%%%%%%%%%%%%%%%%%%%%%%%%%%%
%%%%%%%%%%%%%%%%%%%%%%%%%%%%%%%%%%%%%%%%%%%%%%%%%%%%%%%%%%%%%%%%%%%%%%%%
%%%%%%%%%%%%%%%%%%%%%%%%%%%%%%%%%%%%%%%%%%%%%%%%%%%%%%%%%%%%%%%%%%%%%%%%
\begin{Exercice} {\bf (La bibliothèque graphique {\sf MLV})}\smallskip
\begin{enumerate}
   \item Visiter la page
    \begin{center}
        \url{http://www-igm.univ-mlv.fr/~boussica/mlv/index.html},
    \end{center}
    télécharger et installer la bibliothèque graphique {\sf MLV}.
    \smallskip

    {\it Indication~: il se peut que cette bibliothèque soit déjà
    installée.}
    \smallskip

    \item Parcourir sa documentation et s'assurer qu'elle a bien été
    installée en créant un programme {\tt Rectangle.c} qui affiche
    un rectangle rouge dans une fenêtre.
    \smallskip

    {\it Indication~: garder la page de documentation en permanence
    ouverte au cours les séances de TP est une bonne idée.}
    \smallskip

    \item Créer un programme {\tt RectangleClic.c} qui affiche un
    rectangle rouge dans une fenêtre. Lorsque l'utilisateur clique
    sur le rectangle, il change de couleur (il passe du rouge au bleu
    et du bleu au rouge).
    \smallskip

    {\it Indication~: bien regarder les exemples d'utilisation de
    la bibliothèque sur le site dédié. Ils sont excellents, bien choisis
    et très instructifs.}
    \smallskip

    \item Créer un programme {\tt Rebond.c} qui affiche un petit
    rectangle bleu au centre d'un grand rectangle blanc. Lorsque
    l'utilisateur appuie sur la touche {\tt Entrée}, le petit rectangle
    se déplace progressivement en direction nord-est. Lorsqu'il atteint
    le bord défini par le grand rectangle, celui-ci rebondit. Le petit
    rectangle continue à bouger tant que l'utilisateur n'appuie à nouveau
    sur {\tt Entrée}, provoquant la fin de l'exécution du programme.
\end{enumerate}
\end{Exercice}
\bigskip

%%%%%%%%%%%%%%%%%%%%%%%%%%%%%%%%%%%%%%%%%%%%%%%%%%%%%%%%%%%%%%%%%%%%%%%%
%%%%%%%%%%%%%%%%%%%%%%%%%%%%%%%%%%%%%%%%%%%%%%%%%%%%%%%%%%%%%%%%%%%%%%%%
%%%%%%%%%%%%%%%%%%%%%%%%%%%%%%%%%%%%%%%%%%%%%%%%%%%%%%%%%%%%%%%%%%%%%%%%
\begin{Exercice} {\bf (Arguments d'un programme)}\smallskip

Un programme {\sf C} peut être exécuté comme une commande habituelle et
peut donc recevoir des arguments. Un argument est une chaîne de
caractères. Pour profiter de cette fonctionnalité, la fonction {\tt main}
doit être écrite en respectant le prototype suivant :
\begin{lstlisting}
int main(int argc, char *argv[]);
\end{lstlisting}
La variable {\tt argc} contient le nombre d'arguments avec lesquels le
programme vient d'être lancé et la variable {\tt argv} est un tableau de
chaînes de caractères, indicé de $0$ à {\tt argc - 1}, qui contient les
arguments dans l'ordre de leur saisie. Il est a noter que le nom du
programme est lui-même considéré comme un argument et occupe de ce fait
invariablement la première position du tableau {\tt argv}.
\smallskip

Par exemple, supposons que l'on souhaite lancer un programme nommé
{\tt Prog} avec les arguments {\tt arg1} et {\tt autreArg}. Pour cela,
nous saisirons la commande {\tt ./Prog arg1 autreArg}, qui a pour effet
d'affecter à la variable {\tt argc} la valeur $3$ et le tableau
{\tt argv} va vérifier {\tt argv[0] = "Prog"}, {\tt argv[1] = "arg1"} et
{\tt argv[2] = "autreArg"}.

\begin{enumerate}
    \item Écrire un programme {\tt LectureArg} qui affiche sur la sortie
    standard le nombre d'arguments avec lesquels il est lancé.
    \smallskip

    \item Écrire un programme {\tt Calc} qui prend trois arguments~:
    un opérateurs parmi {\tt +}, {\tt -}, {\tt *} et {\tt /} et deux
    entiers. Le programme affiche ensuite le résultat de l'opération
    spécifiée par le 1\ier{} argument entre ses 2\ieme{} et 3\ieme{}
    arguments. Il est important de capturer ici tous les cas d'erreur
    et de renseigner l'utilisateur lorsque ceux-ci surviennent.
\end{enumerate}
\end{Exercice}
\bigskip

%%%%%%%%%%%%%%%%%%%%%%%%%%%%%%%%%%%%%%%%%%%%%%%%%%%%%%%%%%%%%%%%%%%%%%%%
%%%%%%%%%%%%%%%%%%%%%%%%%%%%%%%%%%%%%%%%%%%%%%%%%%%%%%%%%%%%%%%%%%%%%%%%
%%%%%%%%%%%%%%%%%%%%%%%%%%%%%%%%%%%%%%%%%%%%%%%%%%%%%%%%%%%%%%%%%%%%%%%%
\begin{Exercice} {\bf (Programmation d'un système de cryptage simple)}
\smallskip

Le but de cet exercice est de programmer un système de cryptage élémentaire
basé sur le cryptage dit de \textit{César}. Il consiste, étant donné un
texte \textit{en clair} (c.-à-d. non crypté), et une constante
$K \in \mathbb{Z}$ appelée \textit{clé de cryptage}, à décaler
circulairement chacune des lettres du texte de $K$ positions par rapport
à sa place dans l'alphabet. Par exemple, pour le texte en clair :
\begin{center}
    {\tt Les TP de programmation sont super !}
\end{center}
et la clé~$K = 3$, on obtient le texte \textit{crypté} suivant :
\begin{center}
    {\tt Ohv WS gh surjudppdwlrq vrqw vxshu !}
\end{center}
Les caractères non alphabétiques ne sont pas modifiés.
La formule qui permet de calculer, étant donné une lettre alphabétique en clair $\ell$, la
lettre cryptée $c$ qui lui correspond est la suivante :
\begin{equation} \label{F1}
    c := (\ell + K) ~\operatorname{mod}~ 26.
\end{equation}
Réciproquement, pour décrypter un texte crypté, il suffit de décrypter chacune
des lettres alphabétiques qu'il contient. Étant donné une
lettre cryptée~$c$, le calcul de la lettre en clair~$\ell$ qui lui
correspond s'effectue par la formule suivante :
\begin{equation} \label{F2}
    \ell := (c - K) ~\operatorname{mod}~ 26.
\end{equation}
\begin{enumerate}
    \item Déterminer ce en quoi est évaluée l'expression {\tt -9 \% 5}
    et en quoi cela est gênant pour la suite de l'exercice.
    \smallskip

    {\it Indication~: regarder la formule \eqref{F2} et s'interroger sur
    quand $\ell$ définit bien un caractère.}
    \smallskip

    \item Programmer une fonction
\begin{lstlisting}
int bon_modulo(int a, int b);
\end{lstlisting}
    qui calcule le reste de la division euclidienne de {\tt a} par {\tt b}
    (la valeur de retour doit être toujours positive).
    \smallskip

    \item Écrire un programme {\tt Crypt}, qui accepte comme arguments
    une valeur numérique (la clé) et un mot (le mot à crypter). Ce programme
    affiche sur la sortie standard le mot crypté selon la clé
    en suivant la formule \eqref{F1}.
    \smallskip

    \item Écrire un autre programme {\tt Decrypt} qui réalise le décryptage.
    Il prend comme arguments une valeur numérique (la clé) et un mot
    (le mot à décrypter). Ce programme affiche sur la sortie standard le
    mot décrypté selon la clé en suivant la formule \eqref{F2}.
    \smallskip

    \item Tester les deux programmes précédents, en particulier en vérifiant
    que le mot obtenu en cryptant puis en décryptant successivement
    un mot avec une même clé est bien égal au mot de départ.
\end{enumerate}
\end{Exercice}
\bigskip

\end{document}
