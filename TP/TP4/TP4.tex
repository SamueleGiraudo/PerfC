% Auteur : Samuele Giraudo
% Création : 26-01-2016
% Modifications : 26-01-2016, 12-03-2016

\documentclass[11pt]{article}

\usepackage[utf8]{inputenc}
\usepackage[french]{babel}
\usepackage{amsmath,amsthm,amsfonts,amssymb}
\usepackage{lmodern}
\usepackage[top=3cm,bottom=3cm,left=3cm,right=3cm]{geometry}
\usepackage{hyperref}
\usepackage{multicol}
\usepackage{enumerate}
\usepackage{listings}
\usepackage{moreverb}
\usepackage[dvipsnames]{xcolor}
\usepackage{tikz}

%\author{Samuele Giraudo}
\date{}
\title{{\bf Perfectionnement à la programmation en {\sf C}} \\
    Fiche de TP 4 \\
    {\small L2 Informatique 2015-2016} \\
    {\it \small Compter les mots}}

\theoremstyle{definition}
\newtheorem{Exercice}{Exercice}

% Configuration de listings.
\lstset{
    language=c,
    basicstyle=\ttfamily\footnotesize,
    identifierstyle=\color{Mahogany},
    keywordstyle=\color{NavyBlue},
    stringstyle=\color{Emerald},
    commentstyle=\it\color{Gray},
    columns=flexible,
    tabsize=4,
    extendedchars=true,
    showspaces=false,
    numbers=left,
    numberstyle=\tiny,
    breaklines=true,
    breakautoindent=true,
    captionpos=b,
    showstringspaces=true
}

\begin{document}

\maketitle

Ce TP se déroule en deux séances et est à faire par binômes. Le travail
réalisé doit être envoyé au plus tard exactement une semaine après le
début de la $2\ieme$ séance de TP. Il sera disposé sur la plate-forme
prévue à cet effet et constitué des programmes répondant aux questions
et des éventuels fichiers annexes qui peuvent être demandés.
\bigskip
\bigskip

L'objectif de ce TP est d'écrire un utilitaire opérant sur des fichiers
texte. Il utilisera des notions de manipulation de fichiers texte et de
chaînes de caractères. Il est voué à renforcer la maîtrise de
l'écriture de projets et de découpage en modules. La notion de test est
introduite dans ce sujet.
\bigskip

Contrairement aux sujets de TP précédents, nous donnons ici une
spécification du projet à lire, à comprendre et, finalement, à satisfaire.
Les définitions de types et de fonctions sont laissées libres de choix.
\bigskip

Voici maintenant la spécification du projet.
\bigskip

\begin{center} *** \\ * \end{center}
\medskip

Il est toujours utile, lors de l'écriture d'un texte, d'éviter d'employer
trop souvent des mêmes mots. Détecter l'usage trop fréquent d'un même
terme est donc utile pour rédiger des textes agréables à lire. C'est pour
cette raison qu'il est demandé ici de concevoir un utilitaire {\tt clm}
(\og {\em compter les mots} \fg) paramétré par le nom d'un fichier texte
(qui ne contient que des caractères ASCII). L'utilitaire compte les
occurrences de chaque mot du fichier et produit en réponse un fichier
qui contient chaque mot du fichier texte de départ et son nombre
d'occurrences.
\medskip

Plus précisément, est considéré comme un mot toute suite de caractères
contigus délimitée à gauche et à droite par des espaces, des fins de
lignes, des tabulations ou des début ou fin de fichier. Toute suite de
caractères contigus qui contient un chiffre ou tout autre caractère non
alphabétique est ignorée. De plus, dans le comptage des occurrences d'un
mot, on ne fait pas de différence entre majuscules et minuscules.
\medskip

La commande {\tt clm X}, où {\tt X} est le nom d'un fichier texte, créé
un fichier texte {\tt X.clm} où chaque ligne contient un mot en
minuscules suivi d'une espace, suivie du nombre (en base dix)
d'occurrences du mot en question dans {\tt X}. Chaque mot y est mentionné
suivant l'ordre d'apparition de sa première occurrence dans le texte.
Par exemple, à partir du fichier texte {\tt texte} de contenu
\begin{center} \footnotesize
\begin{boxedverbatim}
TROIS Un un un1un deux DeUx
deux deuxdeux un quatre4
\end{boxedverbatim}
\end{center}
la commande {\tt clm texte} créé le fichier {\tt texte.clm} qui contient
\begin{center} \footnotesize
\begin{boxedverbatim}
trois 1
un 3
deux 3
deuxdeux 1
\end{boxedverbatim}
\end{center}
\medskip

Il figurera une option {\tt -a} faisant en sorte de trier les mots du
fichier de sortie dans l'ordre lexicographique. Par exemple, la commande
{\tt clm texte -a} créé le fichier {\tt texte.clm} de contenu
\begin{center} \footnotesize
\begin{boxedverbatim}
deux 3
deuxdeux 1
trois 1
un 3
\end{boxedverbatim}
\end{center}
Il figurera aussi une option {\tt -n} faisant en sorte de trier les mots
de fichier de sortie selon leur nombre d'occurrences de manière
décroissante. Les mots ayant un même nombre d'occurrences apparaissent
dans l'ordre lexicographique. Par exemple, la commande {\tt clm texte -n}
créé le fichier {\tt texte.clm} de contenu
\begin{center} \footnotesize
\begin{boxedverbatim}
deux 3
un 3
deuxdeux 1
trois 1
\end{boxedverbatim}
\end{center}
\medskip

Tout ceci constitue la première partie de l'application. Dans cette
deuxième partie, nous souhaitons mettre en place un mécanisme pour
renseigner sur les mots qui suivent ou qui précèdent d'autres. En
pratique, ceci permet de détecter des erreurs de grammaire.
\medskip

Ainsi, la commande {\tt clm X -s MOT}, où {\tt X} est le nom d'un fichier
texte et {\tt MOT} un mot, créé un fichier texte {\tt X.clm} où chaque
ligne contient chaque mot {\tt S} qui suit {\tt MOT} dans {\tt X} suivi
d'une espace, suivie du nombre (en base dix) d'occurrences de {\tt S}
situées après {\tt MOT}. Chaque mot y est mentionné suivant l'ordre
d'apparition de sa première occurrence dans le texte qui suit une
occurrence de {\tt MOT}. Par exemple, à partir du fichier texte
{\tt texte} de contenu
\begin{center} \footnotesize
\begin{boxedverbatim}
ab aaa abc a ab
aaa ab a ab aaa
ab ab b b ab a ab b
\end{boxedverbatim}
\end{center}
la commande {\tt clm texte -s ab} créé le fichier {\tt texte.clm} qui
contient
\begin{center} \footnotesize
\begin{boxedverbatim}
aaa 3
a 2
ab 1
b 2
\end{boxedverbatim}
\end{center}
Comme dans la première partie de la spécification, on ajoutera les options
{\tt -a} et {\tt -n} pour trier les mots qui apparaissent dans le
fichier résultat. Par exemple, {\tt clm texte -s ab -n} affiche
\begin{center} \footnotesize
\begin{boxedverbatim}
aaa 3
a 2
b 2
ab 1
\end{boxedverbatim}
\end{center}

Il figurera également dans l'application une option {\tt -p} qui possède
la même spécification que l'option {\tt -s} mais qui prend en compte
les mots qui précèdent le mot spécifié dans la commande.
\medskip

La dernière partie de l'application est la plus intéressante à utiliser
et à mettre en \oe uvre. Nous souhaitons maintenant recenser les
expressions composées d'un nombre fixé de mots. Plus précisément, est
considérée comme une expression toute suite de mots séparés par une ou
plusieurs espaces ou une ou plusieurs fins de lignes.
\medskip

Il est ainsi question de disposer d'une commande {\tt clm X -e N}, où
{\tt X} est le nom d'un fichier texte et {\tt N} un entier strictement
positif exprimé en base dix. Cette commande créé un fichier texte
{\tt X.clm} où chaque ligne contient chaque expression composée de
{\tt N} mots extraite de {\tt X} suivie d'une espace, suivie du nombre
(en base dix) d'occurrences de l'expression en question dans {\tt X}.
Par exemple, à partir du fichier texte {\tt texte} de contenu
\begin{center} \footnotesize
\begin{boxedverbatim}
un un un deux trois trois
un un trois deux trois
\end{boxedverbatim}
\end{center}
la commande {\tt clm texte -e 2} créé le fichier {\tt texte.clm} qui
contient
\begin{center} \footnotesize
\begin{boxedverbatim}
un un 3
un deux 1
deux trois 2
trois trois 1
trois un 1
un trois 1
trois deux 1
\end{boxedverbatim}
\end{center}
Comme dans les autres parties de la spécification, on ajoutera les options
{\tt -a} et {\tt -n} pour trier les mots qui apparaissent dans le
fichier résultat. Par exemple, {\tt clm texte -e 2 -a} affiche
\begin{center} \footnotesize
\begin{boxedverbatim}
deux trois 2
trois deux 1
trois trois 1
trois un 1
un deux 1
un trois 1
un un 3
\end{boxedverbatim}
\end{center}
\medskip

\begin{center} *** \\ * \end{center}
\bigskip
\bigskip

%%%%%%%%%%%%%%%%%%%%%%%%%%%%%%%%%%%%%%%%%%%%%%%%%%%%%%%%%%%%%%%%%%%%%%%%
%%%%%%%%%%%%%%%%%%%%%%%%%%%%%%%%%%%%%%%%%%%%%%%%%%%%%%%%%%%%%%%%%%%%%%%%
%%%%%%%%%%%%%%%%%%%%%%%%%%%%%%%%%%%%%%%%%%%%%%%%%%%%%%%%%%%%%%%%%%%%%%%%
\begin{Exercice} {\bf (Conception du projet)}\smallskip

L'objectif de cet exercice est de concevoir une architecture viable pour
le projet présenté.
\smallskip

\begin{enumerate}
    \item Lire attentivement {\bf l'intégralité du sujet} avant de
    commencer à répondre aux questions. La description du sujet constitue
    une spécification de projet.
    \smallskip

    \item Une fois ceci fait, proposer un découpage en modules cohérent
    du projet. Pour chaque module proposé, décrire les types qu'il apporte
    ainsi que ses objectifs principaux.
    \smallskip

    \item Au fil de l'écriture du projet (dans le travail demandé par les
    prochains exercices), il est possible de se rendre compte que le
    découpage initialement prévu n'est pas complet ou adapté. Si c'est
    le cas, mentionner l'historique de ses modifications.
    \smallskip

    \item Maintenir un {\tt Makefile} pour compiler le projet.
\end{enumerate}
\end{Exercice}
\bigskip

%%%%%%%%%%%%%%%%%%%%%%%%%%%%%%%%%%%%%%%%%%%%%%%%%%%%%%%%%%%%%%%%%%%%%%%%
%%%%%%%%%%%%%%%%%%%%%%%%%%%%%%%%%%%%%%%%%%%%%%%%%%%%%%%%%%%%%%%%%%%%%%%%
%%%%%%%%%%%%%%%%%%%%%%%%%%%%%%%%%%%%%%%%%%%%%%%%%%%%%%%%%%%%%%%%%%%%%%%%
\begin{Exercice} {\bf (Tests)}\smallskip
\label{exo:test}

Les tests sont des éléments de première importance dans le processus
d'écriture d'un projet. Ils sont également primordiaux pour la
maintenance d'un projet. Ils servent à s'assurer que chaque partie du
programme fonctionne et sont utiles pour capturer des erreurs de
programmation à leur source.
\smallskip

\begin{enumerate}
    \item Avant de commencer l'écriture du projet, créer un module
    {\tt Test}. Ce module va inclure tous les autres modules (exception
    faite pour le module principal) et va servir à tester les fonctions
    du projet. Il contient pour le moment une fonction {\tt test} de
    corps vide, sans paramètre et à type de retour {\tt int}.
    \smallskip

    \item Au cours de l'écriture du projet, pour chaque fonction
    {\tt fct} nouvellement écrite, ajouter dans le module {\tt Test} une
    fonction {\tt test\_fct} qui réalise des tests cohérents de la
    fonction {\tt fct}. Les tests sont construits en appelant {\tt fct}
    avec des arguments pour lesquels la réponse est connue. Le test
    consiste à vérifier si la fonction répond correctement. La fonction
    {\tt test\_fct} renvoie {\tt 1} si tous les tests qu'elle exécute se
    sont bien déroulés et {\tt 0} sinon. Pour chaque test mal déroulé,
    la fonction affiche sur la sortie standard des informations pour
    identifier le test problématique.
    \smallskip

    \item La fonction {\tt test} du module {\tt Test} est sa seule
    fonction visible depuis l'extérieur. Elle appelle chaque fonction
    de test et vérifie que toutes se déroulent correctement. Elle
    renvoie ainsi {\tt 1} si toutes les fonctions de test renvoient
    {\tt 1} et {\tt 0} sinon.
    \smallskip

    \item Le programme dispose d'une option {\tt --test} qui exécute
    la fonction {\tt test} du module {\tt Test}. Il est important, lors
    de la phase de programmation du projet ou encore lors d'une de ses
    mises-à-jour, de lancer ces tests pour vérifier que tout fonctionne
    comme attendu.
\end{enumerate}
\end{Exercice}
\bigskip
\bigskip

{\bf Remarque importante 1~:} {\it le bon découpage en modules du projet
est une étape essentielle. Cette étape, si elle est bien menée, fait
gagner beaucoup de temps et facilite l'écriture du projet. Il est ainsi
fortement déconseillé de commencer à écrire du code avant d'avoir conçu
l'architecture du projet.}
\bigskip
\bigskip

{\bf Remarque importante 2~:} {\it l'exercice~\ref{exo:test} compte
beaucoup dans l'évaluation de ce TP. Il est important de proposer pour
chaque fonction un jeu de tests cohérent et complet. Il est possible que
l'écriture des tests occupent une proportion significative et majoritaire
de temps de développement.}
\bigskip
\bigskip

{\bf Remarque importante 3~:} {\it il n'est pas question dans ce TP
d'utiliser les structures de données et les algorithmes les plus
efficaces qui répondent au problème (il est en effet possible de
proposer des solutions très efficaces en étant astucieux). L'objectif
principal est de concevoir un projet sans faille dans son architecture
globale. Cependant, un programme efficace sera apprécié.}

\end{document}
