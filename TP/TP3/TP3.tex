% Auteur : Michel Landschoot, Samuele Giraudo
% Création : 18/09/12
% Modifications :  aout 2014, janvier 2016

\documentclass[12pt]{article}

\usepackage[utf8]{inputenc}
\usepackage[french]{babel}
\usepackage{amsmath,amsthm,amsfonts,amssymb}
\usepackage{lmodern}
\usepackage[top=2.4cm,bottom=2.4cm,left=2cm,right=2cm]{geometry}
\usepackage{hyperref}
\usepackage{multicol}
\usepackage{enumerate}
\usepackage{listings}
\usepackage[dvipsnames]{xcolor}
\usepackage{tikz}

%\author{Samuele Giraudo}
\date{}
\title{{\bf Perfectionnement à la programmation en {\sf C}} \\
    Fiche de TP 4 \\
    {\small L2 Informatique 2015-2016} \\
    {\it \small Le jeu du serpent}}

% Couleurs.
\definecolor{Noir}{RGB}{0,0,0}
\definecolor{Rouge}{RGB}{205,35,38}
\definecolor{Bleu}{RGB}{2,60,195}
\definecolor{Bleu1}{RGB}{121,176,197}
\definecolor{Vert}{RGB}{23,103,1}
\definecolor{Orange}{RGB}{245,113,15}
\definecolor{Blanc}{RGB}{255,255,255}
\definecolor{Marron}{RGB}{193,88,50}
\definecolor{Jaune}{RGB}{255,180,30}
\definecolor{Violet}{RGB}{181,18,225}

\theoremstyle{definition}
\newtheorem{Exercice}{Exercice}

% Configuration de listings.
\lstset{
    language=c,
    basicstyle=\ttfamily\footnotesize,
    identifierstyle=\color{Mahogany},
    keywordstyle=\color{NavyBlue},
    stringstyle=\color{Emerald},
    commentstyle=\it\color{Gray},
    columns=flexible,
    tabsize=4,
    extendedchars=true,
    showspaces=false,
    numbers=left,
    numberstyle=\tiny,
    breaklines=true,
    breakautoindent=true,
    captionpos=b,
    showstringspaces=true
}

\tikzstyle{Quadr} = [rectangle,draw=Marron!100,fill=Marron!15,
thick,inner sep=0pt,minimum size=10mm,line width=1pt,font=\Huge]
\tikzstyle{Pomme} = [circle,draw=Rouge!100,fill=Rouge!60,
thick,inner sep=0pt,minimum size=6mm,line width=1pt,font=\Huge]
\tikzstyle{Tete} = [circle,draw=Noir!100,fill=Noir!60,
thick,inner sep=0pt,minimum size=6mm,line width=1pt,font=\Huge]
\tikzstyle{Corps} = [rectangle,draw=Vert!100,fill=Vert!60,
thick,inner sep=0pt,minimum size=6mm,line width=1pt,font=\Huge]

\begin{document}

\maketitle

Ce TP se déroule en une seule séance et est à faire par binômes.
Le travail réalisé doit être envoyé au plus tard exactement une
semaine après le début de la séance de TP. Il sera disposé sur
la plate-forme prévue à cet effet et constitué des programmes
répondant aux questions et des éventuels fichiers annexes qui
peuvent être demandés.
\bigskip
\bigskip

L'objectif de ce TP est de programmer le très célèbre {\em jeu du serpent}.
Le principe est de piloter un serpent (vu du dessus) pour attraper des
pommes et grandir progressivement. La difficulté principale consiste en
ce que le serpent est perpétuellement en mouvement et le contact avec
les bords de l'écran ou avec lui-même provoque la défaite.
\smallskip

Le programme à réaliser se décrit de la manière suivante. Au lancement
du jeu, un quadrillage régulier (de dimensions modulables) est affiché
et un nombre raisonnable de pommes y sont disséminées aléatoirement. De
plus, un serpent de petite taille figure au centre du quadrillage. Il est
constitué d'une tête (qui occupe un carré du quadrillage) et d'un corps
(qui occupe plusieurs carrés). Les pommes sont dessinées par des disques
rouges, la tête est dessinée par un disque noir et son corps par
des carrés verts. Une illustration d'une situation d'une situation
de jeu est donnée par la figure~\ref{fig:quadrillage}.
\begin{figure}[ht]
    \centering
    \scalebox{.6}{\begin{tikzpicture}
        \foreach \x in {0, ..., 10} {
            \foreach \y in {0, ..., 7} {
                \node[Quadr](\x,\y)at(\x,\y){};
            }
        }
        \node[Pomme]at(2,3){};
        \node[Pomme]at(3,4){};
        \node[Pomme]at(8,6){};
        \node[Tete]at(6,2){};
        \node[Corps]at(6,1){};
        \node[Corps]at(6,0){};
        \node[Corps]at(5,0){};
        \node[Corps]at(4,0){};
        \node[Corps]at(3,0){};
        \node[Corps]at(2,0){};
    \end{tikzpicture}}
    \caption{Une situation de jeu du serpent qui respecte les conventions
    graphiques établies. Le quadrillage est de dimensions $11 \times 8$.}
    \label{fig:quadrillage}
\end{figure}
\smallskip

Le serpent se déplace à une vitesse d'une case par seconde selon
une direction (nord, sud, est ou ouest) et l'utilisateur le pilote
à l'aide des flèches directionnelles pour modifier sa direction. À
chaque seconde, la tête du serpent se déplace vers la case indiquée
par sa direction et la dernière case de son corps disparaît. La
figure~\ref{fig:deplacement} illustre ce mécanisme de déplacement.
\begin{figure}[ht]
    \centering
    \begin{equation*}
    \begin{split}
    \scalebox{.6}{\begin{tikzpicture}
        \foreach \x in {0, ..., 6} {
            \foreach \y in {0, ..., 4} {
                \node[Quadr](\x,\y)at(\x,\y){};
            }
        }
        \node[Tete]at(4,2){};
        \node[Corps]at(3,2){};
        \node[Corps]at(3,1){};
        \node[Corps]at(2,1){};
        \node[Corps]at(1,1){};
        \node[Corps]at(1,0){};
    \end{tikzpicture}}
    \end{split}
    \begin{split} \qquad \longrightarrow \qquad \end{split}
    \begin{split}
    \scalebox{.6}{\begin{tikzpicture}
        \foreach \x in {0, ..., 6} {
            \foreach \y in {0, ..., 4} {
                \node[Quadr](\x,\y)at(\x,\y){};
            }
        }
        \node[Tete]at(5,2){};
        \node[Corps]at(4,2){};
        \node[Corps]at(3,2){};
        \node[Corps]at(3,1){};
        \node[Corps]at(2,1){};
        \node[Corps]at(1,1){};
    \end{tikzpicture}}
    \end{split}
    \end{equation*}
    \caption{Le déplacement d'un serpent d'une case à une autre.
    Le serpent se dirige vers l'est. Le 1\ier{} quadrillage
    est la situation à un temps $t$ et le 2\ieme{}, au temps $t + 1$.}
    \label{fig:deplacement}
\end{figure}
\smallskip

Lorsque le serpent arrive sur une case qui contient une pomme, celle-ci
disparaît et le corps du serpent grandit en évitant de faire disparaître
la dernière case de son corps lors du déplacement. La
figure~\ref{fig:croissance} illustre ce mécanisme de croissance.
\begin{figure}[ht]
    \centering
    \begin{equation*}
    \begin{split}
    \scalebox{.6}{\begin{tikzpicture}
        \foreach \x in {0, ..., 6} {
            \foreach \y in {0, ..., 4} {
                \node[Quadr](\x,\y)at(\x,\y){};
            }
        }
        \node[Tete]at(4,2){};
        \node[Corps]at(3,2){};
        \node[Corps]at(3,1){};
        \node[Corps]at(2,1){};
        \node[Corps]at(1,1){};
        \node[Corps]at(1,0){};
        \node[Pomme]at(5,2){};
    \end{tikzpicture}}
    \end{split}
    \begin{split} \qquad \longrightarrow \qquad \end{split}
    \begin{split}
    \scalebox{.6}{\begin{tikzpicture}
        \foreach \x in {0, ..., 6} {
            \foreach \y in {0, ..., 4} {
                \node[Quadr](\x,\y)at(\x,\y){};
            }
        }
        \node[Tete]at(5,2){};
        \node[Corps]at(4,2){};
        \node[Corps]at(3,2){};
        \node[Corps]at(3,1){};
        \node[Corps]at(2,1){};
        \node[Corps]at(1,1){};
        \node[Corps]at(1,0){};
    \end{tikzpicture}}
    \end{split}
    \end{equation*}
    \caption{La croissance d'un serpent provoquée par la consommation
    d'une pomme. Le serpent se dirige vers l'est. Le 1\ier{} quadrillage
    est la situation à un temps $t$ et le 2\ieme{}, au temps $t + 1$.}
    \label{fig:croissance}
\end{figure}
Lorsque le serpent arrive sur une case occupée par son corps ou bien
s'il sort du quadrillage, la partie se termine et le score (qui est le
nombre de pommes mangées) est affiché.
\bigskip
\bigskip

{\bf Conseils~:}
\begin{itemize}
    \item pour chaque fonction programmée, s'interroger sur les arguments
    qui peuvent poser des problèmes. Capturer ces cas à l'aide de
    pré-assertions~;
    \item utiliser au maximum les fonctions déjà programmées dans les
    nouvelles à écrire. Il faut éviter au maximum la duplication de code~;
    \item il ne suffit pas de programmer uniquement les fonctions qui sont
    demandées. Pour mener le sujet à bien, des fonctions supplémentaires
    et non mentionnées explicitement dans le sujet seront utiles. Soigner
    leur nom~;
    \item dès qu'une fonction est écrite, la tester de manière à
    recouvrir tous les cas significatifs possibles et conserver les tests
    (le but étant de les relancer pour vérifier l'intégrité du
    programme lors de modifications futures)~;
    \item commenter le code en évitant absolument les commentaires inutiles.
    Il faut commenter chaque fonction écrite, juste avant son prototype.
    Il faut y mentionner les trois points suivants~: (1) une description
    du rôle de la fonction, (2) une description de ses paramètres et (3)
    une description de ce qu'elle renvoie~;
    \item préférer la concision au maximum. Un code concis et lisible
    possède une grande valeur.
\end{itemize}
\bigskip
\bigskip

%%%%%%%%%%%%%%%%%%%%%%%%%%%%%%%%%%%%%%%%%%%%%%%%%%%%%%%%%%%%%%%%%%%%%%%%
%%%%%%%%%%%%%%%%%%%%%%%%%%%%%%%%%%%%%%%%%%%%%%%%%%%%%%%%%%%%%%%%%%%%%%%%
%%%%%%%%%%%%%%%%%%%%%%%%%%%%%%%%%%%%%%%%%%%%%%%%%%%%%%%%%%%%%%%%%%%%%%%%
\begin{Exercice} {\bf (Conception du projet)}\smallskip

L'objectif de cet exercice est de concevoir une architecture viable pour
le projet présenté.
\begin{enumerate}
    \item Lire attentivement l'intégralité du sujet avant de commencer
    à répondre aux questions. La description du sujet constitue une 
    spécification de projet. 
    \smallskip
    
    \item Une fois ceci fait, proposer un découpage en modules cohérent 
    du projet. Pour chaque module proposé, décrire les types qu'il apporte 
    ainsi que ses objectifs principaux.
    \smallskip
    
    \item Au fil de l'écriture du projet (dans le travail demandé par les 
    prochains exercices), il est possible de se rendre compte que le 
    découpage initialement prévu n'est pas complet ou adapté. Si c'est 
    le cas, mentionner l'historique de ses modifications.
    \smallskip
    
    \item Maintenir un {\tt Makefile} pour compiler le projet.
\end{enumerate}
\end{Exercice}
\bigskip

%%%%%%%%%%%%%%%%%%%%%%%%%%%%%%%%%%%%%%%%%%%%%%%%%%%%%%%%%%%%%%%%%%%%%%%%
%%%%%%%%%%%%%%%%%%%%%%%%%%%%%%%%%%%%%%%%%%%%%%%%%%%%%%%%%%%%%%%%%%%%%%%%
%%%%%%%%%%%%%%%%%%%%%%%%%%%%%%%%%%%%%%%%%%%%%%%%%%%%%%%%%%%%%%%%%%%%%%%%
\begin{Exercice} {\bf (Définitions des types)}\smallskip
\begin{enumerate}
    \item Définir un type {\tt Case} qui permet de représenter une
    case d'un quadrillage.
    \smallskip

    \item Définir un type {\tt Direction} qui permet de représenter
    les quatre directions cardinales.
    \smallskip

    \item Définir un type {\tt Serpent} qui permet de représenter un
    serpent. Un serpent est codé par une suite (un tableau dynamique ou
    bien une liste) de cases, la 1\iere{} étant celle de la tête et les
    autres celles du corps et une direction.
    \smallskip

    \item Définir un type {\tt Pomme}. Une pomme est codée par la
    case qui la contient.
    \smallskip

    \item Définir un type {\tt Monde} qui permet de représenter une
    situation de jeu. Elle comprend les dimensions du quadrillage, le
    serpent, les positions des pommes présentes et le nombre de pommes
    mangées.
\end{enumerate}
\end{Exercice}
\bigskip

%%%%%%%%%%%%%%%%%%%%%%%%%%%%%%%%%%%%%%%%%%%%%%%%%%%%%%%%%%%%%%%%%%%%%%%%
%%%%%%%%%%%%%%%%%%%%%%%%%%%%%%%%%%%%%%%%%%%%%%%%%%%%%%%%%%%%%%%%%%%%%%%%
%%%%%%%%%%%%%%%%%%%%%%%%%%%%%%%%%%%%%%%%%%%%%%%%%%%%%%%%%%%%%%%%%%%%%%%%
\begin{Exercice} {\bf (Mécaniques du jeu)}\smallskip
\begin{enumerate}
    \item Écrire une fonction
\begin{lstlisting}
Pomme pomme_gen_alea(int n, int m);
\end{lstlisting}
    qui renvoie une pomme générée aléatoirement de sorte qu'elle tienne
    dans un quadrillage ${\tt n} \times {\tt m}$.
    \smallskip

    \item Écrire une fonction
\begin{lstlisting}
void ajouter_pomme_monde(Monde *mon);
\end{lstlisting}
    qui ajoute une nouvelle pomme au monde {\tt mon}. Cette fonction
    doit utiliser la fonction précédente. Attention, il faut veiller
    à ce que la nouvelle pomme n'apparaisse pas sur une pomme déjà
    existante ou bien sur le serpent.
    \smallskip

    \item Écrire une fonction
\begin{lstlisting}
Serpent init_serpent(int n, int m);
\end{lstlisting}
    qui renvoie un serpent dont la tête est située au centre (ou dans
    l'une des cases centrales suivant la parité de {\tt n} et {\tt m})
    d'un quadrillage ${\tt n} \times {\tt m}$. Il dispose d'exactement
    une partie pour son corps, située à gauche de sa tête. Sa direction
    est vers l'est.
    \smallskip

    \item Écrire une fonction
\begin{lstlisting}
Monde init_monde(int n, int m, int nb_pommes);
\end{lstlisting}
    qui renvoie un monde dans une configuration initiale pour le jeu.
    Le quadrillage est de dimensions ${\tt n} \times {\tt m}$ et
    {\tt nb\_pommes} y sont disposées. Le serpent se trouve dans
    sa configuration initiale fournie par la fonction {\tt init\_serpent}.
    \smallskip

    \item Écrire une fonction
\begin{lstlisting}
int deplacer_serpent(Monde *mon);
\end{lstlisting}
    qui modifie le serpent du monde {\tt mon} de sorte à le faire avancer
    suivant sa direction vers une case vide. Cette fonction renvoie {\tt 1}
    si la tête du serpent arrive bien sur une case vide du quadrillage.
    Si la tête du serpent arrive sur une case non vide (occupée par une
    pomme ou bien par une partie de son corps ou bien à l'extérieur du
    quadrillage), cette fonction ne modifie par le serpent et renvoie {\tt 0}.
    \smallskip

    \item Écrire une fonction
\begin{lstlisting}
int manger_pomme_serpent(Monde *mon);
\end{lstlisting}
    qui modifie le serpent du monde {\tt mon} de sorte à le faire avancer
    suivant sa direction vers une case occupée par une pomme. Cette fonction
    renvoie {\tt 1} si la tête du serpent arrive bien sur une case occupée
    par une pomme du quadrillage. Il faut supprimer la pomme ainsi
    mangée du monde {\tt mon}. Si la tête du serpent arrive sur une
    case qui n'est pas occupée par une pomme (occupée par une partie
    de son corps ou bien une case vide ou bien à l'extérieur du quadrillage),
    cette fonction ne modifie par le serpent et renvoie~{\tt 0}.
    \smallskip

    \item Écrire une fonction
\begin{lstlisting}
int mort_serpent(Monde *mon);
\end{lstlisting}
    qui renvoie {\tt 1} si le serpent du monde {\tt mon} va se trouver
    dans une situation perdante (la tête du serpent sort du quadrillage
    ou bien se trouve une case occupée par son corps) à l'issue de son
    déplacement imposé par sa direction et {\tt 0} sinon.
\end{enumerate}
\end{Exercice}
\bigskip

%%%%%%%%%%%%%%%%%%%%%%%%%%%%%%%%%%%%%%%%%%%%%%%%%%%%%%%%%%%%%%%%%%%%%%%%
%%%%%%%%%%%%%%%%%%%%%%%%%%%%%%%%%%%%%%%%%%%%%%%%%%%%%%%%%%%%%%%%%%%%%%%%
%%%%%%%%%%%%%%%%%%%%%%%%%%%%%%%%%%%%%%%%%%%%%%%%%%%%%%%%%%%%%%%%%%%%%%%%
\begin{Exercice} {\bf (Graphisme)}\smallskip

Toutes les fonctions d'affichage suivantes utilisent la bibliothèque
graphique {\sf MLV} et agissent sur la fenêtre graphique ouverte.

\begin{enumerate}
    \item Écrire une fonction
\begin{lstlisting}
void afficher_quadrillage(Monde *mon);
\end{lstlisting}
    qui affiche le quadrillage sous-jacent au monde {\tt mon}.
    \smallskip

    \item Écrire une fonction
\begin{lstlisting}
void afficher_pomme(Pomme *pom);
\end{lstlisting}
    qui affiche la pomme {\tt pom} selon sa position.
    \smallskip

    \item Écrire une fonction
\begin{lstlisting}
void afficher_serpent(Serpent *ser);
\end{lstlisting}
    qui affiche le serpent {\tt ser} selon sa position.
    \smallskip

    \item Écrire une fonction
\begin{lstlisting}
void afficher_monde(Monde *mon);
\end{lstlisting}
    qui affiche le monde {\tt mon}. Ceci dessine la situation de jeu
    au complet~: le quadrillage, le serpent, les pommes et le nombre
    de pommes mangées.
\end{enumerate}
\end{Exercice}
\bigskip

%%%%%%%%%%%%%%%%%%%%%%%%%%%%%%%%%%%%%%%%%%%%%%%%%%%%%%%%%%%%%%%%%%%%%%%%
%%%%%%%%%%%%%%%%%%%%%%%%%%%%%%%%%%%%%%%%%%%%%%%%%%%%%%%%%%%%%%%%%%%%%%%%
%%%%%%%%%%%%%%%%%%%%%%%%%%%%%%%%%%%%%%%%%%%%%%%%%%%%%%%%%%%%%%%%%%%%%%%%
\begin{Exercice} {\bf (Assemblage du jeu)}\smallskip

Utiliser les tout ce qui a été fait jusqu'à présent pour construire le
programme {\tt Serpent.c} qui permet de jouer au jeu du serpent.
L'algorithme suivant est utilisé~:
\smallskip

\begin{enumerate}[(I)]
    \item Initialiser le monde~;
    \item afficher le monde et attendre qu'une touche soit pressée pour
    commencer~;
    \item \label{item:boucle_principale}
    tant que la partie n'est pas terminée~:
    \begin{enumerate}[(1)]
        \item faire évoluer le serpent en fonction de sa direction~:
        \begin{enumerate}[(a)]
            \item vérifier si le serpent va mourir ({\tt mort\_serpent}).
            Si c'est le cas, la partie est terminée~;
            \item sinon, tenter de faire avancer le serpent vers une case vide
            ({\tt deplacer\_serpent})~;
            \item sinon, tenter de faire manger une pomme au serpent
            ({\tt manger\_pomme\_serpent}).
            Dans ce cas, ajouter une nouvelle pomme au monde~;
        \end{enumerate}
        \item mettre à jour l'affichage (serpent, pommes et score)~;
        \item mettre à jour la direction du serpent en fonction de la
        touche pressée~;
    \end{enumerate}
    \item afficher que la partie est terminée et le score.
\end{enumerate}
\medskip

Il ne faut pas oublier de temporiser de manière adéquate cet
algorithme. En effet, un tour de boucle
principale \eqref{item:boucle_principale} doit durer une seconde.
\end{Exercice}
\bigskip

%%%%%%%%%%%%%%%%%%%%%%%%%%%%%%%%%%%%%%%%%%%%%%%%%%%%%%%%%%%%%%%%%%%%%%%%
%%%%%%%%%%%%%%%%%%%%%%%%%%%%%%%%%%%%%%%%%%%%%%%%%%%%%%%%%%%%%%%%%%%%%%%%
%%%%%%%%%%%%%%%%%%%%%%%%%%%%%%%%%%%%%%%%%%%%%%%%%%%%%%%%%%%%%%%%%%%%%%%%
\begin{Exercice} {\bf (Améliorations)}\smallskip

{\it Cet exercice est optionnel, il n'est a envisager que lorsque
tout le reste fonctionne parfaitement.}
\smallskip

\begin{enumerate}
    \item \label{item:ajout_parametres}
    Rendre le programme paramétrable~: le joueur peut choisir
    les dimensions du quadrillage, le nombre de pommes, la taille
    initiale du serpent et la durée d'un tour de boucle (exprimée
    en ms). Ces paramètres doivent figurer dans un fichier de configuration
    {\tt Serpent.ini} situé dans le même répertoire que l'exécutable.
    Il est constitué de lignes de la forme
\begin{small}
\begin{verbatim}
largeur = 32
hauteur = 16
nombre_pommes = 4
taille_serpent = 7
duree_tour = 2500
\end{verbatim}
\end{small}
    \smallskip

    \item Mettre au point un système pour mettre en pause la partie
    en appuyant sur une touche dédiée.
    \smallskip

    \item Mettre au point un système de sauvegarde des meilleurs scores.
    Au lancement du jeu, figurera un menu permettant de visualiser
    les meilleurs scores, ligne par ligne. Une ligne contient le nom
    du joueur, les paramètres de la partie (voir \ref{item:ajout_parametres})
    et le score obtenu.
    \smallskip

    \item Ajouter la possibilité de générer des pommes empoisonnées
    qui provoquent la défaite si mangées. Au moins une pomme parmi
    celles présentes dans le monde ne doit pas être empoisonnée.
\end{enumerate}

\end{Exercice}
\bigskip


\end{document}
