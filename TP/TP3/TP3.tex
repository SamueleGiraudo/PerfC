% Auteur : Grégory Châtel Samuele Giraudo
% Création : 18/09/12
% Modifications : 02/10/12, oct. 2013 aout 2014, janvier 2016

\documentclass[12pt]{article}

\usepackage[utf8]{inputenc}
\usepackage[french]{babel}
\usepackage{amsmath,amsthm,amsfonts,amssymb}
\usepackage{lmodern}
\usepackage[top=2.4cm,bottom=2.4cm,left=2cm,right=2cm]{geometry}
\usepackage{hyperref}
\usepackage{multicol}
\usepackage{enumitem}
\usepackage{listings}
\usepackage[dvipsnames]{xcolor}
\usepackage[ps]{skak}

\usepackage{tikz}

%\author{Samuele Giraudo}
\date{}
\title{{\bf Perfectionnement à la programmation en {\sf C}} \\
    Fiche de TP 3 \\
    {\small L2 Informatique 2015-2016} \\
    {\it \small Le problème des huit dames}}
    
% Couleurs.
\definecolor{Noir}{RGB}{0,0,0}
\definecolor{Rouge}{RGB}{205,35,38}
\definecolor{Bleu}{RGB}{2,60,195}
\definecolor{Bleu1}{RGB}{121,176,197}
\definecolor{Vert}{RGB}{23,103,1}
\definecolor{Orange}{RGB}{245,113,15}
\definecolor{Blanc}{RGB}{255,255,255}
\definecolor{Marron}{RGB}{193,88,50}
\definecolor{Jaune}{RGB}{255,180,30}
\definecolor{Violet}{RGB}{181,18,225}

\theoremstyle{definition}
\newtheorem{Exercice}{Exercice}

% Configuration de listings.
\lstset{
    language=c,
    basicstyle=\ttfamily\footnotesize,
    identifierstyle=\color{Mahogany},
    keywordstyle=\color{NavyBlue},
    stringstyle=\color{Emerald},
    commentstyle=\it\color{Gray},
    columns=flexible,
    tabsize=4,
    extendedchars=true,
    showspaces=false,
    numbers=left,
    numberstyle=\tiny,
    breaklines=true,
    breakautoindent=true,
    captionpos=b,
    showstringspaces=true
}

\begin{document}

\maketitle

Ce TP se déroule en deux séances et est à faire par binômes.
Le travail réalisé doit être envoyé au plus tard exactement une
semaine après le début de la 2\ieme{} séance de TP traitant ce sujet. Il
sera disposé sur la plate-forme prévue à cet effet et constitué des
programmes répondant aux questions et des éventuels fichiers annexes qui
peuvent être demandés.
\bigskip
\bigskip

Un célèbre problème d'échecs consiste à placer huit dames
(aux échecs, on dit \og dame \fg{} et non pas \og reine \fg{}~!) sur un
échiquier sans qu'aucune n'en menace une autre. Voici, par exemple, une
configuration gagnante~:
\fenboard{3Q4/6Q1/Q7/7Q/4Q3/1Q6/5Q2/2Q5 w - - 0 1}
\begin{center}\showboard\end{center}
\smallskip

L'objectif de ce TP est de réaliser une interface graphique permettant
à un utilisateur de placer à la souris des dames sur un échiquier. Le
programme signale si le placement est gagnant ou non. Pour mener tout
ceci à bien, il sera question de manipulation de bits et d'opérateurs
sur les bits.
\bigskip
\bigskip

{\bf Conseils~:}
\begin{itemize}
    \item pour chaque fonction programmée, s'interroger sur les arguments
    qui peuvent poser des problèmes. Capturer ces cas à l'aide de
    pré-assertions~;
    \item utiliser au maximum les fonctions déjà programmées dans les
    nouvelles à écrire. Il faut éviter au maximum la duplication de code~;
    \item il ne suffit pas de programmer uniquement les fonctions qui sont
    demandées. Pour mener le sujet à bien, des fonctions supplémentaires
    et non mentionnées explicitement dans le sujet seront utiles. Soigner
    leur nom~;
    \item dès qu'une fonction est écrite, la tester de manière à
    recouvrir tous les cas significatifs possibles et conserver les tests
    (le but étant de les relancer pour vérifier l'intégrité du
    programme lors de modifications futures)~;
    \item commenter le code en évitant absolument les commentaires inutiles.
    Il faut commenter chaque fonction écrite, juste avant son prototype.
    Il faut y mentionner les trois points suivants~: (1) une description
    du rôle de la fonction, (2) une description de ses paramètres et (3)
    une description de ce qu'elle renvoie~;
    \item préférer la concision au maximum. Un code concis et lisible
    possède une grande valeur.
\end{itemize}
\bigskip
\bigskip

%%%%%%%%%%%%%%%%%%%%%%%%%%%%%%%%%%%%%%%%%%%%%%%%%%%%%%%%%%%%%%%%%%%%%%%%
%%%%%%%%%%%%%%%%%%%%%%%%%%%%%%%%%%%%%%%%%%%%%%%%%%%%%%%%%%%%%%%%%%%%%%%%
%%%%%%%%%%%%%%%%%%%%%%%%%%%%%%%%%%%%%%%%%%%%%%%%%%%%%%%%%%%%%%%%%%%%%%%%
\begin{Exercice} {\bf (Conception du projet)}\smallskip

L'objectif de cet exercice est de concevoir une architecture viable pour
le projet présenté.
\begin{enumerate}
    \item Lire attentivement l'intégralité du sujet avant de commencer
    à répondre aux questions. La description du sujet constitue une 
    spécification de projet. 
    \smallskip
    
    \item Une fois ceci fait, proposer un découpage en modules cohérent 
    du projet. Pour chaque module proposé, décrire les types qu'il apporte 
    ainsi que ses objectifs principaux.
    \smallskip
    
    \item Au fil de l'écriture du projet (dans le travail demandé par les 
    prochains exercices), il est possible de se rendre compte que le 
    découpage initialement prévu n'est pas complet ou adapté. Si c'est 
    le cas, mentionner l'historique de ses modifications.
    \smallskip
    
    \item Maintenir un {\tt Makefile} pour compiler le projet.
\end{enumerate}
\end{Exercice}
\bigskip

%%%%%%%%%%%%%%%%%%%%%%%%%%%%%%%%%%%%%%%%%%%%%%%%%%%%%%%%%%%%%%%%%%%%%%%%
%%%%%%%%%%%%%%%%%%%%%%%%%%%%%%%%%%%%%%%%%%%%%%%%%%%%%%%%%%%%%%%%%%%%%%%%
%%%%%%%%%%%%%%%%%%%%%%%%%%%%%%%%%%%%%%%%%%%%%%%%%%%%%%%%%%%%%%%%%%%%%%%%
\begin{Exercice} {\bf (Opérations bit à bit)}\smallskip
\label{ex:operations}

Nous utilisons dans cet exercice les opérateurs bit à bit pour
modifier des mots binaires de façon adéquate. Ces opérateurs vont être
utilisés de manière récurrente dans tout le TP et le but de cet exercice
est de se familiariser avec cette nouvelle sorte d'opérateurs. 
\smallskip

À titre d'exemple, l'expression
\begin{lstlisting}
~(1 << 7);
\end{lstlisting}
a pour valeur l'entier possédant son $8\ieme$ bit à {\tt 0} et les autres 
à {\tt 1}. En effet, la sous-expression {\tt 1 << 7} a pour valeur l'entier 
dont le $8\ieme$ bit est à {\tt 1} et les autres à {\tt 0}. Le fait 
de considérer son complémentaire (opérateur ${\tt \sim}$) produit 
le résultat attendu.
\smallskip

Toutes les questions de cet exercice doivent {\bf être résolues en une ligne} 
en {\sf C} à la manière de l'exemple donné.
\begin{enumerate}
    \item Créer un entier tel que son $15\ieme$ bit est à {\tt 1} et les
    autres sont à {\tt 0}.
    \smallskip

    \item Créer un entier tel que son $14\ieme$ bit est à {\tt 0} et les
    autres sont à {\tt 1}.
    \smallskip

    \item Mettre le $13\ieme$ bit d'un entier {\tt x} à {\tt 1}.
    \smallskip

    \item Mettre le $12\ieme$ bit d'un entier {\tt x} à {\tt 0}.
    \smallskip

    \item Tester si le $11\ieme$ bit d'un entier {\tt x} est un {\tt 1}.
    \smallskip

    \item Tester l'égalité des $10\ieme$ bits de deux entiers {\tt x} et
    {\tt y}.
    \smallskip

    \item Tester si deux entiers {\tt x} et {\tt y} possèdent au moins
    un bit à {\tt 1} en une même position.
\end{enumerate}
\end{Exercice}
\bigskip

%%%%%%%%%%%%%%%%%%%%%%%%%%%%%%%%%%%%%%%%%%%%%%%%%%%%%%%%%%%%%%%%%%%%%%%%
%%%%%%%%%%%%%%%%%%%%%%%%%%%%%%%%%%%%%%%%%%%%%%%%%%%%%%%%%%%%%%%%%%%%%%%%
%%%%%%%%%%%%%%%%%%%%%%%%%%%%%%%%%%%%%%%%%%%%%%%%%%%%%%%%%%%%%%%%%%%%%%%%
\begin{Exercice} {\bf (Représentations des données)}\smallskip
\label{ex:representation}

Le but de cet exercice est de définir la façon dont va être représenté
un échiquier en mémoire. L'information contenue dans un échiquier est
la présence ou l'absence, case par case, de dames.
\smallskip

Un échiquier est un quadrillage de taille $8 \times 8$ dont les cases
sont, en première approche, indexées par des couples lettre/entier. Il
est possible d'\og aplanir \fg{} un échiquier en le voyant comme un
tableau de $64$ cases à une seule dimension. Chaque case est ainsi
représentée par un entier de $0$ à $63$ comme le montre le schéma suivant~:
\fenboard{8/8/8/8/8/8/8/8 w - - 0 1}
\begin{equation*}\begin{split}
    \begin{split}\showboard\end{split}
    \qquad \longrightarrow \qquad
    \renewcommand{\arraystretch}{1.35}
    \begin{split}
    \begin{tabular}{|c|c|c|c|c|c|c|c|} \hline
        56 & 57 & 58 & 59 & 60 & 61 & 62 & 63 \\ \hline
        48 & 49 & 50 & 51 & 52 & 53 & 54 & 55 \\ \hline
        40 & 41 & 42 & 43 & 44 & 45 & 46 & 47 \\ \hline
        32 & 33 & 34 & 35 & 36 & 37 & 38 & 39 \\ \hline
        24 & 25 & 26 & 27 & 28 & 29 & 30 & 31 \\ \hline
        16 & 17 & 18 & 19 & 20 & 21 & 22 & 23 \\ \hline
        8 & 9 & 10 & 11 & 12 & 13 & 14 & 15 \\ \hline
        0 & 1 & 2 & 3 & 4 & 5 & 6 & 7 \\ \hline
    \end{tabular}
    \end{split} \\[.75em]
    \qquad \longrightarrow \qquad
    \begin{tabular}{|c|c|c|c|c|c|c|c|c|c|c|c|c|ccccc|c|c|} \hline
        0 & 1 & 2 & 3 & 4 & 5 & 6 & 7 & 8 & 9 & 10 & 11 & 12 &
        \dots & \dots & \dots & \dots & & 62 & 63 \\ \hline
    \end{tabular}
\end{split}\end{equation*}
\smallskip

Toute case d'un échiquier peut prendre deux états~: occupée par une dame
ou bien libre. On décide de représenter une position d'échecs par un entier
de $64$ bits~: chaque bit code l'état de la case indiquée par sa position
dans l'entier. Un bit à {\tt 1} (resp. {\tt 0}) spécifie que la case
correspondante est occupée par une dame (resp. vide). De plus, le bit de
poids fort de l'entier de $64$ bits code pour la case d'indice $0$. Par
exemple, la position située en première page est codée par l'entier de
$64$ bits suivant~:
\begin{equation*}
{\tt 00100000 00000100 01000000 00001000 00000001 10000000 00000010 00010000}.
\end{equation*}

\begin{enumerate}
    \item Définir un type énuméré {\tt Case} qui contient, de {\tt A1} à
    {\tt H8} toutes les cases d'un échiquier. Il faut que la valeur
    numérique de la constante {\tt A1} soit {\tt 0}, celle de {\tt B1}
    soit {\tt 1}, \dots, et finalement celle de {\tt H8} soit {\tt 63}.
    Utiliser les fonctions rechercher/remplacer de l'éditeur de texte
    pour ne pas perdre de temps ici.
    \smallskip

    \item Définir un type synonyme {\tt Position} qui permet de
    représenter des positions selon la convention expliquée.
    \smallskip

    \item Définir une fonction
\begin{lstlisting}
int est_case_occupee(Position pos, Case c);
\end{lstlisting}
    qui renvoie {\tt 1} si la case {\tt c} de la position {\tt pos} est
    occupée par une dame et {\tt 0} sinon.
\end{enumerate}
\end{Exercice}
\bigskip

%%%%%%%%%%%%%%%%%%%%%%%%%%%%%%%%%%%%%%%%%%%%%%%%%%%%%%%%%%%%%%%%%%%%%%%%
%%%%%%%%%%%%%%%%%%%%%%%%%%%%%%%%%%%%%%%%%%%%%%%%%%%%%%%%%%%%%%%%%%%%%%%%
%%%%%%%%%%%%%%%%%%%%%%%%%%%%%%%%%%%%%%%%%%%%%%%%%%%%%%%%%%%%%%%%%%%%%%%%
\begin{Exercice}{\bf (C\oe ur du programme)}\smallskip

Dans cet exercice, nous allons écrire les fonctions de base permettant
de faire les différents tests nécessaires à la résolution du problème
des huit dames. Nous utiliserons les opérations bit à bit vues dans
l'exercice \ref{ex:operations} et la représentation d'une position
introduite dans l'exercice \ref{ex:representation}.
\begin{enumerate}
    \item Écrire une fonction
\begin{lstlisting}
Position placer_dame(Case c);
\end{lstlisting}
    qui renvoie une position contenant uniquement une dame placée sur
    la case spécifiée par {\tt c}.
    \smallskip

    \item Écrire une fonction
\begin{lstlisting}
void afficher_position(Position pos);
\end{lstlisting}
    qui réalise un affichage de la position {\tt pos} sur la sortie
    standard. La présence d'une pièce sera notée par un {\tt '+'} et
    l'absence, par un {\tt '.'}.
    \smallskip

    \item Une dame {\em menace} toute les cases situées sur sa même
    ligne, sa même colonne et en diagonale. Voici par exemple les
    cases menacées (case contenant une $\times$) par une dame placée
    en d5~:
    \fenboard{8/8/8/3Q4/8/8/8/8 w - - 0 1}
    \begin{center}\showboard\end{center}
    \highlight[x]{c4,b3,a2,e4,f3,g2,h1,c6,b7,a8,e6,f7,g8}
    \highlight[x]{a5,b5,c5,e5,f5,g5,h5}
    \highlight[x]{d6,d7,d8,d4,d3,d2,d1}

    Écrire une fonction
\begin{lstlisting}
Position calculer_cases_attaquees(Case c);
\end{lstlisting}
    qui renvoie une position (et donc un entier de $64$ bits) qui
    contient, pour chaque case menacée par une dame sur la case
    {\tt c}, des bits à {\tt 1} aux endroits correspondants et des
    bits à {\tt 0} ailleurs.
    \smallskip

    {\it Attention~: une dame ne menace pas la case sur laquelle elle
    se trouve.}
    \smallskip

    \item On souhaite à présent, pour gagner en efficacité,
    sauvegarder les résultats de la fonction précédente pour toutes
    les valeurs possibles de son paramètre {\tt c}. Pour cela,
    déclarer une variable globale
\begin{lstlisting}
Position tab_cases_attaquees[64];
\end{lstlisting}
    Ce tableau doit être initialisé à l'aide de la fonction
    {\tt calculer\_cases\_attaquees} avant toute utilisation.
    \smallskip

    {\it Attention~: l'utilisation de variables globales est à
    proscrire dans le cas général. Son utilisation est cependant ici
    justifiée.}
    \smallskip

    \item Écrire une fonction
\begin{lstlisting}
int est_sans_menace_mutuelle(Position pos);
\end{lstlisting}
    qui retourne {\tt 1} si les dames de la position {\tt pos} ne se
    menacent pas mutuellement et {\tt 0} sinon.
    \smallskip

    {\it Indication~: Il est possible d'utiliser l'algorithme
    suivant. On commence par calculer un entier {\tt menaces} de
    $64$ bits qui contient l'ensemble des cases menacées par toutes
    les dames de {\tt pos}. Ensuite, comparer par le bon opérateur
    bit à bit les entiers {\tt menaces} et {\tt pos}. De cette
    comparaison, en déduire le résultat à renvoyer.}
\end{enumerate}
\end{Exercice}
\bigskip

%%%%%%%%%%%%%%%%%%%%%%%%%%%%%%%%%%%%%%%%%%%%%%%%%%%%%%%%%%%%%%%%%%%%%%%%
%%%%%%%%%%%%%%%%%%%%%%%%%%%%%%%%%%%%%%%%%%%%%%%%%%%%%%%%%%%%%%%%%%%%%%%%
%%%%%%%%%%%%%%%%%%%%%%%%%%%%%%%%%%%%%%%%%%%%%%%%%%%%%%%%%%%%%%%%%%%%%%%%
\begin{Exercice}{\bf (Interface graphique)}\smallskip

Le but de cet exercice est de réaliser une interface graphique
utilisant la bibliothèque graphique {\sf MLV} qui permet de s'essayer
au problème des huit dames.

\begin{enumerate}
    \item Écrire une fonction paramétrée par une position et qui
    permet de l'afficher.
    \smallskip

    \item Écrire une fonction dont le rôle est de récupérer les
    coordonnées de la case sélectionnée à la souris par l'utilisateur.
    \smallskip

    \item Écrire une fonction qui permet à l'utilisateur d'essayer
    de résoudre le problème. L'interface indiquera par un moyen de
    quelconque (visuel, sonore, {\em etc.}) si un placement
    illégal (c.-à-d. lorsque deux dames se menacent mutuellement)
    a été fait. Dans le cas où les huit dames ont été placées sans
    qu'elles ne se menacent mutuellement, l'interface indiquera
    que l'utilisateur a gagné.
    \smallskip

    {\em Optionnel}~: on permettra à l'utilisateur d'annuler un
    nombre de coups arbitraire en utilisant un bouton {\em précédent}.
    Cette fonction peut par exemple utiliser un tableau de huit
    positions représentant les différentes étapes de la résolution.
\end{enumerate}
\end{Exercice}
\bigskip

%%%%%%%%%%%%%%%%%%%%%%%%%%%%%%%%%%%%%%%%%%%%%%%%%%%%%%%%%%%%%%%%%%%%%%%%
%%%%%%%%%%%%%%%%%%%%%%%%%%%%%%%%%%%%%%%%%%%%%%%%%%%%%%%%%%%%%%%%%%%%%%%%
%%%%%%%%%%%%%%%%%%%%%%%%%%%%%%%%%%%%%%%%%%%%%%%%%%%%%%%%%%%%%%%%%%%%%%%%
\begin{Exercice}{\bf (Des cavaliers en plus)}\smallskip

{\em Cet exercice est optionnel, il n'est a envisager que lorsque
tout le reste fonctionne parfaitement.}
\smallskip

Un cavalier (aux échecs, on dit \og cavalier \fg{} et non pas
\og cheval \fg{}~!) menace toutes les cases situées \og en L \fg\,
par rapport à celle où il se trouve. Voici par exemple les cases
menacées (cases contenant une $\times$) par un cavalier situé
en d5~:
\fenboard{8/8/8/3N4/8/8/8/8 w - - 0 1}
\begin{center}\showboard\end{center}
\highlight[x]{c7,e7,f6,f4,e3,c3,b4,b6}

\begin{enumerate}
    \item Refaire l'ensemble du projet avec des cavaliers à la place
    de dames. Le but du jeu est maintenant de placer le maximum de
    cavaliers sans qu'aucun ne menace un autre.
    \smallskip

    \item Mélanger maintenant les dames et les cavaliers dans une même
    position. L'utilisateur peut placer, selon ses envies, des dames
    et des cavaliers dans une même position. L'objectif est de
    construire des configurations ayant un maximum de dames et de
    cavaliers qui ne se menacent pas mutuellement.
    \smallskip

    {\it Indication~: une position est maintenant représentée non
    pas par un seul nombre de $64$ bits mais par deux. L'un contient
    les cases occupées par les dames tandis que l'autre contient
    les cases occupées par les cavaliers.}
\end{enumerate}
\end{Exercice}
\bigskip

{\bf Pour ceux qui veulent aller plus loin}, ces entiers de $64$ bits
que nous avons utilisés pour représenter un ensemble de pièces aux échecs
sont connus sous le nom de {\em bitboards}. Les programmes de classe
mondiale actuels qui jouent aux échecs utilisent des bitboards (sous
certains détails plus ou moins sophistiqués) pour représenter des
positions de manière extrêmement efficace.

\end{document}
