% Auteur : Samuele Giraudo
% Création : 30-01-16
% Modifications : 30-01-16

\documentclass[12pt]{article}

\usepackage[utf8]{inputenc}
\usepackage[french]{babel}
\usepackage{amsmath,amsthm,amsfonts,amssymb}
\usepackage{lmodern}
\usepackage[top=3cm,bottom=3cm,left=3cm,right=3cm]{geometry}
\usepackage{hyperref}
\usepackage{multicol}
\usepackage{enumitem}
\usepackage{listings}
\usepackage[dvipsnames]{xcolor}
\usepackage{tikz}

%\author{Samuele Giraudo}
\date{}
\title{{\bf Perfectionnement à la programmation en {\sf C}} \\
    Fiche de TP 0 \\
    {\small L2 Informatique 2015-2016} \\
    {\it \small Instructions générales}}

\theoremstyle{definition}
\newtheorem{Exercice}{Exercice}

% Configuration de listings.
\lstset{
    language=c,
    basicstyle=\ttfamily\footnotesize,
    identifierstyle=\color{Mahogany},
    keywordstyle=\color{NavyBlue},
    stringstyle=\color{Emerald},
    commentstyle=\it\color{Gray},
    columns=flexible,
    tabsize=4,
    extendedchars=true,
    showspaces=false,
    numbers=left,
    numberstyle=\tiny,
    breaklines=true,
    breakautoindent=true,
    captionpos=b,
    showstringspaces=true
}

\begin{document}

\maketitle

Ce document décrit des éléments à respecter scrupuleusement au cours
des TP du module. S'y référer en cas de doute ou avant de poser une
question.
\medskip

%%%%%%%%%%%%%%%%%%%%%%%%%%%%%%%%%%%%%%%%%%%%%%%%%%%%%%%%%%%%%%%%%%%%%%%%
%%%%%%%%%%%%%%%%%%%%%%%%%%%%%%%%%%%%%%%%%%%%%%%%%%%%%%%%%%%%%%%%%%%%%%%%
%%%%%%%%%%%%%%%%%%%%%%%%%%%%%%%%%%%%%%%%%%%%%%%%%%%%%%%%%%%%%%%%%%%%%%%%
\section{Sur le déroulement des séances}
La présence aux séances de TP est primordiale et un émargement aura lieu
au début de chacune. Aucun retard non justifié n'est accepté.
\smallskip

Lors de la première séance de TP, des binômes seront constitués. Chaque
TP est à faire ainsi à deux, les groupes restant les mêmes pour toute la
durée du module.
\smallskip

Il est vivement conseillé d'échanger pendant les séances des informations
avec son coéquipier. En revanche, il est seulement toléré de communiquer
avec des étudiants d'un autre binôme. Dans tous les cas, tout échange 
doit se faire le plus discrètement possible.
\medskip

%%%%%%%%%%%%%%%%%%%%%%%%%%%%%%%%%%%%%%%%%%%%%%%%%%%%%%%%%%%%%%%%%%%%%%%%
%%%%%%%%%%%%%%%%%%%%%%%%%%%%%%%%%%%%%%%%%%%%%%%%%%%%%%%%%%%%%%%%%%%%%%%%
%%%%%%%%%%%%%%%%%%%%%%%%%%%%%%%%%%%%%%%%%%%%%%%%%%%%%%%%%%%%%%%%%%%%%%%%
\section{À propos du travail à rendre}
Chaque TP est à réaliser en une ou plusieurs séances et certains 
autoriseront un délai de rendu supplémentaire. Les modalités propres à
chaque TP seront précisées.
\smallskip

Dans tous les cas, chaque rendu du TP est a déposer avant échéance du 
temps imparti sur la plate-forme prévue à cet effet. Le rendu est 
composé des sources (fichiers {\tt .c} et {\tt .h}), de fichiers textes 
éventuels et de rapports au format {\tt .pdf} exclusivement. Il est 
autorisé également de joindre des fichiers image et son si cela est 
nécessaire au rendu. Il ne doit figurer dans le rendu aucun fichier 
régénérable (pas de {\tt .o} ni d'exécutable). Le seul format d'archivage 
autorisé est le format {\tt .zip}.
\medskip

%%%%%%%%%%%%%%%%%%%%%%%%%%%%%%%%%%%%%%%%%%%%%%%%%%%%%%%%%%%%%%%%%%%%%%%%
%%%%%%%%%%%%%%%%%%%%%%%%%%%%%%%%%%%%%%%%%%%%%%%%%%%%%%%%%%%%%%%%%%%%%%%%
%%%%%%%%%%%%%%%%%%%%%%%%%%%%%%%%%%%%%%%%%%%%%%%%%%%%%%%%%%%%%%%%%%%%%%%%
\section{À propos du rapport}
Un rapport doit accompagner chaque rendu de TP. Il est consitué, au 
minimum, des parties suivantes~:
\begin{itemize}
    \item une introduction, présentant les objectifs du projet et ses 
    enjeux~;
    \smallskip
    
    \item le corps du rapport, présentant les moyens mis en \oe uvre,
    les réponses aux questions posées explicitement dans les énoncés des 
    sujets, et toute autre information jugée utile à mentionner. Cette
    partie devra être structurée de manière adéquate en fonction de la
    nature du TP~;
    \smallskip
    
    \item une conclusion, évoquant les difficultés rencontrées, ce que 
    le sujet a apporté~;
    \smallskip
    
    \item une première annexe, contenant les instructions pour compiler
    le projet~;
    \smallskip
    
    \item une deuxième annexe, contenant la documentation pour utiliser 
    le projet.
\end{itemize}
\medskip

%%%%%%%%%%%%%%%%%%%%%%%%%%%%%%%%%%%%%%%%%%%%%%%%%%%%%%%%%%%%%%%%%%%%%%%%
%%%%%%%%%%%%%%%%%%%%%%%%%%%%%%%%%%%%%%%%%%%%%%%%%%%%%%%%%%%%%%%%%%%%%%%%
%%%%%%%%%%%%%%%%%%%%%%%%%%%%%%%%%%%%%%%%%%%%%%%%%%%%%%%%%%%%%%%%%%%%%%%%
\section{Conseils à suivre}
\begin{itemize}
    \item lire entièrement chaque énoncé de TP avant de commencer à y
    répondre~;
    \smallskip
    
    \item pour chaque fonction programmée, s'interroger sur les arguments
    qui peuvent poser des problèmes. Capturer ces cas à l'aide de
    pré-assertions~;
    \smallskip
    
    \item utiliser au maximum les fonctions déjà programmées dans les
    nouvelles à écrire. Il faut éviter au maximum la duplication de code~;
    \smallskip
    
    \item il ne suffit pas de programmer uniquement les fonctions qui sont
    demandées. Pour mener le sujet à bien, des fonctions supplémentaires
    et non mentionnées explicitement dans le sujet seront utiles. Soigner
    leurs noms~;
    \smallskip
    
    \item dès qu'une fonction est écrite, la tester de manière à
    recouvrir tous les cas significatifs possibles et conserver les tests
    (le but étant de les relancer pour vérifier l'intégrité du
    programme lors de modifications futures)~;
    \smallskip
    
    \item commenter le code en évitant absolument les commentaires inutiles.
    Il faut commenter chaque fonction écrite, juste avant son prototype.
    Il faut y mentionner les trois points suivants~: (1) une description
    du rôle de la fonction, (2) une description de ses paramètres et (3)
    une description de ce qu'elle renvoie~;
    \smallskip
    
    \item préférer la concision au maximum. Un code concis et lisible
    possède une grande valeur.
\end{itemize}


\end{document}
