% Auteur : Samuele Giraudo
% Création : 11-12-15
% Modifications : 11-12-15, déc. 2015

\documentclass[12pt]{article}

\usepackage[utf8]{inputenc}
\usepackage[french]{babel}
\usepackage{amsmath,amsthm,amsfonts,amssymb}
\usepackage{lmodern}
\usepackage[top=2.4cm,bottom=2.4cm,left=2cm,right=2cm]{geometry}
\usepackage{hyperref}
\usepackage{multicol}
\usepackage{enumerate}
\usepackage{listings}
\usepackage[dvipsnames]{xcolor}
\usepackage{tikz}

%\author{Samuele Giraudo}
\date{}
\title{{\bf Perfectionnement à la programmation en {\sf C}} \\
    Fiche de TP 2 \\
    {\small L2 Informatique 2015-2016} \\
    {\it \small Le jeu du Chomp}}

\theoremstyle{definition}
\newtheorem{Exercice}{Exercice}

% Configuration de listings.
\lstset{
    language=c,
    basicstyle=\ttfamily\footnotesize,
    identifierstyle=\color{Mahogany},
    keywordstyle=\color{NavyBlue},
    stringstyle=\color{Emerald},
    commentstyle=\it\color{Gray},
    columns=flexible,
    tabsize=4,
    extendedchars=true,
    showspaces=false,
    numbers=left,
    numberstyle=\tiny,
    breaklines=true,
    breakautoindent=true,
    captionpos=b,
    showstringspaces=true
}

\tikzstyle{Boite} = [rectangle,draw=blue!100,fill=blue!20,
thick,inner sep=0pt,minimum size=10mm,line width=1pt,font=\Huge]

\begin{document}

\maketitle

Ce TP se déroule en deux séances et est à faire par binômes.
Le travail réalisé doit être envoyé au plus tard exactement une
semaine après le début de la 2\ieme{} séance de TP traitant ce sujet. Il
sera disposé sur la plate-forme prévue à cet effet et constitué des
programmes répondant aux questions et des éventuels fichiers annexes qui
peuvent être demandés.
\bigskip
\bigskip

L'objectif de ce TP est d'implanter le {\em Chomp}, un jeu combinatoire
abstrait à deux joueurs. Les règles sont les suivantes. Deux joueurs se
disputent une tablette de chocolat de dimension~$n \times m$ où~$n$ et~$m$
sont des entiers supérieurs à un (voir figure~\ref{fig:Chomp3x5}).
\begin{figure}[ht]
    \centering
    \scalebox{.8}{\begin{tikzpicture}
        \node[Boite]at(0,0){};
        \node[Boite]at(1,0){};
        \node[Boite]at(2,0){};
        \node[Boite]at(3,0){};
        \node[Boite]at(4,0){};
        \node[Boite]at(0,1){};
        \node[Boite]at(1,1){};
        \node[Boite]at(2,1){};
        \node[Boite]at(3,1){};
        \node[Boite]at(4,1){};
        \node[Boite]at(0,2){$\cdot$};
        \node[Boite]at(1,2){};
        \node[Boite]at(2,2){};
        \node[Boite]at(3,2){};
        \node[Boite]at(4,2){};
    \end{tikzpicture}}
    \caption{La tablette de chocolat du {\em Chomp}~$3 \times 5$
    dans sa configuration initiale. Le carré empoisonné contient un
    point~$\cdot$.}
    \label{fig:Chomp3x5}
\end{figure}
Chaque joueur choisit alternativement un carré de chocolat, le mange, et
mange aussi tous les carrés qui se trouvent en bas et à sa droite
(voir figure~\ref{fig:MangerCarre}).
\begin{figure}[ht]
    \centering
    \begin{equation*}
    \begin{split}\scalebox{.8}{\begin{tikzpicture}
        \node[Boite]at(0,0){};
        \node[Boite]at(1,0){};
        \node[Boite]at(2,0){};
        \node[Boite]at(3,0){};
        \node[Boite]at(4,0){};
        \node[Boite]at(0,1){};
        \node[Boite]at(1,1){};
        \node[Boite]at(2,1){$\times$};
        \node[Boite]at(3,1){};
        \node[Boite]at(4,1){};
        \node[Boite]at(0,2){$\cdot$};
        \node[Boite]at(1,2){};
        \node[Boite]at(2,2){};
        \node[Boite]at(3,2){};
        \node[Boite]at(4,2){};
    \end{tikzpicture}}\end{split}
    \begin{split}\qquad \longrightarrow \qquad\end{split}
    \begin{split}\scalebox{.8}{\begin{tikzpicture}
        \node[Boite]at(0,0){};
        \node[Boite]at(1,0){};
        \node[Boite]at(0,1){};
        \node[Boite]at(1,1){};
        \node[Boite]at(0,2){$\cdot$};
        \node[Boite]at(1,2){};
        \node[Boite]at(2,2){};
        \node[Boite]at(3,2){};
        \node[Boite]at(4,2){};
    \end{tikzpicture}}\end{split}
    \end{equation*}
    \caption{Le joueur mange un carré de chocolat --- celui qui contient une
    croix~$\times$. Ceci a l'effet de manger tous les carrés de chocolat
    situés en bas et à sa droite.}
    \label{fig:MangerCarre}
\end{figure}
La partie s'arrête lorsque l'un des deux joueurs mange le carré de chocolat
en position~$(0, 0)$. Ce carré est en effet empoisonné et ce joueur perd
la partie.
\smallskip

L'un des objectifs de ce TP est de parfaire l'utilisation de la bibliothèque
graphique {\sf MLV} qui permet d'afficher la tablette de chocolat et
de faire en sorte que deux joueurs puissent jouer au {\em Chomp} à la
souris. On suit ici une approche ascendante (dit {\em bottom-up}). Celle-ci
consiste à réaliser le projet en constituant une à une ses pièces pour
les assembler finalement.
\bigskip
\bigskip

{\bf Conseils~:}
\begin{itemize}
    \item pour chaque fonction programmée, s'interroger sur les arguments
    qui peuvent poser des problèmes. Capturer ces cas à l'aide de
    pré-assertions~;
    \item utiliser au maximum les fonctions déjà programmées dans les
    nouvelles à écrire. Il faut éviter au maximum la duplication de code~;
    \item il ne suffit pas de programmer uniquement les fonctions qui sont
    demandées. Pour mener le sujet à bien, des fonctions supplémentaires
    et non mentionnées explicitement dans le sujet seront utiles. Soigner
    leur nom~;
    \item dès qu'une fonction est écrite, la tester de manière à
    recouvrir tous les cas significatifs possibles et conserver les tests
    (le but étant de les relancer pour vérifier l'intégrité du
    programme lors de modifications futures)~;
    \item commenter le code en évitant absolument les commentaires inutiles.
    Il faut commenter chaque fonction écrite, juste avant son prototype.
    Il faut y mentionner les trois points suivants~: (1) une description
    du rôle de la fonction, (2) une description de ses paramètres et (3)
    une description de ce qu'elle renvoie~;
    \item préférer la concision au maximum. Un code concis et lisible
    possède une grande valeur.
\end{itemize}
\bigskip
\bigskip

{\bf Remarque importante~:} {\it le programme décrit dans les 
exercices~\ref{ex:1}, \ref{ex:2} et~\ref{ex:3} est a écrire dans un 
unique fichier nommé {\tt Chomp.c}. Il sera demandé de le modulariser 
dans l'exercice~\ref{ex:4}. Cette approche, consistant à concevoir un 
projet dans un unique fichier source puis à le modulariser ensuite, est 
incorrecte en pratique. On la considère ici pour apprécier les 
désavantages de la programmation dans un fichier unique au détriment de 
la programmation modulaire.}
\bigskip
\bigskip

%%%%%%%%%%%%%%%%%%%%%%%%%%%%%%%%%%%%%%%%%%%%%%%%%%%%%%%%%%%%%%%%%%%%%%%%
%%%%%%%%%%%%%%%%%%%%%%%%%%%%%%%%%%%%%%%%%%%%%%%%%%%%%%%%%%%%%%%%%%%%%%%%
%%%%%%%%%%%%%%%%%%%%%%%%%%%%%%%%%%%%%%%%%%%%%%%%%%%%%%%%%%%%%%%%%%%%%%%%
\begin{Exercice} {\bf (Définitions des types)}\smallskip
\label{ex:1}

    \begin{enumerate}
        \item Définir un type {\tt Tablette} qui représente une tablette de
        chocolat de dimension~$n \times m$. Utiliser un tableau~$n \times m$
        qui contient des entiers. Un contenu à {\tt 1} signifie
        que le carré de chocolat correspondant existe encore tandis qu'un contenu
        à {\tt 0} signifie qu'il a été mangé. Toute variable de type
        {\tt Tablette} connaît sa dimension, c.-à-d. les entiers~$n$
        et~$m$.
        \smallskip

        \item Définir un type énuméré {\tt Joueur} qui permet de modéliser
        les deux joueurs.
        \smallskip

        \item Définir un type {\tt Position} qui permet de représenter une
        position de jeu de {\em Chomp}. Une position est déterminée par une tablette
        de chocolat et le joueur dont c'est le tour de jouer.
        \smallskip

        \item Définir un type {\tt Coup} qui permet de modéliser un coup joué.
        Un coup est entièrement spécifié par les coordonnées {\tt x} et {\tt y}
        du carré de chocolat que le joueur souhaite manger.
    \end{enumerate}
\end{Exercice}
\bigskip

%%%%%%%%%%%%%%%%%%%%%%%%%%%%%%%%%%%%%%%%%%%%%%%%%%%%%%%%%%%%%%%%%%%%%%%%
%%%%%%%%%%%%%%%%%%%%%%%%%%%%%%%%%%%%%%%%%%%%%%%%%%%%%%%%%%%%%%%%%%%%%%%%
%%%%%%%%%%%%%%%%%%%%%%%%%%%%%%%%%%%%%%%%%%%%%%%%%%%%%%%%%%%%%%%%%%%%%%%%
\begin{Exercice} {\bf (Manipulation des objets)}\smallskip
\label{ex:2}

    \begin{enumerate}
        \item Écrire une fonction
\begin{lstlisting}
Tablette creer_tablette(int n, int m);
\end{lstlisting}
        qui crée et renvoie une variable de type {\tt Tablette} de
        dimension~$n \times m$. La tablette renvoyée possède tous ses
        carrés de chocolat.
        \smallskip

        \item Écrire une fonction
\begin{lstlisting}
void manger(Tablette *t, int x, int y);
\end{lstlisting}
        qui modifie la tablette {\tt t} de sorte à manger le carré de chocolat
        en position $({\tt x}, {\tt y})$ ainsi que tous ceux qui sont en dessous de lui
        et à sa droite.
        \smallskip

        \item Un coup dans une position donnée est légal s'il ordonne de manger
        un carré de chocolat qui existe encore. Écrire une fonction
\begin{lstlisting}
int est_legal(Position *pos, Coup *coup);
\end{lstlisting}
        qui renvoie {\tt 1} si le coup {\tt coup} est légal dans la position
        {\tt pos} et {\tt 0} sinon.
        \smallskip

        \item La partie est terminée lorsque le carré de chocolat empoisonné
        vient d'être mangé. Dans ce cas, c'est le joueur qui vient de jouer
        qui a perdu et l'autre qui a gagné. Écrire une fonction
\begin{lstlisting}
int est_jeu_termine(Position *pos, Joueur *joueur_gagnant);
\end{lstlisting}
        qui renvoie {\tt 1} si la partie représentée par la position {\tt pos}
        est terminée et {\tt 0} sinon. Dans le cas où la partie est terminée,
        la fonction affecte à la variable pointée par {\tt joueur\_gagnant}
        le joueur qui gagne la partie.
        \smallskip

        \item Écrire une fonction
\begin{lstlisting}
void jouer_coup(Position *pos, Coup *coup);
\end{lstlisting}
        qui joue le coup {\tt coup} dans la position {\tt pos}. Il ne faut pas
        oublier de modifier le champ qui contient le joueur dont c'est le tour de jouer.
    \end{enumerate}
\end{Exercice}
\bigskip

%%%%%%%%%%%%%%%%%%%%%%%%%%%%%%%%%%%%%%%%%%%%%%%%%%%%%%%%%%%%%%%%%%%%%%%%
%%%%%%%%%%%%%%%%%%%%%%%%%%%%%%%%%%%%%%%%%%%%%%%%%%%%%%%%%%%%%%%%%%%%%%%%
%%%%%%%%%%%%%%%%%%%%%%%%%%%%%%%%%%%%%%%%%%%%%%%%%%%%%%%%%%%%%%%%%%%%%%%%
\begin{Exercice} {\bf (Assemblage du jeu)}\smallskip
\label{ex:3}

    \begin{enumerate}
        \item Écrire une fonction
\begin{lstlisting}
void afficher_position(Position *pos);
\end{lstlisting}
        qui affiche en utilisant la bibliothèque {\sf MLV} la position {\tt pos}.
        \smallskip

        \item Écrire une fonction
\begin{lstlisting}
Coup lire_coup(Position *pos);
\end{lstlisting}
        qui attend un clic de l'utilisateur sur la tablette de chocolat
        dans la fenêtre graphique et calcule et renvoie le coup spécifié
        par l'utilisateur. Si l'utilisateur ne clique pas sur la tablette,
        ou bien clique sur un carré de chocolat déjà mangé, la fonction
        ne fait rien et attend de traiter le prochain clic.
        \smallskip

        \item Utiliser les fonctions précédentes pour construire le
        programme {\tt Chomp.c} qui permet de jouer au {\em Chomp}.
        Utiliser pour cela l'algorithme suivant~:
        \smallskip

        \begin{enumerate}[(1)]
            \item lire la dimension de la tablette de chocolat sur
            laquelle le jeu va se dérouler (passée en argument au programme)~;
            \item initialiser la position {\tt pos}~;
            \item tant que la position {\tt pos} ne représente pas une
            partie terminée~:
            \begin{enumerate}[(a)]
                \item afficher la position {\tt pos} sur la
                fenêtre graphique~;
                \item lire un coup {\tt c} sur la fenêtre graphique~;
                \item jouer le coup {\tt c} dans la position {\tt pos}~;
            \end{enumerate}
            \item afficher le numéro du joueur gagnant.
        \end{enumerate}
    \end{enumerate}
\end{Exercice}
\bigskip

%%%%%%%%%%%%%%%%%%%%%%%%%%%%%%%%%%%%%%%%%%%%%%%%%%%%%%%%%%%%%%%%%%%%%%%%
%%%%%%%%%%%%%%%%%%%%%%%%%%%%%%%%%%%%%%%%%%%%%%%%%%%%%%%%%%%%%%%%%%%%%%%%
%%%%%%%%%%%%%%%%%%%%%%%%%%%%%%%%%%%%%%%%%%%%%%%%%%%%%%%%%%%%%%%%%%%%%%%%
\begin{Exercice} {\bf (Modularisation)}\smallskip
\label{ex:4}

\begin{enumerate}
    \item Imaginer un bon découpage en modules du programme réalisé. 
    Justifier le choix de découpage adopté.
    \smallskip
    
    \item Reprendre le fichier {\tt Chomp.c} et le découper en plusieurs
    modules conformément à la question précédente ({\it Note~: le fichier 
    {\tt Chomp.c} doit tout de même figurer dans le rendu du TP.}).
    \smallskip
    
    \item Écrire un {\tt Makefile} complet pour ce projet.
    \smallskip
    
    \item Comparer les deux version du projet, la première en un unique 
    fichier et la seconde modularisée. Citer les avantages et les 
    inconvénients des deux méthodes ({\it Note~: à partir de maintenant, 
    on adoptera uniquement l'approche de programmation par modules dans 
    les futurs projets.}).
\end{enumerate}
\end{Exercice}
\bigskip

%%%%%%%%%%%%%%%%%%%%%%%%%%%%%%%%%%%%%%%%%%%%%%%%%%%%%%%%%%%%%%%%%%%%%%%%
%%%%%%%%%%%%%%%%%%%%%%%%%%%%%%%%%%%%%%%%%%%%%%%%%%%%%%%%%%%%%%%%%%%%%%%%
%%%%%%%%%%%%%%%%%%%%%%%%%%%%%%%%%%%%%%%%%%%%%%%%%%%%%%%%%%%%%%%%%%%%%%%%
\begin{Exercice} {\bf (Améliorations)}\smallskip

{\it Cet exercice est optionnel, il n'est a envisager que lorsque
tout le reste fonctionne parfaitement. Les modifications et ajouts
proposés sont à apporter à la version modularisée du projet.}
\smallskip

\begin{enumerate}
    \item Proposer une option au programme pour gérer des {\em matchs}.
    Un match est une succession de parties dans laquelle le nombre
    de victoires de chaque joueur est pris en compte. Le déroulement est
    le suivant~: chacun des deux joueurs entre un nombre souhaité de
    parties~; ensuite, un nombre de parties égal à la moyenne (arrondie
    à l'entier supérieur) entre ces deux entiers sont jouées.
    À l'issue de ces parties, le résultat du match est affiché
    (victoire pour l'un ou l'autre des joueurs ou bien match nul).
    \smallskip

    \item On souhaite concevoir un système de sauvegarde de partie. Une
    partie est encodée par la taille du plateau et la suite des coups
    joués depuis le début de la partie. Proposer un format de fichier pour
    coder une partie de cette façon. Ajouter au programme la possibilité
    de sauvegarder une partie (et donc écrire son encodage dans un fichier)
    et la possibilité de charger une partie (et donc de lire son
    encodage depuis un fichier).
    \smallskip

    \item Proposer un mode de jeu contre l'ordinateur. Suggérer plusieurs
    niveaux de difficulté~: dans le plus facile, l'ordinateur joue
    ses coup au hasard~; dans l'autre, imaginer un moyen de faire jouer
    l'ordinateur avec une certaine force.
\end{enumerate}
\end{Exercice}
\bigskip

\end{document}
