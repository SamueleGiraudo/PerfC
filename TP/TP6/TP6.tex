% Auteur : Samuele Giraudo
% Création : 29-01-2016
% Modifications : 29-01-2016, avril 2016

\documentclass[12pt]{article}

\usepackage[utf8]{inputenc}
\usepackage[french]{babel}
\usepackage{amsmath,amsthm,amsfonts,amssymb}
\usepackage{lmodern}
\usepackage[top=3cm,bottom=3cm,left=3cm,right=3cm]{geometry}
\usepackage{hyperref}
\usepackage{multicol}
\usepackage{enumerate}
\usepackage{listings}
\usepackage{moreverb}
\usepackage[dvipsnames]{xcolor}
\usepackage{tikz}

%\author{Samuele Giraudo}
\date{}
\title{{\bf Perfectionnement à la programmation en {\sf C}} \\
    Fiche de TP 6 \\
    {\small L2 Informatique 2015-2016} \\
    {\it \small Le voyageur de commerce}}

\theoremstyle{definition}
\newtheorem{Exercice}{Exercice}

\newcommand{\Mut}{\mathrm{mut}}

% Configuration de listings.
\lstset{
    language=c,
    basicstyle=\ttfamily\footnotesize,
    identifierstyle=\color{Mahogany},
    keywordstyle=\color{NavyBlue},
    stringstyle=\color{Emerald},
    commentstyle=\it\color{Gray},
    columns=flexible,
    tabsize=4,
    extendedchars=true,
    showspaces=false,
    numbers=left,
    numberstyle=\tiny,
    breaklines=true,
    breakautoindent=true,
    captionpos=b,
    showstringspaces=true
}

\renewcommand{\arraystretch}{1.4}

\begin{document}

\maketitle

Ce TP se déroule en trois séances et est à faire par binômes. Le travail
réalisé doit être envoyé au plus tard exactement une semaine après le
début de la $3\ieme$ séance de TP. Il sera disposé sur la plate-forme
prévue à cet effet et constitué des programmes répondant aux questions
et des éventuels fichiers annexes qui peuvent être demandés.
\bigskip
\bigskip

Contrairement à certains sujets de TP précédents, nous donnons ici une
spécification du projet à lire et à mener à bien. Les définitions de
types et de fonctions sont laissées libres de choix.
\bigskip

Voici maintenant la spécification du projet.
\bigskip

\begin{center} *** \\ * \end{center}

\subsection*{\centering  Description générale}
Le {\em problème du voyageur de commerce} décrit une situation qui prend
en entrée un ensemble de villes et qui fournit en réponse un ordre de
visite de ces villes qui soit le plus court possible.
\medskip

Plus précisément, l'utilitaire à concevoir, nommé {\tt pvc}, prend en
entrée un fichier texte dont chaque ligne est de la forme
\begin{verbatim}
    VILLE_A VILLE_B DIST_A_B
\end{verbatim}
où {\tt VILLE\_A} et {\tt VILLE\_B} sont deux noms de villes différents
et {\tt DIST\_A\_B} est la distance entière exprimée en base dix et en
kilomètres entre les villes {\tt VILLE\_A} et {\tt VILLE\_B}. Le nom
d'une ville ne contient que des caractères alphabétiques ou le caractère
{\tt '-'}. Pour décrire une situation de manière exhaustive, le fichier
doit renseigner, pour toutes villes $x$ et $y$ y intervenant, la distance
entre $x$ et $y$ (ou bien par symétrie, entre $y$ et $x$). On appelle
{\em carte} la situation ainsi décrite par un tel fichier.
\medskip

Considérons un exemple. Soient les villes suivantes, dont les distances
relatives sont données par la matrice symétrique et de diagonale nulle
\begin{center}
    \begin{tabular}{c|ccccc}
                 & Moscou & Irkoutsk & Perm & Omsk & Tomsk \\ \hline
        Moscou   & 0      & 4202     & 1158 & 2237 & 2878 \\
        Irkoutsk & 4202   & 0        & 3047 & 2041 & 1330 \\
        Perm     & 1158   & 3047     & 0    & 1099 & 1720 \\
        Omsk     & 2237   & 2041     & 1099 & 0    & 742 \\
        Tomsk    & 2878   & 1330     & 1720 & 742  & 0
    \end{tabular}
\end{center}
Cette situation est modélisée par le fichier texte de contenu
\begin{center} \small
\begin{boxedverbatim}
Moscou Irkoutsk 4202
Moscou Perm 1158
Omsk Moscou 2237
Tomsk Moscou 2878
Irkoutsk Perm 3047
Irkoutsk Omsk 2041
Tomsk Irkoutsk 1330
Tomsk Perm 1720
Omsk Perm 1099
Omsk Tomsk 742
\end{boxedverbatim}
\end{center}
\medskip

Une {\em visite} est un ordre de parcours des villes de la carte. La
{\em longueur} d'une visite est le nombre total de kilomètres que
demande la visite pour être effectuée. Les visites sont enregistrées
dans des fichiers texte sous la forme d'une ligne contenant les villes à
visiter dans l'ordre, séparées par une espace. À la fin de la ligne,
figure la longueur totale de la visite~:
\begin{verbatim}
    VILLE_1 VILLE_2 ... VILLE_3 LONGUEUR
\end{verbatim}
\medskip

Avec la carte de l'exemple précédent, l'ordre de parcours Tomsk, Perm,
Irkoutsk, Omsk, Moscou  est une viste. Sa longueur est de
$1720 + 3047 + 2041 + 2237 = 9045$ kilomètres. Elle est enregistrée dans
le fichier texte de contenu
\begin{center} \small
\begin{boxedverbatim}
Tomsk Perm Irkoutsk Omsk Moscou 9045
\end{boxedverbatim}
\end{center}
On observe, bien évidemment, que toutes les visites n'ont pas la même
longueur. Par exemple, on vérifie sans peine que la visite Omsk, Tomsk,
Perm, Moscou, Irkoutsk est meilleure que la précédente car sa longueur,
de $7822$ kilomètres, est plus petite.
\medskip

L'objectif de l'application est de fournir une visite dont la longueur
est la plus petite possible en travaillant sur des cartes qui
contiennent potentiellement un grand nombre de villes.
\medskip

\subsection*{\centering Fonctionnement}
Une fois lancé, l'utilitaire {\tt pvc} va calculer la meilleure visite
possible selon une méthode expliquée plus loin. Dès qu'il trouve une
visite meilleure que celle qu'il a précédemment calculée, il l'affiche
sur la sortie standard et l'écrit dans un fichier texte de nom
{\tt CARTE.pvc} où {\tt CARTE} désigne le nom du fichier contenant la
carte. Le programme poursuit ainsi ses calculs tant qu'il n'est
pas interrompu. Une fois que l'utilisateur considère qu'il a laissé
suffisamment de temps de calcul au programme, il peut l'interrompre
et lire une solution au problème du voyageur de commerce sur la sortie
standard ou bien dans le fichier ainsi créé.
\medskip

\subsection*{\centering Méthode de calcul}
Expliquons à présent une manière efficace pour calculer une visite à
partir d'une carte donnée. La première chose à observer est qu'il n'est,
en pratique, pas possible d'essayer toutes les visites possibles d'une
carte et de ne garder que la meilleure. En effet, une carte de taille
$30$ demanderait $30! = 265252859812191058636308480000000$ traitements
de visites, ce qui est prohibitif. Nous n'allons donc pas chercher la
meilleure solution mais seulement {\em une bonne solution}, compromis
entre temps de calcul raisonnable et longueur faible de la visite.
\medskip

Appelons $C$ la carte sur laquelle nous travaillons et $n$ le nombre de
villes qu'elle contient. La méthode de recherche d'une bonne visite sur
$C$ se décrit de la manière suivante~:
\begin{enumerate}
    \item {\bf générer} de manière aléatoire une liste $V = [v_1, \dots, v_k]$
    de visites sur $C$. L'entier $k$ est fixé (une valeur habituelle pour
    $k$ est de $128$)~;
    \smallskip

    \item \label{item:etape_tri}
    {\bf trier} $V$ dans l'ordre décroissant selon la longueur des visites.
    De cette manière, la meilleure visite de $V$ est celle située en
    tête. Elle se note $v_1$~;
    \smallskip

    \item {\bf construire} une nouvelle liste $V'$ de villes de la manière
    suivante. Placer dans $V'$~:
    \begin{enumerate}
        \item les visites $\Mut(v_1)$, $\Mut(v_2)$, \dots,
        $\Mut(v_{\alpha})$, où $\Mut$ est une opération de mutation
        appliquée sur visite. Son fonctionnement est expliqué plus loin~;
        \item les visites $v_1$, $v_2$, \dots, $v_\beta$ provenant de $V$~;
        \item un nombre $\gamma$ de visites générées aléatoirement~;
    \end{enumerate}
    Les entiers $\alpha$, $\beta$ et $\gamma$ sont fixés et vérifient
    $\alpha + \beta + \gamma = k$ de sorte que $V'$ contiennent autant
    de visites que $V$~;
    \smallskip

    \item la {\bf nouvelle liste} de visites $V$ est la liste $V'$~;
    \smallskip

    \item {\bf tant que} le programme n'est pas interrompu, aller à
    l'étape~\ref{item:etape_tri}.
\end{enumerate}
\medskip

Décrivons à présent l'opérateur de {\em mutation} $\Mut$. L'idée
consiste à modifier légèrement et aléatoirement une visite pour la
rendre potentiellement meilleure. Une visite est vue ici comme une liste
de villes. Ainsi, si $v$ est une visite, $v = [a_1, \dots, a_n]$ où
$a_1$, \dots, $a_n$ sont les $n$ villes de la carte. La visite $\Mut(v)$
est la visite calculée par le procédé suivant~:
\begin{enumerate}
    \item générer aléatoirement un entier $\ell$ compris entre $1$
    et $\lfloor\frac{n}{2}\rfloor$~;
    \smallskip

    \item générer aléatoirement deux entiers $i$ et $j$ (vérifiant des
    conditions de cohérence à déterminer)~;
    \smallskip

    \item remplacer dans $v$ le facteur
    $a_i, a_{i + 1}, \dots, a_{i + \ell - 1}$ par le facteur
    $a_j, a_{j + 1}, \dots, a_{j + \ell - 1}$. Ces deux facteurs ne
    doivent pas se chevaucher. Cette nouvelle
    visite constitue la ville calculée par $\Mut(v)$.
\end{enumerate}
Par exemple, considérons la visite symbolisée par la permutation
\begin{equation*}
    \sigma := 4 \textcolor{red}{\bf 782} 19 \textcolor{blue}{\bf 365}.
\end{equation*}
Si les entiers $\ell := 3$, $i = 2$ et $j = 7$ ont été générés,
on obtient
\begin{equation*}
    \Mut(\sigma) =
    4 \textcolor{blue}{\bf 365} 19 \textcolor{red}{\bf 782}.
\end{equation*}
\medskip

\subsection*{\centering Précisions}
Les paramètres $\alpha$, $\beta$ et $\gamma$ doivent pouvoir être
choisis par l'utilisateur (des valeurs par défaut sont proposées qui
peuvent dépendre du nombre $n$ de villes de la carte). Une phase
d'expérimentation est importante pour trouver de bonnes valeurs par
défaut.
\medskip

Une gestion efficace d'erreur doit protéger l'application. Tout ce qui
peut provoquer une erreur doit être anticipé (format de carte non valide,
distances proposées non cohérentes, {\em etc.}).
\medskip

Tout ajout ou toute optimisation pertinente est le bienvenu. Par exemple,
\begin{itemize}
    \item l'ajout d'un mécanisme pour visualiser une carte dans une
    fenêtre graphique~;
    \smallskip

    \item la mise au point d'un mécanisme pour visualiser graphiquement
    et en temps réel la meilleure visite calculée pour le moment~;
    \smallskip

    \item la possibilité de saisir une carte en cliquant dans une fenêtre
    graphique de sorte à créer des coordonnées $(x, y)$ pour chaque ville~;
    \smallskip

    \item la connexion de l'application à un site internet contenant une
    base de données de distances entre villes~;
    \smallskip

    \item la prise en compte, en plus des distances entre villes, d'un
    coût de transport. La minimisation devra se faire par une fonction
    linéaire $v \mapsto \lambda_d d + \lambda_c c$ sur la distance et
    le coût. Elle associe à chaque visite $v$ la quantité mentionnée,
    où $d$ est la longueur de la visite, $c$ son coût total, et
    $\lambda_d$ et $\lambda_c$ des coefficients rationnels choisis par
    l'utilisateur.
\end{itemize}

\begin{center} *** \\ * \end{center}
\bigskip
\bigskip

%%%%%%%%%%%%%%%%%%%%%%%%%%%%%%%%%%%%%%%%%%%%%%%%%%%%%%%%%%%%%%%%%%%%%%%%
%%%%%%%%%%%%%%%%%%%%%%%%%%%%%%%%%%%%%%%%%%%%%%%%%%%%%%%%%%%%%%%%%%%%%%%%
%%%%%%%%%%%%%%%%%%%%%%%%%%%%%%%%%%%%%%%%%%%%%%%%%%%%%%%%%%%%%%%%%%%%%%%%
\begin{Exercice} {\bf (Architecture du projet)}\smallskip

L'objectif de cet exercice est de concevoir une architecture viable pour
le projet présenté.
\smallskip

\begin{enumerate}
    \item Lire attentivement {\bf l'intégralité du sujet} avant de
    commencer à répondre aux questions. Synthétiser ce qui est demandé
    de sorte à résumer l'information contenue dans le sujet en moins de
    la moitié d'une page.
    \smallskip

    \item Une fois ceci fait, proposer un découpage en modules cohérent
    du projet. Pour chaque module proposé, décrire les types qu'il
    apporte ainsi que ses objectifs principaux.
\end{enumerate}
\end{Exercice}
\bigskip

%%%%%%%%%%%%%%%%%%%%%%%%%%%%%%%%%%%%%%%%%%%%%%%%%%%%%%%%%%%%%%%%%%%%%%%%
%%%%%%%%%%%%%%%%%%%%%%%%%%%%%%%%%%%%%%%%%%%%%%%%%%%%%%%%%%%%%%%%%%%%%%%%
%%%%%%%%%%%%%%%%%%%%%%%%%%%%%%%%%%%%%%%%%%%%%%%%%%%%%%%%%%%%%%%%%%%%%%%%
\begin{Exercice} {\bf (Conception du projet)}\smallskip

\begin{enumerate}
    \item Implanter les modules mis en évidence dans l'exercice
    précédent. Au fil de l'écriture du projet, il est possible de se
    rendre compte que le découpage initialement prévu n'est pas complet
    ou adapté. Si c'est le cas, mentionner l'historique de ses
    modifications.
    \smallskip

    \item Maintenir un {\tt Makefile} pour compiler le projet.
    \smallskip

    \item Dans le rendu final, inclure des cartes d'exemple assez
    conséquentes (entre $30$ et $50$ villes) pour que l'application
    puisse être testée et son efficacité mesurée.
\end{enumerate}
\end{Exercice}
\bigskip

%%%%%%%%%%%%%%%%%%%%%%%%%%%%%%%%%%%%%%%%%%%%%%%%%%%%%%%%%%%%%%%%%%%%%%%%
%%%%%%%%%%%%%%%%%%%%%%%%%%%%%%%%%%%%%%%%%%%%%%%%%%%%%%%%%%%%%%%%%%%%%%%%
%%%%%%%%%%%%%%%%%%%%%%%%%%%%%%%%%%%%%%%%%%%%%%%%%%%%%%%%%%%%%%%%%%%%%%%%
\begin{Exercice} {\bf (Travail théorique)}\smallskip

\begin{enumerate}
    \item Il n'y a pas grand chose qui soit imposé dans l'implantation
    de ce projet. Cette grande liberté implique de justifier avec
    précision les choix effectués dans le rapport, que ce soit pour
    les algorithmes employés ou les structures de données choisies.
    \smallskip

    \item Le problème du voyageur de commerce est un problème célèbre.
    Pour cette raison, une grande bibliographie est présente à son sujet.
    On inclura ainsi dans le rapport une synthèse d'informations
    relatives à ce problème (sa description, une explication de sa
    difficulté, algorithmes d'approximation, {\em etc.}). Ne pas oublier
    de citer les sources consultées (et ne s'appuyer que sur celles qui
    ont de la valeur).
\end{enumerate}
\end{Exercice}

\end{document}
