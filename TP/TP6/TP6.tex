% Auteur : Samuele Giraudo
% Création : 29-01-2016
% Modifications : 29-01-2016

\documentclass[12pt]{article}

\usepackage[utf8]{inputenc}
\usepackage[french]{babel}
\usepackage{amsmath,amsthm,amsfonts,amssymb}
\usepackage{lmodern}
\usepackage[top=2.4cm,bottom=2.4cm,left=2cm,right=2cm]{geometry}
\usepackage{hyperref}
\usepackage{multicol}
\usepackage{enumerate}
\usepackage{listings}
\usepackage[dvipsnames]{xcolor}
\usepackage{tikz}

%\author{Samuele Giraudo}
\date{}
\title{{\bf Perfectionnement à la programmation en {\sf C}} \\
    Fiche de TP 6 \\
    {\small L2 Informatique 2015-2016} \\
    {\it \small Le problème du voyageur de commerce}}

\theoremstyle{definition}
\newtheorem{Exercice}{Exercice}

\newcommand{\Mut}{\mathrm{mut}}

% Configuration de listings.
\lstset{
    language=c,
    basicstyle=\ttfamily\footnotesize,
    identifierstyle=\color{Mahogany},
    keywordstyle=\color{NavyBlue},
    stringstyle=\color{Emerald},
    commentstyle=\it\color{Gray},
    columns=flexible,
    tabsize=4,
    extendedchars=true,
    showspaces=false,
    numbers=left,
    numberstyle=\tiny,
    breaklines=true,
    breakautoindent=true,
    captionpos=b,
    showstringspaces=true
}

\begin{document} 

\maketitle

Ce TP se déroule en trois séances et est à faire par binômes.
Le travail réalisé doit être envoyé au plus tard exactement une
semaine après le début de la séance de TP. Il sera disposé sur
la plate-forme prévue à cet effet et constitué des programmes
répondant aux questions et des éventuels fichiers annexes qui
peuvent être demandés.
\bigskip
\bigskip

Contrairement aux sujets de TP précédents, nous donnons ici une 
spécification du projet à lire et à mener à bien. Les définitions de 
types et de fonctions sont laissées libres de choix.
\bigskip

Voici maintenant la spécification du projet.
\bigskip

\begin{center} *** \\ * \end{center}
Le {\em problème du voyageur de commerce} décrit une situation qui
prend en entrée un ensemble de villes et qui fournit en réponse un ordre 
de visite de ces villes qui soit le plus court possible.
\medskip

Plus précisément, l'utilitaire à concevoir, nommé {\tt pvc}, va prendre 
en entrée un fichier texte dont chaque ligne est de la forme 
\begin{verbatim}
    VILLE_A VILLE_B DIST_A_B
\end{verbatim}
où {\tt VILLE\_A} et {\tt VILLE\_B} sont deux noms de villes différents
et {\tt DIST\_A\_B} est la distance exprimée en base dix entre les villes
{\tt VILLE\_A} et {\tt VILLE\_B}. Le nom d'une ville ne contient
que des caractères alphabétiques ou le caractère {\tt '-'}.
Pour décrire une sitation de manière
exhaustive, le fichier doit renseigner, pour toute villes $x$ et $y$ y 
intervenant, la distance entre $x$ et $y$ (ou bien pas symétrie, entre
$y$ et $x$). Par exemple, le fichier texte de contenu
% Mettre plutôt des villes russes, c'est plus drole.
\begin{verbatim}
    Paris Bordeaux 600
    Bordeaux Lyon 400
    Lyon Paris 700
    Paris Rouen 160
    Rouen Lyon 800
    Rouen Bordeaux 700
\end{verbatim}
décrit de manière exhaustive une situation faisant intervenir les 
quatre villes {\tt Paris}, {\tt Bordeaux}, {\tt Lyon} et {\tt Rouen}.
On appelle {\em carte} la situation ainsi décrite par un tel fichier.
\medskip

Une {\em visite}
est un ordre de parcours des villes de la carte qui minimise la distance 
totale. Les visites sont enregistrées dans des fichiers textes sous la 
forme d'une ligne contenant les villes à visiter dans l'ordre, séparées 
par une espace. À la fin de la ligne, figure la longueur totale de la
visite. Avec la carte de l'exemple précédent, le fichier de contenu
\begin{verbatim}
    Bordeaux Lyon Rouen Paris ??
\end{verbatim}
représente une visite dont la longueur totale est ??.
\medskip

Une fois lancé, l'utilitaire {\tt pvc} va calculer la meilleure visite 
possible selon une méhtode expliquée plus loin. Dès qu'il trouve une
visite meilleure que celle qu'il a précédemment calculée, il l'affiche
sur la sortie standard et l'écrit dans un fichier texte de nom 
{\tt CARTE.pvc} où {\tt CARTE} désigne le nom du fichier contenant la 
carte. Le programme poursuit ainsi ses calculs tant qu'il n'est
pas interrompu. Une fois que l'utilisateur considère qu'il a laissé
suffisemment de temps de calcul au programme, il peut l'interrompre
et lire une solution au problème du 
voyageur de commerce sur la sortie standard ou bien dans le fichier
ainsi créé.
\medskip

Expliquons à présent une manière efficace pour calculer une viste 
à partir d'une carte donnée. La première chose à observer est qu'il
n'est pas possible en pratique d'essayer toutes les visites possibles
d'une carte et de ne garder que la meilleure. En effet, une carte 
de taille $20$ demanderait $20!$ traitements de visites, ce qui est
prohibitif. Nous n'allons donc pas chercher la meilleure solution mais
seulement {\em une bonne solution}, compromis entre temps de calcul
raisonnable et longueur faible de la visite.
\medskip

Appellons $C$ la carte sur laquelle nous travaillons. La méthode de 
recherche d'une bonne visite sur $C$ se décrit de la manière suivante~:
\begin{enumerate}
    \item générer de manière aléatoire une liste $V = [v_1, \dots, v_k]$ 
    de visites sur $C$. L'entier $k$ est fixé~;
    \item \label{item:etape_tri}
    trier $V$ dans l'ordre décroissant selon la longueur des visites.
    De cette manière, la meilleure visite de $V$ est celle située en
    tête~;
    \item construire une nouvelle liste $V'$ de villes de la manière 
    suivante. Placer dans $V'$~:
    \begin{enumerate}
        \item les villes $\Mut(v_1)$, $\Mut(v_2)$, \dots, 
        $\Mut(v_{\alpha})$, où $\Mut$ est une opération de mutation
        appliquée sur visite. Son fonctionnement est expliqué plus loin~;
        \item les villes $v_1$, $v_2$, \dots, $v_\beta$ provenant de $V$~;
        \item un nombre $\gamma$ de visites générées aléatoirement~;
    \end{enumerate}
    Les entiers $\alpha$, $\beta$ et $\gamma$ sont fixés et vérifient
    $\alpha + \beta + \gamma = k$~;
    \item la nouvelle liste de visites $V$ est la liste $V'$~;
    \item tant que le programme n'est pas interrompu, aller à 
    l'étape~\ref{item:etape_tri}.
\end{enumerate}
\medskip

Décrivons à présent l'opérateur de {\em mutation} $\Mut$. L'idée
consiste à modifier légèrement une visite pour la rendre potentiellement
meilleure. Une visite est vue ici comme une liste de villes. Ainsi,
si $v$ est une visite, $v = [a_1, \dots, a_n]$ où $a_1$, \dots, $a_n$
sont les $n$ villes de la carte. La visite
$\Mut(v)$ est la visite calculée par 
le procédé suivant~:
\begin{enumerate}
    \item générer aléatoirement un entier $\ell$ compris entre $1$
    et $\lfloor\frac{n}{2}\rfloor$~;
    \item générer aléatoirement deux entiers $i$ et $j$ (vérifiant des
    conditions à déterminer)~;
    \item remplacer dans $v$ le facteur 
    $a_i, a_{i + 1}, \dots, a_{i + \ell - 1}$ par le facteur
    $a_j, a_{j + 1}, \dots, a_{j + \ell - 1}$. Cette nouvelle 
    visite constitue la ville calculée par $\Mut(v)$.
\end{enumerate}






\begin{center} *** \\ * \end{center}
\bigskip

{\bf Conseils~:}
\begin{itemize}
    \item pour chaque fonction programmée, s'interroger sur les arguments
    qui peuvent poser des problèmes. Capturer ces cas à l'aide de
    pré-assertions~;
    \item utiliser au maximum les fonctions déjà programmées dans les
    nouvelles à écrire. Il faut éviter au maximum la duplication de code~;
    \item commenter le code en évitant absolument les commentaires inutiles.
    Il faut commenter chaque fonction écrite, juste avant son prototype.
    Il faut y mentionner les trois points suivants~: (1) une description
    du rôle de la fonction, (2) une description de ses paramètres et (3)
    une description de ce qu'elle renvoie~;
    \item préférer la concision au maximum. Un code concis et lisible
    possède une grande valeur.
\end{itemize}
\bigskip
\bigskip

%%%%%%%%%%%%%%%%%%%%%%%%%%%%%%%%%%%%%%%%%%%%%%%%%%%%%%%%%%%%%%%%%%%%%%%%
%%%%%%%%%%%%%%%%%%%%%%%%%%%%%%%%%%%%%%%%%%%%%%%%%%%%%%%%%%%%%%%%%%%%%%%%
%%%%%%%%%%%%%%%%%%%%%%%%%%%%%%%%%%%%%%%%%%%%%%%%%%%%%%%%%%%%%%%%%%%%%%%%
\begin{Exercice} {\bf (Conception du projet)}\smallskip

L'objectif de cet exercice est de concevoir une architecture viable pour
le projet présenté.
\begin{enumerate}
    \item Lire attentivement l'intégralité du sujet avant de commencer
    à répondre aux questions. Il est important de bien analyser la 
    spécification du projet et d'en distiller le plan du travail à 
    accomplir.
    \smallskip
    
    \item Une fois ceci fait, proposer un découpage en modules cohérent 
    du projet. Pour chaque module proposé, décrire les types qu'il apporte 
    ainsi que ses objectifs principaux.
    \smallskip
    
    \item Au fil de l'écriture du projet, il est possible de se rendre 
    compte que le 
    découpage initialement prévu n'est pas complet ou adapté. Si c'est 
    le cas, mentionner l'historique de ses modifications.
    \smallskip
    
    \item Maintenir un {\tt Makefile} pour compiler le projet.
\end{enumerate}
\end{Exercice}
\bigskip

%%%%%%%%%%%%%%%%%%%%%%%%%%%%%%%%%%%%%%%%%%%%%%%%%%%%%%%%%%%%%%%%%%%%%%%%
%%%%%%%%%%%%%%%%%%%%%%%%%%%%%%%%%%%%%%%%%%%%%%%%%%%%%%%%%%%%%%%%%%%%%%%%
%%%%%%%%%%%%%%%%%%%%%%%%%%%%%%%%%%%%%%%%%%%%%%%%%%%%%%%%%%%%%%%%%%%%%%%%
\begin{Exercice} {\bf (Tests)}\smallskip
\label{exo:test}

Les tests sont des éléments de première importance dans le processus 
d'écriture d'un projet. Ils sont également primordiaux pour la maintenance
d'un projet. Ils servent à s'assurer que chaque partie du 
programme fonctione et sont utiles pour capturer des erreurs de 
programmation à leur source.
\smallskip

\begin{enumerate}
    \item Avant de commencer l'écriture du projet, créer un module 
    {\tt Test}. Ce module va inclure tous les autres modules (exception
    faite pour le module principal) et va servir à tester les fonctions
    du projet. Il contient pour le moment une fonction {\tt test} de 
    corps vide, sans paramètre et à type de retour {\tt int}.
    \smallskip
    
    \item Au cours de l'écriture du projet, pour chaque fonction {\tt fct}
    nouvellement écrite, ajouter dans le module {\tt Test} une fonction
    {\tt test\_fct} qui réalise des tests cohérents de la fonction {\tt fct}.
    Les tests sont construits en appelant {\tt fct} avec des 
    arguments pour lesquels la réponse est connue. Le test consiste à 
    vérifier si la fonction répond correctement. La fonction {\tt test\_fct}
    renvoie {\tt 1} si tous les tests qu'elle exécute se sont bien 
    déroulés et {\tt 0} sinon. Pour chaque test mal déroulé, la
    fonction affiche sur la sortie standard des informations pour identifier
    le test problématique.
    \smallskip
    
    \item La fonction {\tt test} du module {\tt Test} est sa seule 
    fonction visible depuis l'extérieur. Elle appelle chaque fonction 
    de test et vérifie que toutes se déroulent correctement. Elle renvoie
    ainsi {\tt 1} si toutes les fonctions de test renvoient {\tt 1} et
    {\tt 0} sinon.
    \smallskip
    
    \item Le programme dispose d'une option {\tt --test} qui exécute
    la fonction {\tt test} du module {\tt Test}. Il est important, lors 
    de la phase de programmation du projet ou encore lors d'une de ses 
    mises-à-jour, de lancer ces tests pour vérifier que tout fonctionne 
    comme attendu.
\end{enumerate}
\end{Exercice}
\bigskip


\end{document}
