% Auteur : Samuele Giraudo
% Création : déc. 2015 (à partir du CM de P3 et P4)
% Modifications : déc. 2015, jan. 2016, fev. 2016

\documentclass[10pt,xcolor={dvipsnames}]{beamer}

\usepackage[utf8]{inputenc}
\usepackage[T1]{fontenc}
\usepackage[francais]{babel}
\usepackage{amsmath,amsthm,amsfonts,amssymb}
\usepackage{lmodern}
\usepackage{multicol}
\usepackage{multirow}
\usepackage{hyperref}
\usepackage{listings}
\usepackage{tikz}
%\usetikzlibrary{fit,calc}
%\usepackage{pgfpages}
%\pgfpagesuselayout{4 on 1}

\title{{\bf Programmation {\sf C}}}
\author{\href{http://igm.univ-mlv.fr/~giraudo/}{\bf Samuele Giraudo} \\[1em]
{\small
Université Paris-Est Marne-la-Vallée \\[.2em]
LIGM, bureau 4B055 \\[.2em]
{\tt samuele.giraudo@u-pem.fr} \\[.2em]
\href{http://igm.univ-mlv.fr/~giraudo/}{http://igm.univ-mlv.fr/\~{}giraudo/}}}
\institute{}
\subtitle{(Perfectionnement à la programmation en {\sf C})}
\date{}

\usetheme{Rochester}
\usecolortheme{rose}

\setbeamertemplate{footline}[frame number]

% Pour supprimer les boutons de commande en bas à droite par défaut.
\setbeamertemplate{navigation symbols}{}

\newenvironment{Exemple}{%
    \setbeamercolor{block body}{fg=black, bg=Vert!12}
    \setbeamercolor{block title}{fg=Vert,bg=Vert!40}
    \begin{block}{Exemple}}{\end{block}}

% Couleurs.
\definecolor{Noir}{RGB}{0,0,0}
\definecolor{Rouge}{RGB}{205,35,38}
\definecolor{Bleu}{RGB}{2,60,195}
\definecolor{Bleu1}{RGB}{121,176,197}
\definecolor{Vert}{RGB}{23,103,1}
\definecolor{Orange}{RGB}{245,113,15}
\definecolor{Blanc}{RGB}{255,255,255}
\definecolor{Marron}{RGB}{193,88,50}
\definecolor{Jaune}{RGB}{255,180,30}
\definecolor{Violet}{RGB}{181,18,225}

\renewcommand{\leq}{\leqslant}
\renewcommand{\geq}{\geqslant}

\renewcommand{\arraystretch}{1.3}

\newcommand{\CRouge}[1]{\textcolor{Rouge}{#1}}
\newcommand{\CBleu}[1]{\textcolor{Bleu}{#1}}
\newcommand{\CVert}[1]{\textcolor{Vert}{#1}}
\newcommand{\COrange}[1]{\textcolor{Orange}{#1}}
\newcommand{\CMarron}[1]{\textcolor{Marron}{#1}}
\newcommand{\CViolet}[1]{\textcolor{Violet}{#1}}

\tikzset{every picture/.style={execute at begin picture={\shorthandoff{:}}}}

\newcommand{\todo}[1]{\CViolet{\it #1}}
\newcommand{\Code}[1]{\CBleu{\tt #1}}
\newcommand{\Sortie}[1]{\CMarron{\tt #1}}

\tikzstyle{Zone}=[rectangle,draw=black,line width=1pt, minimum size=3.5cm,
font=\large,minimum height=1.5cm]

% Configuration de listings.
\lstset{
    language=c,
    basicstyle=\ttfamily\footnotesize,
    identifierstyle=\color{Mahogany},
    keywordstyle=\color{NavyBlue},
    stringstyle=\color{Emerald},
    commentstyle=\it\color{Gray},
    columns=flexible,
    tabsize=4,
    extendedchars=true,
    showspaces=false,
    numbers=left,
    numberstyle=\tiny,
    breaklines=true,
    breakautoindent=true,
    captionpos=b,
    showstringspaces=true
}

\AtBeginSection[]{
    \frame<handout:0>{
        \frametitle{Plan}
        \tableofcontents[sectionstyle=show/hide,
            subsectionstyle=show/show/hide]}}

\AtBeginSubsection[]{
    \frame<handout:0>{
        \frametitle{Plan}
        \tableofcontents[sectionstyle=show/hide,
            subsectionstyle=show/shaded/hide]}}

%\includeonly{Contenu/Bases,Contenu/Habitudes}

%%%%%%%%%%%%%%%%%%%%%%%%%%%%%%%%%%%%%%%%%%%%%%%%%%%%%%%%%%%%%%%%%%%%%%%%
%%%%%%%%%%%%%%%%%%%%%%%%%%%%%%%%%%%%%%%%%%%%%%%%%%%%%%%%%%%%%%%%%%%%%%%%
%%%%%%%%%%%%%%%%%%%%%%%%%%%%%%%%%%%%%%%%%%%%%%%%%%%%%%%%%%%%%%%%%%%%%%%%
\begin{document}

%%%%%%%%%%%%%%%%%%%%%%%%%%%%%%%%%%%%%%%%%%%%%%%%%%%%%%%%%%%%%%%%%%%%%%%%
\begin{frame}
    \titlepage
\end{frame}

%%%%%%%%%%%%%%%%%%%%%%%%%%%%%%%%%%%%%%%%%%%%%%%%%%%%%%%%%%%%%%%%%%%%%%%%
% Auteur : Samuele Giraudo
% Création : déc. 2015
% Modifications : déc. 2015

%%%%%%%%%%%%%%%%%%%%%%%%%%%%%%%%%%%%%%%%%%%%%%%%%%%%%%%%%%%%%%%%%%%%%%%%
%%%%%%%%%%%%%%%%%%%%%%%%%%%%%%%%%%%%%%%%%%%%%%%%%%%%%%%%%%%%%%%%%%%%%%%%
%%%%%%%%%%%%%%%%%%%%%%%%%%%%%%%%%%%%%%%%%%%%%%%%%%%%%%%%%%%%%%%%%%%%%%%%
\section*{Introduction}

%%%%%%%%%%%%%%%%%%%%%%%%%%%%%%%%%%%%%%%%%%%%%%%%%%%%%%%%%%%%%%%%%%%%%%%%
\begin{frame} \frametitle{Introduction}
L'objectif de ce module est d'approfondir les concepts de base et 
d'approcher certains concepts plus avancés de la programmation en {\sf C}.
\bigskip

Celui-ci est organisé en trois axes.
\medskip

\begin{enumerate}
    \item {\bf Axe~1}~: \alert{écrire un projet maintenable}.
    \medskip
    
    Bonnes habitudes de programmation, documentation, pré-assertions,
    gestion des erreurs, programmation modulaire, compilation.
    \bigskip
    
    \item {\bf Axe~2}~: \alert{comprendre les mécanismes de base}.
    \medskip
    
    Pointeurs, allocation dynamique, types, types structurés, entrées/sorties.
    \bigskip
    
    \item {\bf Axe~3}~: \alert{utiliser quelques techniques avancées}.
    \medskip
    
    Opérateurs bit à bit, mémoïsation, génération aléatoire, pointeurs
    de fonction, généricité.
\end{enumerate}
\end{frame}

%%%%%%%%%%%%%%%%%%%%%%%%%%%%%%%%%%%%%%%%%%%%%%%%%%%%%%%%%%%%%%%%%%%%%%%%
\begin{frame} \frametitle{Contenu du cours}
\begin{multicols}{3}
    \begin{footnotesize}
        {\large \bf Axe~1.}
        \smallskip
        
        \tableofcontents[hideallsubsections,part=1]
        \bigskip

        {\large \bf Axe~2.}
        \smallskip
        
        \tableofcontents[hideallsubsections,part=2]
        \bigskip
        
        {\large \bf Axe~3.}
        \smallskip
        
        \tableofcontents[hideallsubsections,part=3]
    \end{footnotesize}
\end{multicols}
\end{frame}

%%%%%%%%%%%%%%%%%%%%%%%%%%%%%%%%%%%%%%%%%%%%%%%%%%%%%%%%%%%%%%%%%%%%%%%%
\begin{frame} \frametitle{Pré-requis}
Ce cours demande les pré-requis suivants~:
\smallskip

\begin{itemize}
    \item des bases en {\bf programmation générale} (notion de programme, 
    d'instruction, de compilation)~;
    \medskip

    \item des bases en {\bf programmation impérative} (notion de variable,
    de structures de contrôle, de fonction)~;
    \medskip
    
    \item des bases en {\bf algorithmique} (manipulation de structures de 
    données élémentaires comme les tableaux, les chaînes de caractères,
    les listes)~;
    \medskip
    
    \item des bases en {\bf programmation en {\sf C}} (écriture de 
    fonctions, manipulation de tableaux, de chaînes de caractères, 
    connaissance des types numériques, des types structurés, 
    notion d'allocation dynamique, de pointeur).
\end{itemize}
\end{frame}

%%%%%%%%%%%%%%%%%%%%%%%%%%%%%%%%%%%%%%%%%%%%%%%%%%%%%%%%%%%%%%%%%%%%%%%%
\begin{frame} \frametitle{Quelques dates}
Le langage {\sf C} est apparu en 1972 dans les laboratoires Bell.
Il a été créé par \alert{D. Ritchie} et \alert{K. Thompson}
au même moment que l'apparition des systèmes {\sf UNIX}.
\medskip

\uncover<2->{
Il est influencé par le langage {\sf B} qui, contrairement au {\sf C},
ne possédait pas de système de type.
\medskip}

\uncover<3->{
Quelques dates sur l'évolution de la spécification du langage~:

\begin{itemize}
    \item 1978~: première description complète du langage ({\sf C} traditionnel)~;
    \smallskip

    \item 1989~:%
    {\only<4->{\bf} publication de la norme ANSI {\sf C} (ou {\sf C}89)}~;

    \smallskip

    \item 1995~: évolution du langage {\sf C}94/95~;
    \smallskip

    \item 1999~: évolution du langage {\sf C}99~;
    \smallskip

    \item 2011~: nouvelle version du standard {\sf C}11.
\end{itemize}}
\end{frame}

%%%%%%%%%%%%%%%%%%%%%%%%%%%%%%%%%%%%%%%%%%%%%%%%%%%%%%%%%%%%%%%%%%%%%%%%
\begin{frame} \frametitle{Quelques caractéristiques}
Le {\sf C} est un \alert{langage de bas niveau}~: il offre la possibilité
de se placer \og proche \fg{} de la machine. Il permet en effet de manipuler
finement des données en mémoire, des adresses et des suites de bits.
\bigskip

\uncover<2->{
Ce langage a été initialement pensé pour la
\alert{conception de systèmes d'exploitation}.
\medskip

Il est cependant suffisamment expressif pour s'adapter à un large
éventail d'utilisations.
\bigskip}

\uncover<3->{
Il se situe à la croisée des chemins du monde des langages de programmation~:
beaucoup de langages modernes sont traduits en {\sf C} pour être compilés.
\bigskip}

\uncover<4->{
L'un de ses compilateurs, {\tt gcc}, fruit de nombreuses optimisations,
fait que le {\sf C} est un langage d'une \alert{efficacité extrême}.}
\end{frame}



%%%%%%%%%%%%%%%%%%%%%%%%%%%%%%%%%%%%%%%%%%%%%%%%%%%%%%%%%%%%%%%%%%%%%%%%
%%%%%%%%%%%%%%%%%%%%%%%%%%%%%%%%%%%%%%%%%%%%%%%%%%%%%%%%%%%%%%%%%%%%%%%%
%%%%%%%%%%%%%%%%%%%%%%%%%%%%%%%%%%%%%%%%%%%%%%%%%%%%%%%%%%%%%%%%%%%%%%%%
\begin{frame} \frametitle{}
\part{Axe~1}
\begin{center} \Large
    {\bf Axe~1}~: écrire un projet maintenable
\end{center}

\begin{footnotesize}
    \tableofcontents[hideallsubsections,part=1]
\end{footnotesize}
\end{frame}

% Auteur : Samuele Giraudo
% Création : oct. 2013
% Modifications : août 2014 sept 2014 oct. 2014, déc. 2015, jan. 2016
% fev. 2016

\tikzstyle{PointDiag}=[rectangle,draw=Marron!100,fill=Marron!20,
    line width=1pt,minimum width=1cm]
\tikzstyle{TestDiag}=[circle,draw=Bleu!100,fill=Bleu!20,
    line width=1pt]
\tikzstyle{FlecheDiag}=[->,draw=Rouge!80,line width=1pt]
\tikzstyle{CasePile}=[rectangle,draw=GreenYellow!100,fill=GreenYellow!20,
    line width=1pt,minimum width=2.5cm, minimum height=1cm]

%%%%%%%%%%%%%%%%%%%%%%%%%%%%%%%%%%%%%%%%%%%%%%%%%%%%%%%%%%%%%%%%%%%%%%%%
%%%%%%%%%%%%%%%%%%%%%%%%%%%%%%%%%%%%%%%%%%%%%%%%%%%%%%%%%%%%%%%%%%%%%%%%
%%%%%%%%%%%%%%%%%%%%%%%%%%%%%%%%%%%%%%%%%%%%%%%%%%%%%%%%%%%%%%%%%%%%%%%%
\section{Bases}

%%%%%%%%%%%%%%%%%%%%%%%%%%%%%%%%%%%%%%%%%%%%%%%%%%%%%%%%%%%%%%%%%%%%%%%%
%%%%%%%%%%%%%%%%%%%%%%%%%%%%%%%%%%%%%%%%%%%%%%%%%%%%%%%%%%%%%%%%%%%%%%%%
\subsection{Généralités}

%%%%%%%%%%%%%%%%%%%%%%%%%%%%%%%%%%%%%%%%%%%%%%%%%%%%%%%%%%%%%%%%%%%%%%%%
\begin{frame} \frametitle{La notion de programme}
Un \alert{programme} en {\sf C} est avant tout un texte (contenu
dans un {\bf fichier}) qui suit certaines règles (dites de {\bf syntaxe}).
\bigskip
\bigskip

\uncover<2->{
C'est une collection de déclarations et de définitions de fonctions,
de types, de variables globales, assorties de commandes pré-processeur.
\bigskip
\bigskip}

\uncover<3->{
Tout programme possède une {\bf fonction principale} nommée \Code{main}.
C'est par elle que commence l'exécution du programme. On appelle ceci
le \alert{point d'entrée} du programme.}
\end{frame}

%%%%%%%%%%%%%%%%%%%%%%%%%%%%%%%%%%%%%%%%%%%%%%%%%%%%%%%%%%%%%%%%%%%%%%%%
\begin{frame} \frametitle{Compiler un programme}
\alert{Compiler} un programme signifie {\bf traduire} le fichier le
contenant en un langage compréhensible et exécutable par la machine cible.
\medskip

\uncover<2->{
On compile un fichier \Code{Prog.c} par la commande
\begin{center} \Code{gcc -o NOM Prog.c} \end{center}
Ceci produit un exécutable nommé \Code{NOM}.
\bigskip}

\uncover<3->{
On compilera obligatoirement avec les {\bf options} \Code{-ansi},
\Code{-pedantic} et \Code{-Wall} au moyen de la commande
\begin{center} \Code{gcc -o NOM -ansi -pedantic -Wall Prog.c} \end{center}}
\begin{itemize}
    \uncover<4->{
    \item \Code{-ansi -pedantic}~: empêche un programme non compatible
    avec la norme {\sf ANSI C} de compiler~;
    \smallskip}

    \uncover<5->{
    \item \Code{-Wall}~: active tous les messages d'avertissement.}
\end{itemize}
\end{frame}

%%%%%%%%%%%%%%%%%%%%%%%%%%%%%%%%%%%%%%%%%%%%%%%%%%%%%%%%%%%%%%%%%%%%%%%%
%%%%%%%%%%%%%%%%%%%%%%%%%%%%%%%%%%%%%%%%%%%%%%%%%%%%%%%%%%%%%%%%%%%%%%%%
\subsection{Expressions et instructions}

%%%%%%%%%%%%%%%%%%%%%%%%%%%%%%%%%%%%%%%%%%%%%%%%%%%%%%%%%%%%%%%%%%%%%%%%
\begin{frame}\frametitle{Expressions}
Une \alert{expression} est définie récursivement comme étant soit

\begin{enumerate}
    \item une constante~;
    \smallskip

    \uncover<2->{
    \item une variable~;
    \smallskip}

    \uncover<3->{
    \item une combinaison d'expressions et d'opérateurs~;
    \smallskip}

    \uncover<4->{
    \item un appel de fonction qui renvoie une valeur, c.-à-d., de
    type de retour autre que \Code{void}.}
\end{enumerate}
\medskip

\uncover<5->{
Une expression n'est donc rien d'autre qu'un {\bf assemblage de symboles}
qui vérifie des règles {\bf syntaxiques} et {\bf sémantiques}.
\bigskip}

\uncover<6->{
Toute expression possède une \alert{valeur} et un \alert{type}. Le 
processus qui consiste à déterminer la valeur d'une expression se nomme 
l'{\bf évaluation}.
\bigskip}
\end{frame}

%%%%%%%%%%%%%%%%%%%%%%%%%%%%%%%%%%%%%%%%%%%%%%%%%%%%%%%%%%%%%%%%%%%%%%%%
\begin{frame}\frametitle{Expressions}
P.ex., les entités suivantes sont des expressions~:
\begin{multicols}{2}
\begin{itemize}\footnotesize
    \item \Code{0}
    \smallskip
    
    Valeur~: \Code{0}, type~: \Code{int}
    \bigskip
    
    \uncover<2->{
    \item \Code{'a'}
    \smallskip
    
    Valeur~: \Code{97}, type~: \Code{int}
    \bigskip}
    
    \uncover<3->{
    \item \Code{"abc"}
    \smallskip
    
    Valeur~: \Code{"abc"}, type~: \Code{char *}
    \bigskip}
    
    \uncover<4->{
    \item \Code{x}
    \smallskip
    
    Valeur~: \Code{x}, type~: le type de la variable \Code{x}
    \bigskip}
    
    \uncover<5->{
    \item \Code{a == 2}
    \smallskip
    
    Valeur~: \Code{0} ou \Code{1}, type~: \Code{int}
    \bigskip}
    
    \uncover<6->{
    \item \Code{b = 3}
    \smallskip
    
    Valeur~: \Code{3}, type~: \Code{int}
    \bigskip}
    
    \uncover<7->{
    \item \Code{f(5)}
    \smallskip
    
    Valeur~: la valeur renvoyée par \Code{f(5)}, type~: le type de retour
    de \Code{f}
    \bigskip}
    
    \uncover<8->{
    \item \Code{(a == 0) \&\& (b >= 3)}
    \smallskip
    
    Valeur~: \Code{0} ou \Code{1}, type~: \Code{int}
    \bigskip}
    
    \uncover<9->{
    \item \Code{a > 'a' ? a : -a}
    \smallskip
    
    Valeur~: \Code{a} ou \Code{-a}, type~: le type de \Code{a}
    \bigskip}
    
    \uncover<10->{
    \item \Code{1. + 7 * 2}
    \smallskip
    
    Valeur~: \Code{15.}, type~: \Code{float}}
\end{itemize}
\end{multicols}
\end{frame}

%%%%%%%%%%%%%%%%%%%%%%%%%%%%%%%%%%%%%%%%%%%%%%%%%%%%%%%%%%%%%%%%%%%%%%%%
\begin{frame}\frametitle{Instructions}
Une \alert{instruction} est définie récursivement comme étant soit 

\begin{enumerate}
    \item une expression \Code{E;} terminée par un point-virgule~;
    \smallskip
    
    \item un bloc \Code{$\lbrace$ I $\rbrace$} où \Code{I} est une 
    instruction~;
    \smallskip
    
    \item une conditionnelle \Code{if (E) I1 else I2}, où
    \Code{E} est une expression et \Code{I1} et \Code{I2} sont des 
    instructions~;
    \smallskip
    
    \item toute autre construction similaire (\Code{switch}, 
    \Code{while}, \Code{do while}, \Code{for}).
\end{enumerate}
\medskip

\uncover<2->{
À la différence des expressions, c'est souvent l'{\bf effet} d'une
instruction qui est sa raison d'être (et non plus uniquement sa valeur ou
son type).
\medskip}

\uncover<3->{
En effet, une instruction peut p.ex. afficher un élément, allouer
une zone mémoire, lire dans un fichier, {\em etc.} La valeur de
l'expression sous-jacente à l'instruction est d'importance secondaire.}
\end{frame}

%%%%%%%%%%%%%%%%%%%%%%%%%%%%%%%%%%%%%%%%%%%%%%%%%%%%%%%%%%%%%%%%%%%%%%%%
\begin{frame}\frametitle{Instructions}
P.ex., les entités suivantes sont des instructions~:
\begin{multicols}{2}
\begin{itemize}
    \item \Code{;} (instruction vide)
    \bigskip
    
    \item \Code{1;}
    \bigskip
    
    \item \Code{a = b;}
    \bigskip
    
    \item \Code{while (1) {a += 1;}}
    \bigskip
    
    \item \Code{printf("abc\textbackslash n");}
    \bigskip
    
    \item \Code{a++;}
    \bigskip
    
    \item \Code{malloc(64);}
    \bigskip
    
    \item \Code{int tab[64];}
    \bigskip
    
    \item \Code{tab[3] = 9;}
    \bigskip
    
    \item \Code{tab[3];}
    \bigskip
\end{itemize}
\end{multicols}
\end{frame}

%%%%%%%%%%%%%%%%%%%%%%%%%%%%%%%%%%%%%%%%%%%%%%%%%%%%%%%%%%%%%%%%%%%%%%%%
\begin{frame}\frametitle{Expressions à effet de bord}
Une {\bf expression} est à \alert{effet de bord}
(ou encore à \alert{effet secondaire}) si son évaluation modifie
la mémoire.
\bigskip

\uncover<2->{
P.ex., 
\begin{multicols}{2}
\begin{itemize}
    \item \Code{0}, 
    \item \Code{'c'}, 
    \item \Code{(a + 21) <\,< 3},
    \item \Code{tab[8]},
\end{itemize}
\end{multicols}
ne sont pas a effet de bord.
\medskip}

\uncover<3->{
En revanche, 
\begin{multicols}{2}
\begin{itemize}
    \item \Code{a = 31}, 
    \item \Code{printf("abc\textbackslash n")},
    \item \Code{char c},
    \item \Code{malloc(64)},
\end{itemize}
\end{multicols}
sont à effet de bord.}
\end{frame}

%%%%%%%%%%%%%%%%%%%%%%%%%%%%%%%%%%%%%%%%%%%%%%%%%%%%%%%%%%%%%%%%%%%%%%%%
\begin{frame}\frametitle{Expressions à effet de bord}
En règle générale, ce sont les éléments suivants dans les expressions
qui produisent des effets de bord~:
\begin{itemize}
    \item les affectations~;
    \item les allocations de mémoire~;
    \item la sollicitation au système de fichiers.
\end{itemize}
\bigskip

\uncover<2->{
En revanche, les éléments suivants dans les expressions ne produisent
pas d'effet de bord~:
\begin{itemize}
    \item lectures de constantes et de variables~;
    \item calculs arithmétiques, logiques ou bit à bit~;
    \item comparaisons.
\end{itemize}}
\end{frame}

%%%%%%%%%%%%%%%%%%%%%%%%%%%%%%%%%%%%%%%%%%%%%%%%%%%%%%%%%%%%%%%%%%%%%%%%
%%%%%%%%%%%%%%%%%%%%%%%%%%%%%%%%%%%%%%%%%%%%%%%%%%%%%%%%%%%%%%%%%%%%%%%%
\subsection{Constructions syntaxiques}

%%%%%%%%%%%%%%%%%%%%%%%%%%%%%%%%%%%%%%%%%%%%%%%%%%%%%%%%%%%%%%%%%%%%%%%%
\begin{frame}[fragile]\frametitle{Blocs}
Un \alert{bloc} est une suite d'instructions délimitée par des accolades.
Il est constitué d'une partie consacrée à la déclaration de variables
et d'une partie consacrée aux instructions.
\medskip

\uncover<2->{
\begin{multicols}{2}
\Code{$\lbrace$} \\
\quad \Code{DECL} \\
\quad \Code{INST} \\
\Code{$\rbrace$}
\medskip

\Code{DECL}~: section de {\bf déclarations}.
\bigskip

\Code{INST}~: section d'{\bf instructions}.
\bigskip
\end{multicols}}
\bigskip

\uncover<3->{
Les parties \Code{DECL} ou \Code{INST} peuvent être vides. 
\bigskip}

\uncover<4->{
Sachant qu'un bloc est une suite d'instructions, il est possible de
placer un bloc dans la section d'instructions d'un bloc et ainsi
d'{\bf imbriquer} plusieurs blocs.}
\end{frame}

%%%%%%%%%%%%%%%%%%%%%%%%%%%%%%%%%%%%%%%%%%%%%%%%%%%%%%%%%%%%%%%%%%%%%%%%
\begin{frame}[fragile]\frametitle{Blocs}
\begin{multicols}{2}
\begin{semiverbatim}\footnotesize
int main() \{
    int b;
    scanf("%d", &b);
    \textcolor{BrickRed}{\{
        int a;
        a = 1 + b;
    \}}
\}
\end{semiverbatim}

La partie rouge est un bloc au sein d'une fonction \Code{main}.
\bigskip

Il y a des règles concernant la visibilité des variables (que nous
verrons plus loin).
\bigskip
\bigskip
\bigskip
\bigskip
\end{multicols}
\bigskip

\begin{multicols}{2}
\begin{semiverbatim}\footnotesize\uncover<2->{
\{
    int a;
    int b;
    scanf(" %d", &a);
    \textcolor{JungleGreen}{\{
        printf("%d\\n", a);
    \}}
    scanf(" %d", &b);
\}}
\end{semiverbatim}
\uncover<2->{Ceci est un bloc constitué d'une section de déclarations
et d'une section d'instructions.
\bigskip

Cette dernière partie contient un bloc (en vert) dont la section de
déclarations est vide.
\bigskip
\bigskip
\bigskip}
\end{multicols}
\end{frame}

%%%%%%%%%%%%%%%%%%%%%%%%%%%%%%%%%%%%%%%%%%%%%%%%%%%%%%%%%%%%%%%%%%%%%%%%
\begin{frame}\frametitle{Opérateur de test {\tt if}}
L'\alert{opérateur de test} se décline en deux versions~:

\begin{multicols}{2}
\Code{if (EXP)} \\
\quad \Code{BLOC}
\bigskip

\Code{if (EXP)} \\
\quad \Code{BLOC\_1} \\
\Code{else} \\
\quad \Code{BLOC\_2}

Ici, \Code{EXP} est une {\bf expression booléenne}~: elle est considérée comme
fausse si elle s'évalue en zéro et comme vraie sinon. De plus, \Code{BLOC},
\Code{BLOC\_1} et \Code{BLOC\_2} sont des blocs d'instructions.
\end{multicols}
\medskip

\uncover<2->{
Diagrammes d'exécution~:
\begin{multicols}{2}
\begin{center}
\scalebox{.65}{
\begin{tikzpicture}
    \node[PointDiag](Debut)at(0,0){\Code{Debut}};
    \node[TestDiag](EXP)at(0,-1.5){\Code{EXP}};
    \node[PointDiag](BLOC)at(0,-3){\Code{BLOC}};
    \node[PointDiag](Fin)at(0,-4){\Code{Fin}};
    \draw[FlecheDiag](Debut)--(EXP);
    \draw[FlecheDiag](BLOC)--(Fin);
    \draw[FlecheDiag](EXP)edge[bend left=60]node[anchor=west]{$= 0$}(Fin);
    \draw[FlecheDiag](EXP)edge[bend right=60]node[anchor=east]{$\ne 0$}(BLOC);
\end{tikzpicture}}
\end{center}
\begin{center}
\scalebox{.65}{
\begin{tikzpicture}
    \node[PointDiag](Debut)at(0,0){\Code{Debut}};
    \node[TestDiag](EXP)at(0,-1.5){\Code{EXP}};
    \node[PointDiag](BLOC_1)at(-2,-3){\Code{BLOC\_1}};
    \node[PointDiag](BLOC_2)at(2,-3){\Code{BLOC\_2}};
    \node[PointDiag](Fin)at(0,-4){\Code{Fin}};
    \draw[FlecheDiag](Debut)--(EXP);
    \draw[FlecheDiag](BLOC_1)edge[bend right=30](Fin);
    \draw[FlecheDiag](BLOC_2)edge[bend left=30](Fin);
    \draw[FlecheDiag](EXP)edge[bend left=30]node[anchor=west]{$= 0$}(BLOC_2);
    \draw[FlecheDiag](EXP)edge[bend right=30]node[anchor=east]{$\ne 0$}(BLOC_1);
\end{tikzpicture}}
\end{center}
\end{multicols}}
\end{frame}

%%%%%%%%%%%%%%%%%%%%%%%%%%%%%%%%%%%%%%%%%%%%%%%%%%%%%%%%%%%%%%%%%%%%%%%%
\begin{frame}\frametitle{Instruction de branchement {\tt switch}}
L'\alert{instruction de branchement} admet la syntaxe

\begin{multicols}{2}
\Code{switch (EXP) $\lbrace$} \\
\quad \Code{case EXP\_1 : INSTR\_1} \\
\quad \Code{case EXP\_2 : INSTR\_2} \\
\quad \Code{...} \\
\quad \Code{case EXP\_N : INSTR\_N} \\
\quad \Code{default : INSTR\_D} \\
\Code{$\rbrace$}

Ici, \Code{EXP}, \Code{EXP\_1}, \Code{EXP\_2}, \dots, \Code{EXP\_N} sont
des expressions qui s'évaluent en des entiers.
\medskip

Les \Code{INSTR\_1}, \Code{INSTR\_2}, \dots, \Code{INSTR\_N} et
\Code{INSTR\_D} sont des suites d'instructions.
\end{multicols}

\uncover<2->{
L'exécution de cette instruction se passe ainsi.
Soit \Code{i} le plus petit entier (s'il existe) tel que
les évaluations de \Code{EXP} et \Code{EXP\_i} sont égales.
Les instructions \Code{INSTR\_i}, \dots, \Code{INSTR\_N}
ainsi que \Code{INSTR\_D} sont exécutées.
\bigskip}

\uncover<3->{
La ligne \Code{default : INSTR\_D} est facultative.
Si elle est présente, \Code{INSTR\_D} est toujours exécutée.}
\end{frame}

%%%%%%%%%%%%%%%%%%%%%%%%%%%%%%%%%%%%%%%%%%%%%%%%%%%%%%%%%%%%%%%%%%%%%%%%
\begin{frame}\frametitle{Instruction de branchement {\tt switch}}
On s'impose de terminer chaque \Code{INSTR\_j} et \Code{INSTR\_D} par le
mot-clé \Code{break}. 
\medskip

Ceci permet de n'exécuter que le \Code{INSTR\_i} tel que les évaluations 
de \Code{EXP} et \Code{EXP\_i} sont égales. Dans ce cas, \Code{INSTR\_D}
n'est exécuté que si aucun des \Code{EXP\_i} ne l'a été.
\bigskip

On obtient ainsi le diagramme d'exécution
\begin{center}
\scalebox{.65}{
\begin{tikzpicture}
    \node[PointDiag](Debut)at(0,1){\Code{Debut}};
    \node[TestDiag](EXP)at(0,-.5){\Code{EXP}};
    \node[PointDiag](INSTR_1)at(-6,-3){\Code{INSTR\_1}};
    \node[PointDiag](INSTR_2)at(-3,-3){\Code{INSTR\_2}};
    \node at(0,-3){\begin{math}\dots\end{math}};
    \node[PointDiag](INSTR_N)at(3,-3){\Code{INSTR\_N}};
    \node[PointDiag](INSTR_D)at(6,-3){\Code{INSTR\_D}};
    \node[PointDiag](Fin)at(0,-5){\Code{Fin}};
    \draw[FlecheDiag](Debut)--(EXP);
    \draw[FlecheDiag](INSTR_1)edge[bend right=30](Fin);
    \draw[FlecheDiag](INSTR_2)edge[bend right=30](Fin);
    \draw[FlecheDiag](INSTR_N)edge[bend left=30](Fin);
    \draw[FlecheDiag](INSTR_D)edge[bend left=30](Fin);
    \draw[FlecheDiag](EXP)edge[bend right=30]node[anchor=east]
        {$= \Code{EXP\_1}$}(INSTR_1);
    \draw[FlecheDiag](EXP)edge[bend right=30]node[anchor=east]
        {$= \Code{EXP\_2}$}(INSTR_2);
    \draw[FlecheDiag](EXP)edge[bend left=30]node[anchor=west]
        {$= \Code{EXP\_N}$}(INSTR_N);
    \draw[FlecheDiag](EXP)edge[bend left=30]node[anchor=west]
        {$\ne \Code{EXP\_i}, 1 \leq \Code{i} \leq \Code{N}$}(INSTR_D);
\end{tikzpicture}}
\end{center}
\end{frame}

%%%%%%%%%%%%%%%%%%%%%%%%%%%%%%%%%%%%%%%%%%%%%%%%%%%%%%%%%%%%%%%%%%%%%%%%
\begin{frame}\frametitle{Instruction itérative {\tt while}}
L'\alert{instruction itérative} \Code{while} admet la syntaxe

\begin{multicols}{2}
\Code{while (EXP)} \\
\quad \Code{BLOC}
\bigskip

Ici, \Code{EXP} est une expression booléenne et \Code{BLOC}
est un bloc d'instructions.
\end{multicols}
\bigskip

\uncover<2->{
Diagramme d'exécution~:
\begin{center}
\scalebox{.75}{
\begin{tikzpicture}
    \node[PointDiag](Debut)at(0,0){\Code{Debut}};
    \node[TestDiag](EXP)at(0,-1.5){\Code{EXP}};
    \node[PointDiag](BLOC)at(-2,-3){\Code{BLOC}};
    \node[PointDiag](Fin)at(2,-3){\Code{Fin}};
    \draw[FlecheDiag](Debut)--(EXP);
    \draw[FlecheDiag](EXP)edge[bend left=60]node[anchor=west]{$= 0$}(Fin);
    \draw[FlecheDiag](EXP)edge[bend right=60]node[anchor=east]{$\ne 0$}(BLOC);
    \draw[FlecheDiag](BLOC)edge[bend right=20](EXP);
\end{tikzpicture}}
\end{center}}
\end{frame}

%%%%%%%%%%%%%%%%%%%%%%%%%%%%%%%%%%%%%%%%%%%%%%%%%%%%%%%%%%%%%%%%%%%%%%%%
\begin{frame}\frametitle{Instruction itérative {\tt do while}}
L'\alert{instruction itérative} \Code{do while} admet la syntaxe

\begin{multicols}{2}
\Code{do} \\
\quad \Code{BLOC} \\
\Code{while (EXP);}
\bigskip

Ici, \Code{EXP} est une expression booléenne et \Code{BLOC}
est un bloc d'instructions.
\end{multicols}
\bigskip

\uncover<2->{
Diagramme d'exécution~:
\begin{center}
\scalebox{.75}{
\begin{tikzpicture}
    \node[PointDiag](Debut)at(0,0){\Code{Debut}};
    \node[PointDiag](BLOC)at(0,-1){\Code{BLOC}};
    \node[TestDiag](EXP)at(0,-2.5){\Code{EXP}};
    \node[PointDiag](Fin)at(0,-4){\Code{Fin}};
    \draw[FlecheDiag](Debut)--(BLOC);
    \draw[FlecheDiag](BLOC)--(EXP);
    \draw[FlecheDiag](EXP)edge[]node[anchor=west]{$= 0$}(Fin);
    \draw[FlecheDiag](EXP)edge[bend right=90]node[anchor=west]{$\ne 0$}(BLOC);
\end{tikzpicture}}
\end{center}}
\end{frame}

%%%%%%%%%%%%%%%%%%%%%%%%%%%%%%%%%%%%%%%%%%%%%%%%%%%%%%%%%%%%%%%%%%%%%%%%
\begin{frame}\frametitle{Instruction itérative {\tt for}}
L'\alert{instruction itérative} \Code{for} admet la syntaxe

\begin{multicols}{2}
\Code{for (EXP\_1 ; EXP\_2 ; EXP\_3)} \\
\quad \Code{BLOC}
\bigskip

Ici, \Code{EXP\_2} est une expression booléenne (le {\bf test}),
\Code{EXP\_1} (l'{\bf initialisation}) et \Code{EXP\_3}
(l'{\bf incrémentation}) sont des expressions et \Code{BLOC}
est un bloc d'instructions.
\end{multicols}
\smallskip

\uncover<2->{
Diagramme d'exécution~:
\begin{center}
\scalebox{.71}{
\begin{tikzpicture}
    \node[PointDiag](Debut)at(0,0){\Code{Debut}};
    \node[PointDiag](EXP_1)at(0,-1){\Code{EXP\_1}};
    \node[TestDiag](EXP_2)at(0,-2.5){\scriptsize\Code{EXP\_2}};
    \node[PointDiag](BLOC)at(2,-4){\Code{BLOC}};
    \node[PointDiag](EXP_3)at(2,-5){\Code{EXP\_3}};
    \node[PointDiag](Fin)at(-2,-5){\Code{Fin}};
    \draw[FlecheDiag](Debut)--(EXP_1);
    \draw[FlecheDiag](EXP_1)--(EXP_2);
    \draw[FlecheDiag](EXP_2)edge[bend left=30]node[anchor=north east]
        {$\ne 0$}(BLOC);
    \draw[FlecheDiag](BLOC)--(EXP_3);
    \draw[FlecheDiag](EXP_3)edge[bend left=30](EXP_2);
    \draw[FlecheDiag](EXP_2)edge[bend right=30]node[anchor=west]{$= 0$}(Fin);
\end{tikzpicture}}
\end{center}}
\end{frame}

%%%%%%%%%%%%%%%%%%%%%%%%%%%%%%%%%%%%%%%%%%%%%%%%%%%%%%%%%%%%%%%%%%%%%%%%
\begin{frame}[fragile]\frametitle{Instruction itérative {\tt for}}\small
\begin{multicols}{2}
\begin{semiverbatim}\small
for (i = 0 ; i < 3 ; ++i) \{
    printf("a");
\}
\end{semiverbatim}
Affiche trois occurrences du caractère \Code{'a'}.
\end{multicols}
%\smallskip

\begin{multicols}{2}
\begin{semiverbatim}\small\uncover<2->{
for (x = lst ; x != NULL ;
        x = x->suiv) \{
    afficher(x->elem);
\}}
\end{semiverbatim}
\uncover<2->{
Parcours une liste simplement chaînée et l'affiche.
La variable \Code{x} est un pointeur sur la cellule courante.}
\end{multicols}
%\smallskip

\begin{multicols}{2}
\begin{semiverbatim}\small\uncover<3->{
for ( ; *str != '\\0'; str++)) \{
    putchar(*str);
\}}
\end{semiverbatim}
\uncover<3->{
Affiche la chaîne de caractères \Code{str}. Il n'y a pas d'initialisation.
L'incrémentation fait pointer \Code{str} sur le prochain caractère.}
\end{multicols}
%\smallskip

\begin{multicols}{2}
\begin{semiverbatim}\small\uncover<4->{
for ( ; *str != '\\0';
        putchar(*str++)) \{
    /* Rien. */
\}}
\end{semiverbatim}
\uncover<4->{
Affiche la chaîne de caractères \Code{str}. Il n'y a pas
d'initialisation. L'affichage est réalisé comme effet de bord
de l'incrémentation.}
\end{multicols}
\end{frame}

%%%%%%%%%%%%%%%%%%%%%%%%%%%%%%%%%%%%%%%%%%%%%%%%%%%%%%%%%%%%%%%%%%%%%%%%
\begin{frame}[fragile]\frametitle{Instructions de court-circuit}
Les \alert{instructions de court-circuit} sont \Code{break} et
\Code{continue}.
\bigskip

\uncover<2->{
L'instruction \Code{break} permet à l'exécution de sortir d'une structure
\Code{switch}, \Code{while}, \Code{do while} ou \Code{for}.
L'exécution continue alors aux instructions qui suivent cette structure.
\bigskip}

\uncover<3->{
L'instruction \Code{continue} permet à l'exécution de sauter à la fin du
bloc \Code{BLOC} d'une structure \Code{while}, \Code{do while} ou 
\Code{for}. L'exécution continue alors à l'évaluation de l'expression 
test.}
\end{frame}

%%%%%%%%%%%%%%%%%%%%%%%%%%%%%%%%%%%%%%%%%%%%%%%%%%%%%%%%%%%%%%%%%%%%%%%%
\begin{frame}[fragile]\frametitle{Instructions de court-circuit}
Dans une boucle \Code{for}, l'instruction \Code{continue}
fait que l'expression d'incrémentation est tout de même évaluée.
\medskip

\begin{multicols}{2}
\begin{semiverbatim}\uncover<2->{
for (i = 1 ; i <= 7 ; ++i) \{
    if (i == 4)
        continue;
    printf("%d ", i);
\}}
\end{semiverbatim}
\uncover<2->{
L'exécution de ces instructions produit sept tours de boucle et 
l'affichage \Code{1 2 3 5 6 7}.
Lorsque \Code{i == 4}, le \Code{continue} relance l'exécution à 
l'incrémentation du \Code{for}.}
\end{multicols}
\bigskip

\uncover<3->{Ce n'est pas le cas pour le \Code{break}.
\medskip}

\begin{multicols}{2}
\begin{semiverbatim}\uncover<4->{
for (i = 1 ; i <= 7 ; ++i) \{
    if (i == 4)
        break;
    printf("%d ", i);
\}}
\end{semiverbatim}
\uncover<4->{
L'exécution de ces instructions produit quatre tours de boucle
et l'affichage \Code{1 2 3}.
Lorsque \Code{i == 4}, le \Code{break} fait sortir de la
boucle et \Code{i} vaut \Code{4}.}
\end{multicols}
\end{frame}

%%%%%%%%%%%%%%%%%%%%%%%%%%%%%%%%%%%%%%%%%%%%%%%%%%%%%%%%%%%%%%%%%%%%%%%%
%%%%%%%%%%%%%%%%%%%%%%%%%%%%%%%%%%%%%%%%%%%%%%%%%%%%%%%%%%%%%%%%%%%%%%%%
\subsection{Variables}

%%%%%%%%%%%%%%%%%%%%%%%%%%%%%%%%%%%%%%%%%%%%%%%%%%%%%%%%%%%%%%%%%%%%%%%%
\begin{frame} \frametitle{Variables}
Une \alert{variable} est une entité constituée des cinq éléments suivants~:
\medskip

\begin{multicols}{2}
\begin{enumerate}
    \uncover<2->{\item un identificateur~;}
    \uncover<3->{\item un type~;}
    \uncover<4->{\item une valeur~;}
    \uncover<5->{\item une adresse~;}
    \uncover<6->{\item une portée lexicale.}
\end{enumerate}
\begin{center}
\begin{tikzpicture}
    \uncover<2->{
    \draw[fill=Rouge!30](0,-.5)rectangle node{\small Identificateur}(2,0);}
    \uncover<3->{
    \draw[fill=Orange!30](2,-.5)rectangle node{\small Type}(3,1);}
    \uncover<4->{
    \draw[fill=Bleu!30](0,0)rectangle node{\small Valeur}(2,1);}
    \uncover<5->{
    \node(adr)at(-2,0){\small Adresse};
    \draw(adr)edge[->,bend right=20](0,-.25);}
\end{tikzpicture}
\end{center}
\end{multicols}
\bigskip

\uncover<7->{
{\bf Intuitivement}, c'est une boîte qui peut contenir un objet (valeur) et
qui dispose d'un nom (identificateur).
\smallskip

Une boîte ne peut contenir que des objets d'une certaine sorte (type).
\smallskip

Elle se situe de plus à un endroit bien précis dans la mémoire (adresse)
et elle n'est visible qu'à partir de certains endroits du code
(portée lexicale).}
\end{frame}

%%%%%%%%%%%%%%%%%%%%%%%%%%%%%%%%%%%%%%%%%%%%%%%%%%%%%%%%%%%%%%%%%%%%%%%%
\begin{frame} \frametitle{Identificateurs de variable}
L'\alert{identificateur} d'une variable est un mot commençant par une
lettre ou bien \Code{'\_'}, suivi par un nombre arbitraire de lettres,
chiffres ou \Code{'\_'}.
\bigskip

De plus, aucun identificateur ne peut-être un \alert{mot réservé} du
langage. En voici la liste complète~:
\begin{center}
    \scriptsize
    \begin{tabular}{cccccccc}
        \Code{auto} & \Code{break} & \Code{case} & \Code{char} &
        \Code{const} & \Code{continue} & \Code{default} & \Code{do} \\[.5em]
        \Code{double} & \Code{else} & \Code{enum} & \Code{extern} &
        \Code{float} & \Code{for} & \Code{goto} & \Code{if} \\[.5em]
        \Code{int} & \Code{long} & \Code{register} & \Code{return} &
        \Code{short} & \Code{signed} & \Code{sizeof} & \Code{static} \\[.5em]
        \Code{struct} & \Code{switch} & \Code{typedef} & \Code{union} &
        \Code{unsigned} & \Code{void} & \Code{volatile} & \Code{while}
    \end{tabular}
\end{center}
\bigskip

\uncover<2->{
L'identificateur d'une variable est \alert{attribué à sa déclaration}.}
\end{frame}

%%%%%%%%%%%%%%%%%%%%%%%%%%%%%%%%%%%%%%%%%%%%%%%%%%%%%%%%%%%%%%%%%%%%%%%%
\begin{frame}[fragile] \frametitle{Valeur d'une variable}
La \alert{valeur} d'une variable est la raison pour laquelle celle-ci
existe. Le rôle premier d'une variable étant en effet de contenir une
valeur.
\bigskip

\uncover<2->{
La valeur d'une variable n'est \alert{pas attribuée à sa déclaration}
(elle contient à ce moment là une valeur mais il ne faut rien supposer
dessus).
\medskip}

\uncover<3->{
On accède à la valeur valeur d'une variable par son identificateur.
\bigskip}

\uncover<4->{
On modifie une variable par une {\bf affectation}. L'occurrence de
l'identificateur de la variable se trouve dans ce cas à gauche de
l'opérateur \Code{=}.}
\medskip

\begin{multicols}{2}
\begin{semiverbatim}\uncover<5->{
int num;

num = 23;
num = num + 32;}
\end{semiverbatim}
\bigskip
\bigskip

\uncover<5->{
L'occurrence de \Code{num} en l. 2 est située à gauche du \Code{=}~:
il s'agit d'une affectation. Il y en a deux en ligne $3$~: la $1\iere$
permet de modifier et la $2\ieme$ de lire sa valeur.}
\end{multicols}
\end{frame}

%%%%%%%%%%%%%%%%%%%%%%%%%%%%%%%%%%%%%%%%%%%%%%%%%%%%%%%%%%%%%%%%%%%%%%%%
\begin{frame}[fragile] \frametitle{{\it L}-values et {\it R}-values}
Nous rencontrons une subtilité~: un identificateur \Code{x} de variable 
peut désigner soit~:
\begin{enumerate}
    \item la valeur de la variable \Code{x}, p.ex., dans 
    \Code{x + 16}~;
    \smallskip
    
    \uncover<2->{
    \item soit la variable \Code{x} elle-même, p.ex., dans 
    \Code{x += 8}.}
\end{enumerate}
\medskip

\uncover<3->{
La terminologie de \og{\it L}-value\fg{} (valeur gauche) et 
\og{\it R}-value\fg{} (valeur droite) permet de mettre en évidence cette
différence.
\bigskip}

\uncover<4->{
Une \alert{{\it L}-value} est une expression qui peut se situer dans le 
membre gauche d'une affectation (l'expression peut {\bf recevoir} une 
valeur).
\bigskip}

\uncover<5->{
Une \alert{{\it R}-value} est une expression qui peut se situer dans le 
membre droit d'une affectation (une valeur peut être {\bf lue} depuis
l'expression).
\bigskip}

\uncover<6->{
{\bf Note~:} toute {\it L}-value est une {\it R}-value, mais pas 
l'inverse.}
\end{frame}

%%%%%%%%%%%%%%%%%%%%%%%%%%%%%%%%%%%%%%%%%%%%%%%%%%%%%%%%%%%%%%%%%%%%%%%%
\begin{frame}[fragile] \frametitle{Adresse d'une variable}
L'\alert{adresse} d'une variable est une valeur entière spécifiant la
position de la variable en mémoire.
\bigskip

\uncover<2->{
L'adresse d'une variable est \alert{attribuée à sa déclaration} par
le système à l'{\bf exécution}. Elle ne peut pas être choisie par le
programmeur ni être modifiée.
\bigskip}

\uncover<3->{
On accède à l'adresse d'une variable par son identificateur précédé
de l'opérateur \Code{$\&$}.}

\begin{multicols}{2}
\begin{semiverbatim}\uncover<4->{
int num;

printf("%p\\n", &num);}
\end{semiverbatim}
\bigskip
\bigskip
\bigskip

\uncover<4->{
Une $1\iere$ exécution de ces instructions affiche \Sortie{0x7fff6a3014fc}.
Une $2\ieme$ affiche \Sortie{0x7fffbdc357dc}. L'adresse de \Code{num}
varie d'une exécution à l'autre.}
\end{multicols}
\end{frame}

%%%%%%%%%%%%%%%%%%%%%%%%%%%%%%%%%%%%%%%%%%%%%%%%%%%%%%%%%%%%%%%%%%%%%%%%
\begin{frame}[fragile]
    \frametitle{Portée lexicale d'une variable et variables locales}
La \alert{portée lexicale} d'une variable désigne la
{\bf zone du programme} dans laquelle la variable peut être utilisée.
\bigskip

\uncover<2->{
Elle dépend de l'endroit dans lequel elle a été déclarée.
\bigskip}

\uncover<3->{
Sa portée lexicale s'étend aux instructions qui sont situées
après sa déclaration dans le plus petit bloc d'instructions qui
la contient.
\bigskip}

\begin{multicols}{2}
\begin{semiverbatim}\small\uncover<4->{
void f(int x) \{
    int a, b;
    ...
\}

int g(int y, int z) \{
    int c;
    ...
\}}
\end{semiverbatim}
\small
\uncover<4->{
La portée lexicale des variables \Code{a} et \Code{b} s'étend aux
instructions du corps de la fonction \Code{f}.
Elle ne s'étend pas aux instructions du corps de \Code{g}. Les variables
\Code{a} et \Code{b} sont des \alert{variables locales} à la fonction \Code{f}
et invisibles ailleurs.
\smallskip}

\uncover<5->{
Le \alert{paramètre} \Code{x} de \Code{f} a pour portée lexicale uniquement
le corps de \Code{f}.}
\end{multicols}
\end{frame}

%%%%%%%%%%%%%%%%%%%%%%%%%%%%%%%%%%%%%%%%%%%%%%%%%%%%%%%%%%%%%%%%%%%%%%%%
\begin{frame}[fragile]
    \frametitle{Portée lexicale d'une variable et variables locales}
\begin{multicols}{2}
\begin{semiverbatim}
...
int f() \{...\}
...
int taille = 31;
...
int g() \{...\}
...
int main() \{...\}
\end{semiverbatim}
La portée lexicale de la variable \Code{taille} s'étend à tout ce qui
suit sa déclaration dans le programme. Elle est donc visible dans les
fonctions \Code{g} et \Code{main} mais pas dans \Code{f}. Étant donné
qu'elle est déclarée en dehors de toute fonction, elle est qualifiée de
\alert{variable globale}.
\end{multicols}
\bigskip

\uncover<2->{
{\bf Attention}~: l'utilisation de variables globales n'est ni élégante
ni indispensable. Elle est bannie pour ces raisons.
\smallskip

On préfère utiliser des définitions préprocesseur pour représenter
leur valeur.}
\end{frame}

%%%%%%%%%%%%%%%%%%%%%%%%%%%%%%%%%%%%%%%%%%%%%%%%%%%%%%%%%%%%%%%%%%%%%%%%
\begin{frame}[fragile]
\frametitle{Portée lexicale d'une variable et blocs d'instructions}
\begin{multicols}{2}
\begin{semiverbatim}
\{
    int a;
    a = 15;
    printf("%d", a);
\}
printf("%d", a);
\end{semiverbatim}
Il y a erreur de compilation~: la portée
lexicale de la variable \Code{a} s'étend de la l. 2 à la l. 5.
Elle n'est pas visible à la l. 6. Il n'existe
pas de variable identifiée par \Code{a} lorsque la l. 7 est évaluée.
\end{multicols}
\smallskip

\begin{multicols}{2}
\begin{semiverbatim}\small\uncover<2->{
\{
    int a;
    a = 15;
    printf("%d", a);
\}
\{
    printf("%d", a);
\}}
\end{semiverbatim}
\uncover<2->{
De la même manière que dans l'exemple précédent, l'occurrence
du symbole \Code{a} dans le second bloc (l. 7) n'est pas
résolue. Elle se situe dans un bloc qui n'est pas contenu
par celui où le symbole \Code{a} est déclaré.}
\end{multicols}
\end{frame}

%%%%%%%%%%%%%%%%%%%%%%%%%%%%%%%%%%%%%%%%%%%%%%%%%%%%%%%%%%%%%%%%%%%%%%%%
\begin{frame}[fragile]
\frametitle{Portée lexicale d'une variable et blocs d'instructions}
\begin{multicols}{2}
\begin{semiverbatim}
int a;
a = 10;
\{
    printf("%d ", a);
\}
printf("%d\\n", a);
\end{semiverbatim}
\medskip

Ces instructions produisent l'affichage \Sortie{10 10}. En effet, la
variable \Code{a} est visible dans le bloc d'instructions dans lequel
elle est définie, ainsi que dans les blocs d'instructions qui se trouvent
à l'intérieur.
\end{multicols}
\smallskip

\begin{multicols}{2}
\begin{semiverbatim}\small\uncover<2->{
int a;
a = 10;
\{
    int a;
    a = 20;
    printf("%d ", a);
\}
printf("%d\\n", a);}
\end{semiverbatim}
\uncover<2->{
Ces instructions produisent l'affichage \Sortie{20 10}. La variable
identifiée par \Code{a} dans le bloc d'instructions de la l. 3 à
la l. 7 est celle déclarée en l. 4. La variable identifiée par
\Code{a} hors de ce bloc d'instructions est celle déclarée en l. 1.}
\end{multicols}
\end{frame}

%%%%%%%%%%%%%%%%%%%%%%%%%%%%%%%%%%%%%%%%%%%%%%%%%%%%%%%%%%%%%%%%%%%%%%%%
\begin{frame}[fragile]
\frametitle{Portée lexicale d'une variable et blocs d'instructions}
Comparons les instructions
\begin{multicols}{2}
\begin{semiverbatim}\small
int a;
a = 10;
\{
    int a;
    a = 20;
    printf("%d ", a);
\}
printf("%d\\n", a);
\end{semiverbatim}
\begin{semiverbatim}\small
int a;
a = 10;
\{
    a = 20;
    printf("%d ", a);
\}
printf("%d\\n", a);
\end{semiverbatim}
\end{multicols}
\bigskip

\uncover<2->{
Dans le cas de gauche (déjà vu), l'affectation \Code{a = 20} n'a d'effet
que sur la variable \Code{a} déclarée à l'intérieur du bloc. Ceci
affiche \Sortie{20 10}.
\bigskip}

\uncover<3->{
En revanche, les instructions de droite affichent \Sortie{20 20} car
il n'y a pas de déclaration de \Code{a} dans le bloc. L'affectation
\Code{a = 20} modifie la variable \Code{a} déclarée en l. 1.}
\end{frame}

%%%%%%%%%%%%%%%%%%%%%%%%%%%%%%%%%%%%%%%%%%%%%%%%%%%%%%%%%%%%%%%%%%%%%%%%
%%%%%%%%%%%%%%%%%%%%%%%%%%%%%%%%%%%%%%%%%%%%%%%%%%%%%%%%%%%%%%%%%%%%%%%%
\subsection{Fonctions et pile}

%%%%%%%%%%%%%%%%%%%%%%%%%%%%%%%%%%%%%%%%%%%%%%%%%%%%%%%%%%%%%%%%%%%%%%%%
\begin{frame}[fragile] \frametitle{Fonctions}
Une \alert{fonction} est constituée
\begin{enumerate}
    \item d'un identificateur (qui suit les mêmes contraintes que ceux des
    variables)~;
    \item d'une signature (la liste de ses paramètres et de leurs types)~;
    \item d'un type de retour (le type de la valeur renvoyée par la fonction)~;
    \item d'instructions (qui forment le corps de la fonction).
\end{enumerate}
\bigskip

\uncover<2->{
La ligne constituée du type de retour, de l'identificateur et
de la signature d'une fonction est son {\bf prototype}.
\bigskip}

\uncover<3->{P.ex.,}
\begin{semiverbatim}\uncover<3->{
int produit(int a, int b, int c) \{
    return a * b * c;
\}}
\end{semiverbatim}
\uncover<3->{
est une fonction d'identificateur \Code{produit}, de signature
\Code{(int a, int b, int c)} et de type de retour \Code{int}.}
\end{frame}

%%%%%%%%%%%%%%%%%%%%%%%%%%%%%%%%%%%%%%%%%%%%%%%%%%%%%%%%%%%%%%%%%%%%%%%%
\begin{frame}[fragile] \frametitle{Définition {\em vs} déclaration}
La \alert{définition} d'une fonction consiste à fournir tous ses 
constituants.
\medskip

\uncover<2->{
La \alert{déclaration} d'une fonction consiste à fournir son
\alert{prototype}.
\medskip}

\uncover<3->{
Déclarer une fonction est utile si l'on souhaite s'en servir avant de
l'avoir définie.
\smallskip}

\uncover<4->{
Voici un exemple~:}
\begin{multicols}{2}
\begin{semiverbatim}\small\uncover<4->{
#include <stdio.h>


/* Declarations. */
void flop(int nb);
void flip(int nb);

/* Definitions. */
int main() \{
    flip(10);
    return 0;
\}

void flip(int nb) \{
    if (nb >= 1) \{
        printf("flip\\n");
        flop(nb - 1);
    \}
\}
void flop(int nb) \{
    if (nb >= 1) \{
        printf("flop\\n");
        flip(nb - 1);
    \}
\}}
\end{semiverbatim}
\end{multicols}
\end{frame}

%%%%%%%%%%%%%%%%%%%%%%%%%%%%%%%%%%%%%%%%%%%%%%%%%%%%%%%%%%%%%%%%%%%%%%%%
\begin{frame}[fragile] \frametitle{Paramètres {\em vs} arguments}
Il faut faire attention à bien distinguer les notions de paramètre et
d'argument qui sont deux choses différentes.
\bigskip

\uncover<2->{
On considère la fonction}
\begin{semiverbatim}\uncover<2->{
int produit(int a, int b, int c) \{
    return a * b * c;
\}}
\end{semiverbatim}
\medskip

\uncover<3->{
Les symboles \Code{a}, \Code{b} et \Code{c} de son prototype sont ses
\alert{paramètres}.}
\bigskip
\bigskip

\uncover<4->{
Lors de l'appel}
\begin{semiverbatim}\uncover<4->{
produit(15, num, -3);}
\end{semiverbatim}
\uncover<4->{
les valeurs \Code{15}, \Code{num} et \Code{-3} sont les \alert{arguments}
de l'appel.
\bigskip}

\uncover<5->{
{\bf Aide-mémoire}~: {\bf p}aramètre $\leftrightarrow$ {\bf p}rototype~;
{\bf a}rgument $\leftrightarrow$ {\bf a}ppel.}
\end{frame}

%%%%%%%%%%%%%%%%%%%%%%%%%%%%%%%%%%%%%%%%%%%%%%%%%%%%%%%%%%%%%%%%%%%%%%%%
\begin{frame}[fragile] \frametitle{Portée lexicale des paramètres}
Les variables locales d'une fonction ont pour portée lexicale la
fonction elle-même (déjà mentionné).
\bigskip

\uncover<2->{
Il en est de même pour ses {\bf paramètres}~: leur portée lexicale est
la fonction elle-même. On peut voir la déclaration des paramètres d'une
fonction dans son en-tête comme une déclaration de variable.
\bigskip}

\begin{multicols}{2}
\begin{semiverbatim}\small\uncover<3->{
int double(int a) \{
    return 2 * a;
\}

int main() \{
    int a;
    a = 10;
    a = double(a + 1);
    return 0;
\}}
\end{semiverbatim}
\uncover<3->{
Il y plusieurs occurrences du symbole \Code{a}.
\smallskip

Celui déclaré dans l'en-tête de \Code{double} a une portée
lexicale qui s'étend de la l. 1 à la l. 3.
\smallskip

Celui déclaré dans le \Code{main} a pour portée lexicale le \Code{main}
tout entier.
\medskip

Ce sont des variables différentes.}
\end{multicols}
\end{frame}

%%%%%%%%%%%%%%%%%%%%%%%%%%%%%%%%%%%%%%%%%%%%%%%%%%%%%%%%%%%%%%%%%%%%%%%%
\begin{frame}[fragile] \frametitle{Pile}
Lors de l'appel d'une fonction, les valeurs de ses arguments sont
\alert{recopiées} dans une zone de la mémoire appelée \alert{pile}.
\bigskip

{\bf Conséquence très importante}~: toute modification des paramètres
dans une fonction ne modifie pas les valeurs des arguments avec lesquels
elle a été appelée.
\bigskip

P.ex.,
\begin{multicols}{2}
\begin{lstlisting}[basicstyle=\ttfamily\scriptsize]
void incr(int a) {
    a = a + 1;
}

int f() {
    int b;
    
    b = 3;
    incr(b);
    printf("%d\n", b);
    return 0;
}
...
f();
\end{lstlisting}
\small
L'appel à \Code{f} en l. 14 produit les configurations de pile
\begin{flushleft}
    \begin{tabular}{ccc}
    \scalebox{.45}{\begin{tikzpicture}
        \node[CasePile](1)at(0,0){\Code{b} (l. 6, val. \Code{3})};
    \end{tikzpicture}}
    &
    \scalebox{.45}{\begin{tikzpicture}
        \node[CasePile,draw=SeaGreen,fill=SeaGreen!30](1)at(0,0)
            {\Code{a} (l. 1, val. \Code{3})};
        \node[CasePile](2)at(0,-1){\Code{b} (l. 6, val. \Code{3})};
    \end{tikzpicture}}
    &
    \scalebox{.45}{\begin{tikzpicture}
        \node[CasePile,draw=SeaGreen,fill=SeaGreen!30](1)at(0,0)
            {\Code{a} (l. 1, val. \Code{4})};
        \node[CasePile](2)at(0,-1){\Code{b} (l. 6, val. \Code{3})};
    \end{tikzpicture}} \\
    l. 8 & l. 1 & l. 2
    \end{tabular}
\end{flushleft}
\begin{flushleft}
    \begin{tabular}{cc}
    \scalebox{.45}{\begin{tikzpicture}
        \node[CasePile](1)at(0,0){\Code{b} (l. 6, val. \Code{3})};
    \end{tikzpicture}}
    &
    \scalebox{.45}{\begin{tikzpicture}
        \node[CasePile,draw=YellowOrange,fill=YellowOrange!30](1)at(0,0)
            {\Code{0} (l. 11, val. ret.)};
    \end{tikzpicture}} \\
    l. 10 & l. 11 
    \end{tabular}
\end{flushleft}
\end{multicols}
\end{frame}

%%%%%%%%%%%%%%%%%%%%%%%%%%%%%%%%%%%%%%%%%%%%%%%%%%%%%%%%%%%%%%%%%%%%%%%%
\begin{frame} \frametitle{Pile}
Les \alert{variables locales} d'une fonction (c.-à-d. les variables déclarées
dans le corps de la fonction) se situent dans la \alert{pile}.
\medskip

De plus, la \alert{valeur renvoyée} (si son type de retour n'est pas 
\Code{void}) se situe dans la \alert{pile}.
\medskip

\uncover<2->{
Après avoir appelé une fonction, c.-à-d. juste après avoir renvoyé
la valeur de retour, la pile se trouve dans le même état qu'avant l'appel.
\bigskip
\bigskip}

\uncover<3->{
{\bf Conséquence très importante}~: toute variable locale à une fonction
est non seulement invisible mais n'existe plus en mémoire hors de la
fonction et après son appel.}
\end{frame}

%%%%%%%%%%%%%%%%%%%%%%%%%%%%%%%%%%%%%%%%%%%%%%%%%%%%%%%%%%%%%%%%%%%%%%%%
\begin{frame}[fragile] \frametitle{Pile}
P.ex., 
\begin{multicols}{2}
\begin{lstlisting}
void fct_1(int a) {
    int b, c;
    b = 1;
    c = 2;
    a = b + c;
}


void fct_2() {
    int x;
    x = 16;
    fct_1(x);
    printf("%d\n", x);
}
...
fct_2();

\end{lstlisting}
\end{multicols}

Configurations de pile~:
\begin{center} \small
\begin{tabular}{cccccccc}
    $\emptyset$
    &
    \scalebox{.45}{\begin{tikzpicture}
        \node[CasePile](1)at(0,0){\Code{x} (l. 10, val. \Code{16})};
    \end{tikzpicture}}
    &
    \scalebox{.45}{\begin{tikzpicture}
        \node[CasePile](1)at(0,0){\Code{x} (l. 10, val. \Code{16})};
        \node[CasePile,draw=SeaGreen,fill=SeaGreen!30](2)at(0,1)
            {\Code{a} (l. 1, val. \Code{16})};
    \end{tikzpicture}}
    &
    \scalebox{.45}{\begin{tikzpicture}
        \node[CasePile](1)at(0,0){\Code{x} (l. 10, val. \Code{16})};
        \node[CasePile,draw=SeaGreen,fill=SeaGreen!30](2)at(0,1)
            {\Code{a} (l. 1, val. \Code{16})};
        \node[CasePile,draw=YellowOrange,fill=YellowOrange!30](3)at(0,2)
            {\Code{b} (l. 2, val. \Code{?})};
    \end{tikzpicture}}
    &
    \scalebox{.45}{\begin{tikzpicture}
        \node[CasePile](1)at(0,0){\Code{x} (l. 10, val. \Code{16})};
        \node[CasePile,draw=SeaGreen,fill=SeaGreen!30](2)at(0,1)
            {\Code{a} (l. 1, val. \Code{16})};
        \node[CasePile,draw=YellowOrange,fill=YellowOrange!30](3)at(0,2)
            {\Code{b} (l. 2, val. \Code{?})};
        \node[CasePile,draw=YellowOrange,fill=YellowOrange!30](4)at(0,3)
            {\Code{c} (l. 2, val. \Code{?})};
    \end{tikzpicture}}
    &
    \scalebox{.45}{\begin{tikzpicture}
        \node[CasePile](1)at(0,0){\Code{x} (l. 10, val. \Code{16})};
        \node[CasePile,draw=SeaGreen,fill=SeaGreen!30](2)at(0,1)
            {\Code{a} (l. 1, val. \Code{3})};
        \node[CasePile,draw=YellowOrange,fill=YellowOrange!30](3)at(0,2)
            {\Code{b} (l. 2, val. \Code{1})};
        \node[CasePile,draw=YellowOrange,fill=YellowOrange!30](4)at(0,3)
            {\Code{c} (l. 2, val. \Code{2})};
    \end{tikzpicture}}
    &
    \scalebox{.45}{\begin{tikzpicture}
        \node[CasePile](1)at(0,0){\Code{x} (l. 10, val. \Code{16})};
    \end{tikzpicture}}
    &
    $\emptyset$ \\
    l. 15 & l. 11 & l. 2 & l. 3 & l. 3 & l. 6 & l. 13 & l. 17
\end{tabular}
\end{center}
\end{frame}

%%%%%%%%%%%%%%%%%%%%%%%%%%%%%%%%%%%%%%%%%%%%%%%%%%%%%%%%%%%%%%%%%%%%%%%%
\begin{frame}[fragile] \frametitle{Pile et fonctions récursives}
Soit la fonction 
\begin{lstlisting}
void fibo(int a) {
    if (a <= 1) 
        return a;
    return fibo(a - 1) + fibo(a - 2);
}
\end{lstlisting}
et la suite d'instructions
\begin{lstlisting}
int x;
x = fibo(4);
\end{lstlisting}

\begin{center}
    \todo{Dessiner l'arbre des appels et la pile au fur et à mesure 
    de l'exécution.}
\end{center}
\end{frame}

%%%%%%%%%%%%%%%%%%%%%%%%%%%%%%%%%%%%%%%%%%%%%%%%%%%%%%%%%%%%%%%%%%%%%%%%
\begin{frame}[fragile]\frametitle{Fonctions à effet de bord}
Une {\bf fonction} est à \alert{effet de bord} s'il existe au
moins un jeu d'arguments qui fait que l'évaluation de l'appel
à la fonction sur ce jeu d'arguments modifie la mémoire par rapport
à son état d'avant l'appel.
\bigskip

\begin{multicols}{2}
\begin{semiverbatim}\uncover<2->{
inf f(int a, int b) \{
    return 21 * a + b;
\}}
\end{semiverbatim}
\uncover<2->{
Cette fonction n'est pas à effet de bord. Elle renvoie une valeur
sans modifier la mémoire.}
\end{multicols}
\medskip

\begin{multicols}{2}
\begin{semiverbatim}\uncover<3->{
float double_val(float *x) \{
    *x = 2 * (*x);
    return *x;
\}}
\end{semiverbatim}
\uncover<3->{
Cette fonction est à effet de bord puisqu'elle modifie
une zone de la mémoire (celle à l'adresse spécifiée
par son argument).}
\end{multicols}
\end{frame}

%%%%%%%%%%%%%%%%%%%%%%%%%%%%%%%%%%%%%%%%%%%%%%%%%%%%%%%%%%%%%%%%%%%%%%%%
\begin{frame}[fragile]\frametitle{Fonctions à effet de bord}
\begin{multicols}{2}
\begin{semiverbatim}
int g(char c) \{
    int b;
    b = 5;
    return b + c;
\}
\end{semiverbatim}
\uncover<2->{
Cette fonction n'est pas à effet de bord. La déclaration (et l'affectation)
de \Code{b} reste locale à la fonction. Après tout appel à \Code{g},
la variable \Code{b} n'existe plus.}
\end{multicols}

\begin{multicols}{2}
\begin{semiverbatim}\small\uncover<3->{
char *allouer(int n) \{
    char *res;
    res = (char *)
    malloc(sizeof(char) * n);
    return res;
\}}
\end{semiverbatim}
\uncover<4->{
Cette fonction est à effet de bord. Elle réserve en effet,
par l'appel interne à la fonction \Code{malloc}, une zone
de la mémoire, ce qui modifie son état.}
\end{multicols}

\begin{multicols}{2}
\begin{semiverbatim}\small\uncover<5->{
int h(int a, int b) \{
    if (a * b == 0)
        printf("z\\n");
    return a - b;
\}}
\end{semiverbatim}
\uncover<6->{
Cette fonction est à effet de bord. En effet, l'appel à \Code{h}
avec, p.ex., les arguments \Code{1} et \Code{0} provoque un
affichage sur la sortie standard, modifiant l'état de la
mémoire.}
\end{multicols}
\end{frame}

%%%%%%%%%%%%%%%%%%%%%%%%%%%%%%%%%%%%%%%%%%%%%%%%%%%%%%%%%%%%%%%%%%%%%%%%
%%%%%%%%%%%%%%%%%%%%%%%%%%%%%%%%%%%%%%%%%%%%%%%%%%%%%%%%%%%%%%%%%%%%%%%%
\subsection{Commandes préprocesseur}

%%%%%%%%%%%%%%%%%%%%%%%%%%%%%%%%%%%%%%%%%%%%%%%%%%%%%%%%%%%%%%%%%%%%%%%%
\begin{frame} \frametitle{Commandes préprocesseur}
Une \alert{commande préprocesseur} (ou directive préprocesseur) est
une ligne qui commence par \Code{\#}.
\bigskip

\uncover<2->{
Le \alert{préprocesseur} est une unité qui intervient lors de la
compilation. Son rôle est de traiter les commandes préprocesseur.
\medskip

Il fonctionne en construisant une nouvelle version du programme
en {\bf remplaçant} chaque commande préprocesseur par des expressions
en {\sf C} adéquates.
\bigskip}

\uncover<3->{
Il existe plusieurs sortes de commandes préprocesseur~:

\begin{itemize}
    \item les inclusions de fichiers~;
    \smallskip

    \item les définitions de symboles~;
    \smallskip

    \item les macro-instructions à paramètres;
    \smallskip
    
    \item les macro-instructions de contrôle de compilation.
\end{itemize}}
\end{frame}

%%%%%%%%%%%%%%%%%%%%%%%%%%%%%%%%%%%%%%%%%%%%%%%%%%%%%%%%%%%%%%%%%%%%%%%%
\begin{frame}[fragile] \frametitle{Inclusions de fichiers}
La commande préprocesseur
\begin{center}
    \Code{\#include <NOM.h>}
\end{center}
permet d'\alert{inclure} le fichier \Code{NOM.h} dans le programme pour
bénéficier des fonctionnalités qu'il apporte.
\medskip

\uncover<2->{
Le préprocesseur résout cette commande en recopiant le contenu de
\Code{NOM.h} à l'endroit où elle est invoquée.
\bigskip}

\uncover<3->{
Il est possible d'enchaîner les inclusions~:}
\begin{semiverbatim}\uncover<3->{
#include <stdio.h>
#include <stdlib.h>
#include <string.h>}
\end{semiverbatim}
\bigskip

\uncover<4->{
Habituellement, les inclusions sont réalisées au début du programme.}
\end{frame}

%%%%%%%%%%%%%%%%%%%%%%%%%%%%%%%%%%%%%%%%%%%%%%%%%%%%%%%%%%%%%%%%%%%%%%%%
\begin{frame}[fragile] \frametitle{Définitions de symboles}
La commande préprocesseur
\begin{center}
    \Code{\#define SYMB EXP}
\end{center}
permet de \alert{définir un alias} \Code{SYMB} pour l'expression
\Code{EXP}. Ceci autorise à faire référence à l'expression \Code{EXP} par
l'intermédiaire du symbole \Code{SYMB}.
\medskip

\uncover<2->{
Le préprocesseur résout tout invocation \Code{SYMB} en la remplaçant
par \Code{EXP}.
\bigskip}

\uncover<3->{
À gauche (resp. à droite), des instructions avant (resp. après) le
passage du préprocesseur~:}
\begin{multicols}{2}
\begin{semiverbatim}\small\uncover<3->{
#define NB 5
#define CHAINE "cba\\n"
...
for (i = 1 ; i <= NB ; ++i) \{
    printf("%s", CHAINE);
\}}
\end{semiverbatim}
\begin{semiverbatim}\small\uncover<3->{
/* Rien. */
/* Rien. */
...
for (i = 1 ; i <= 5 ; ++i) \{
    printf("%s", "cba\\n");
\}}
\end{semiverbatim}
\end{multicols}

\uncover<4->{
Par convention, tout alias est constitué de lettres majuscules,
de chiffres ou de tirets bas.}
\end{frame}

%%%%%%%%%%%%%%%%%%%%%%%%%%%%%%%%%%%%%%%%%%%%%%%%%%%%%%%%%%%%%%%%%%%%%%%%
\begin{frame}[fragile] \frametitle{Macro-instructions à paramètres}
La commande préprocesseur
\begin{center}
    \Code{\#define SYMB(P1, P2, ..., Pn) EXP}
\end{center}
permet de définir une \alert{macro-instruction à paramètres} \Code{SYMB}.
Ceci autorise à faire référence à l'expression \Code{EXP} par
l'intermédiaire du symbole \Code{SYMB} paramétrable par des
paramètres \Code{P1}, \Code{P2}, \dots, \Code{Pn}.
\medskip

\uncover<2->{
Le préprocesseur résout toute invocation \Code{SYMB(A1, A2, ..., An)}
en la remplaçant par l'expression obtenue en substituant \Code{Ai}
à toute occurrence du paramètre \Code{Pi} dans \Code{EXP}.
\bigskip}

\uncover<3->{
À gauche (resp. à droite), des instructions avant (resp. après) le
passage du préprocesseur~:}
\begin{multicols}{2}
\begin{semiverbatim}\small\uncover<3->{
#define MAX(a, b) a > b ? a : b
...
int x;
x = MAX(10, 14);}
\end{semiverbatim}
\begin{semiverbatim}\small\uncover<3->{
/* Rien. */
...
int x;
x = 10 > 14 ? 10 : 14;}
\end{semiverbatim}
\end{multicols}
\end{frame}

%%%%%%%%%%%%%%%%%%%%%%%%%%%%%%%%%%%%%%%%%%%%%%%%%%%%%%%%%%%%%%%%%%%%%%%%
\begin{frame}[fragile] \frametitle{Macro-instructions à paramètres}
{\bf Problème}~: étudions comment le préprocesseur transforme les
instructions suivantes~:
\begin{semiverbatim}
#define CARRE(a) a * a
...
x = 2;
y = 3;
z = CARRE(x + y);
\end{semiverbatim}
\uncover<2->{
La l. 5 est remplacée par \Code{z = x + y * x + y;}.

Ainsi, la valeur \Code{2 + 3 * 2 + 3 = 13} est affectée à \Code{z} au
lieu de \Code{25} comme attendu.
\bigskip}

\uncover<3->{
{\bf Solution}~: il faut placer des parenthèses autour des paramètres
des macro-instructions à paramètres~:}
\begin{semiverbatim}\uncover<3->{
#define CARRE(a) (a) * (a)}
\end{semiverbatim}
\uncover<4->{
De cette façon, \Code{CARRE(x + y)} est remplacée par
\Code{(x + y) * (x + y)} comme désiré.}
\end{frame}

%%%%%%%%%%%%%%%%%%%%%%%%%%%%%%%%%%%%%%%%%%%%%%%%%%%%%%%%%%%%%%%%%%%%%%%%
\begin{frame}[fragile] \frametitle{Macro-instructions à paramètres}
{\bf Problème}~: étudions comment le préprocesseur transforme les
instructions suivantes~:
\begin{semiverbatim}
#define DOUBLE(a) (a) + (a)
...
x = 3;
z = 5 * DOUBLE(x);
\end{semiverbatim}
\uncover<2->{
La l. 4 est remplacée par \Code{z = 5 * (x) + (x);}.

Ainsi, la valeur \Code{5 * 3 + 3 = 18} est affectée à \Code{z} au lieu de
\Code{30} comme attendu.
\bigskip}

\uncover<3->{
{\bf Solution}~: il faut placer des parenthèses autour de l'expression
toute entière~:}
\begin{semiverbatim}\uncover<3->{
#define DOUBLE(a) ((a) + (a))}
\end{semiverbatim}
\uncover<4->{
De cette façon, \Code{5 * DOUBLE(x)} est remplacée par
\Code{5 * ((x) + (x))} comme désiré.}
\end{frame}

%%%%%%%%%%%%%%%%%%%%%%%%%%%%%%%%%%%%%%%%%%%%%%%%%%%%%%%%%%%%%%%%%%%%%%%%
\begin{frame} \frametitle{Macro-instructions de contrôle de compilation}
Les \alert{macro-instructions de contrôle de compilation} permettent 
d'ignorer, lors de la compilation, une partie du programme.
\medskip

Ceci est utile pour {\bf sélectionner} les parties à prendre en compte 
dans un programme, sans avoir à les (dé)commenter.
\bigskip

On dispose ainsi des constructions
\begin{multicols}{3}
\Code{\#ifdef SYMB} \\
\Code{\dots} \\
\Code{\#endif}
\bigskip
\bigskip

\Code{\#ifndef SYMB} \\
\Code{\dots} \\
\Code{\#endif}
\bigskip
\bigskip

\Code{\#ifdef SYMB} \\
\Code{\dots} \\
\Code{\#else} \\
\Code{\dots} \\
\Code{\#endif}
\end{multicols}

À gauche, le code \Code{\dots} n'est considéré que si l'alias \Code{SYMB}
est défini.
\medskip

Au centre, le code \Code{\dots} n'est considéré que si l'alias \Code{SYMB}
n'est pas défini.
\end{frame}

%%%%%%%%%%%%%%%%%%%%%%%%%%%%%%%%%%%%%%%%%%%%%%%%%%%%%%%%%%%%%%%%%%%%%%%%
\begin{frame}[fragile] 
\frametitle{Macro-instructions de contrôle de compilation}
\begin{multicols}{2}
\begin{lstlisting}[basicstyle=\ttfamily\scriptsize]
#include <stdio.h>


#define GAUCHE_DROITE

#ifdef GAUCHE_DROITE

int rechercher(char *tab, 
        int n, char x) {
    int i;
    for (i = 0 ; i <= n ; ++i)
        if (tab[i] == x)
            return i;
    return -1;
}

#else


int rechercher(char *tab, 
        int n, char x) {
    int i;
    for (i = n - 1 ; i >= 0 ; --i)
        if (tab[i] == x)
            return i;
    return -1;
}

#endif

int main() {
    int res;
    char tab[] = "chaine de test";
    
    res = rechercher(tab, 14, 't');
    printf("%d\n", res);
}
\end{lstlisting}
\end{multicols}

\begin{footnotesize}
Ici, on donne deux algorithmes pour localiser la première occurrence
d'une lettre dans un tableau~: de la gauche vers la droite, ou bien
de la droite vers la gauche.
\smallskip

Ce programme affiche \Code{10}~; si on renomme \Code{GAUCHE\_DROITE}
(en l. 4), il affiche \Code{13}.
\end{footnotesize}
\end{frame}

% Auteur : Samuele Giraudo
% Création : déc. 2015
% Modifications : déc. 2015, fev. 2016

%%%%%%%%%%%%%%%%%%%%%%%%%%%%%%%%%%%%%%%%%%%%%%%%%%%%%%%%%%%%%%%%%%%%%%%%
%%%%%%%%%%%%%%%%%%%%%%%%%%%%%%%%%%%%%%%%%%%%%%%%%%%%%%%%%%%%%%%%%%%%%%%%
%%%%%%%%%%%%%%%%%%%%%%%%%%%%%%%%%%%%%%%%%%%%%%%%%%%%%%%%%%%%%%%%%%%%%%%%
\section{Habitudes}

%%%%%%%%%%%%%%%%%%%%%%%%%%%%%%%%%%%%%%%%%%%%%%%%%%%%%%%%%%%%%%%%%%%%%%%%
%%%%%%%%%%%%%%%%%%%%%%%%%%%%%%%%%%%%%%%%%%%%%%%%%%%%%%%%%%%%%%%%%%%%%%%%
\subsection{Mise en page}

%%%%%%%%%%%%%%%%%%%%%%%%%%%%%%%%%%%%%%%%%%%%%%%%%%%%%%%%%%%%%%%%%%%%%%%%
\begin{frame} \frametitle{Mise en page d'un programme}
Pour écrire un programme de valeur, il faut soigner les points suivants~:
\medskip

\begin{enumerate}
    \item l'indentation~;
    \medskip
    
    \item l'organisation des espaces autour des caractères~;
    \medskip
    
    \item le choix des identificateurs~;
    \medskip
    
    \item la documentation.
\end{enumerate}
\end{frame}

%%%%%%%%%%%%%%%%%%%%%%%%%%%%%%%%%%%%%%%%%%%%%%%%%%%%%%%%%%%%%%%%%%%%%%%%
\begin{frame}[fragile] \frametitle{Indentation}
L'\alert{indentation} consiste à disposer des caractères blancs au début 
de certaines lignes d'un programme. 
\medskip

Contrairement au {\sf Python}, celle-ci ne modifie pas le comportement 
d'un programme.
\medskip

L'objectif est d'augmenter sa lisibilité.
\medskip

Règle~: on place {\bf quatre espaces} (pas de tabulation) avant chaque 
instruction d'un bloc et on incrémente cet espacement de quatre en 
quatre en fonction de la profondeur des blocs.
\medskip

\begin{multicols}{2}
\begin{lstlisting}[showspaces=true]
/* Correct. */
a = 8;
if (b >= 0) {
    printf("%d\n", a);
    a = 0;
    for (i = 0 ; i <= b ; ++i)
        a /= 2;
}
\end{lstlisting}

\begin{lstlisting}[showspaces=true]
/* Incorrect. */
a = 8;
if (b >= 0) {
  printf("%d\n", a);
  a = 0;
  for (i = 0 ; i <= b ; ++i)
      a /= 2;
}
\end{lstlisting}
\end{multicols}
\end{frame}

%%%%%%%%%%%%%%%%%%%%%%%%%%%%%%%%%%%%%%%%%%%%%%%%%%%%%%%%%%%%%%%%%%%%%%%%
\begin{frame}[fragile] \frametitle{Organisation des blocs}
Nous avons vu qu'une \alert{bloc} des une suite d'instructions délimitée 
par des accolades. Il se trouve attaché à une instruction de branchement 
ou de boucle. Il peut aussi être indépendant.
\medskip

On {\bf revient à la ligne} après une {\bf accolade ouvrante}.
\medskip

L'{\bf accolade fermante} se trouve horizontalement au {\bf niveau du 
début} de la ligne qui contient l'accolade ouvrante.

\begin{multicols}{2}
\begin{lstlisting}
/* Correct. */
if (valeur >= 1) {
    valeur -= 1;
}
\end{lstlisting}
\bigskip

\begin{lstlisting}
/* Incorrect. */
if (valeur >= 1) 
{
    valeur -= 1;
}
\end{lstlisting}
\end{multicols}

\begin{multicols}{2}
\begin{lstlisting}
/* Correct. */
valeur = 1;
{
    int a;
    valeur = 10;
}
\end{lstlisting}

\begin{lstlisting}
/* Incorrect. */
valeur = 1; {
    int a;
    valeur = 10;
}
\end{lstlisting}

\end{multicols}
\end{frame}

%%%%%%%%%%%%%%%%%%%%%%%%%%%%%%%%%%%%%%%%%%%%%%%%%%%%%%%%%%%%%%%%%%%%%%%%
\begin{frame}[fragile] \frametitle{Organisation des espaces}
On place une {\bf espace avant et après} chaque opérateur.
\begin{multicols}{2}
\begin{lstlisting}[showspaces=true]
/* Correct. */
a = b * 2 + 5;
\end{lstlisting}

\begin{lstlisting}[showspaces=true]
/* Incorrect. */
a = b*2 + 5;
\end{lstlisting}
\end{multicols}
\medskip

On utilise les règles habituelles de {\bf typographie} pour l'usage des 
{\bf virgules}.
\begin{multicols}{2}
\begin{lstlisting}[showspaces=true]
/* Correct. */
f(a, b, c, 16);
\end{lstlisting}

\begin{lstlisting}[showspaces=true]
/* Incorrect. */
f(a,b,c,16);
\end{lstlisting}
\end{multicols}
\medskip

On ne place {\bf pas d'espace} après une {\bf parenthèse ouvrante} ou 
avant une {\bf parenthèse fermante}.
\begin{multicols}{2}
\begin{lstlisting}[showspaces=true]
/* Correct. */
a = (f(a, 3) + a) * 2; 
\end{lstlisting}

\begin{lstlisting}[showspaces=true]
/* Incorrect. */
a = (f( a, 3) + a ) * 2;
\end{lstlisting}
\end{multicols}
\end{frame}

%%%%%%%%%%%%%%%%%%%%%%%%%%%%%%%%%%%%%%%%%%%%%%%%%%%%%%%%%%%%%%%%%%%%%%%%
\begin{frame}[fragile] \frametitle{Choix des identificateurs}
Les \alert{identificateurs} doivent à la fois {\bf renseigner sur le rôle} 
des entités auxquelles ils appartiennent (variables, paramètres, 
fonctions, modules, {\em etc.}) et être {\bf concis}. 
\begin{lstlisting}
/* Identificateur non explicite. */
v
\end{lstlisting}

\begin{lstlisting}
/* Identificateur trop long. */
valeur_choisie_pour_le_nombre_parties
\end{lstlisting}

\begin{lstlisting}
/* Identificateur acceptable. */
nb_parties
\end{lstlisting}
\medskip

On fixe la langue au {\bf français} pour leur construction.
\end{frame}

%%%%%%%%%%%%%%%%%%%%%%%%%%%%%%%%%%%%%%%%%%%%%%%%%%%%%%%%%%%%%%%%%%%%%%%%
\begin{frame}[fragile] \frametitle{Choix des identificateurs}
Les seuls identificateurs d'une lettre autorisés sont \Code{i},
\Code{j}, \Code{k}, {\em etc.}, pour les indices de boucles.
\medskip

Les majuscules sont interdites dans les identificateurs. Dans un
identificateur composé de plusieurs mots, ces derniers sont séparés par 
des sous-tirets.

\begin{multicols}{2}
\begin{lstlisting}
/* Correct. */
nb_parties
\end{lstlisting}

\begin{lstlisting}
/* Incorrect. */
nbParties
\end{lstlisting}
\end{multicols}

Exception~: les identificateurs de constantes (définies par une 
instruction pré-processeur) sont écrits exclusivement en majuscules.
\begin{lstlisting}
/* Correct. */

#define TAILLE_MAX 1024
#define DEBUG
\end{lstlisting}
\end{frame}

%%%%%%%%%%%%%%%%%%%%%%%%%%%%%%%%%%%%%%%%%%%%%%%%%%%%%%%%%%%%%%%%%%%%%%%%
\begin{frame}[fragile] \frametitle{Documentation}
Un programme est documenté par des \alert{commentaires}. Ce sont des 
{\bf phrases}, placées entre \Code{/*} et \Code{*/}.
\medskip

On documente chaque \alert{fichier de programme}, au tout début, par

\begin{itemize}
    \item les prénoms et noms des auteurs~;
    \smallskip
    
    \item la date de création (au format jour-mois-année) ~;
    \smallskip
    
    \item la date de modification (au format précédent).
\end{itemize}

\begin{lstlisting}
/* Auteur : L. W. Polsfuss
 * Creation : 01-01-1952
 * Modification : 12-08-2009 */
\end{lstlisting}
\bigskip

On commentera le moins possible les instructions.
\smallskip

Il faut éviter les commentaires inutiles.
\begin{lstlisting}
/* Incorrect. */
/* Affiche la valeur de `a`. */
printf("%d\n", a); 
\end{lstlisting}
\end{frame}

%%%%%%%%%%%%%%%%%%%%%%%%%%%%%%%%%%%%%%%%%%%%%%%%%%%%%%%%%%%%%%%%%%%%%%%%
\begin{frame}[fragile] \frametitle{Documentation des fonctions}
On documente la plupart des \alert{fonctions} par des commentaires 
situés avant leur déclaration (ou leur définition).
\medskip 

Un commentaire de fonction explique

\begin{itemize}
    \item le {\bf rôle} de chaque {\bf paramètre}~;
    \smallskip
    
    \item ce que {\bf renvoie} la fonction~; 
    \smallskip
     
    \item l'{\bf effet} produit par la fonction.
\end{itemize}

\begin{multicols}{2}
\begin{lstlisting}
/* Correct. */
/* Renvoie le plus grand entier 
 * parmi `a` et `b`. */
int max(int a, int b) {
    if (a >= b) 
        return a;
    return b;
}
\end{lstlisting}

\begin{lstlisting}
/* Incorrect. */
/* Calcule le maximum de deux 
 * entiers */
int max(int a, int b) {
    if (a >= b) 
        return a;
    return b;
}
\end{lstlisting}
\end{multicols}
\end{frame}

%%%%%%%%%%%%%%%%%%%%%%%%%%%%%%%%%%%%%%%%%%%%%%%%%%%%%%%%%%%%%%%%%%%%%%%%
%%%%%%%%%%%%%%%%%%%%%%%%%%%%%%%%%%%%%%%%%%%%%%%%%%%%%%%%%%%%%%%%%%%%%%%%
\subsection{Gestion d'erreurs}

%%%%%%%%%%%%%%%%%%%%%%%%%%%%%%%%%%%%%%%%%%%%%%%%%%%%%%%%%%%%%%%%%%%%%%%%
\begin{frame} \frametitle{Mécanisme de gestion d'erreurs}
On écrira la plupart des fonctions selon le schéma suivant~:

\begin{itemize}
    \item le type de retour est \Code{int} et la valeur de retour, le 
    \alert{code d'erreur}, renseigne si l'exécution de la fonction s'est 
    bien déroulée~;
    \smallskip
    
    \item la (ou les) valeur(s) \og renvoyée(s) \fg{} par la fonction 
    se fait par un (des) passage(s) par adresse.
\end{itemize}
\bigskip

Schématiquement, une telle fonction admet ainsi le prototype
\begin{center}
    \Code{int FCT(T1 E1, \dots, TN EN, U1 S1, \dots, UK SK);}
\end{center}
dit \alert{prototype standard} où

\begin{itemize}
    \item les \Code{Ei} sont les paramètres d'entrée~;
    \smallskip
    
    \item les \Code{Ti} sont des types (potentiellement des adresses)~;
    \smallskip
    
    \item les \Code{Sj} sont les paramètres de sortie~;
    \smallskip
    
    \item les \Code{Uj} sont des types (potentiellement des adresses).
\end{itemize}
\end{frame}

%%%%%%%%%%%%%%%%%%%%%%%%%%%%%%%%%%%%%%%%%%%%%%%%%%%%%%%%%%%%%%%%%%%%%%%%
\begin{frame} \frametitle{Mécanisme de gestion d'erreurs}
Le \alert{code d'erreur} d'une fonction ayant un prototype standard 
suit la spécification suivante~:

\begin{itemize}
    \item une valeur nulle \Code{0} signifie que l'exécution de la 
    fonction a échoué~;
    \smallskip
    
    \item une valeur positive signifie que l'exécution de la fonction 
    s'est bien déroulée et apporte une information supplémentaire 
    exploitable~;
    \smallskip
    
    \item une valeur négative renseigne sur une erreur particulière.
\end{itemize}
\bigskip
\bigskip

{\bf Attention}~: il y a des fonctions de la librairie standard qui ne 
suivent pas cette spécification.
\smallskip

De notre côté, nous allons la suivre à la lettre dans les fonctions que 
nous écrirons.
\end{frame}

%%%%%%%%%%%%%%%%%%%%%%%%%%%%%%%%%%%%%%%%%%%%%%%%%%%%%%%%%%%%%%%%%%%%%%%%
\begin{frame}[fragile] \frametitle{Exemple 1 de gestion d'erreur}
Considérons la fonction
\begin{lstlisting}
int division(float x, float y, float *res) {
    if (y == 0)
        return 0;
    *res = x / y;
    return 1;
}
\end{lstlisting}

Les entrées sont les flottants \Code{x} et \Code{y}. La sortie est 
\Code{res}~; c'est une adresse qui pointera sur le résultat de la 
division de \Code{x} par \Code{y}.
\smallskip

La valeur de retour est un \alert{code d'erreur}~: il vaut \Code{0}
lorsque la division ne peut pas être calculée (\Code{y} nul) et vaut
\Code{1} sinon.
\smallskip

On remarque que l'on ne modifie pas \Code{*res} lorsque le calcul ne 
peut pas être réalisé.
\end{frame}

%%%%%%%%%%%%%%%%%%%%%%%%%%%%%%%%%%%%%%%%%%%%%%%%%%%%%%%%%%%%%%%%%%%%%%%%
\begin{frame}[fragile] \frametitle{Exemple 2 de gestion d'erreur}
\footnotesize
Considérons la fonction
\begin{lstlisting}
int nb_min_maj(char *chaine, int *res_min, int *res_maj)
    int i;
    *res_min = 0;
    *res_maj = 0;
    i = 0;
    while (chaine[i] != '\0') {
        if ('a' <= chaine[i] <= 'z')
            *res_min += 1;
        else if ('A' <= chaine[i] <= 'Z')
            *res_maj += 1;
        else
            return 0;
        i += 1;
    }
    return i;
}
\end{lstlisting}

L'entrée est la chaîne de caractères \Code{chaine}. Les sorties sont 
\Code{res\_min} et \Code{res\_maj} ; ces adresses pointeront
sur le nombre de minuscules et de majuscules dans \Code{chaine}.
\smallskip

La valeur de retour est un \alert{code d'erreur}~: il vaut \Code{0}
si un caractère non alphabétique apparaît dans \Code{chaine} et vaut 
la longueur de \Code{chaine} sinon.
\end{frame}

%%%%%%%%%%%%%%%%%%%%%%%%%%%%%%%%%%%%%%%%%%%%%%%%%%%%%%%%%%%%%%%%%%%%%%%%
\begin{frame} \frametitle{Fonctions classiques à gestion d'erreurs}
La librairie standard du {\sf C} contient beaucoup de fonctions à
gestion d'erreurs. Par exemple~:

\begin{itemize} \small
    \item \Code{printf} renvoie le nombre de caractères écrits (sans 
    compter \Code{'\textbackslash 0'})~;
    \smallskip
    
    \item \Code{scanf} renvoie le nombre d'affectations réalisées lors 
    de la lecture. La valeur \Code{EOF} est renvoyée si une erreur de 
    lecture a lieu~;
    \smallskip
    
    \item \Code{malloc} renvoie un pointeur vers la zone de la mémoire 
    allouée. Lorsque l'allocation échoue, la valeur \Code{NULL} est 
    renvoyée~;
    \smallskip
    
    \item \Code{fopen} renvoie un pointeur sur le fichier ouvert. 
    Lorsque l'ouverture échoue,  la valeur \Code{NULL} est renvoyée~;
    \smallskip
    
    \item \Code{fclose} renvoie \Code{0} si la fermeture du fichier 
    s'est bien déroulée (attention à ce cas particulier). Lorsque 
    la fermeture échoue, la valeur \Code{EOF} est renvoyée.
\end{itemize}

{\bf Remarque}~: certaines de ces fonctions ont une gestion d'erreurs
encore plus sophistiquée et modifient des variables globales 
comme \Code{errno} (de l'en-tête \Code{errno.h}) pour renseigner 
précisément sur l'erreur survenue.
\end{frame}

%%%%%%%%%%%%%%%%%%%%%%%%%%%%%%%%%%%%%%%%%%%%%%%%%%%%%%%%%%%%%%%%%%%%%%%%
\begin{frame}[fragile] \frametitle{Emploi des fonctions à gestion d'erreurs}
On combine l'appel d'une fonction à gestion d'erreurs avec un test pour 
traiter l'erreur éventuelle.
\begin{lstlisting}
if (division(8, a, &b) == 0) {
    /* Traitement de l'erreur lors
     * de la division par zero. */
}
/* Instructions suivantes. */
\end{lstlisting}

\begin{lstlisting}
if (nb_min_maj("UnDeuxTrois", &a, &b) == 0) {
    /* Traitement de l'erreur lorsque
     * la chaine de caracteres contient des 
     * caracteres non alphabetiques. */
}
/* Instructions suivantes. */
\end{lstlisting}
\end{frame}

%%%%%%%%%%%%%%%%%%%%%%%%%%%%%%%%%%%%%%%%%%%%%%%%%%%%%%%%%%%%%%%%%%%%%%%%
\begin{frame}[fragile] \frametitle{Interruption de l'exécution}
Dans certains cas où une erreur survient, celle-ci peut être 
irrécupérable. Il faut donc \alert{interrompre l'exécution} du programme. 
On utilise pour cela la fonction
\begin{lstlisting}
    void exit(int status);
\end{lstlisting}
de \Code{stdlib.h}, appelée avec l'argument \Code{EXIT\_FAILURE}.
\medskip

\begin{lstlisting}
/* Allocation dynamique. */
tab = (int *) malloc(sizeof(int) * 1024);

/* Verification de son succes. */
if (NULL == tab)
    /* Sur son echec, on interrompt
     * l'execution immediatement. */
    exit(EXIT_FAILURE);
/* Instructions suivantes. */
\end{lstlisting}
\end{frame}

%%%%%%%%%%%%%%%%%%%%%%%%%%%%%%%%%%%%%%%%%%%%%%%%%%%%%%%%%%%%%%%%%%%%%%%%
%%%%%%%%%%%%%%%%%%%%%%%%%%%%%%%%%%%%%%%%%%%%%%%%%%%%%%%%%%%%%%%%%%%%%%%%
\subsection{Assertions d'entrée}

%%%%%%%%%%%%%%%%%%%%%%%%%%%%%%%%%%%%%%%%%%%%%%%%%%%%%%%%%%%%%%%%%%%%%%%%
\begin{frame}[fragile] \frametitle{Assertions d'entrée}
Lors d'un appel à une fonction, certains arguments peuvent être dans un 
état incohérent.
\medskip

Au lieu de gérer ces cas de figure par l'usage de codes d'erreur, il est
possible de tester l'état des arguments.
\medskip

Une \alert{pré-assertion} (ou assertion d'entrée) est un test réalisé 
dans une fonction pour vérifier si elle appelée avec des arguments 
adéquats.
\bigskip

On utilise la fonction 
\begin{lstlisting}
void assert(int a);
\end{lstlisting} 
du fichier d’en-tête \Code{assert.h}. Elle fonction de la manière 
suivante~:

\begin{itemize}
    \item lorsque l'assertion \Code{a} est fausse, l’exécution du 
    programme est interrompue et diverses informations utiles sont 
    affichées~;
    \smallskip
    
    \item lorsque \Code{a} est vraie, l'exécution continue.
\end{itemize}
\end{frame}

%%%%%%%%%%%%%%%%%%%%%%%%%%%%%%%%%%%%%%%%%%%%%%%%%%%%%%%%%%%%%%%%%%%%%%%%
\begin{frame}[fragile] 
    \frametitle{Exemple 1 de fonction avec pré-assertions}
Considérons la fonction
\begin{lstlisting}
void afficher_tab(int tab[], int nb) {
    int i;
    assert(tab != NULL);
    assert(nb >= 0);
    for (i = 1 ; i <= nb ; ++i)
        printf("%d\n", tab[i]);
}
\end{lstlisting}

Elle possède deux pré-assertions~:

\begin{enumerate}
    \item la première teste si le tableau \Code{tab} est bien un pointeur
    valide (différent de \Code{NULL})~;
    \smallskip
    
    \item la seconde teste si la taille \Code{nb} donnée est bien 
    positive.
\end{enumerate}
\end{frame}

%%%%%%%%%%%%%%%%%%%%%%%%%%%%%%%%%%%%%%%%%%%%%%%%%%%%%%%%%%%%%%%%%%%%%%%%
\begin{frame}[fragile] \frametitle{Conception de pré-assertions}
Il est important de \alert{munir ses fonctions de pré-assertions} les 
plus précises et complètes possibles. Quelques règles~:

\begin{itemize}
    \item la condition testée ne doit dépendre que des arguments d'une 
    fonction (elle ne dépend pas de données apprises à l'exécution)~;
    \smallskip
    
    \item la condition testée doit la plus atomique possible.
    \begin{multicols}{2}
        \begin{lstlisting}
/* Correct. */
assert(nb >= 0);
assert(nb <= 1024);
        \end{lstlisting}
  
        \begin{lstlisting}
/* Incorrect. */
assert(0 <= nb <= 1024);
        \end{lstlisting}
    \end{multicols}
    \smallskip
    
    \item Pour les concevoir, il faut imaginer les pires cas possibles
    à capturer qui peuvent survenir (p.ex., pointeurs nuls, quantités
    négatives, chaînes de caractères vides, {\em etc.}).
    \smallskip
    
    \item Elles sont situées juste après les déclarations de variables
    dans le corps d'une fonction.
\end{itemize}
\end{frame}

%%%%%%%%%%%%%%%%%%%%%%%%%%%%%%%%%%%%%%%%%%%%%%%%%%%%%%%%%%%%%%%%%%%%%%%%
\begin{frame}[fragile]
    \frametitle{Redondance nécessaire des pré-assertions}
Considérons les fonctions 
\begin{multicols}{2}
\begin{lstlisting}
int div(int a, int b) {
    assert(b != 0);
    return a / b;
}

\end{lstlisting}
\begin{lstlisting}
int somme_div(int a, int b) {
    return div(a, b)
        + div(b, a + 1);
}
\end{lstlisting}
\end{multicols}
{\bf Raisonnement}~: il n'y a pas de pré-assertion dans \Code{somme\_div}
mais cela n'est pas grave car les cas problématiques sont capturés par
\Code{div} qui contient une pré-assertion.
\medskip

Ceci est une {\bf fausse bonne idée}~: chaque fonction doit faire ses
propres pré-assertions. Toute erreur doit être capturée le plus en amont
possible.
\medskip

\begin{multicols}{2}
La bonne version de \Code{somme\_div} consiste à capturer les
mauvaises valeurs possibles de ses arguments de la manière suivante~:
\bigskip

\begin{lstlisting}
int somme_div(int a, int b) {
    assert(b != 0);
    assert(a + 1 != 0);
    return div(a, b)
        + div(b, a + 1);
}
\end{lstlisting}
\end{multicols}
\end{frame}

%%%%%%%%%%%%%%%%%%%%%%%%%%%%%%%%%%%%%%%%%%%%%%%%%%%%%%%%%%%%%%%%%%%%%%%%
\begin{frame}[fragile] \frametitle{Exemple 2 de fonction avec pré-assertion}
La fonction \Code{nb\_min\_maj}, à code d'erreur, doit être pourvue
de pré-assertions~:
\begin{lstlisting}
int nb_min_maj(char *chaine, int *res_min, int *res_maj)
    int i;
    
    assert(chaine != NULL);
    assert(res_min != NULL);
    assert(res_maj != NULL);
    
    /* Suite inchangee. */
}
\end{lstlisting}

On observe que le mécanisme de gestion d'erreurs par valeur de retour
teste des comportement incohérents complexes qui se déroulent à 
l'exécution, tandis que le mécanisme de pré-assertion permet de capturer 
des erreurs évidentes.
\end{frame}

%%%%%%%%%%%%%%%%%%%%%%%%%%%%%%%%%%%%%%%%%%%%%%%%%%%%%%%%%%%%%%%%%%%%%%%%
\begin{frame}[fragile] 
    \frametitle{Les pré-assertions pour corriger un programme}
Considérons le programme
\begin{multicols}{2}
\begin{lstlisting}    
#include <stdio.h>
#include <assert.h>

int div(int a, int b) {
    assert(b != 0);

    return a / b;
}

int main() {
    int a;

    a = div(17, 0);
    printf("%d\n", a);

    return 0;
}
\end{lstlisting}
\begin{small}
La compilation \Code{gcc -ansi -pedantic -Wall Prgm.c} donne 
l'exécutable \Code{a.out}.
\medskip

Son exécution \Code{./a.out} est interrompue en l.5. Elle produit la 
réponse 
\smallskip

\Sortie{a.out: Prgm.c:5: div: Assertion `b != 0' failed. \\
Aborted (core dumped)}
\medskip

On récolte la précieuse information sur le numéro de ligne de la 
pré-assertion non satisfaite qui produit l'arrêt précipité de 
l'exécution.
\end{small}
\end{multicols}
\end{frame}

% Auteur : Samuele Giraudo
% Création : jan. 2014, jan. 2015, fév. 2015 , déc. 2015

\tikzstyle{Module}=[rectangle,draw=Violet!100,fill=Violet!20,
    line width=1pt,font=\scriptsize\tt]
\tikzstyle{ModuleC}=[Module,draw=Bleu!100,fill=Bleu!20]
\tikzstyle{ModuleH}=[Module,draw=Vert!100,fill=Vert!20]
\tikzstyle{Fleche}=[->,draw=Rouge,line width=1.5pt]
\tikzstyle{Sommet}=[circle,draw=Marron!100,fill=Marron!10,line width=1.5pt]

%%%%%%%%%%%%%%%%%%%%%%%%%%%%%%%%%%%%%%%%%%%%%%%%%%%%%%%%%%%%%%%%%%%%%%%%
%%%%%%%%%%%%%%%%%%%%%%%%%%%%%%%%%%%%%%%%%%%%%%%%%%%%%%%%%%%%%%%%%%%%%%%%
%%%%%%%%%%%%%%%%%%%%%%%%%%%%%%%%%%%%%%%%%%%%%%%%%%%%%%%%%%%%%%%%%%%%%%%%
\section{Modules}

%%%%%%%%%%%%%%%%%%%%%%%%%%%%%%%%%%%%%%%%%%%%%%%%%%%%%%%%%%%%%%%%%%%%%%%%
%%%%%%%%%%%%%%%%%%%%%%%%%%%%%%%%%%%%%%%%%%%%%%%%%%%%%%%%%%%%%%%%%%%%%%%%
\subsection{Notion de modularité}

%%%%%%%%%%%%%%%%%%%%%%%%%%%%%%%%%%%%%%%%%%%%%%%%%%%%%%%%%%%%%%%%%%%%%%%%
\begin{frame}[fragile]
\frametitle{Modules}
\alert{Modulariser} un projet signifie le découper de manière cohérente
en plusieurs parties plus petites.
\medskip

Un \alert{module} est un ensemble de données et d'instructions qui
permettent de gérer une partie bien ciblée d'un projet.
\bigskip

Il existe deux manières de concevoir un projet~:
\smallskip

\begin{enumerate}
    \item programmer dans un unique fichier contenant tout
    le code nécessaire~;
    \smallskip

    \item programmer dans divers fichiers qui fractionnent le projet
    en plusieurs sous-parties.
\end{enumerate}
\medskip

À partir de maintenant, on adoptera la $2\ieme$ manière.
\bigskip

Il reste à savoir comment {\bf découper un projet de manière cohérente}
et comment {\bf utiliser les outils offerts par le langage} pour gérer ce
découpage.
\end{frame}

%%%%%%%%%%%%%%%%%%%%%%%%%%%%%%%%%%%%%%%%%%%%%%%%%%%%%%%%%%%%%%%%%%%%%%%%
\begin{frame}[fragile]
\frametitle{Avantages offerts par la modularité}
La modularité, illustration du principe stratégique universel
\begin{center}
    \og {\em diviser pour régner} \fg,
\end{center}
offre les avantages suivants.
\medskip

\begin{enumerate}
    \item La {\bf lisibilité} du code est accrue, ainsi que la facilité
    de son {\bf entretien}.
    \medskip

    \item Permet de {\bf regrouper} les types et les fonctions selon leurs
    objectifs.
    \medskip

    \item Il devient possible de {\bf réutiliser} dans un nouveau projet
    un module créé dans un projet antérieur.
    \medskip

    \item Permet de {\bf cacher des fonctions} (notion de fonctions privées).
    \medskip

    \item Facilite le {\bf travail par équipe}.
    \medskip

    \item Permet de rendre la {\bf compilation localisée}
    (compilation module par module).
\end{enumerate}
\end{frame}

%%%%%%%%%%%%%%%%%%%%%%%%%%%%%%%%%%%%%%%%%%%%%%%%%%%%%%%%%%%%%%%%%%%%%%%%
%%%%%%%%%%%%%%%%%%%%%%%%%%%%%%%%%%%%%%%%%%%%%%%%%%%%%%%%%%%%%%%%%%%%%%%%
\subsection{Découpage d'un projet}

%%%%%%%%%%%%%%%%%%%%%%%%%%%%%%%%%%%%%%%%%%%%%%%%%%%%%%%%%%%%%%%%%%%%%%%%
\begin{frame}[fragile]
\frametitle{Spécification d'un projet}
Considérons le \alert{projet spécifié} de la manière suivante~:
\medskip

\begin{itemize}
    \item le but est de fournir un programme qui permet de décider si des
    formules logiques sans quantificateur sont valides ou contradictoires.
    \medskip

    \item La syntaxe d'une formule est la suivante~: on dispose
    du jeu de formules atomiques $a$, $b$, \dots, $z$ et on écrit les
    formules de manière infixe et totalement parenthésée.
    Par exemple, la formule $(A \to (B \vee \neg C)) \wedge A$ s'écrit
    \Code{((a IMP (b OU (NON c))) ET a)}.
    \medskip

    \item L'interaction utilisateur/programme se fait de la manière
    suivante~:
    \begin{enumerate} \normalsize
        \item l'utilisateur fournit un fichier en entrée contenant une
        formule par ligne~;
        \smallskip

        \item le programme produit un fichier en sortie contenant, ligne
        par ligne, la réponse \Sortie{erreur} si la formule correspondante
        est syntaxiquement erronée ou bien \Sortie{valide},
        \Sortie{contradictoire} ou \Sortie{rien} selon la nature de la
        formule.
    \end{enumerate}
\end{itemize}
\end{frame}

%%%%%%%%%%%%%%%%%%%%%%%%%%%%%%%%%%%%%%%%%%%%%%%%%%%%%%%%%%%%%%%%%%%%%%%%
\begin{frame}[fragile]
\frametitle{Découpage du projet}
Il y a deux parties bien distinctes dans ce projet~:
\smallskip

\begin{enumerate}
    \item la {\bf représentation des formules} et le test de
    validité/contradiction~;
    \smallskip

    \item la {\bf gestion syntaxique des formules} (lecture/écriture
    d'une formule dans un fichier).
\end{enumerate}

Ces deux parties dictent le découpage suivant~:
\begin{center}
\begin{tikzpicture}[every text node part/.style={align=left}]
    \node[Module](Formule)at(0,0){
        -- type formule \\
        -- test validité \\
        -- test contradiction \\[.5em]
        {\small \bf Formule}};
    %
    \node[Module](Parseur)at(5,0){
        -- test syntaxique \\
        \phantom{--} d'une formule \\
        -- chaîne vers formule \\
        -- formule vers chaîne \\[.5em]
        {\small \bf Parseur}};
    %
    \node[Module](Main)at(2.5,-3){
        -- lecture fichier de \\
        \phantom{--} formules et écriture \\
        \phantom{--} des résultats \\[.5em]
        {\small \bf Main}};
    %
    \draw[Fleche](Parseur)--(Formule);
    \draw[Fleche](Main)edge[bend left](Formule);
    \draw[Fleche](Main)edge[bend right](Parseur);
\end{tikzpicture}
\end{center}
Toute flèche \hspace{-1em}
\raisebox{-.4em}{
\begin{tikzpicture}
    \node(1)at(0,0){$A$};
    \node(2)at(1,0){$B$};
    \draw[Fleche](1)--(2);
\end{tikzpicture}}
signifie que le module $A$ \alert{dépend} du module $B$.
\end{frame}

%%%%%%%%%%%%%%%%%%%%%%%%%%%%%%%%%%%%%%%%%%%%%%%%%%%%%%%%%%%%%%%%%%%%%%%%
%%%%%%%%%%%%%%%%%%%%%%%%%%%%%%%%%%%%%%%%%%%%%%%%%%%%%%%%%%%%%%%%%%%%%%%%
\subsection{Création de modules}

%%%%%%%%%%%%%%%%%%%%%%%%%%%%%%%%%%%%%%%%%%%%%%%%%%%%%%%%%%%%%%%%%%%%%%%%
\begin{frame}[fragile]
\frametitle{Composition d'un module}
Un module est composé de deux fichiers~:
\begin{enumerate}
    \item un \alert{fichier d'en-tête} d'extension \Code{.h}~;
    \item un \alert{fichier source} d'extension \Code{.c}.
\end{enumerate}
\medskip

Les noms de ces deux fichiers sont les mêmes (extension mise à part).
\medskip

\begin{center}
\begin{tikzpicture}[every text node part/.style={align=left}]
    \node[ModuleH](Ah)at(0,0){
        -- ... \\
        -- ... \\[.5em]
        {\small \bf A.h}};
    %
    \node[ModuleC](Ac)at(1.5,0){
        -- ... \\
        -- ... \\[.5em]
        {\small \bf A.c}};
\end{tikzpicture}
\end{center}

Seul le module principal est constitué d'un seul fichier source \Code{Main.c}.
\begin{center}
\begin{tikzpicture}[every text node part/.style={align=left}]
    \node[ModuleC](Mainc)at(0,0){
        -- ... \\
        -- ... \\[.5em]
        {\small \bf Main.c}};
\end{tikzpicture}
\end{center}
\bigskip

Par exemple, le projet précédent est constitué des fichiers
\Code{Formule.h}, \Code{Formule.c}, \Code{Parseur.h}, \Code{Parseur.c}
et \Code{Main.c}.
\end{frame}

%%%%%%%%%%%%%%%%%%%%%%%%%%%%%%%%%%%%%%%%%%%%%%%%%%%%%%%%%%%%%%%%%%%%%%%%
\begin{frame}[fragile]
\frametitle{Fichiers d'en-tête}
Un fichier d'en-tête contient des \alert{déclarations} de types et
de fonctions (prototypes).
\bigskip

Les prototypes qui y figurent sont ceux des fonctions que l'on
souhaite rendre visibles (utilisables) par d'autres modules.
\bigskip

Par exemple, un en-tête possible du module \Code{Formule} est
\medskip

\begin{minipage}[c]{.4\textwidth}
\begin{lstlisting}[frame=single,numbers=none]
/* Formule.h */

typedef struct {
...
} Form;

int est_valide(Form *f);
int est_contra(Form *f);
\end{lstlisting}
\end{minipage}
\end{frame}

%%%%%%%%%%%%%%%%%%%%%%%%%%%%%%%%%%%%%%%%%%%%%%%%%%%%%%%%%%%%%%%%%%%%%%%%
\begin{frame}[fragile]
\frametitle{Fichiers d'en-tête}
Les fichiers d'en-tête sont ceux que le programmeur regarde en premier
pour connaître le \alert{rôle d'un type ou d'une fonction}.
\medskip

De ce fait, c'est dans les fichiers d'en-tête que l'{\bf on commente chaque
type et fonction} pour préciser leur rôle.
\medskip

Par exemple, l'en-tête du module \Code{Formule} devrait être de la forme
\medskip

\begin{minipage}[c]{.9\textwidth}
\begin{lstlisting}[frame=single,numbers=none]
/* Formule.h */

/* Representation des formules logiques sans quantificateur. */
typedef struct {
...
} Form;

/* Renvoie `1` si la formule pointee par `f` est valide et
 * `0` sinon. */
int est_valide(Form *f);
...
\end{lstlisting}
\end{minipage}
\end{frame}

%%%%%%%%%%%%%%%%%%%%%%%%%%%%%%%%%%%%%%%%%%%%%%%%%%%%%%%%%%%%%%%%%%%%%%%%
\begin{frame}[fragile]
\frametitle{Fichiers source}
Un fichier source contient une \alert{implantation} de l'en-tête du
module auquel il appartient.
\medskip

Il contient de ce fait les définitions de fonctions déclarées dans
l'en-tête.
\medskip

Par exemple, une implantation possible du module \Code{Formule} est
\medskip

\begin{minipage}[c]{.4\textwidth}
\begin{lstlisting}[frame=single,numbers=none]
/* Formule.c */

...
int est_valide(Form *f) {
    int i, j;
    assert(f != NULL);
    ...
}

int est_contra(Form *f) {
    ...
}
\end{lstlisting}
\end{minipage}
\end{frame}

%%%%%%%%%%%%%%%%%%%%%%%%%%%%%%%%%%%%%%%%%%%%%%%%%%%%%%%%%%%%%%%%%%%%%%%%
\begin{frame}[fragile]
\frametitle{Fichiers source}
Il faut impérativement que toutes les fonctions déclarées dans le fichier
d'en-tête du module soient \alert{définies} dans le fichier source
correspondant.
\bigskip
\bigskip

Il est en revanche possible de définir dans un fichier source des fonctions
qui ne sont pas déclarées dans l'en-tête correspondant.
\bigskip
\bigskip

Une fonction définie dans un fichier source mais pas dans l'en-tête
correspondant s'appelle \alert{fonction privée}.
\bigskip
\bigskip

L'intérêt des fonctions privées est d'être des {\bf fonctions outils}
dont le champ d'application est local au module dans lequel elles sont
définies. On ne souhaite pas les rendre utilisables en dehors.
\end{frame}

%%%%%%%%%%%%%%%%%%%%%%%%%%%%%%%%%%%%%%%%%%%%%%%%%%%%%%%%%%%%%%%%%%%%%%%%
\begin{frame}[fragile]
\frametitle{Fichiers source et fonctions privées}
Une fonction privée se définit avec le mot clé \Code{static} (à ne pas
confondre avec le \Code{static} pour la déclaration de variables).
\bigskip

Par exemple, on peut avoir besoin d'une fonction privée appartenant au
module \Code{Formule} qui permet de compter le nombre d'occurrences d'un
atome dans une formule. On la définit alors dans \Code{Formule.c} par
\begin{lstlisting}
static int nb_occ(Form *f, char atome) {
    ...
    assert(f != NULL);
    assert('a' <= atome && atome <= 'z');
    ...
}
\end{lstlisting}

La portée lexicale de cette fonction s'étend à tout ce qui suit
sa définition dans le fichier \Code{Formule.c}. Elle est invisible
ailleurs.
\end{frame}

%%%%%%%%%%%%%%%%%%%%%%%%%%%%%%%%%%%%%%%%%%%%%%%%%%%%%%%%%%%%%%%%%%%%%%%%
\begin{frame}[fragile]
\frametitle{Fichiers source et fonctions privées}
C'est un contresens que de définir une fonction dans un fichier source
sans le mot clé \Code{static} et sans l'avoir déclarée dans le fichier
d'en-tête.
\bigskip

C'est aussi un contresens que de définir dans un fichier source une
fonction déclarée dans le fichier d'en-tête avec \Code{static}.
\bigskip

Ainsi, pour résumer, toute fonction définie dans un fichier source est
\begin{enumerate}
    \item soit déclarée dans le fichier d'en-tête~;
    \item soit non déclarée dans le fichier d'en-tête mais définie
    par \Code{static}.
\end{enumerate}
\end{frame}

%%%%%%%%%%%%%%%%%%%%%%%%%%%%%%%%%%%%%%%%%%%%%%%%%%%%%%%%%%%%%%%%%%%%%%%%
\begin{frame}[fragile]
\frametitle{Allure d'un projet}
Il y a deux manières d'organiser un projet en termes de fichiers et de
répertoires~:
\medskip

\begin{enumerate}
    \item la 1\iere{} consiste à regrouper l'ensemble des fichiers
    d'en-tête et des fichiers sources dans un même répertoire. Il y figure
    donc un nombre impair de fichiers (le module principal et les
    paires en-tête/source pour chaque module)~;
    \bigskip

    \item la 2\ieme{} consiste à séparer les fichiers du projet
    en deux répertoires frères, \Code{include} et \Code{src}, le
    premier contenant les fichiers d'en-tête et l'autre, les fichiers
    sources et le module principal du projet.
\end{enumerate}

\end{frame}

%%%%%%%%%%%%%%%%%%%%%%%%%%%%%%%%%%%%%%%%%%%%%%%%%%%%%%%%%%%%%%%%%%%%%%%%
%%%%%%%%%%%%%%%%%%%%%%%%%%%%%%%%%%%%%%%%%%%%%%%%%%%%%%%%%%%%%%%%%%%%%%%%
\subsection{Utilisation des modules}

%%%%%%%%%%%%%%%%%%%%%%%%%%%%%%%%%%%%%%%%%%%%%%%%%%%%%%%%%%%%%%%%%%%%%%%%
\begin{frame}[fragile]
\frametitle{Inclusion de modules}
Pour utiliser un module \Code{Module} dans un fichier \Code{F}, on doit
l'y \alert{inclure}.
\bigskip

On utilise pour cela dans \Code{F} la commande pré-processeur
\begin{center}
    \Code{\#include "Module.h"}
\end{center}
\bigskip

Celle-ci sera remplacée par le pré-processeur par le contenu de
\Code{Module.h}.
\bigskip

Cette commande peut se trouver
\smallskip

\begin{itemize}
    \item dans un fichier source pour bénéficier des fonctions définies et
    des types déclarés par le module~;
    \medskip

    \item dans un fichier d'en-tête pour bénéficier des types déclarés
    par le module.
\end{itemize}
\end{frame}

%%%%%%%%%%%%%%%%%%%%%%%%%%%%%%%%%%%%%%%%%%%%%%%%%%%%%%%%%%%%%%%%%%%%%%%%
\begin{frame}[fragile]
\frametitle{Inclusion de modules}
Par exemple, si \Code{A} est un module définissant une fonction
\Code{f}, pour utiliser \Code{f} dans un fichier \Code{B.c}
 situé dans le même répertoire que \Code{A.c} et \Code{A.h},
on écrit
\medskip

\begin{minipage}[c]{.3\textwidth}
\begin{lstlisting}[frame=single,numbers=none]
/* B.c */

#include "A.h"
...
\end{lstlisting}
\end{minipage}
\medskip

Il est possible d'inclure à la suite plusieurs modules dans un même fichier~:
\medskip

\begin{minipage}[c]{.4\textwidth}
\begin{lstlisting}[frame=single,numbers=none]
/* Fichier.c ou Fichier.h */

#include "A.h"
#include "B.h"
#include "C.h"
...
\end{lstlisting}
\end{minipage}
\qquad
\begin{minipage}[c]{.5\textwidth}
De cette manière, \Code{Fichier.c} ou \Code{Fichier.h} bénéficie de tout
ce qui est déclaré et défini dans les modules \Code{A}, \Code{B} et
\Code{C}.
\medskip

{\bf Attention}~: les modules inclus ne doivent pas déclarer/définir des
éléments d'un identificateur commun.
\end{minipage}
\end{frame}

%%%%%%%%%%%%%%%%%%%%%%%%%%%%%%%%%%%%%%%%%%%%%%%%%%%%%%%%%%%%%%%%%%%%%%%%
\begin{frame}[fragile]
\frametitle{Inclusion de modules}
Le fichier incluant n'a \alert{accès} qu'au fichier d'en-tête du module,
et donc qu'aux \alert{déclarations effectuées}.
\medskip

Le fichier incluant n'a pas besoin de connaître l'implantation du module.
\bigskip

Considérons par exemple le module \Code{Couple} défini par
\medskip

\begin{minipage}[c]{.38\textwidth}
\begin{lstlisting}[frame=single,numbers=none]
/* Couple.h */

typedef int Couple[2];

int est_zero(Couple c);
void afficher(Couple c);
\end{lstlisting}
\end{minipage}
\quad
\begin{minipage}[c]{.55\textwidth}
\begin{lstlisting}[frame=single,numbers=none]
/* Couple.c */
...
int est_zero(Couple c) {
  return (c[0] == 0) && (c[1]== 0);
}
void afficher(Couple c) {
    printf("(%d, %d)", c[0], c[1]);
}
\end{lstlisting}
\end{minipage}
\end{frame}

%%%%%%%%%%%%%%%%%%%%%%%%%%%%%%%%%%%%%%%%%%%%%%%%%%%%%%%%%%%%%%%%%%%%%%%%
\begin{frame}[fragile]
\frametitle{Inclusion de modules}
Supposons que l'on ait besoin d'inclure le module \Code{Couple} dans un
fichier \Code{Fichier.c}.
\medskip

Le pré-processeur aura donc l'effet suivant sur \Code{Fichier.c}~:
\medskip

\begin{minipage}[c]{.3\textwidth}
\begin{lstlisting}[frame=single,numbers=none]
/* Fichier.c avant
 * la passe du
 * pre-processeur */

#include "Couple.h"
...

if (!est_zero(c))
    afficher(c);
...
\end{lstlisting}
\end{minipage}
\qquad \qquad
\begin{minipage}[c]{.4\textwidth}
\begin{lstlisting}[frame=single,numbers=none]
/* Fichier.c apres la passe
 * du pre-processeur */

typedef int Couple[2];
int est_zero(Couple c);
void afficher(Couple c);
...

if (!est_zero(c))
    afficher(c);
...
\end{lstlisting}
\end{minipage}
\medskip

\Code{Fichier.c} a besoin uniquement de connaître les types de retour et
les signatures des fonctions qu'il invoque (connus à leur déclaration).
Il n'a à ce stade pas besoin de connaître les définitions de ces fonctions.
\end{frame}

%%%%%%%%%%%%%%%%%%%%%%%%%%%%%%%%%%%%%%%%%%%%%%%%%%%%%%%%%%%%%%%%%%%%%%%%
\begin{frame}[fragile]
\frametitle{Création complète d'un module}
Pour créer un module \Code{A}, il faut inclure son fichier d'en-tête
\Code{A.h} dans son fichier source \Code{A.c}.
\medskip

\begin{minipage}[c]{.3\textwidth}
\begin{lstlisting}[frame=single,numbers=none]
/* A.h */
...
\end{lstlisting}
\end{minipage}
\qquad \qquad
\begin{minipage}[c]{.3\textwidth}
\begin{lstlisting}[frame=single,numbers=none]
/* A.c */

#include "A.h"
...
\end{lstlisting}
\end{minipage}
\medskip

De cette manière,
\begin{itemize}
    \item d'une part, \Code{A.c} a accès aux types et
    aux prototypes de fonctions déclarés dans \Code{A.h}~;
    \smallskip

    \item d'autre part, cela permet d'implanter les fonctions déclarées
    dans \Code{A.h} sans contrainte d'ordre.
\end{itemize}
\end{frame}

%%%%%%%%%%%%%%%%%%%%%%%%%%%%%%%%%%%%%%%%%%%%%%%%%%%%%%%%%%%%%%%%%%%%%%%%
\begin{frame}[fragile]
\frametitle{Création complète d'un module}
Considérons la situation suivante~:
\bigskip

\begin{minipage}[c]{.22\textwidth}
\begin{lstlisting}[frame=single,numbers=none]
/* A.h */
...
#include "C.h"
...
\end{lstlisting}
\end{minipage}
\enspace
\begin{minipage}[c]{.22\textwidth}
\begin{lstlisting}[frame=single,numbers=none]
/* A.c */

#include "A.h"
...
\end{lstlisting}
\end{minipage}
\quad
\begin{minipage}[c]{.22\textwidth}
\begin{lstlisting}[frame=single,numbers=none]
/* B.h */
...
#include "C.h"
...
\end{lstlisting}
\end{minipage}
\enspace
\begin{minipage}[c]{.22\textwidth}
\begin{lstlisting}[frame=single,numbers=none]
/* B.c */

#include "B.h"
...
\end{lstlisting}
\end{minipage}
\medskip

\begin{minipage}[c]{.22\textwidth}
\begin{lstlisting}[frame=single,numbers=none]
/* C.h */
...
int f();
...
\end{lstlisting}
\end{minipage}
\enspace
\begin{minipage}[c]{.22\textwidth}
\begin{lstlisting}[frame=single,numbers=none]
/* C.c */

#include "C.h"
...
\end{lstlisting}
\end{minipage}
\qquad \qquad
\begin{minipage}[c]{.22\textwidth}
\begin{lstlisting}[frame=single,numbers=none]
/* D.c */

#include "A.h"
#include "B.h"
...
\end{lstlisting}
\end{minipage}
\end{frame}

%%%%%%%%%%%%%%%%%%%%%%%%%%%%%%%%%%%%%%%%%%%%%%%%%%%%%%%%%%%%%%%%%%%%%%%%
\begin{frame}[fragile]
\frametitle{Création complète d'un module}
Le pré-processeur transforme ces fichiers en
\bigskip

\begin{minipage}[c]{.22\textwidth}
\begin{lstlisting}[frame=single,numbers=none]
/* A.h */
...
int f();
...
\end{lstlisting}
\end{minipage}
\enspace
\begin{minipage}[c]{.22\textwidth}
\begin{lstlisting}[frame=single,numbers=none]
/* A.c */

/* Copie A.h */
...
\end{lstlisting}
\end{minipage}
\quad
\begin{minipage}[c]{.22\textwidth}
\begin{lstlisting}[frame=single,numbers=none]
/* B.h */
...
int f();
...
\end{lstlisting}
\end{minipage}
\enspace
\begin{minipage}[c]{.22\textwidth}
\begin{lstlisting}[frame=single,numbers=none]
/* B.c */

/* Copie B.h */
...
\end{lstlisting}
\end{minipage}
\medskip

\begin{minipage}[c]{.22\textwidth}
\begin{lstlisting}[frame=single,numbers=none]
/* C.h */
...
int f();
...
\end{lstlisting}
\end{minipage}
\enspace
\begin{minipage}[c]{.22\textwidth}
\begin{lstlisting}[frame=single,numbers=none]
/* C.c */

/* Copie C.h */
...
\end{lstlisting}
\end{minipage}
\qquad \qquad
\begin{minipage}[c]{.22\textwidth}
\begin{lstlisting}[frame=single,numbers=none]
/* D.c */

int f();
int f();
...
\end{lstlisting}
\end{minipage}
\medskip

{\bf Problème}~: le contenu de \Code{C.h} est copié deux fois dans
\Code{D.c}. Ceci n'est pas accepté par le compilateur car il y a
\alert{multiple déclaration} d'un même symbole (\Code{f} ici).
\end{frame}

%%%%%%%%%%%%%%%%%%%%%%%%%%%%%%%%%%%%%%%%%%%%%%%%%%%%%%%%%%%%%%%%%%%%%%%%
\begin{frame}[fragile]
\frametitle{Création complète d'un module}
La parade consiste à \alert{inclure} un fichier d'en-tête de
\alert{manière conditionnelle}~: on procède à l'inclusion que s'il n'a
pas déjà été inclus.
\medskip

On utilise pour cela les commandes \Code{\#define}, \Code{\#ifndef} et
\Code{\#endif} du pré-processeur.
\medskip

Le schéma général est
\medskip

\begin{minipage}[c]{.49\textwidth}
\begin{lstlisting}[frame=single,numbers=none]
/* A.h */
#ifndef __A__
#define __A__

    /* Declaration de types */

    /* Declaration de fonctions */

#endif
\end{lstlisting}
\end{minipage}\quad
\begin{minipage}[c]{.45\textwidth}
    Ainsi, lors d'une inclusion de \Code{A.h}, le pré-processeur
    vérifie si la macro \Code{\_\_A\_\_} n'existe pas.
    \begin{itemize}
        \item Si elle n'existe pas, alors on la définit
        (\Code{\#define \_\_A\_\_}) et le contenu du module est pris
        en compte~;
        \item sinon, cela signifie que le contenu a déjà été pris en
        compte. Celui-ci n'est pas repris en compte une 2\ieme{} fois.
    \end{itemize}
\end{minipage}
\end{frame}

%%%%%%%%%%%%%%%%%%%%%%%%%%%%%%%%%%%%%%%%%%%%%%%%%%%%%%%%%%%%%%%%%%%%%%%%
\begin{frame}[fragile]
\frametitle{Squelette d'un module}
Pour résumer, tous les modules doivent avoir le squelette suivant~:
\begin{minipage}[c]{.42\textwidth}
\begin{lstlisting}[frame=single,numbers=none]
/* A.h */
#ifndef __A__
#define __A__

  /* Inclusions eventuelles
   * de modules */

  /* Definitions eventuelles
   * de macros */

  /* Declarations eventuelles
   * de types */

  /* Declarations eventuelles
   * de fonctions */

#endif
\end{lstlisting}
\end{minipage}
\qquad
\begin{minipage}[c]{.45\textwidth}
\begin{lstlisting}[frame=single,numbers=none]
/* A.c */
#include "A.h"

/* Inclusions eventuelles
 * de modules */

/* Definitions eventuelles
 * de fonctions privees */

/* Definitions de toutes les
 * fonctions declarees dans
 * le fichier d'en-tete */
\end{lstlisting}
\end{minipage}
\end{frame}

%%%%%%%%%%%%%%%%%%%%%%%%%%%%%%%%%%%%%%%%%%%%%%%%%%%%%%%%%%%%%%%%%%%%%%%%
\begin{frame}[fragile]
\frametitle{Exemple du module {\tt Parseur}}
\begin{minipage}[c]{.9\textwidth}
\begin{lstlisting}[frame=single,numbers=none,basicstyle=\ttfamily\scriptsize]
/* Parseur.h */
#ifndef __PARSEUR__
#define __PARSEUR__

#include "Formule.h"

    /* Convertit la chaine de caracteres `ch` sensee representer une formule
     * en une variable de type `Form`, qui va etre ecrite dans `f`.
     * Renvoie `1` si `ch` represente bien une formule et `0` sinon. */
    int chaine_vers_form(Form *f, char *ch);

#endif
\end{lstlisting}
\end{minipage}

\begin{minipage}[c]{.5\textwidth}
\begin{lstlisting}[frame=single,numbers=none,basicstyle=\ttfamily\scriptsize]
/* Parseur.c */
#include "Parseur.h"

#include <stdio.h>
#include <stdlib.h>
#include <assert.h>

int chaine_vers_form(Form *f, char *ch) {
    ...
    assert(f != NULL);
    ...
}
\end{lstlisting}
\end{minipage}
\end{frame}

%%%%%%%%%%%%%%%%%%%%%%%%%%%%%%%%%%%%%%%%%%%%%%%%%%%%%%%%%%%%%%%%%%%%%%%%
%%%%%%%%%%%%%%%%%%%%%%%%%%%%%%%%%%%%%%%%%%%%%%%%%%%%%%%%%%%%%%%%%%%%%%%%
\subsection{Graphes d'inclusions}

%%%%%%%%%%%%%%%%%%%%%%%%%%%%%%%%%%%%%%%%%%%%%%%%%%%%%%%%%%%%%%%%%%%%%%%%
\begin{frame}[fragile]
\frametitle{Graphes d'inclusions}
On rappelle qu'un module \Code{A} \alert{dépend} d'un module \Code{B} si
le {\bf fichier d'en-tête} de \Code{A} inclut \Code{B}.
\medskip

Pour visualiser l'allure d'un projet, on trace son \alert{graphe d'inclusions}.
On représente pour cela chacun des modules qui le composent dans des cercles
({\bf sommets}) et on trace des flèches ({\bf arcs}) de \Code{A} vers
\Code{B} pour tout module \Code{A} dépendant de \Code{B}.
\medskip

Par exemple, le graphe d'inclusions
\begin{multicols}{2}
\begin{center}
\scalebox{.8}{\begin{tikzpicture}
    \node[Sommet](A)at(0,0){\Code{A}};
    \node[Sommet](B)at(3,0){\Code{B}};
    \node[Sommet](C)at(0,-2){\Code{C}};
    \node[Sommet](D)at(3,-2){\Code{D}};
    \draw[Fleche](A)edge[bend left=20]node{}(B);
    \draw[Fleche](A)edge[bend right=20]node{}(D);
    \draw[Fleche](B)edge[bend left=20]node{}(D);
    \draw[Fleche](C)edge[bend right=20]node{}(D);
\end{tikzpicture}}
\end{center}
signifie que
\begin{itemize}
    \item \Code{A.h} inclut \Code{B.h} et \Code{D.h}~;
    \item \Code{B.h} inclut \Code{D.h}~;
    \item \Code{C.h} inclut \Code{D.h}~;
    \item \Code{D.h} n'inclut rien.
\end{itemize}
\end{multicols}
\medskip

On ne mentionne pas dans les graphes d'inclusions les inclusions aux
fichiers d'en-tête standards (\Code{stdio.h}, \Code{stdlib.h},
\Code{assert.h}, {\em etc.}).
\end{frame}

%%%%%%%%%%%%%%%%%%%%%%%%%%%%%%%%%%%%%%%%%%%%%%%%%%%%%%%%%%%%%%%%%%%%%%%%
\begin{frame}[fragile]
\frametitle{Graphe d'inclusions et fichier principal}
Le graphe d'inclusions d'un projet consistant à faire jouer l'ordinateur
aux échecs contre un humain peut être le suivant~:
\begin{center}
\scalebox{.75}{\begin{tikzpicture}
    \node[Sommet,minimum size=2cm](Piece)at(-4,0){\Code{Piece}};
    \node[Sommet,minimum size=2cm](Case)at(-4,-4){\Code{Case}};
    \node[Sommet,minimum size=2cm](Position)at(0,-2){\Code{Position}};
    \node[Sommet,minimum size=2cm](IA)at(4,0){\Code{IA}};
    \node[Sommet,minimum size=2cm](IGraph)at(4,-4){\Code{IGraph}};
    \node[Sommet,minimum size=2cm,draw=Vert,fill=Vert!20](Main)at(7,-2){\Code{Main}};
    \draw[Fleche](Main)edge[bend right=20]node{}(IA);
    \draw[Fleche](Main)edge[bend left=20]node{}(IGraph);
    \draw[Fleche](IGraph)edge[bend left=20]node{}(Case);
    \draw[Fleche](IGraph)edge[bend left=20]node{}(Position);
    \draw[Fleche](IGraph)edge[bend right=40]node{}(Piece);
    \draw[Fleche](IA)edge[bend left=40]node{}(Case);
    \draw[Fleche](IA)edge[bend right=20]node{}(Position);
    \draw[Fleche](IA)edge[bend right=20]node{}(Piece);
    \draw[Fleche](Position)edge[bend right=20]node{}(Case);
    \draw[Fleche](Position)edge[bend left=20]node{}(Piece);
\end{tikzpicture}}
\end{center}
\medskip

Tout projet contient un fichier source \Code{Main.c},
\alert{le fichier principal} du projet, où figure la fonction \Code{main}
(le {\bf point d'entrée} de l'exécution du programme). Celui-ci apparaît
dans le graphe d'inclusions.
\end{frame}

%%%%%%%%%%%%%%%%%%%%%%%%%%%%%%%%%%%%%%%%%%%%%%%%%%%%%%%%%%%%%%%%%%%%%%%%
\begin{frame}[fragile]
\frametitle{Inclusions circulaires}
Les graphes d'inclusions permettent d'avoir une vision globale de
l'architecture d'un projet.
\bigskip

Ils permettent aussi de mettre en évidence des problèmes de conception
et notamment les problèmes d'\alert{inclusion circulaire}. Ce type de
problème s'observe par la présence d'un {\bf cycle} dans le graphe
d'inclusions~:
\begin{center}
\scalebox{.8}{\begin{tikzpicture}
    \node[Sommet](A)at(0,0){\Code{A}};
    \node[Sommet](B)at(2,0){\Code{B}};
    \node[Sommet](C)at(0,-2){\Code{C}};
    \node[Sommet](D)at(2,-2){\Code{D}};
    \node[Sommet](E)at(4,-1){\Code{E}};
    \draw[Fleche,draw=Violet,line width=3pt](A)edge[bend left=20]node{}(B);
    \draw[Fleche,draw=Violet,line width=3pt](B)edge[bend left=20]node{}(D);
    \draw[Fleche,draw=Violet,line width=3pt](D)edge[bend left=20]node{}(C);
    \draw[Fleche,draw=Violet,line width=3pt](C)edge[bend left=20]node{}(A);
    \draw[Fleche](E)edge[bend right=20]node{}(B);
    \draw[Fleche](E)edge[bend left=20]node{}(D);
    \draw[Fleche](B)edge[bend left=20]node{}(C);
\end{tikzpicture}}
\end{center}
\medskip

{\bf Règle importante}~: il ne doit jamais y avoir de cycle dans le
graphe d'inclusions d'un projet. S'il y a un cycle, c'est que le projet
est mal découpé en modules.
\end{frame}

%%%%%%%%%%%%%%%%%%%%%%%%%%%%%%%%%%%%%%%%%%%%%%%%%%%%%%%%%%%%%%%%%%%%%%%%
\begin{frame}[fragile]
\frametitle{Limiter les inclusions}
La plupart des inclusions circulaires peuvent être évitées en
\alert{réduisant au maximum les inclusions} de modules dans les
\alert{fichiers d'en-tête} en les faisant plutôt si possible dans les
fichiers source.
\bigskip
\bigskip
\bigskip

Par exemple, supposons que l'on dispose d'un module \Code{Tri} qui
permet de trier des tableaux génériques. Il est de la forme~:
\medskip

\begin{minipage}[c]{.45\textwidth}
\begin{lstlisting}[frame=single,numbers=none,basicstyle=\ttfamily\scriptsize]
/* Tri.h */
#ifndef __TRI__
#define __TRI__

    void trier_tab(void **t, int n,
        int (*est_inf)(void *, void *));
    ...
#endif
\end{lstlisting}
\end{minipage}
\quad
\begin{minipage}[c]{.45\textwidth}
\begin{lstlisting}[frame=single,numbers=none,basicstyle=\ttfamily\scriptsize]
/* Tri.c */
#include "Tri.h"
...
void trier_tab(void **t, int n,
    int (*est_inf)(void *, void *)) {
    ...
}
...
\end{lstlisting}
\end{minipage}
\end{frame}

%%%%%%%%%%%%%%%%%%%%%%%%%%%%%%%%%%%%%%%%%%%%%%%%%%%%%%%%%%%%%%%%%%%%%%%%
\begin{frame}[fragile]
\frametitle{Limiter les inclusions}
On souhaite maintenant écrire un module \Code{TabInt} pour gérer des
tableaux d'entiers.
\medskip

On n'écrira pas

\begin{minipage}[c]{.55\textwidth}
\begin{lstlisting}[frame=single,numbers=none,basicstyle=\ttfamily\scriptsize]
/* TabInt.h */
#ifndef __TAB_INT__
#define __TAB_INT__

#include "Tri.h"

    typedef struct {int n; int *tab;} TabInt;
    int trier_tab_int(TabInt *t);
#endif
\end{lstlisting}
\end{minipage}
\quad
\begin{minipage}[c]{.35\textwidth}
\begin{lstlisting}[frame=single,numbers=none,basicstyle=\ttfamily\scriptsize]
/* TabInt.c */
#include "TabInt.h"

int trier_tab_int(TabInt *t) {
    /* Util. de `trier_tab` */
}
\end{lstlisting}
\end{minipage}

mais plutôt

\begin{minipage}[c]{.55\textwidth}
\begin{lstlisting}[frame=single,numbers=none,basicstyle=\ttfamily\scriptsize]
/* TabInt.h */
#ifndef __TAB_INT__
#define __TAB_INT__

    typedef struct {int n; int *tab;} TabInt;
    int trier_tab_int(TabInt *t);
#endif
\end{lstlisting}
\end{minipage}
\quad
\begin{minipage}[c]{.35\textwidth}
\begin{lstlisting}[frame=single,numbers=none,basicstyle=\ttfamily\scriptsize]
/* TabInt.c */
#include "TabInt.h"

#include "Tri.h"

int trier_tab_int(TabInt *t) {
    /* Util. de `trier_tab` */
}
\end{lstlisting}
\end{minipage}
\end{frame}

% Auteur : Samuele Giraudo
% Création : fév. 2014, jan 2015, mars 2015, déc. 2015, fév. 2016

\tikzstyle{Module}=[rectangle,draw=Violet!100,fill=Violet!20,
    line width=1pt,font=\scriptsize\tt]
\tikzstyle{FObj}=[rectangle,draw=Rouge!100,fill=Rouge!20,
    line width=1pt,font=\scriptsize\tt]
\tikzstyle{Bib}=[rectangle,draw=Vert!100,fill=Vert!20,
    line width=1pt,font=\scriptsize\tt]
\tikzstyle{Exec}=[rectangle,draw=Bleu!100,fill=Bleu!20,
    line width=1pt,font=\scriptsize\tt]
\tikzstyle{Fleche}=[->,draw=Rouge,line width=1.5pt]
\tikzstyle{Sommet}=[circle,draw=Marron!100,fill=Marron!10,line width=1.5pt]

%%%%%%%%%%%%%%%%%%%%%%%%%%%%%%%%%%%%%%%%%%%%%%%%%%%%%%%%%%%%%%%%%%%%%%%%
%%%%%%%%%%%%%%%%%%%%%%%%%%%%%%%%%%%%%%%%%%%%%%%%%%%%%%%%%%%%%%%%%%%%%%%%
%%%%%%%%%%%%%%%%%%%%%%%%%%%%%%%%%%%%%%%%%%%%%%%%%%%%%%%%%%%%%%%%%%%%%%%%
\section{Compilation}

%%%%%%%%%%%%%%%%%%%%%%%%%%%%%%%%%%%%%%%%%%%%%%%%%%%%%%%%%%%%%%%%%%%%%%%%
%%%%%%%%%%%%%%%%%%%%%%%%%%%%%%%%%%%%%%%%%%%%%%%%%%%%%%%%%%%%%%%%%%%%%%%%
\subsection{Étapes}

%%%%%%%%%%%%%%%%%%%%%%%%%%%%%%%%%%%%%%%%%%%%%%%%%%%%%%%%%%%%%%%%%%%%%%%%
\begin{frame}[fragile]
\frametitle{Compilation d'un projet d'un fichier}
La \alert{compilation} d'un projet constitué d'un \alert{unique fichier}
\Code{Fichier.c} contenant la fonction principale \Code{main} se fait
par la commande
\begin{center}\Code{gcc Fichier.c}\end{center}
\medskip

Cette commande réalise à la suite les étapes suivantes~:
\begin{enumerate}
    \item traitement préliminaire par le {\bf pré-processeur}~;
    \smallskip

    \item compilation en {\bf langage assembleur}~;
    \smallskip

    \item traduction du langage assembleur en {\bf langage machine}~;
    \smallskip

    \item {\bf édition des liens}.
\end{enumerate}
\medskip

Elle permet d'obtenir un fichier {\bf exécutable}.
\end{frame}

%%%%%%%%%%%%%%%%%%%%%%%%%%%%%%%%%%%%%%%%%%%%%%%%%%%%%%%%%%%%%%%%%%%%%%%%
\begin{frame}[fragile]
\frametitle{Compilation d'un projet d'un fichier}
\begin{center}
    \begin{tikzpicture}
        \node(c)at(0,0){\Code{Fichier.c}};
        \node(i)at(0,-3){\Code{Fichier.i}};
        \node(s)at(4,-3){\Code{Fichier.s}};
        \node(o)at(8,-3){\Code{Fichier.o}};
        \node(a)at(8,-6){\Code{a.out}};
        \draw[->](c)edge[anchor=west,font=\scriptsize] node{pré-processeur}(i);
        \draw[->](i)edge[anchor=north,font=\scriptsize] node{assembleur}(s);
        \draw[->](s)edge[anchor=north,font=\scriptsize] node{langage machine}(o);
        \draw[->](o)edge[anchor=east,font=\scriptsize] node{exécutable}(a);
    \end{tikzpicture}
\end{center}
\end{frame}

%%%%%%%%%%%%%%%%%%%%%%%%%%%%%%%%%%%%%%%%%%%%%%%%%%%%%%%%%%%%%%%%%%%%%%%%
\begin{frame}[fragile]
\frametitle{Pré-processeur}
Le \alert{pré-processeur} réalise un pré-traitement du fichier source
pour le rendre traduisible en langage machine.
\medskip

Il procède en
\begin{enumerate}
    \item supprimant les commentaires~;
    \smallskip

    \item incluant les fichiers d'en-tête
    (copie/colle les fichiers \Code{.h} inclus)~;
    \smallskip

    \item traitant les définitions de symboles par un
    mécanisme de substitution (\Code{\#define})~;
    \smallskip

    \item traitant les macro-instructions de contrôle de compilation
    (\Code{\#ifndef}, \Code{\#endif}, {\em etc.}).
\end{enumerate}
\medskip

Il est possible de récupérer le fichier d'extension \Code{.i} ainsi
obtenu par la commande
\begin{center} \Code{gcc -E Fichier.c >\,> Fichier.i} \end{center}
\end{frame}

%%%%%%%%%%%%%%%%%%%%%%%%%%%%%%%%%%%%%%%%%%%%%%%%%%%%%%%%%%%%%%%%%%%%%%%%
\begin{frame}[fragile]
\frametitle{Compilation en assembleur}
Après avoir été traité par le pré-processeur, le fichier \Code{Fichier.i}
est \alert{traduit en assembleur}.
\medskip

Il est possible de récupérer le fichier d'extension \Code{.s} ainsi
obtenu par la commande
\begin{center} \Code{gcc -S Fichier.c} \end{center}
\bigskip

L'assembleur est un langage très proche de la machine. Il peut se traduire
assez facilement en un langage directement exécutable par le procésseur.
\medskip

Il existe plusieurs langages d'assemblage différents~: au moins un par
architecture.
\end{frame}

%%%%%%%%%%%%%%%%%%%%%%%%%%%%%%%%%%%%%%%%%%%%%%%%%%%%%%%%%%%%%%%%%%%%%%%%
\begin{frame}[fragile]
\frametitle{Compilation en assembleur}
Par exemple, avec le fichier \Code{Fichier.c} suivant~:

\begin{minipage}[c]{.4\textwidth}
\begin{lstlisting}[frame=single,numbers=none,basicstyle=\ttfamily\scriptsize]
/* Fichier.c */
#include <stdio.h>

int main() {
    printf("Bonjour");
    return 0;
}
\end{lstlisting}
\end{minipage}

on obtient le fichier assembleur \Code{Fichier.s} suivant~:
\begin{multicols}{3}
\begin{lstlisting}[language={[x86masm]Assembler},numbers=none,
    basicstyle=\ttfamily\scriptsize]
    .file "Fichier.c"
    .section .rodata
.LC0:
    .string	"Bonjour"
    .text
    .globl main
    .type main, @function
main:
.LFB0:
    .cfi_startproc
    pushq %rbp
    .cfi_def_cfa_offset 16
    .cfi_offset 6, -16
    movq %rsp, %rbp
    .cfi_def_cfa_register 6
    movl $.LC0, %edi
    movl $0, %eax
    call printf
    movl $0, %eax
    popq %rbp
    .cfi_def_cfa 7, 8
    ret
    .cfi_endproc
.LFE0:
    .size main, .-main
    .ident "GCC: (Ubuntu/Linaro 4.8.1-10ubuntu9) 4.8.1"
    .section .note.GNU-stack,"",@progbits

\end{lstlisting}
\end{multicols}
\begin{math} \end{math}
\end{frame}

%%%%%%%%%%%%%%%%%%%%%%%%%%%%%%%%%%%%%%%%%%%%%%%%%%%%%%%%%%%%%%%%%%%%%%%%
\begin{frame}[fragile]
\frametitle{Traduction en langage machine}
Le code assembleur \Code{Fichier.s} est traduit en \alert{langage machine}.
\medskip

On obtient ce fichier d'extension \Code{.o} par la commande
\begin{center}\Code{gcc -c Fichier.c}\end{center}
\bigskip

Ce fichier s'appelle {\bf fichier objet}. Il est illisible pour un humain
mais peut cependant être affiché au moyen de la commande
\begin{center}\Code{od -x Fichier.o} ou bien \Code{od -a Fichier.o} \end{center}
\medskip

Le langage machine est directement compris par le processeur qui peut
de ce fait exécuter directement les instructions qu'il contient.
\end{frame}

%%%%%%%%%%%%%%%%%%%%%%%%%%%%%%%%%%%%%%%%%%%%%%%%%%%%%%%%%%%%%%%%%%%%%%%%
\begin{frame}[fragile]
\frametitle{Traduction en langage machine}
Par exemple, avec le programme précédent, le contenu de \Code{Fichier.o}
est
\medskip

{\tt \footnotesize
\setlength{\tabcolsep}{.07cm}
\begin{tabular}{ccccccccccccccccc}
0000000 &del &E &L &F &stx &soh &soh &nul &nul &nul &nul &nul &nul &nul &nul &nul \\
0000020 &soh &nul &> &nul &soh &nul &nul &nul &nul &nul &nul &nul &nul &nul &nul &nul \\
0000040 &nul &nul &nul &nul &nul &nul &nul &nul &0 &soh &nul &nul &nul &nul &nul &nul \\
0000060 &nul &nul &nul &nul &@ &nul &nul &nul &nul &nul &@ &nul &cr &nul &nl &nul \\
0000100 &U &H &ht &e &? &nul &nul &nul &nul &8 &nul &nul &nul &nul &h &nul \\
0000120 &nul &nul &nul &8 &nul &nul &nul &nul &] &C &B &o &n &j &o &u \\
0000140 &r &nul &nul &G &C &C &: &sp &( &U &b &u &n &t &u &/ \\
0000160 &L &i &n &a &r &o &sp &4 &. &8 &. &1 &- &1 &0 &u \\
0000200 &b &u &n &t &u &9 &) &sp &4 &. &8 &. &1 &nul &nul &nul \\
0000220 &dc4 &nul &nul &nul &nul &nul &nul &nul &soh &z &R &nul &soh &x &dle &soh \\
0000240 &esc &ff &bel &bs &dle &soh &nul &nul &fs &nul &nul &nul &fs &nul &nul &nul \\
0000260 &nul &nul &nul &nul &sub &nul &nul &nul &nul &A &so &dle &ack &stx &C &cr \\
0000300 &ack &U &ff &bel &bs &nul &nul &nul &nul &. &s &y &m &t &a &b \\
0000320 &nul &. &s &t &r &t &a &b &nul &. &s &h &s &t &r &t \\
0000340 &a &b &nul &. &r &e &l &a &. &t &e &x &t &nul &. &d \\
0000360 &a &t &a &nul &. &b &s &s &nul &. &r &o &d &a &t &a \\
\end{tabular}}
\end{frame}

%%%%%%%%%%%%%%%%%%%%%%%%%%%%%%%%%%%%%%%%%%%%%%%%%%%%%%%%%%%%%%%%%%%%%%%%
\begin{frame}[fragile]
\frametitle{Traduction en langage machine}
{\tt \footnotesize
\setlength{\tabcolsep}{.07cm}
\begin{tabular}{ccccccccccccccccc}
0000400 &nul &. &c &o &m &m &e &n &t &nul &. &n &o &t &e &. \\
0000420 &G &N &U &- &s &t &a &c &k &nul &. &r &e &l &a &. \\
0000440 &e &h &\_ &f &r &a &m &e &nul &nul &nul &nul &nul &nul &nul &nul \\
0000460 &nul &nul &nul &nul &nul &nul &nul &nul &nul &nul &nul &nul &nul &nul &nul &nul \\
* \\
0000560 &sp &nul &nul &nul &soh &nul &nul &nul &ack &nul &nul &nul &nul &nul &nul &nul \\
0000600 &nul &nul &nul &nul &nul &nul &nul &nul &@ &nul &nul &nul &nul &nul &nul &nul \\
0000620 &sub &nul &nul &nul &nul &nul &nul &nul &nul &nul &nul &nul &nul &nul &nul &nul \\
0000640 &soh &nul &nul &nul &nul &nul &nul &nul &nul &nul &nul &nul &nul &nul &nul &nul \\
0000660 &esc &nul &nul &nul &eot &nul &nul &nul &nul &nul &nul &nul &nul &nul &nul &nul \\
0000700 &nul &nul &nul &nul &nul &nul &nul &nul &dle &enq &nul &nul &nul &nul &nul &nul \\
0000720 &0 &nul &nul &nul &nul &nul &nul &nul &vt &nul &nul &nul &soh &nul &nul &nul \\
0000740 &bs &nul &nul &nul &nul &nul &nul &nul &can &nul &nul &nul &nul &nul &nul &nul \\
0000760 &\& &nul &nul &nul &soh &nul &nul &nul &etx &nul &nul &nul &nul &nul &nul &nul \\
0001000 &nul &nul &nul &nul &nul &nul &nul &nul &Z &nul &nul &nul &nul &nul &nul &nul \\
0001020 &nul &nul &nul &nul &nul &nul &nul &nul &nul &nul &nul &nul &nul &nul &nul &nul \\
0001040 &soh &nul &nul &nul &nul &nul &nul &nul &nul &nul &nul &nul &nul &nul &nul &nul \\
\end{tabular}}
\end{frame}

%%%%%%%%%%%%%%%%%%%%%%%%%%%%%%%%%%%%%%%%%%%%%%%%%%%%%%%%%%%%%%%%%%%%%%%%
\begin{frame}[fragile]
\frametitle{Traduction en langage machine}
{\tt \footnotesize
\setlength{\tabcolsep}{.07cm}
\begin{tabular}{ccccccccccccccccc}
0001060 &, &nul &nul &nul &bs &nul &nul &nul &etx &nul &nul &nul &nul &nul &nul &nul \\
0001100 &nul &nul &nul &nul &nul &nul &nul &nul &Z &nul &nul &nul &nul &nul &nul &nul \\
0001120 &nul &nul &nul &nul &nul &nul &nul &nul &nul &nul &nul &nul &nul &nul &nul &nul \\
0001140 &soh &nul &nul &nul &nul &nul &nul &nul &nul &nul &nul &nul &nul &nul &nul &nul \\
0001160 &1 &nul &nul &nul &soh &nul &nul &nul &stx &nul &nul &nul &nul &nul &nul &nul \\
0001200 &nul &nul &nul &nul &nul &nul &nul &nul &Z &nul &nul &nul &nul &nul &nul &nul \\
0001220 &bs &nul &nul &nul &nul &nul &nul &nul &nul &nul &nul &nul &nul &nul &nul &nul \\
0001240 &soh &nul &nul &nul &nul &nul &nul &nul &nul &nul &nul &nul &nul &nul &nul &nul \\
0001260 &9 &nul &nul &nul &soh &nul &nul &nul &0 &nul &nul &nul &nul &nul &nul &nul \\
0001300 &nul &nul &nul &nul &nul &nul &nul &nul &b &nul &nul &nul &nul &nul &nul &nul \\
0001320 &, &nul &nul &nul &nul &nul &nul &nul &nul &nul &nul &nul &nul &nul &nul &nul \\
0001340 &soh &nul &nul &nul &nul &nul &nul &nul &soh &nul &nul &nul &nul &nul &nul &nul \\
0001360 &B &nul &nul &nul &soh &nul &nul &nul &nul &nul &nul &nul &nul &nul &nul &nul \\
0001400 &nul &nul &nul &nul &nul &nul &nul &nul &so &nul &nul &nul &nul &nul &nul &nul \\
0001420 &nul &nul &nul &nul &nul &nul &nul &nul &nul &nul &nul &nul &nul &nul &nul &nul \\
0001440 &soh &nul &nul &nul &nul &nul &nul &nul &nul &nul &nul &nul &nul &nul &nul &nul \\
0001460 &W &nul &nul &nul &soh &nul &nul &nul &stx &nul &nul &nul &nul &nul &nul &nul \\
\end{tabular}}
\end{frame}

%%%%%%%%%%%%%%%%%%%%%%%%%%%%%%%%%%%%%%%%%%%%%%%%%%%%%%%%%%%%%%%%%%%%%%%%
\begin{frame}[fragile]
\frametitle{Traduction en langage machine}
{\tt \footnotesize
\setlength{\tabcolsep}{.07cm}
\begin{tabular}{ccccccccccccccccc}
0001500 &nul &nul &nul &nul &nul &nul &nul &nul &dle &nul &nul &nul &nul &nul &nul &nul \\
0001520 &8 &nul &nul &nul &nul &nul &nul &nul &nul &nul &nul &nul &nul &nul &nul &nul \\
0001540 &bs &nul &nul &nul &nul &nul &nul &nul &nul &nul &nul &nul &nul &nul &nul &nul \\
0001560 &R &nul &nul &nul &eot &nul &nul &nul &nul &nul &nul &nul &nul &nul &nul &nul \\
0001600 &nul &nul &nul &nul &nul &nul &nul &nul &@ &enq &nul &nul &nul &nul &nul &nul \\
0001620 &can &nul &nul &nul &nul &nul &nul &nul &vt &nul &nul &nul &bs &nul &nul &nul \\
0001640 &bs &nul &nul &nul &nul &nul &nul &nul &can &nul &nul &nul &nul &nul &nul &nul \\
0001660 &dc1 &nul &nul &nul &etx &nul &nul &nul &nul &nul &nul &nul &nul &nul &nul &nul \\
0001700 &nul &nul &nul &nul &nul &nul &nul &nul &H &nul &nul &nul &nul &nul &nul &nul \\
0001720 &a &nul &nul &nul &nul &nul &nul &nul &nul &nul &nul &nul &nul &nul &nul &nul \\
0001740 &soh &nul &nul &nul &nul &nul &nul &nul &nul &nul &nul &nul &nul &nul &nul &nul \\
0001760 &soh &nul &nul &nul &stx &nul &nul &nul &nul &nul &nul &nul &nul &nul &nul &nul \\
0002000 &nul &nul &nul &nul &nul &nul &nul &nul &p &eot &nul &nul &nul &nul &nul &nul \\
0002020 &bs &soh &nul &nul &nul &nul &nul &nul &ff &nul &nul &nul &ht &nul &nul &nul \\
0002040 &bs &nul &nul &nul &nul &nul &nul &nul &can &nul &nul &nul &nul &nul &nul &nul \\
0002060 &ht &nul &nul &nul &etx &nul &nul &nul &nul &nul &nul &nul &nul &nul &nul &nul \\
0002100 &nul &nul &nul &nul &nul &nul &nul &nul &x &enq &nul &nul &nul &nul &nul &nul \\
\end{tabular}}
\end{frame}

%%%%%%%%%%%%%%%%%%%%%%%%%%%%%%%%%%%%%%%%%%%%%%%%%%%%%%%%%%%%%%%%%%%%%%%%
\begin{frame}[fragile]
\frametitle{Traduction en langage machine}
{\tt \footnotesize
\setlength{\tabcolsep}{.07cm}
\begin{tabular}{ccccccccccccccccc}
0002120 &etb &nul &nul &nul &nul &nul &nul &nul &nul &nul &nul &nul &nul &nul &nul &nul \\
0002140 &soh &nul &nul &nul &nul &nul &nul &nul &nul &nul &nul &nul &nul &nul &nul &nul \\
0002160 &nul &nul &nul &nul &nul &nul &nul &nul &nul &nul &nul &nul &nul &nul &nul &nul \\
0002200 &nul &nul &nul &nul &nul &nul &nul &nul &soh &nul &nul &nul &eot &nul &q &del \\
0002220 &nul &nul &nul &nul &nul &nul &nul &nul &nul &nul &nul &nul &nul &nul &nul &nul \\
0002240 &nul &nul &nul &nul &etx &nul &soh &nul &nul &nul &nul &nul &nul &nul &nul &nul \\
0002260 &nul &nul &nul &nul &nul &nul &nul &nul &nul &nul &nul &nul &etx &nul &etx &nul \\
0002300 &nul &nul &nul &nul &nul &nul &nul &nul &nul &nul &nul &nul &nul &nul &nul &nul \\
0002320 &nul &nul &nul &nul &etx &nul &eot &nul &nul &nul &nul &nul &nul &nul &nul &nul \\
0002340 &nul &nul &nul &nul &nul &nul &nul &nul &nul &nul &nul &nul &etx &nul &enq &nul \\
0002360 &nul &nul &nul &nul &nul &nul &nul &nul &nul &nul &nul &nul &nul &nul &nul &nul \\
0002400 &nul &nul &nul &nul &etx &nul &bel &nul &nul &nul &nul &nul &nul &nul &nul &nul \\
0002420 &nul &nul &nul &nul &nul &nul &nul &nul &nul &nul &nul &nul &etx &nul &bs &nul \\
0002440 &nul &nul &nul &nul &nul &nul &nul &nul &nul &nul &nul &nul &nul &nul &nul &nul \\
0002460 &nul &nul &nul &nul &etx &nul &ack &nul &nul &nul &nul &nul &nul &nul &nul &nul \\
0002500 &nul &nul &nul &nul &nul &nul &nul &nul &vt &nul &nul &nul &dc2 &nul &soh &nul \\
0002520 &nul &nul &nul &nul &nul &nul &nul &nul &sub &nul &nul &nul &nul &nul &nul &nul \\
\end{tabular}}
\end{frame}

%%%%%%%%%%%%%%%%%%%%%%%%%%%%%%%%%%%%%%%%%%%%%%%%%%%%%%%%%%%%%%%%%%%%%%%%
\begin{frame}[fragile]
\frametitle{Traduction en langage machine}
{\tt \footnotesize
\setlength{\tabcolsep}{.07cm}
\begin{tabular}{ccccccccccccccccc}
0002540 &dle &nul &nul &nul &dle &nul &nul &nul &nul &nul &nul &nul &nul &nul &nul &nul \\
0002560 &nul &nul &nul &nul &nul &nul &nul &nul &nul &F &i &c &h &i &e &r \\
0002600 &. &c &nul &m &a &i &n &nul &p &r &i &n &t &f &nul &nul \\
0002620 &enq &nul &nul &nul &nul &nul &nul &nul &nl &nul &nul &nul &enq &nul &nul &nul \\
0002640 &nul &nul &nul &nul &nul &nul &nul &nul &si &nul &nul &nul &nul &nul &nul &nul \\
0002660 &stx &nul &nul &nul &nl &nul &nul &nul &| &del &del &del &del &del &del &del \\
0002700 &sp &nul &nul &nul &nul &nul &nul &nul &stx &nul &nul &nul &stx &nul &nul &nul \\
0002720 &nul &nul &nul &nul &nul &nul &nul &nul \\
0002730 \\
\end{tabular}}
\end{frame}

%%%%%%%%%%%%%%%%%%%%%%%%%%%%%%%%%%%%%%%%%%%%%%%%%%%%%%%%%%%%%%%%%%%%%%%%
\begin{frame}[fragile]
\frametitle{Édition des liens}
L'\alert{édition des liens} réunit le fichier objet et le code propre
aux fonctions et types de la librairie standard utilisés (comme
\Code{printf}, \Code{scanf}, {\em etc.}) pour produire l'exécutable
complet.
\bigskip

C'est dans cette phase de la compilation que la
\alert{résolution des symboles} a lieu. C'est l'étape qui consiste à
associer aux identificateurs de fonctions leur implantation.
\end{frame}

%%%%%%%%%%%%%%%%%%%%%%%%%%%%%%%%%%%%%%%%%%%%%%%%%%%%%%%%%%%%%%%%%%%%%%%%
\begin{frame}[fragile]
\frametitle{Compilation d'un projet de plusieurs fichiers}
On suppose que l'on travaille sur un projet constitué de trois modules
\Code{A}, \Code{B} et \Code{C} et d'un fichier principal \Code{Main.c}
contenant la fonction \Code{main}.
\medskip

La compilation de ce projet se réalise au moyen des étapes suivantes~:
\begin{enumerate}
    \item obtenir les fichiers objets de chaque module~;
    \item obtenir le fichier objet de \Code{Main.c}~:
    \item lier les fichiers objets ainsi obtenus en un exécutable.
\end{enumerate}
\medskip

\begin{center}
    \begin{tikzpicture}[every text node part/.style={align=left}]
        \node[Module](A)at(0,0){A.h \\ A.c};
        \node[Module](B)at(0,-1){B.h \\ B.c};
        \node[Module](C)at(0,-2){C.h \\ C.c};
        \node[Module](Main)at(0,-3){Main.c};
        \node[FObj](Ao)at(3.5,0){A.o};
        \node[FObj](Bo)at(3.5,-1){B.o};
        \node[FObj](Co)at(3.5,-2){C.o};
        \node[FObj](Maino)at(3.5,-3){Main.o};
        \node[Exec](Ex)at(9,-1.5){a.out};
        \draw(A)edge[->,anchor=south,font=\scriptsize \tt] node{gcc -c A.c A.h}(Ao);
        \draw(B)edge[->,anchor=south,font=\scriptsize \tt] node{gcc -c B.c B.h}(Bo);
        \draw(C)edge[->,anchor=south,font=\scriptsize \tt] node{gcc -c C.c C.h}(Co);
        \draw(Main)edge[->,anchor=south,font=\scriptsize \tt] node{gcc -c Main.c}(Maino);
        \draw(4,-1.5)edge[->,anchor=south,font=\scriptsize \tt] node{gcc Main.o A.o B.o C.o}(Ex);
    \end{tikzpicture}
\end{center}
\end{frame}

%%%%%%%%%%%%%%%%%%%%%%%%%%%%%%%%%%%%%%%%%%%%%%%%%%%%%%%%%%%%%%%%%%%%%%%%
\begin{frame}[fragile]
\frametitle{Création des fichiers objets}
Pour compiler le projet, on commence par créer un fichier objet pour
chaque module \Code{M}. On utilise pour cela la commande
\begin{center} \Code{gcc -c M.c M.h} \end{center}
\medskip

Cette commande est équivalente à
\begin{center} \Code{gcc -c M.c} \end{center}
car le fichier source \Code{M.c} inclut le fichier d'en-tête \Code{M.h}.
\bigskip

On utilisera donc de préférence cette 2\ieme{} commande.
\bigskip
\bigskip

Chaque module est ainsi \alert{compilé séparément}.
\end{frame}

%%%%%%%%%%%%%%%%%%%%%%%%%%%%%%%%%%%%%%%%%%%%%%%%%%%%%%%%%%%%%%%%%%%%%%%%
\begin{frame}[fragile]
\frametitle{Symboles non résolus}
Pour compiler un module \Code{A}, il n'est pas nécessaire que \Code{A}
ait connaissance des définitions des symboles qu'il utilise.
\medskip

Seules {\bf leurs déclarations sont suffisantes}. Celles-ci se trouvent
dans les fichiers d'en-tête inclus dans \Code{A}.
\bigskip

On dit qu'un symbole n'est pas \alert{résolu} à un stade donné de la
compilation si sa définition n'est pas encore connue.
\bigskip

\begin{minipage}[c]{.3\textwidth}
\begin{lstlisting}[frame=single,numbers=none]
/* Fichier.c */
int g(int x);

int f(int x) {
    return g(x);
}
\end{lstlisting}
\end{minipage}\qquad
\begin{minipage}[c]{.6\textwidth}
Par exemple, ce fichier permet de produire un fichier objet sur la commande
\Code{gcc -c Fichier.c} même si le symbole \Code{g} est non résolu pour
le moment.
\smallskip

Sa déclaration (dans le fichier lui-même ou dans un fichier inclus) est
cependant nécessaire.
\end{minipage}
\end{frame}

%%%%%%%%%%%%%%%%%%%%%%%%%%%%%%%%%%%%%%%%%%%%%%%%%%%%%%%%%%%%%%%%%%%%%%%%
\begin{frame}[fragile]
\frametitle{Résolution des symboles}
Lors de l'édition des liens, un exécutable est créé. On utilise pour
cela la commande
\begin{center} \Code{gcc Main.o A1.o  ... An.o} \end{center}
dans le cadre d'un projet constitué des modules \Code{A1.o}, \dots, \Code{An.o}
et du module principal \Code{Main.o}.
\bigskip

Cette étape \alert{lie à chaque symbole sa définition}.
\bigskip
\bigskip

Tous les symboles utilisés dans le projet doivent être résolus (sinon,
un message d'erreur est produit et l'exécutable ne peut pas être
construit).
\end{frame}

%%%%%%%%%%%%%%%%%%%%%%%%%%%%%%%%%%%%%%%%%%%%%%%%%%%%%%%%%%%%%%%%%%%%%%%%
\begin{frame}[fragile]
\frametitle{Résolution des symboles --- exemple}
\begin{minipage}[c]{.18\textwidth}
\begin{lstlisting}[frame=single,numbers=none,basicstyle=\scriptsize\tt]
/* A.h */
#ifndef __A__
#define __A__
  int f(int x);
#endif
\end{lstlisting}
\end{minipage}
\enspace
\begin{minipage}[c]{.18\textwidth}
\begin{lstlisting}[frame=single,numbers=none,basicstyle=\scriptsize\tt]
/* A.c */
#include "A.h"
int f(int x) {
  return x * x;
}
\end{lstlisting}
\end{minipage}
\enspace
\begin{minipage}[c]{.18\textwidth}
\begin{lstlisting}[frame=single,numbers=none,basicstyle=\scriptsize\tt]
/* B.h */
#ifndef __B__
#define __B__
  int g(int x);
#endif
\end{lstlisting}
\end{minipage}
\enspace
\begin{minipage}[c]{.18\textwidth}
\begin{lstlisting}[frame=single,numbers=none,basicstyle=\scriptsize\tt]
/* B.c */
#include "B.h"
#include "A.h"
int g(int x) {
  return f(x);
}
\end{lstlisting}
\end{minipage}
\enspace
\begin{minipage}[c]{.17\textwidth}
\begin{lstlisting}[frame=single,numbers=none,basicstyle=\scriptsize\tt]
/* Main.c */
#include "B.h"
int main() {
  g(5);
  return 0;
}
\end{lstlisting}
\end{minipage}

Pour compiler ce projet, on emploie les commandes
\medskip

\Code{gcc -c A.c} \\
\Code{gcc -c B.c} \\
\Code{gcc -c Main.c} \\
\Code{gcc Main.o A.o B.o}
\bigskip

L'ordre d'exécution des trois 1\ieres{} commandes n'a aucune incidence
sur le résultat produit.
\end{frame}

%%%%%%%%%%%%%%%%%%%%%%%%%%%%%%%%%%%%%%%%%%%%%%%%%%%%%%%%%%%%%%%%%%%%%%%%
\begin{frame}[fragile]
\frametitle{Résolution des symboles --- exemple}
\begin{minipage}[c]{.18\textwidth}
\begin{lstlisting}[frame=single,numbers=none,basicstyle=\scriptsize\tt]
/* A.h */
#ifndef __A__
#define __A__
  int f(int x);
#endif
\end{lstlisting}
\end{minipage}
\enspace
\begin{minipage}[c]{.18\textwidth}
\begin{lstlisting}[frame=single,numbers=none,basicstyle=\scriptsize\tt]
/* A.c */
#include "A.h"
int f(int x) {
  return x * x;
}
\end{lstlisting}
\end{minipage}
\enspace
\begin{minipage}[c]{.18\textwidth}
\begin{lstlisting}[frame=single,numbers=none,basicstyle=\scriptsize\tt]
/* B.h */
#ifndef __B__
#define __B__
  int g(int x);
#endif
\end{lstlisting}
\end{minipage}
\enspace
\begin{minipage}[c]{.18\textwidth}
\begin{lstlisting}[frame=single,numbers=none,basicstyle=\scriptsize\tt]
/* B.c */
#include "B.h"
#include "A.h"
int g(int x) {
  return f(x);
}
\end{lstlisting}
\end{minipage}
\enspace
\begin{minipage}[c]{.17\textwidth}
\begin{lstlisting}[frame=single,numbers=none,basicstyle=\scriptsize\tt]
/* Main.c */
#include "B.h"
int main() {
  g(5);
  return 0;
}
\end{lstlisting}
\end{minipage}

Lors de la création de \Code{B.o}, le compilateur {\bf ignore} ce que fait
le symbole \Code{f}. Il sait seulement (grâce à l'inclusion de \Code{A}
dans \Code{B}) que \Code{f} est un symbole de fonction paramétrée par un
entier et renvoyant un entier et peut donc {\bf vérifier la correspondance
des types}.
\medskip

C'est au moment de l'édition des liens que le compilateur va
{\bf chercher l'implantation} du symbole \Code{f} pour créer l'exécutable
de la bonne manière.
\bigskip

{\bf Observation importante}~: la compilation d'un projet à plusieurs
fichiers ne dépend pas de la manière dont ses modules sont inclus les
uns dans les autres. Le schéma de compilation est toujours le même.
\end{frame}

%%%%%%%%%%%%%%%%%%%%%%%%%%%%%%%%%%%%%%%%%%%%%%%%%%%%%%%%%%%%%%%%%%%%%%%%
%%%%%%%%%%%%%%%%%%%%%%%%%%%%%%%%%%%%%%%%%%%%%%%%%%%%%%%%%%%%%%%%%%%%%%%%
\subsection{{\tt Makefile}}

%%%%%%%%%%%%%%%%%%%%%%%%%%%%%%%%%%%%%%%%%%%%%%%%%%%%%%%%%%%%%%%%%%%%%%%%
\begin{frame}\frametitle{Compilation séparée --- intuition}
{\bf Fait}~: pour compiler un module \Code{A}, il n'est pas nécessaire
d'avoir les fichiers objets des autres modules du projet dont
\Code{A} ne dépend pas (de manière étendue ou non).
\bigskip

\uncover<2->{%
{\bf Conséquence}~: si un module \Code{B} est modifié, il n'est nécessaire
de recompiler que \Code{B} et l'ensemble des modules qui dépendent
(de manière étendue) à \Code{B}.
\bigskip
\bigskip}

\uncover<3->{%
Attention à ne pas oublier de recompiler le module principal \Code{Main}
si celui-ci dépend de manière étendue à des modules modifiés.}
\end{frame}

%%%%%%%%%%%%%%%%%%%%%%%%%%%%%%%%%%%%%%%%%%%%%%%%%%%%%%%%%%%%%%%%%%%%%%%%
\begin{frame}
\frametitle{Compilation séparée --- exemple introductif}
Considérons par exemple le projet suivant~:
\begin{multicols}{2}
\begin{center}
\scalebox{.8}{\begin{tikzpicture}
    \node[Sommet](A)at(0,0){\Code{A}};
    \node[Sommet](B)at(0,-1){\Code{B}};
    \node[Sommet](C)at(0,-2){\Code{C}};
    \node[Sommet](Main)at(2,-1){\Code{Main}};
    \draw[Fleche](Main)edge[bend right=20,dashed,draw=Vert]node{}(A);
    \draw[Fleche](Main)edge[bend left=0,dashed,draw=Vert](B);
    \draw[Fleche](Main)edge[bend left=20,dashed,draw=Vert]node{}(C);
    \draw[Fleche](B)--(A);
    \draw[Fleche](A)edge[bend right=60]node{}(C);
\end{tikzpicture}}
\end{center}
On le compile pour la 1\iere{} fois par \\
\begin{footnotesize}
    \Code{gcc -c Main.c} \\
    \Code{gcc -c A.c} \\
    \Code{gcc -c B.c} \\
    \Code{gcc -c C.c} \\
    \Code{gcc Main.o A.o B.o C.o}
\end{footnotesize}
\end{multicols}
\medskip

\uncover<2->{%
Si on modifie par la suite le module \Code{A}, il suffit d'exécuter les
commandes
\begin{multicols}{2}
    \Code{gcc -c A.c} \\
    \Code{gcc -c B.c} \\
    \Code{gcc -c Main.c} \\
    \Code{gcc Main.o A.o B.o C.o} \\
    pour mettre à jour l'exécutable du projet.
    \smallskip

    {\bf Note}~: les $3$ premières lignes commutent.
\end{multicols}
\medskip

Il est inutile de recompiler \Code{C} car il ne dépend pas (de manière
étendue) à~\Code{A}.}
\end{frame}

%%%%%%%%%%%%%%%%%%%%%%%%%%%%%%%%%%%%%%%%%%%%%%%%%%%%%%%%%%%%%%%%%%%%%%%%
\begin{frame}
\frametitle{Schéma opérationnel de la compilation d'un projet}
À l'appui de cette observation, la compilation d'un projet
s'organise de la manière suivante.
\medskip

\begin{enumerate}[(1)]
    \uncover<2->{%
    \item Pour chaque module \Code{A} du projet~:
    \smallskip}

    \begin{enumerate}[(a)] \normalsize
        \uncover<3->{%
        \item compiler \Code{A} si au moins l'une des conditions
        suivante est vérifiée~:
        \smallskip}

        \begin{itemize} \normalsize
            \uncover<4->{%
            \item \Code{A.o} n'existe pas~;
            \smallskip}

            \uncover<5->{%
            \item \Code{A.c} ou \Code{A.h} ont été modifiés {\bf après}
            \Code{A.o}~;
            \smallskip}

            \uncover<6->{%
            \item il existe un module \Code{B} dont \Code{A} dépend
            (de manière étendue) et tel que \Code{B.h} a été modifié
            {\bf après} \Code{A.o}~;}
        \end{itemize}
    \end{enumerate}
    \smallskip

    \uncover<7->{%
    \item si au moins un module du projet a été (re)compilé, reconstruire
    l'exécutable.}
\end{enumerate}
\bigskip

\uncover<8->{%
Nous allons utiliser l'utilitaire \alert{\Code{make}} et les fichiers
\alert{\Code{Makefile}} pour rendre cette procédure automatique.}
\end{frame}

%%%%%%%%%%%%%%%%%%%%%%%%%%%%%%%%%%%%%%%%%%%%%%%%%%%%%%%%%%%%%%%%%%%%%%%%
\begin{frame}\frametitle{Fichiers {\tt Makefile}}
L'utilitaire \Code{make} est paramétré par un fichier dont le nom est
imposé~:
\begin{center}
    \og \Code{Makefile} \fg{} ou \og \Code{makefile} \fg.
\end{center}
Il doit se trouver dans le répertoire de travail (là où se trouvent les
autres fichiers du projet).
\medskip

Ce fichier fait partie intégrante du projet.
\bigskip

Tout fichier \Code{Makefile} est constitué de \alert{règles}.
Elles sont de la forme
\smallskip

\Code{CIBLE: DEPENDANCES} \\
\Code{$\rightarrow$ COMMANDE} \\
\Code{\vdots} \\
\Code{$\rightarrow$ COMMANDE}
\bigskip

Le symbole \og\Code{$\rightarrow$}\fg{} désigne une tabulation.
\end{frame}

%%%%%%%%%%%%%%%%%%%%%%%%%%%%%%%%%%%%%%%%%%%%%%%%%%%%%%%%%%%%%%%%%%%%%%%%
\begin{frame}[fragile]
\frametitle{Fichiers {\tt Makefile} et fonctionnement}
Considérons le projet et le \Code{Makefile} suivants.
\begin{multicols}{2}
\begin{center}
\scalebox{.8}{\begin{tikzpicture}
    \node[Sommet](A)at(0,0){\Code{A}};
    \node[Sommet](B)at(0,-2){\Code{B}};
    \node[Sommet](Main)at(2,-1){\Code{Main}};
    \draw[Fleche,dashed,draw=Vert](Main)edge[bend right=20]node{}(A);
    \draw[Fleche,dashed,draw=Vert](Main)edge[bend left=20]node{}(B);
\end{tikzpicture}}
\end{center}
\bigskip
\bigskip
\begin{lstlisting}[language=make]
Prog: A.o B.o Main.o
    gcc -o Prog Main.o A.o B.o
Main.o: Main.c
    gcc -c Main.c
A.o: A.c A.h
    gcc -c A.c
B.o: B.c B.h
    gcc -c B.c
\end{lstlisting}
\end{multicols}
\bigskip

Lorsque l'on exécute la commande \Code{make}, \Code{make} tente de
résoudre la 1\iere{} règle (l. 1). Pour cela, il doit résoudre ses
dépendances. Ensuite, il exécute les commandes de la règle si la cible
n'est pas à jour.
\medskip

Concrètement, pour pouvoir créer l'exécutable \Code{Prog}, il est
nécessaire que \Code{A.o}, \Code{B.o} et \Code{Main.o} soient à jour.
Une fois qu'ils le sont, il suffit de procéder à l'édition des liens
(l. 2).
\end{frame}

%%%%%%%%%%%%%%%%%%%%%%%%%%%%%%%%%%%%%%%%%%%%%%%%%%%%%%%%%%%%%%%%%%%%%%%%
\begin{frame}\frametitle{Fichiers {\tt Makefile} et fonctionnement}
Pour savoir si une cible est à jour, \Code{make} regarde les dépendances
de la règle et~:
\begin{itemize}
    \item si la dépendance est la cible d'une autre règle, \Code{make}
    procède {\bf récursivement} à sa résolution~;
    \smallskip

    \item si la dépendance est le nom d'un fichier, \Code{make} compare
    la {\bf date de dernière modification} de la cible par rapport à
    celle du fichier.
\end{itemize}
\smallskip

S'il y a au moins un fichier dans les dépendances avec une date supérieure
à celle de la cible, les commandes de la règle sont exécutées.
\bigskip

On peut imposer à \Code{make} de commencer par la résolution de la
règle dont la cible est \Code{CIBLE} par
\begin{center}
    \Code{make CIBLE}
\end{center}
\end{frame}

%%%%%%%%%%%%%%%%%%%%%%%%%%%%%%%%%%%%%%%%%%%%%%%%%%%%%%%%%%%%%%%%%%%%%%%%
\begin{frame}[fragile]\frametitle{Déclarations de types et dépendances}
Soient \Code{A} et \Code{B} deux modules tels que
\Code{B} dépend (de manière étendue) à \Code{A}, qu'un type \Code{T}
soit déclaré dans \Code{A} et que \Code{B} utilise ce type.
\bigskip

Toute modification de \Code{A.h} doit être suivie d'une nouvelle
compilation de \Code{B}. En effet, si la déclaration de \Code{T} a été
modifiée, \Code{B} doit être recompilé pour la prendre en compte.
\medskip

En conséquence, dans le \Code{Makefile} du projet doit figurer la règle
\begin{multicols}{2}
\begin{lstlisting}[language=make]
B.o: B.c B.h A.h
    gcc -c B.c
\end{lstlisting}
pour déclarer que la construction de \Code{B.o} {\bf dépend aussi}
de \Code{A.h}.
\end{multicols}
\bigskip

{\bf Attention}~: ceci ne s'applique pas aux modifications de l'implantation
des fonctions de \Code{A} dans \Code{A.c} (comme nous l'avons déjà vu).
La cible \Code{B.o} ne dépend ainsi pas de \Code{A.c}. Elle ne dépend
que des {\bf déclarations} du module \Code{A} (et donc de \Code{A.h}).
\end{frame}

%%%%%%%%%%%%%%%%%%%%%%%%%%%%%%%%%%%%%%%%%%%%%%%%%%%%%%%%%%%%%%%%%%%%%%%%
\begin{frame}[fragile]
\frametitle{Exemple complet de {\tt Makefile}}
\begin{multicols}{2}
\begin{center}
\scalebox{.4}{\begin{tikzpicture}[scale=.8]
    \node[Sommet,minimum size=2cm,font=\Large](Piece)at(-4,0)
        {\small \Code{Piece}};
    \node[Sommet,minimum size=2cm,font=\Large](Case)at(-4,-4)
        {\small \Code{Case}};
    \node[Sommet,minimum size=2cm,font=\Large](Position)at(0,-2)
        {\small \Code{Position}};
    \node[Sommet,minimum size=2cm,font=\Large](IA)at(4,0)
        {\small \Code{IA}};
    \node[Sommet,minimum size=2cm,font=\Large](IGraph)at(4,-4)
        {\small \Code{IGraph}};
    \node[Sommet,minimum size=2cm,draw=Vert,fill=Vert!20,font=\Large]
        (Main)at(7,-2){\small \Code{Main}};
    \draw[Fleche,dashed,draw=Vert](Main)edge[bend right=20]node{}(IA);
    \draw[Fleche,dashed,draw=Vert](Main)edge[bend left=20]node{}(IGraph);
    \draw[Fleche](IGraph)edge[bend left=20]node{}(Case);
    \draw[Fleche](IGraph)edge[bend left=20]node{}(Position);
    \draw[Fleche](IGraph)edge[bend right=40]node{}(Piece);
    \draw[Fleche](IA)edge[bend left=40]node{}(Case);
    \draw[Fleche](IA)edge[bend right=20]node{}(Position);
    \draw[Fleche](IA)edge[bend right=20]node{}(Piece);
    \draw[Fleche](Position)edge[bend right=20]node{}(Case);
    \draw[Fleche](Position)edge[bend left=20]node{}(Piece);
\end{tikzpicture}}
\end{center}
Le \Code{Makefile} du projet dont le graphe d'inclusions est donné
ci-contre est
\end{multicols}

\begin{lstlisting}[language=make,basicstyle=\footnotesize\tt]
Echecs: Main.o Piece.o Case.o Position.o IA.o IGraph.o
    gcc -o Echecs Main.o Piece.o Case.o Position.o IA.o IGraph.o
Main.o: Main.c IA.h IGraph.h
    gcc -c Main.c
Piece.o: Piece.c Piece.h
    gcc -c Piece.c
Case.o: Case.c Case.h
    gcc -c Case.c
Position.o: Position.c Position.h Piece.h Case.h
    gcc -c Position.c
IA.o: IA.c IA.h Piece.h Case.h Position.h
    gcc -c IA.c
IGraph.o: IGraph.c IGraph.h Piece.h Case.h Position.h
    gcc -c IGraph.c
\end{lstlisting}
\end{frame}

%%%%%%%%%%%%%%%%%%%%%%%%%%%%%%%%%%%%%%%%%%%%%%%%%%%%%%%%%%%%%%%%%%%%%%%%
\begin{frame}[fragile]
\frametitle{Écriture de {\tt Makefile} simples --- résumé}
Le \Code{Makefile} d'un projet contenant des modules \Code{A1}, \dots,
\Code{An} et un module principal \Code{Main} est génériquement de la
forme
\begin{lstlisting}[language=make]
NOM: Main.o A1.o ... An.o
    gcc -o NOM Main.o A1.o ... An.o

Main.o: Main.c EXTRAmain
    gcc -c Main.c

A1.o: A1.c A1.h EXTRA1
    gcc -c A1.c

...

An.o: An.c An.h EXTRAn
    gcc -c An.c
\end{lstlisting}
où \Code{EXTRAmain} est la suite des noms des fichiers \Code{.h} que
\Code{Main.c} inclut et pour tout \Code{$1 \leq k \leq n$}, \Code{EXTRAk}
est la suite des noms des fichiers \Code{.h} dont le module \Code{Ak}
dépend (de manière étendue).
\end{frame}

%%%%%%%%%%%%%%%%%%%%%%%%%%%%%%%%%%%%%%%%%%%%%%%%%%%%%%%%%%%%%%%%%%%%%%%%
\begin{frame}[fragile]
\frametitle{Variables dans les {\tt Makefile}}
Il est possible de \alert{définir des variables} dans un \Code{Makefile}
par
\begin{center}\Code{ID=VAL}\end{center}
Ceci définit une variable identifiée par \Code{ID}. Sa valeur est
la {\bf chaîne de caractères} \Code{VAL}.
\bigskip

On accède à la valeur d'une variable identifiée par \Code{ID} par
\begin{center}\Code{\$(ID)}\end{center}
Ceci substitue à l'occurrence de \Code{\$(ID)} la chaîne de caractères qui
lui a été attribuée lors de sa définition.
\end{frame}

%%%%%%%%%%%%%%%%%%%%%%%%%%%%%%%%%%%%%%%%%%%%%%%%%%%%%%%%%%%%%%%%%%%%%%%%
\begin{frame}[fragile]
\frametitle{Variables dans les {\tt Makefile} --- exemple}

Les variables permettent de factoriser les règles d'un \Code{Makefile}~:
\medskip

\begin{lstlisting}[language=make,basicstyle=\footnotesize\tt]
Main: Main.o A.o
    gcc -o Main Main.o A.o -ansi -pedantic -Wall

Main.o: Main.c
    gcc -c Main.c -ansi -pedantic -Wall

A.o : A.c A.h
    gcc -c A.c -ansi -pedantic -Wall
\end{lstlisting}
\medskip

s'écrit plus simplement par
\medskip

\begin{lstlisting}[language=make,basicstyle=\footnotesize\tt]
CFLAGS=-ansi -pedantic -Wall

Main: Main.o A.o
    gcc -o Main Main.o A.o $(CFLAGS)

Main.o: Main.c
    gcc -c Main.c $(CFLAGS)

A.o : A.c A.h
    gcc -c A.c $(CFLAGS)
\end{lstlisting}
\begin{math}\end{math}
\end{frame}

%%%%%%%%%%%%%%%%%%%%%%%%%%%%%%%%%%%%%%%%%%%%%%%%%%%%%%%%%%%%%%%%%%%%%%%%
\begin{frame}[fragile]
\frametitle{Variables dans les {\tt Makefile}}
On utilise en général les noms de variable suivants~:
\begin{itemize}
    \item \Code{CFLAGS} pour les options de compilation, p.ex.,
\begin{lstlisting}[language=make,numbers=none,basicstyle=\footnotesize\tt]
CFLAGS=-ansi -pedantic -Wall
\end{lstlisting}

    \item \Code{LDFLAGS} pour l'inclusion de bibliothèques, p.ex.,
\begin{lstlisting}[language=make,numbers=none,basicstyle=\footnotesize\tt]
LDFLAGS=-lm -lMLV
\end{lstlisting}

    \item \Code{CC} pour le compilateur utilisé, p.ex.,

\begin{minipage}[c]{.16\textwidth}
\begin{lstlisting}[language=make,numbers=none,basicstyle=\footnotesize\tt]
CC=gcc
\end{lstlisting}
\end{minipage}
ou bien \hspace{2.5em}
\begin{minipage}[c]{.16\textwidth}
\begin{lstlisting}[language=make,numbers=none,basicstyle=\footnotesize\tt]
CC=colorgcc
\end{lstlisting}
\end{minipage}

    \item \Code{OPT} pour les option d'optimisation de code

\begin{minipage}[c]{.16\textwidth}
\begin{lstlisting}[language=make,numbers=none,basicstyle=\footnotesize\tt]
OPT=-O1
\end{lstlisting}
\end{minipage}
ou bien \hspace{2.5em}
\begin{minipage}[c]{.16\textwidth}
\begin{lstlisting}[language=make,numbers=none,basicstyle=\footnotesize\tt]
OPT=-O2
\end{lstlisting}
\end{minipage}
ou encore \hspace{2.5em}
\begin{minipage}[c]{.16\textwidth}
\begin{lstlisting}[language=make,numbers=none,basicstyle=\footnotesize\tt]
OPT=-O3
\end{lstlisting}
\end{minipage}
\end{itemize}
\end{frame}

%%%%%%%%%%%%%%%%%%%%%%%%%%%%%%%%%%%%%%%%%%%%%%%%%%%%%%%%%%%%%%%%%%%%%%%%
\begin{frame}[fragile]
\frametitle{Règles courantes}
{\bf Observation}~: la plupart des règles des \Code{Makefile} sont sous
l'une de ces deux formes~:
\medskip

\begin{enumerate}
\item
\Code{M.o: M.c DEP2... DEPn \\
$\rightarrow$ gcc -c M.c}
\medskip

\item
\Code{EXEC: DEP1 ... DEPn \\
$\rightarrow$ gcc -o EXEC DEP1 ... DEPn}
\end{enumerate}
\bigskip

La 1\iere{} forme de règle a pour but de construire le fichier objet
d'un module \Code{M}. Dans ce cas,
\Code{DEP2}, \dots, \Code{DEPn} sont les \Code{.h} dont le module
\Code{M} dépend.
\bigskip

La 2\ieme{} forme de règle a pour but de construire l'exécutable \Code{EXEC}
du projet. Les dépendances \Code{DEP1}, \dots, \Code{DEPn} sont dans ce
cas les fichiers \Code{.o} du projet.
\end{frame}

%%%%%%%%%%%%%%%%%%%%%%%%%%%%%%%%%%%%%%%%%%%%%%%%%%%%%%%%%%%%%%%%%%%%%%%%
\begin{frame}[fragile]
\frametitle{Variables internes dans les {\tt Makefile}}
Il est possible de simplifier l'écriture de ces règles courantes au moyen
des \alert{variables internes}. Il y en a trois principales et deux
secondaires~:
\begin{center}
    \begin{tabular}{c|c}
        Symbole & Ce qu'il désigne \\ \hline \hline
        \Code{\$@} & Nom de la cible \\
        \Code{\$<} & Nom de la 1\iere{} dép. \\
        \Code{\$\textasciicircum} & Noms de toutes les dép. \\ \hline
        \Code{\$?} & Noms des dép. plus récentes que la cible \\
        \Code{\$*} & Nom de la cible sans extension
    \end{tabular}
\end{center}
\medskip

En les utilisant, les deux règles précédentes deviennent

\begin{minipage}[c]{.4\textwidth}
\begin{lstlisting}[language=make,numbers=none,basicstyle=\footnotesize\tt]
M.o: M.c DEP2 ... DEPn
    gcc -c M.c
\end{lstlisting}
\end{minipage}
\hspace{1em}
$\longrightarrow$
\hspace{2em}
\begin{minipage}[c]{.35\textwidth}
\begin{lstlisting}[language=make,numbers=none,basicstyle=\footnotesize\tt]
M.o: M.c DEP2 ... DEPn
    gcc -c $<
\end{lstlisting}
\begin{math}\end{math}
\end{minipage}

\begin{minipage}[c]{.4\textwidth}
\begin{lstlisting}[language=make,numbers=none,basicstyle=\footnotesize\tt]
EXEC: DEP1 ... DEPn
    gcc -o EXEC DEP1 ... DEPn
\end{lstlisting}
\end{minipage}
\hspace{1em}
$\longrightarrow$
\hspace{2em}
\begin{minipage}[c]{.35\textwidth}
\begin{lstlisting}[language=make,numbers=none,basicstyle=\footnotesize\tt]
EXEC: DEP1 ... DEPn
    gcc -o $@ $^
\end{lstlisting}
\begin{math}\end{math}
\end{minipage}
\end{frame}

%%%%%%%%%%%%%%%%%%%%%%%%%%%%%%%%%%%%%%%%%%%%%%%%%%%%%%%%%%%%%%%%%%%%%%%%
\begin{frame}[fragile]
\frametitle{Variables internes dans les {\tt Makefile} --- exemple}
\begin{multicols}{2}
\begin{center}
\scalebox{.35}{\begin{tikzpicture}[scale=.8]
    \node[Sommet,minimum size=2cm,font=\Large](Piece)at(-4,0)
        {\small \Code{Piece}};
    \node[Sommet,minimum size=2cm,font=\Large](Case)at(-4,-4)
        {\small \Code{Case}};
    \node[Sommet,minimum size=2cm,font=\Large](Position)at(0,-2)
        {\small \Code{Position}};
    \node[Sommet,minimum size=2cm,font=\Large](IA)at(4,0)
        {\small \Code{IA}};
    \node[Sommet,minimum size=2cm,font=\Large](IGraph)at(4,-4)
        {\small \Code{IGraph}};
    \node[Sommet,minimum size=2cm,draw=Vert,fill=Vert!20,font=\Large]
        (Main)at(7,-2){\small \Code{Main}};
    \draw[Fleche,dashed,draw=Vert](Main)edge[bend right=20]node{}(IA);
    \draw[Fleche,dashed,draw=Vert](Main)edge[bend left=20]node{}(IGraph);
    \draw[Fleche](IGraph)edge[bend left=20]node{}(Case);
    \draw[Fleche](IGraph)edge[bend left=20]node{}(Position);
    \draw[Fleche](IGraph)edge[bend right=40]node{}(Piece);
    \draw[Fleche](IA)edge[bend left=40]node{}(Case);
    \draw[Fleche](IA)edge[bend right=20]node{}(Position);
    \draw[Fleche](IA)edge[bend right=20]node{}(Piece);
    \draw[Fleche](Position)edge[bend right=20]node{}(Case);
    \draw[Fleche](Position)edge[bend left=20]node{}(Piece);
\end{tikzpicture}}
\end{center}
L'utilisation des variables et variables internes permet de simplifier
les \Code{Makefile}.
\end{multicols}

\begin{multicols}{2}
\begin{lstlisting}[language=make,basicstyle=\footnotesize\tt]
CC=colorgcc
CFLAGS=-ansi -pedantic -Wall
OBJ=Main.o Piece.o Case.o
    Position.o IA.o IGraph.o

Echecs: $(OBJ)
    $(CC) -o $@ $^ $(CFLAGS)


Main.o: Main.c IA.h IGraph.h
    $(CC) -c $< $(CFLAGS)

Piece.o: Piece.c Piece.h
    $(CC) -c $< $(CFLAGS)

Case.o: Case.c Case.h
    $(CC) -c $< $(CFLAGS)

Position.o: Position.c Position.h
            Piece.h Case.h
    $(CC) -c $< $(CFLAGS)

IA.o: IA.c IA.h
      Piece.h Case.h Position.h
    $(CC) -c $< $(CFLAGS)

IGraph.o: IGraph.c IGraph.h
          Piece.h Case.h Position.h
    $(CC) -c $< $(CFLAGS)
\end{lstlisting}
\begin{math}\end{math}
\end{multicols}
\end{frame}

%%%%%%%%%%%%%%%%%%%%%%%%%%%%%%%%%%%%%%%%%%%%%%%%%%%%%%%%%%%%%%%%%%%%%%%%
\begin{frame}[fragile]
\frametitle{Règles génériques}
Il est possible de simplifier encore d'avantage l'écriture des \Code{Makefile}
au moyen des \alert{règles génériques}.
\medskip

Ce sont des règles de la forme
\smallskip

\Code{\%.o: \%.c \\
$\rightarrow$ COMMANDE}
\smallskip

où \Code{COMMANDE} est une commande.
\medskip

Cette syntaxe simule une règle
\smallskip

\Code{M.o: M.c \\
$\rightarrow$ COMMANDE}
\smallskip

pour chaque fichier \Code{M.c} du projet.
\bigskip

{\bf Intérêt principal}~: l'unique règle
\begin{lstlisting}[language=make,numbers=none,basicstyle=\footnotesize\tt]
%.o: %.c
    gcc -c $<
\end{lstlisting}
\begin{math}\end{math}
permet de construire chaque fichier objet du projet.
\end{frame}

%%%%%%%%%%%%%%%%%%%%%%%%%%%%%%%%%%%%%%%%%%%%%%%%%%%%%%%%%%%%%%%%%%%%%%%%
\begin{frame}[fragile]
\frametitle{Règles génériques}
{\bf Attention}~: la règle
\begin{lstlisting}[language=make,numbers=none,basicstyle=\footnotesize\tt]
%.o: %.c
    gcc -c $<
\end{lstlisting}%
\begin{math}\end{math}%
ne prend pas en compte des dépendances des modules aux fichiers \Code{.h}
concernés.
\bigskip

Il faut les mentionner explicitement de la manière suivante~:
\smallskip

\Code{M.o: M.c M.h DEP1 ... DEPn}
\smallskip

pour chaque module \Code{M} du projet. \Code{DEP1}, \dots, \Code{DEPn}
sont les \Code{.h} dont le module \Code{M} dépend.
\end{frame}

%%%%%%%%%%%%%%%%%%%%%%%%%%%%%%%%%%%%%%%%%%%%%%%%%%%%%%%%%%%%%%%%%%%%%%%%
\begin{frame}[fragile]
\frametitle{Règles génériques --- exemple}
\begin{multicols}{2}
\begin{center}
\scalebox{.35}{\begin{tikzpicture}[scale=.8]
    \node[Sommet,minimum size=2cm,font=\Large](Piece)at(-4,0)
        {\small \Code{Piece}};
    \node[Sommet,minimum size=2cm,font=\Large](Case)at(-4,-4)
        {\small \Code{Case}};
    \node[Sommet,minimum size=2cm,font=\Large](Position)at(0,-2)
        {\small \Code{Position}};
    \node[Sommet,minimum size=2cm,font=\Large](IA)at(4,0)
        {\small \Code{IA}};
    \node[Sommet,minimum size=2cm,font=\Large](IGraph)at(4,-4)
        {\small \Code{IGraph}};
    \node[Sommet,minimum size=2cm,draw=Vert,fill=Vert!20,font=\Large]
        (Main)at(7,-2){\small \Code{Main}};
    \draw[Fleche,dashed,draw=Vert](Main)edge[bend right=20]node{}(IA);
    \draw[Fleche,dashed,draw=Vert](Main)edge[bend left=20]node{}(IGraph);
    \draw[Fleche](IGraph)edge[bend left=20]node{}(Case);
    \draw[Fleche](IGraph)edge[bend left=20]node{}(Position);
    \draw[Fleche](IGraph)edge[bend right=40]node{}(Piece);
    \draw[Fleche](IA)edge[bend left=40]node{}(Case);
    \draw[Fleche](IA)edge[bend right=20]node{}(Position);
    \draw[Fleche](IA)edge[bend right=20]node{}(Piece);
    \draw[Fleche](Position)edge[bend right=20]node{}(Case);
    \draw[Fleche](Position)edge[bend left=20]node{}(Piece);
\end{tikzpicture}}
\end{center}

L'utilisation des règles génériques permet de simplifier encore
les \Code{Makefile}.
\end{multicols}

\begin{multicols}{2}
\begin{lstlisting}[language=make,basicstyle=\footnotesize\tt]
CC=colorgcc
CFLAGS=-ansi -pedantic -Wall
OBJ=Main.o Piece.o Case.o
    Position.o IA.o IGraph.o

Echecs: $(OBJ)
    $(CC) -o $@ $^ $(CFLAGS)


Main.o: Main.c IA.h IGraph.h

Piece.o: Piece.c Piece.h

Case.o: Case.c Case.h

Position.o: Position.c Position.h
            Piece.h Case.h

IA.o: IA.c IA.h
      Piece.h Case.h Position.h

IGraph.o: IGraph.c IGraph.h
          Piece.h Case.h Position.h

%.o: %.c
    $(CC) -c $< $(CFLAGS)
\end{lstlisting}
\begin{math}\end{math}
\end{multicols}
\end{frame}

%%%%%%%%%%%%%%%%%%%%%%%%%%%%%%%%%%%%%%%%%%%%%%%%%%%%%%%%%%%%%%%%%%%%%%%%
\begin{frame}[fragile]
\frametitle{Règles de nettoyage}
{\bf Règle de nettoyage}~:
\begin{multicols}{2}
\begin{lstlisting}[language=make]
clean:
    rm -f *.o
\end{lstlisting}
Cette règle permet, lorsque l'on saisit la commande \Code{make clean},
de supprimer les fichiers \Code{.o} du répertoire courant.
\end{multicols}
\bigskip
\bigskip

{\bf Règle de nettoyage total}~:
\begin{multicols}{2}
\begin{lstlisting}[language=make]
mrproper: clean
    rm -f EXEC
\end{lstlisting}
Cette règle,
où \Code{EXEC} est le nom de l'exécutable du projet, permet de supprimer
tous les {\bf fichiers regénérables} (c.-à-d. les fichiers \Code{.o} et
l'exécutable) à partir des fichiers \Code{.c} et \Code{.h} du projet.
\end{multicols}
\end{frame}

%%%%%%%%%%%%%%%%%%%%%%%%%%%%%%%%%%%%%%%%%%%%%%%%%%%%%%%%%%%%%%%%%%%%%%%%
\begin{frame}[fragile]
\frametitle{Règles d'installation / désinstallation}

{\bf Règle d'installation}~:
\begin{multicols}{2}
\begin{lstlisting}[language=make]
install: EXEC
    mkdir ../bin
    mv EXEC ../bin/EXEC
    make mrproper
\end{lstlisting}
Cette règle permet de compiler le projet, de placer son exécutable
\Code{EXEC} dans un répertoire \Code{bin} et de supprimer les fichiers
regénérables.
\end{multicols}
\bigskip
\bigskip

{\bf Règle de désinstallation}~:
\begin{multicols}{2}
\begin{lstlisting}[language=make]
uninstall: mrproper
    rm -f ../bin/EXEC
    rm -rf ../bin
\end{lstlisting}
Cette règle permet de supprimer les fichiers regénérables,
l'exécutable du projet, ainsi que le répertoire \Code{bin} le contenant.
\end{multicols}
\end{frame}

%%%%%%%%%%%%%%%%%%%%%%%%%%%%%%%%%%%%%%%%%%%%%%%%%%%%%%%%%%%%%%%%%%%%%%%%
%%%%%%%%%%%%%%%%%%%%%%%%%%%%%%%%%%%%%%%%%%%%%%%%%%%%%%%%%%%%%%%%%%%%%%%%
\subsection{Bibliothèques}

%%%%%%%%%%%%%%%%%%%%%%%%%%%%%%%%%%%%%%%%%%%%%%%%%%%%%%%%%%%%%%%%%%%%%%%%
\begin{frame}\frametitle{Principes généraux}
Une \alert{bibliothèque} est un regroupement de modules offrant des
fonctionnalités allant vers un même objectif.
\bigskip

\uncover<2->{%
Par exemple, la bibliothèque graphique {\sf MLV} est un ensemble de
plusieurs modules (\Code{MLV\_audio}, \Code{MLV\_keyboard},
\Code{MLV\_image}, {\em etc.}) réunis dans le but d'offrir des fonctions
d'affichage graphique et de gérer les entrées/sorties
(son, clavier, souris, {\em etc.}).
\bigskip}

\uncover<3->{%
L'intérêt des bibliothèques est double~:
\begin{enumerate}
    \item il suffit de ne réaliser qu'{\bf une seule inclusion} pour
    bénéficier des fonctionnalités d'une bibliothèque, plutôt que
    plusieurs inclusions sans être sûr du fichier d'en-tête à inclure~;
    \smallskip}

    \uncover<4->{%
    \item sous réserve de savoir comment créer des bibliothèques, il est
    possible de {\bf partager et réutiliser} son propre code entre
    plusieurs de ses projets, sans avoir à le recompiler.}
\end{enumerate}
\end{frame}

%%%%%%%%%%%%%%%%%%%%%%%%%%%%%%%%%%%%%%%%%%%%%%%%%%%%%%%%%%%%%%%%%%%%%%%%
\begin{frame} \frametitle{Bibliothèques statiques}
Une \alert{bibliothèque statique} est un fichier d'extension \Code{.a}.
\bigskip

\uncover<2->{%
Lors de son utilisation, son code est inclus dans l'exécutable pendant
l'édition des liens.
\bigskip}

\begin{itemize}
    \uncover<3->{%
    \item {\bf Avantage}~: tout projet qui utilise une bibliothèque
    statique peut être exécuté sur une machine où la bibliothèque n'est
    pas installée.
    \medskip}

    \uncover<4->{%
    \item {\bf Inconvénient}~: l'exécutable est plus volumineux.}
\end{itemize}
\end{frame}

%%%%%%%%%%%%%%%%%%%%%%%%%%%%%%%%%%%%%%%%%%%%%%%%%%%%%%%%%%%%%%%%%%%%%%%%
\begin{frame} \frametitle{Bibliothèques dynamiques}
Une \alert{bibliothèque dynamique} est un fichier d'extension \Code{.so}.
\bigskip

\uncover<2->{%
Lors de son utilisation, son code n'est pas inclus dans l'exécutable.
C'est lors de l'exécution que les symboles provenant de la bibliothèque
sont résolus au moyen de l'éditeur de liens dynamique.
\bigskip}

\begin{itemize}
    \uncover<3->{%
    \item {\bf Avantages}~: l'exécutable est moins volumineux. Il n'y a
    pas de duplication du code de la bibliothèque sur un même système si
    plusieurs projets l'utilisent.
    \medskip}

    \uncover<4->{%
    \item {\bf Inconvénient}~: tout projet qui utilise une bibliothèque
    dynamique ne peut être exécuté que sur un système où cette
    dernière est installée.}
\end{itemize}
\end{frame}

%%%%%%%%%%%%%%%%%%%%%%%%%%%%%%%%%%%%%%%%%%%%%%%%%%%%%%%%%%%%%%%%%%%%%%%%
\begin{frame}
\frametitle{Compilation d'un projet utilisant des bibliothèques}
On suppose que l'on travaille sur un projet constitué de deux modules
\Code{A} et \Code{B} et d'un fichier principal \Code{Main.c}.
Ce projet utilise deux bibliothèques statiques~: \Code{libc.a} et
\Code{libMLV.a}.
\medskip

La compilation de ce projet se réalise de manière habituelle. La
différence porte sur l'étape d'{\bf édition des liens} dans laquelle il
est nécessaire de {\bf signaler les bibliothèques utilisées}.
\vspace{-1.5em}
\begin{center}
    \begin{tikzpicture}[every text node part/.style={align=left},yscale=.9]
        \node[Module](A)at(0,0){A.h \\ A.c};
        \node[Module](B)at(0,-1){B.h \\ B.c};
        \node[Module](Main)at(0,-2){Main.c};
        \node[FObj](Ao)at(3.5,0){A.o};
        \node[FObj](Bo)at(3.5,-1){B.o};
        \node[FObj](Maino)at(3.5,-2){Main.o};
        \node[Bib](libm)at(6,.7){libm.a};
        \node[Bib](libMLV)at(8,.7){libMLV.a};
        \node[Exec](Ex)at(10,-1){a.out};
        \node[font=\scriptsize \tt](edliens)at(7,-1)
            {gcc Main.o A.o B.o -lm -lMLV};
        \draw(A)edge[->,anchor=south,font=\scriptsize \tt] node
            {gcc -c A.c}(Ao);
        \draw(B)edge[->,anchor=south,font=\scriptsize \tt] node
            {gcc -c B.c}(Bo);
        \draw(Main)edge[->,anchor=south,font=\scriptsize \tt] node
            {gcc -c Main.c}(Maino);
        \draw[->](Ao)edge[bend left=20]node{}(edliens);
        \draw[->](Bo)--(edliens);
        \draw[->](Maino)edge[bend right=20]node{}(edliens);
        \draw[->](libm)edge[bend left=30]node{}(edliens);
        \draw[->](libMLV)edge[bend left=30]node{}(edliens);
        \draw[->](edliens)--(Ex);
    \end{tikzpicture}
\end{center}
\medskip

Pour utiliser des bibliothèques qui ne sont pas situées dans le répertoire
standard des bibliothèques, on précisera leur chemin \Code{chem} lors de
l'édition des liens au moyen de l'option \Code{-Lchem}.
\end{frame}

%%%%%%%%%%%%%%%%%%%%%%%%%%%%%%%%%%%%%%%%%%%%%%%%%%%%%%%%%%%%%%%%%%%%%%%%
\begin{frame}[fragile]
\frametitle{Compilation d'un projet utilisant des bibliothèques}
Le nom d'une bibliothèque commence par \Code{lib}. On signale l'utilisation
d'une bibliothèque à l'éditeur de liens par \Code{-lNOM} où \Code{libNOM}
est le nom de la bibliothèque.
%\medskip

\begin{multicols}{2}
\footnotesize
En supposant que le graphe d'inclusions du projet précédent est celui
ci-contre, un \Code{Makefile} possible est
\begin{center}
\scalebox{.6}{\begin{tikzpicture}
    \node[Sommet](A)at(0,-.5){\Code{A}};
    \node[Sommet](B)at(0,-1.5){\Code{B}};
    \node[Sommet](Main)at(2,-1){\Code{Main}};
    \draw[Fleche](B)--(A);
    \draw[Fleche,dashed,draw=Vert](Main)edge[bend right=20]node{}(A);
    \draw[Fleche,dashed,draw=Vert](Main)edge[bend left=20]node{}(B);
\end{tikzpicture}}
\end{center}
\end{multicols}\vspace{-1.5em}
\begin{multicols}{2}
\begin{lstlisting}[language=make,basicstyle=\scriptsize\tt]
CC=colorgcc
CFLAGS=-ansi -pedantic -Wall
LDFLAGS=-lm -lMLV
OBJ=Main.o A.o B.o

Projet: $(OBJ)
    $(CC) -o $@ $^ $(CFLAGS) $(LDFLAGS)

Main.o: Main.c A.h B.h

A.o: A.c A.h

B.o: B.c B.h A.h

%.o: %.c
    $(CC) -c $^ $(CFLAGS)

clean:
    rm -f *.o

mrproper: clean
    rm -f Projet

install: Projet
    mkdir ../bin
    mv Projet ../bin/Projet
    make mrproper

uninstall: mrproper
    rm -f ../bin/Projet
    rm -rf ../bin
\end{lstlisting}
\begin{math}\end{math}
\end{multicols}
\end{frame}

%%%%%%%%%%%%%%%%%%%%%%%%%%%%%%%%%%%%%%%%%%%%%%%%%%%%%%%%%%%%%%%%%%%%%%%%
\begin{frame}[fragile]
\frametitle{La bibliothèque standard}
La \alert{bibliothèque standard} \Code{libc.a} (\Code{libc.so}) regroupe
les vingt-quatre modules
\begin{center}
    \begin{tabular}{cccc}
    \Code{assert}, &\quad \Code{complex}, &\quad \Code{ctype}, &\quad \Code{errno}, \\
    \Code{fenv}, &\quad \Code{float}, &\quad \Code{inttypes}, &\quad \Code{iso646}, \\
    \Code{limits}, &\quad \Code{locale}, &\quad \Code{math}, &\quad \Code{setjmp}, \\
    \Code{signal}, &\quad \Code{stdarg}, &\quad \Code{stdbool}, &\quad \Code{stddef}, \\
    \Code{stdint}, &\quad \Code{stdio}, &\quad \Code{stdlib}, &\quad \Code{string}, \\
    \Code{tgmath}, &\quad \Code{time}, &\quad \Code{wchar}, &\quad \Code{wctype}.
    \end{tabular}
\end{center}
\medskip

Cette bibliothèque est {\bf liée implicitement} lors de toute édition
des liens.
\medskip

Il est donc possible d'utiliser les modules de la bibliothèque standard
juste en les incluant (\Code{\#include <NOM.h>}).
\end{frame}

%%%%%%%%%%%%%%%%%%%%%%%%%%%%%%%%%%%%%%%%%%%%%%%%%%%%%%%%%%%%%%%%%%%%%%%%
\begin{frame}[fragile]
\frametitle{Création de bibliothèques statiques}
Pour \alert{créer une bibliothèque statique} \Code{X}, on suit les étapes
suivantes~:
\smallskip

\begin{small}
\begin{enumerate}
    \item {\bf écriture des modules} \Code{M1}, \dots, \Code{Mn} qui vont
    constituer la bibliothèque~;
    \medskip

    \item {\bf compilation des modules} et création des fichiers objets
    \Code{M1.o}, \dots, \Code{Mn.o}~;
    \medskip

    \item {\bf création de l'archive} \Code{libX.a} par
    \begin{center}\Code{ar r libX.a M1.o ... Mn.o}\end{center}
    \medskip

    \item {\bf génération de l'index} de la bibliothèque par
    \begin{center} \Code{ranlib libX.a} \end{center}
    \medskip

    \item (étape facultative) {\bf écriture d'un fichier d'en-tête global}
    \begin{center}
    \begin{minipage}[c]{.3\textwidth}
\begin{lstlisting}[frame=single,numbers=none,basicstyle=\scriptsize\tt]
/* X.h */
#ifndef __X__
#define __X__

#include "M1.h"
...
#include "Mn.h"

#endif
\end{lstlisting}
    \end{minipage}
    \end{center}
\end{enumerate}
\end{small}
\end{frame}


%%%%%%%%%%%%%%%%%%%%%%%%%%%%%%%%%%%%%%%%%%%%%%%%%%%%%%%%%%%%%%%%%%%%%%%%
\begin{frame}[fragile]
\frametitle{Création de bibliothèques statiques --- exemple}
On suppose que l'on a créé trois modules~:
\begin{enumerate}
    \item \Code{Liste} pour la représentation des listes chaînées~;
    \smallskip

    \item \Code{Arbre} pour la représentation des arbres binaires de
    recherche~;
    \smallskip

    \item \Code{Tri} pour l'implantation d'algorithmes de tris de tableaux.
\end{enumerate}
\bigskip
\bigskip

On souhaite regrouper ces modules en une bibliothèque nommée \Code{algo}.
\medskip

Celle-ci sera constituée de deux fichiers~;
\begin{enumerate}
    \item \Code{libalgo.a}, l'implantation de la bibliothèque~;
    \smallskip

    \item \Code{Algo.h}, l'en-tête de la bibliothèque.
\end{enumerate}
\end{frame}

%%%%%%%%%%%%%%%%%%%%%%%%%%%%%%%%%%%%%%%%%%%%%%%%%%%%%%%%%%%%%%%%%%%%%%%%
\begin{frame}[fragile]
\frametitle{Création de bibliothèques statiques --- exemple}
Pour créer de la bibliothèque \Code{algo}, on saisit les commandes
\smallskip

\begin{small}
\Code{gcc -c Liste.c ; gcc -c Arbre.c ; gcc -c Tri.c} \\
\Code{ar r libalgo.a Liste.o Arbre.o Tri.o} \\
\Code{ranlib libalgo.a}
\end{small}
\smallskip

et on écrit l'en-tête global
\begin{center}
\begin{minipage}[c]{.3\textwidth}
\begin{lstlisting}[frame=single,numbers=none,basicstyle=\scriptsize\tt]
/* Algo.h */
#ifndef __ALGO__
#define __ALGO__

#include "Liste.h"
#include "Arbre.h"
#include "Tri.h"

#endif
\end{lstlisting}
\end{minipage}
\end{center}
\medskip

Pour utiliser la bibliothèque \Code{algo} dans un fichier \Code{F} d'un
projet \Code{P}, il faut inclure dans \Code{F} son en-tête, réaliser
l'édition des liens de \Code{P} avec l'option \Code{-lalgo} et indiquer
son chemin \Code{chem} avec l'option \Code{-Lchem}.
\end{frame}

%%%%%%%%%%%%%%%%%%%%%%%%%%%%%%%%%%%%%%%%%%%%%%%%%%%%%%%%%%%%%%%%%%%%%%%%
\begin{frame}[fragile]
\frametitle{Index}
Il est possible de \alert{consulter l'index} d'une bibliothèque statique
\Code{LIB} par la commande
\begin{center}
\Code{nm -s libLIB.a}
\end{center}
\medskip

\begin{center}
\begin{minipage}[c]{.22\textwidth}
\begin{lstlisting}[frame=single,numbers=none,basicstyle=\scriptsize\tt]
/* A.h */
#ifndef __A__
#define __A__
    char h(int a);
#endif
\end{lstlisting}
\end{minipage}
\enspace
\begin{minipage}[c]{.23\textwidth}
\begin{lstlisting}[frame=single,numbers=none,basicstyle=\scriptsize\tt]
/* A.c */
#include "A.h"
char h(int a) {
    return a % 256;
}
\end{lstlisting}
\end{minipage}
\quad
\begin{minipage}[c]{.23\textwidth}
\begin{lstlisting}[frame=single,numbers=none,basicstyle=\scriptsize\tt]
/* B.h */
#ifndef __B__
#define __B__
    typedef int S;

    int f(S s);
    char g(int a);
#endif
\end{lstlisting}
\end{minipage}
\enspace
\begin{minipage}[c]{.22\textwidth}
\begin{lstlisting}[frame=single,numbers=none,basicstyle=\scriptsize\tt]
/* B.c */
#include "B.h"
int f(S s) {
    return s;
}
char g(int a) {
    return a % 64;
}
\end{lstlisting}
\end{minipage}
\end{center}

\begin{center}
\begin{minipage}[c]{.4\textwidth}
{\bf Création}~: \\
\begin{small}
\Code{gcc -c A.c ; gcc -c B.c} \\
\Code{ar r libAB.a A.o B.o} \\
\Code{ranlib libAB.a}
\end{small}
\medskip

{\bf Affichage}~: \\
\begin{small}
\Code{nm -s libAB.a}
\end{small}
\end{minipage}
\qquad
\begin{minipage}[c]{.2\textwidth}
\begin{lstlisting}[frame=single,numbers=none,basicstyle=\ttfamily\tiny,
showstringspaces=false]
Indexe de l'archive:
h in A.o
f in B.o
g in B.o

A.o:
0000000000000000 T h

B.o:
0000000000000000 T f
000000000000000c T g
\end{lstlisting}
\end{minipage}
\end{center}
\end{frame}


%%%%%%%%%%%%%%%%%%%%%%%%%%%%%%%%%%%%%%%%%%%%%%%%%%%%%%%%%%%%%%%%%%%%%%%%
%%%%%%%%%%%%%%%%%%%%%%%%%%%%%%%%%%%%%%%%%%%%%%%%%%%%%%%%%%%%%%%%%%%%%%%%
%%%%%%%%%%%%%%%%%%%%%%%%%%%%%%%%%%%%%%%%%%%%%%%%%%%%%%%%%%%%%%%%%%%%%%%%
\begin{frame} \frametitle{}
\part{Axe~2}
\begin{center} \Large
    {\bf Axe~2}~: comprendre les mécanismes de base
\end{center}

\begin{footnotesize}
    \tableofcontents[hideallsubsections,part=2]
\end{footnotesize}
\end{frame}

% Auteur : Samuele Giraudo
% Création : oct. 2013
% Modifications : août 2014, oct. 2014, déc. 2015

\tikzstyle{Case}=[rectangle,draw=black,line width=1pt, minimum size=2cm,
minimum height=1cm]
\tikzstyle{Bloc}=[rectangle,draw=Vert!100,fill=Vert!20,
    line width=1pt,minimum width=1cm,minimum height=0.75cm]
\tikzstyle{BlocM}=[Bloc,draw=Marron=!100,fill=Marron!10]

%%%%%%%%%%%%%%%%%%%%%%%%%%%%%%%%%%%%%%%%%%%%%%%%%%%%%%%%%%%%%%%%%%%%%%%%
%%%%%%%%%%%%%%%%%%%%%%%%%%%%%%%%%%%%%%%%%%%%%%%%%%%%%%%%%%%%%%%%%%%%%%%%
%%%%%%%%%%%%%%%%%%%%%%%%%%%%%%%%%%%%%%%%%%%%%%%%%%%%%%%%%%%%%%%%%%%%%%%%
\section{Allocation dynamique}

%%%%%%%%%%%%%%%%%%%%%%%%%%%%%%%%%%%%%%%%%%%%%%%%%%%%%%%%%%%%%%%%%%%%%%%%
%%%%%%%%%%%%%%%%%%%%%%%%%%%%%%%%%%%%%%%%%%%%%%%%%%%%%%%%%%%%%%%%%%%%%%%%
\subsection{Pointeurs}

%%%%%%%%%%%%%%%%%%%%%%%%%%%%%%%%%%%%%%%%%%%%%%%%%%%%%%%%%%%%%%%%%%%%%%%%
\begin{frame} \frametitle{La mémoire}
Lors de l'exécution d'un programme, une portion (de taille variable) de
la mémoire lui est dédiée. Cette zone lui est exclusive. On l'appelle
la \alert{mémoire}.
\medskip

\uncover<2->{
On peut voir la mémoire comme un {\bf très grand tableau d'octets}.}

\uncover<3->{
\begin{multicols}{2} \small
La mémoire est segmentée en plusieurs parties~:
\begin{itemize}
    \item la {\bf zone statique} qui contient le code et les
    données statiques~;
    \item le {\bf tas}, de taille variable au fil de l'exécution~;
    \item la {\bf pile}, de taille variable au fil de l'exécution.
\end{itemize}
Il y a d'autres zones (non repr. ici).
\begin{center}
\scalebox{.6}{\begin{tikzpicture}
    \node[Zone,fill=Rouge!30](1)at(0,0){Zone statique};
    \node[Zone,fill=Bleu!30](2)at(0,-1.5){Tas\phantom{q}};
    \node[Zone,fill=Vert!30](3)at(0,-3){\dots\phantom{Tq}};
    \node[Zone,fill=Marron!30](4)at(0,-4.5){Pile\phantom{q}};
    \node[Zone,fill=gray!30,minimum height=.5cm](5)at(0,-5.5){Autre\phantom{q}};
    \draw[->,line width=1.5pt](-2,-1.5)--(-2,-2.5);
    \draw[->,line width=1.5pt](-2,-4.5)--(-2,-3.5);
\end{tikzpicture}}
\end{center}
\end{multicols}}
\medskip

\uncover<4->{
Le tas contient les variables allouées dynamiquement.
\medskip

La pile contient les variables locales lors des appels aux fonctions.}
\end{frame}

%%%%%%%%%%%%%%%%%%%%%%%%%%%%%%%%%%%%%%%%%%%%%%%%%%%%%%%%%%%%%%%%%%%%%%%%
\begin{frame} \frametitle{Pointeurs}
Nous avons vu que chaque variable possède une \alert{adresse}.
\bigskip

Connaître l'adresse d'une variable est suffisant pour la manipuler
(c.-à-d., lire et modifier sa valeur). L'objet qui permet de représenter
et de manipuler des adresses est le \alert{pointeur}.
\bigskip

\uncover<2->{
Un \alert{pointeur} est une entité constituée des deux éléments suivants~:

\begin{enumerate}
    \item une adresse vers une zone de la mémoire~;
    \smallskip
    
    \item un type.
\end{enumerate}}
\end{frame}

%%%%%%%%%%%%%%%%%%%%%%%%%%%%%%%%%%%%%%%%%%%%%%%%%%%%%%%%%%%%%%%%%%%%%%%%
\begin{frame} \frametitle{Pointeurs}
{\bf Intuitivement}, un pointeur est une \alert{flèche} munie d'un mode 
d'emploi, le tout pointant vers une zone de la mémoire.
\smallskip

Le \alert{mode d'emploi} décrit le type de la zone de la mémoire ainsi 
adressée en renseignant sur la taille de la zone.
\medskip

\begin{multicols}{2}
\uncover<2->{Si \Code{ptr\_1} est un pointeur sur une donnée de type 
\Code{int} ($4$ octets)~:
\begin{center}
    \scalebox{.5}{\begin{tikzpicture}
        \node[Case,fill=Gray!20](1)at(0,-1){};
        \node[Case,fill=Gray!20](2)at(0,-2){};
        \node[Case,fill=BrickRed!40](3)at(0,-3){};
        \node[Case,fill=BrickRed!40](4)at(0,-4){};
        \node[Case,fill=BrickRed!40](5)at(0,-5){};
        \node[Case,fill=BrickRed!40](6)at(0,-6){};
        \node[Case,fill=Gray!20](7)at(0,-7){};
        \node[right of=1,node distance=1.5cm]{\tt 1000};
        \node[right of=2,node distance=1.5cm]{\tt 1001};
        \node[right of=3,node distance=1.5cm]{\tt 1002};
        \node[right of=4,node distance=1.5cm]{\tt 1003};
        \node[right of=5,node distance=1.5cm]{\tt 1004};
        \node[right of=6,node distance=1.5cm]{\tt 1005};
        \node[right of=7,node distance=1.5cm]{\tt 1006};
        \node(p)at(-3,-4){\Code{ptr\_1}};
        \draw(p)edge[thick,->,bend right=-20](3);
        \node(p_val)at(-3,-4.5){valeur : {\tt 1002}};
        \node(p_type)at(-3,-5){type : \Code{int *}};
    \end{tikzpicture}}
\end{center}}

\uncover<3->{Si \Code{ptr\_2} est un pointeur sur une donnée de type 
\Code{char} ($2$ octets)~:
\begin{center}
    \scalebox{.5}{\begin{tikzpicture}
        \node[Case,fill=Gray!20](1)at(0,-1){};
        \node[Case,fill=Gray!20](2)at(0,-2){};
        \node[Case,fill=BrickRed!40](3)at(0,-3){};
        \node[Case,fill=BrickRed!40](4)at(0,-4){};
        \node[Case,fill=Gray!20](5)at(0,-5){};
        \node[Case,fill=Gray!20](6)at(0,-6){};
        \node[Case,fill=Gray!20](7)at(0,-7){};
        \node[right of=1,node distance=1.5cm]{\tt 1000};
        \node[right of=2,node distance=1.5cm]{\tt 1001};
        \node[right of=3,node distance=1.5cm]{\tt 1002};
        \node[right of=4,node distance=1.5cm]{\tt 1003};
        \node[right of=5,node distance=1.5cm]{\tt 1004};
        \node[right of=6,node distance=1.5cm]{\tt 1005};
        \node[right of=7,node distance=1.5cm]{\tt 1006};
        \node(p)at(-3,-4){\Code{ptr\_2}};
        \draw(p)edge[thick,->,bend right=-20](3);
        \node(p_val)at(-3,-4.5){valeur : {\tt 1002}};
        \node(p_type)at(-3,-5){type : \Code{char *}};
    \end{tikzpicture}}
\end{center}}
\end{multicols}
\end{frame}

%%%%%%%%%%%%%%%%%%%%%%%%%%%%%%%%%%%%%%%%%%%%%%%%%%%%%%%%%%%%%%%%%%%%%%%%
\begin{frame} \frametitle{Déclaration de pointeurs}
On \alert{déclare} un pointeur sur une zone de la mémoire de type \Code{T} par
\begin{center} \Code{T *ptr;} \end{center}
\medskip

\uncover<2->{
On \alert{accède à la valeur} de la zone mémoire pointée par \Code{ptr}
par
\begin{center} \Code{*ptr} \end{center}}

\uncover<3->{
On \alert{accède à l'adresse} de la zone mémoire pointée par \Code{ptr} par
\begin{center} \Code{ptr} \end{center}
\bigskip}

\uncover<4->{
{\bf Attention}~: le même symbole \Code{'*'} est utilisé pour la
déclaration d'un pointeur et pour accéder à la valeur pointée.
C'est une imperfection du langage {\sf C} qui peut porter à confusion.}
\end{frame}

%%%%%%%%%%%%%%%%%%%%%%%%%%%%%%%%%%%%%%%%%%%%%%%%%%%%%%%%%%%%%%%%%%%%%%%%
\begin{frame} \frametitle{Manipulation de pointeurs}
On suppose que \Code{ptr} est un pointeur pointant sur une zone de la
mémoire de type \Code{T}.
\bigskip
\bigskip

\uncover<2->{
Il est possible de \alert{changer l'endroit} où \Code{ptr} pointe par
\begin{center} \Code{ptr = ADR;} \end{center}
où \Code{ADR} est une adresse de la mémoire de type \Code{T}.
\bigskip
\bigskip}

\uncover<3->{
Il est possible d'\alert{affecter une valeur} \Code{VAL} de type \Code{T}
à \Code{*ptr} par
\begin{center} \Code{*ptr = VAL;} \end{center}
De cette manière, la zone mémoire d'adresse \Code{ptr} est modifiée et
devient de valeur \Code{VAL}.}
\end{frame}

%%%%%%%%%%%%%%%%%%%%%%%%%%%%%%%%%%%%%%%%%%%%%%%%%%%%%%%%%%%%%%%%%%%%%%%%
\begin{frame}[fragile] \frametitle{Manipulation de pointeurs --- exemple}
\begin{multicols}{2}
\begin{lstlisting}
/* (1) */
int num1, num2;
int *ptr1, *ptr2;
num1 = 10;
num2 = 20;

/* (2) */
ptr1 = &num1;
ptr2 = &num2;
*ptr1 = 100;
*ptr2 = 200;

/* (3) */
ptr2 = ptr1;
*ptr1 = 300;
*ptr2 = 400;
\end{lstlisting}

\scalebox{.6}{\begin{tikzpicture}
    \node[Case](num_1_nom)at(0,0){\Code{num\_1}};
    \node[Case](num_1_val)at(0,1){\Code{10}};
    %
    \node[Case](num_2_nom)at(2.5,0){\Code{num\_2}};
    \node[Case](num_2_val)at(2.5,1){\Code{20}};
    %
    \node[Case](ptr_1_nom)at(5,0){\Code{ptr\_1}};
    \node[Case](ptr_1_val)at(5,1){\Code{?}};
    \node(croix_1)at(5,2){\begin{math}\times\end{math}};
    \draw(ptr_1_val)edge[thick,->](croix_1);
    %
    \node[Case](ptr_2_nom)at(7.5,0){\Code{ptr\_2}};
    \node[Case](ptr_2_val)at(7.5,1){\Code{?}};
    \node(croix_2)at(7.5,2){\begin{math}\times\end{math}};
    \draw(ptr_2_val)edge[thick,->](croix_2);
\end{tikzpicture}}

\scalebox{.6}{\begin{tikzpicture}
    \node[Case](num_1_nom)at(0,0){\Code{num\_1}};
    \node[Case](num_1_val)at(0,1){\Code{100}};
    %
    \node[Case](num_2_nom)at(2.5,0){\Code{num\_2}};
    \node[Case](num_2_val)at(2.5,1){\Code{200}};
    %
    \node[Case](ptr_1_nom)at(5,0){\Code{ptr\_1}};
    \node[Case](ptr_1_val)at(5,1){\Code{\&num\_1}};
    \draw(ptr_1_val)edge[thick,->,bend right=45](num_1_val);
    %
    \node[Case](ptr_2_nom)at(7.5,0){\Code{ptr\_2}};
    \node[Case](ptr_2_val)at(7.5,1){\Code{\&num\_2}};
    \draw(ptr_2_val)edge[thick,->,bend right=45](num_2_val);
\end{tikzpicture}}

\scalebox{.6}{\begin{tikzpicture}
    \node[Case](num_1_nom)at(0,0){\Code{num\_1}};
    \node[Case](num_1_val)at(0,1){\Code{400}};
    %
    \node[Case](num_2_nom)at(2.5,0){\Code{num\_2}};
    \node[Case](num_2_val)at(2.5,1){\Code{200}};
    %
    \node[Case](ptr_1_nom)at(5,0){\Code{ptr\_1}};
    \node[Case](ptr_1_val)at(5,1){\Code{\&num\_1}};
    \draw(ptr_1_val)edge[thick,->,bend right=45](num_1_val);
    %
    \node[Case](ptr_2_nom)at(7.5,0){\Code{ptr\_2}};
    \node[Case](ptr_2_val)at(7.5,1){\Code{\&num\_1}};
    \draw(ptr_2_val)edge[thick,->,bend right=45](num_1_val);
\end{tikzpicture}}
\end{multicols}
\end{frame}

%%%%%%%%%%%%%%%%%%%%%%%%%%%%%%%%%%%%%%%%%%%%%%%%%%%%%%%%%%%%%%%%%%%%%%%%
\begin{frame} \frametitle{Pointeurs et tableaux statiques}
Un \alert{tableau} est un pointeur vers le $1\ier$ élément qui le
constitue. Les autres éléments du tableau sont contigus en mémoire et
situés à des adresses plus élevées.
\medskip

\uncover<2->{
On \alert{déclare} un tableau statique de taille \Code{N} de valeurs de
type \Code{T} par
\begin{center} \Code{T tab[N];} \end{center}

\begin{center}
\scalebox{.6}{\begin{tikzpicture}
    \node[Bloc](1)at(0,0){};
    \node at(1,0){$\dots$};
    \node[Bloc](2)at(2,0){};
    \node[Bloc](3)at(3,0){};
    \node at(4,0){$\dots$};
    \node[Bloc](4)at(5,0){};
    \node[Bloc](5)at(6,0){};
    \node at(7,0){$\dots$};
    \node[Bloc](6)at(8,0){};    
    \node at(9,0){\Large $\dots$};
    \node[Bloc](7)at(10,0){};
    \node at(11,0){$\dots$};
    \node[Bloc](8)at(12,0){};
    \draw(-0.5,0.75)edge[anchor=south,<->,line width=1.5pt]
        node{\small 1 octet}(0.5,0.75);
    \draw(-0.5,-0.75)edge[anchor=north,<->,line width=1.5pt]
        node{\small \Code{tab[0]}}(2.5,-0.75);
    \draw(2.5,-0.75)edge[anchor=north,<->,line width=1.5pt]
        node{\small \Code{tab[1]}}(5.5,-0.75);
    \draw(5.5,-0.75)edge[anchor=north,<->,line width=1.5pt]
        node{\small \Code{tab[2]}}(8.5,-0.75);
    \draw(9.5,-0.75)edge[anchor=north,<->,line width=1.5pt]
        node{\small \Code{tab[N - 1]}}(12.5,-0.75);
    %
    \node(p)at(-3, 0){\Code{tab}};
    \draw(p)edge[thick,->,bend right=-15](1);
\end{tikzpicture}}
\end{center}

\medskip}

\uncover<3->{
On \alert{accède à la valeur} du \Code{i}\ieme\, élément de \Code{tab} par
\begin{center} \Code{tab[i]} \end{center}
\medskip}

\uncover<4->{
On \alert{affecte une valeur} \Code{VAL} de type \Code{T} en
\Code{i}\ieme\, position dans \Code{tab} par
\begin{center} \Code{tab[i] = VAL;} \end{center}}
\end{frame}

%%%%%%%%%%%%%%%%%%%%%%%%%%%%%%%%%%%%%%%%%%%%%%%%%%%%%%%%%%%%%%%%%%%%%%%%
\begin{frame} \frametitle{Pointeurs et tableaux statiques}
Sachant qu'un tableau \Code{tab} est un pointeur et que ses éléments sont
contigus en mémoire, la syntaxe
\begin{center} \Code{tab[i]} \end{center}
est équivalente à
\begin{center} \Code{*(tab + i)} \end{center}
\bigskip

\uncover<2->{
Le type du pointeur \Code{tab} permet de faire un décalage correct en
fonction de la taille en mémoire de ses éléments.
\medskip}

\uncover<3->{
En effet, si \Code{ptr} est un pointeur pointant sur une zone mémoire de
type \Code{T}, la valeur de l'expression \Code{ptr + i} dépend de la
taille nécessaire pour représenter une valeur de type \Code{T} (c.-à-d.
de \Code{sizeof(T)}).}
\end{frame}

%%%%%%%%%%%%%%%%%%%%%%%%%%%%%%%%%%%%%%%%%%%%%%%%%%%%%%%%%%%%%%%%%%%%%%%%
\begin{frame}[fragile] \frametitle{Pointeurs et tableaux statiques}
Considérons les instructions suivantes~:
\begin{multicols}{2}
\begin{semiverbatim}\small
int tab[2];
char *ptr;
tab[0] = 300;
tab[1] = 60;
printf("%p %p\\n", tab + 0,
  tab + 1);
printf("%d %d\\n", tab[0],
  tab[1]);
ptr = (char *) tab;
printf("%p %p\\n", ptr + 0,
  ptr + 1);
printf("%d %d\\n", ptr[0],
  ptr[1]);
\end{semiverbatim}
\uncover<2->{
\begin{center} \todo{Faire les dessins de la mémoire.} \end{center}}

\uncover<3->{
Elles affichent \\
\Sortie{\small 0x7fffc38b5d60 0x7fffc38b5d64 \\
300 60 \\
0x7fffc38b5d60 0x7fffc38b5d61 \\
44 1}
\smallskip}

\uncover<4->{
Le pointeur \Code{ptr} interprète les éléments du tableau \Code{tab}
comme des valeurs de taille $1$ octet (= \Code{sizeof(char)}).}
\end{multicols}
\end{frame}

%%%%%%%%%%%%%%%%%%%%%%%%%%%%%%%%%%%%%%%%%%%%%%%%%%%%%%%%%%%%%%%%%%%%%%%%
\begin{frame}[fragile] 
    \frametitle{Pointeurs et tableaux statiques --- exemple}

\end{frame}

%%%%%%%%%%%%%%%%%%%%%%%%%%%%%%%%%%%%%%%%%%%%%%%%%%%%%%%%%%%%%%%%%%%%%%%%
\begin{frame}[fragile] \frametitle{Récapitulatif}
Dans cette sous-partie, nous avons vu~:

\begin{enumerate}
    \item la mémoire et sa segmentation en plusieurs parties~;
    \smallskip

    \item la notion d'adresse et de pointeur~;
    \smallskip

    \item les tableaux statiques~;
    \smallskip

    \item l'accès à une partie d'un tableau (statique) via son adresse.
\end{enumerate}
\end{frame}

%%%%%%%%%%%%%%%%%%%%%%%%%%%%%%%%%%%%%%%%%%%%%%%%%%%%%%%%%%%%%%%%%%%%%%%%
%%%%%%%%%%%%%%%%%%%%%%%%%%%%%%%%%%%%%%%%%%%%%%%%%%%%%%%%%%%%%%%%%%%%%%%%
\subsection{Passage par adresse}

%%%%%%%%%%%%%%%%%%%%%%%%%%%%%%%%%%%%%%%%%%%%%%%%%%%%%%%%%%%%%%%%%%%%%%%%
\begin{frame}[fragile] \frametitle{Passage par valeur}
Nous savons qu'il est impossible de modifier la valeur d'un argument
passé à une fonction car celle-ci travaille sur une {\bf copie de sa
valeur}.
\bigskip

\begin{multicols}{2}
\begin{semiverbatim}\uncover<2->{
void incrementer(int nb) \{
    nb = nb + 1;
\}
...
num = 5;
incrementer(num);
printf("%d\\n", num);}
\end{semiverbatim}
\uncover<2->{
\begin{center} \todo{Dessiner le contenu de la pile.} \end{center}
Ces instructions affichent \Sortie{5}.
\smallskip}

\uncover<3->{
En ligne 6, c'est la {\bf valeur} de \Code{num} qui est transmise et
non pas la variable elle-même.}
\vspace{3em}
\end{multicols}
\end{frame}

%%%%%%%%%%%%%%%%%%%%%%%%%%%%%%%%%%%%%%%%%%%%%%%%%%%%%%%%%%%%%%%%%%%%%%%%
\begin{frame}[fragile] \frametitle{Passage par adresse}
Cependant, si l'on donne l'adresse et le type d'une zone mémoire
(ce qui revient à donner un pointeur vers la zone mémoire considérée),
il devient possible de la modifier.
\bigskip

\begin{multicols}{2}
\begin{semiverbatim}\small\uncover<2->{
void incrementer(int *ptr_nb) \{
    *ptr_nb = *ptr_nb + 1;
\}
...
num = 5;
incrementer(&num);
printf("%d\\n", num);}
\end{semiverbatim}
\uncover<2->{
\begin{center} \todo{Dessiner le contenu de la pile.} \end{center}
Ces instructions affichent \Sortie{6}.
\smallskip}

\uncover<3->{
En ligne 6, c'est (la valeur de) l'{\bf adresse} de \Code{num} qui est
transmise.}
\vspace{3em}
\end{multicols}
\bigskip

\uncover<4->{
C'est un \alert{passage par adresse}.}
\end{frame}

%%%%%%%%%%%%%%%%%%%%%%%%%%%%%%%%%%%%%%%%%%%%%%%%%%%%%%%%%%%%%%%%%%%%%%%%
\begin{frame}[fragile] \frametitle{Qualificateur {\tt const}}
Dans certains cas, un passage par adresse n'est pas fait pour modifier
la valeur située à l'adresse spécifiée.
\bigskip

\begin{multicols}{2}
\uncover<2->{
P.ex., cette fonction affiche, sans la modifier, la chaîne de caractères
\Code{chaine} en remplaçant ses caractères \Code{c} par des étoiles
\Code{'*'}.}
\bigskip
\bigskip

\uncover<3->{
On souhaite {\bf protéger} \Code{chaine} de toute modification sur son
contenu.}
\vspace{3em}
\begin{semiverbatim}\footnotesize\uncover<2->{
void eto(\uncover<4->{const} char *chaine, char c) \{
    int i = 0;
    while (chaine[i] != '\\0') \{
        if (chaine[i] == c)
            putchar('*');
        else
            putchar(chaine[i]);
        i += 1;
    \}
\}}
\end{semiverbatim}
\end{multicols}
\bigskip

\uncover<4->{
Pour cela, on place le \alert{qualificateur \Code{const}}
devant la déclaration de \Code{chaine}.}
\end{frame}

%%%%%%%%%%%%%%%%%%%%%%%%%%%%%%%%%%%%%%%%%%%%%%%%%%%%%%%%%%%%%%%%%%%%%%%%
\begin{frame}[fragile] \frametitle{Qualificateur {\tt const}}
Ainsi, plus généralement,
\begin{center}
    \Code{const T *ID}
\end{center}
déclare un paramètre \Code{ID}, pointeur sur une zone mémoire de
type \Code{T}, dont le \alert{contenu est protégé en écriture}.
\medskip

\begin{multicols}{2}
\begin{semiverbatim}\small\uncover<2->{
void exemple_1(const int *x) \{
    *x = 40;
\}}
\end{semiverbatim} \small
\uncover<2->{
Cette tentative directe pour modifier la valeur à l'adresse \Code{x} est
sanctionnée par le compilateur par une erreur.}
\end{multicols}
\smallskip

\begin{multicols}{2}
\begin{semiverbatim}\small\uncover<3->{
void exemple_2(const int *x) \{
    int *tmp;
    tmp = x;
    *tmp = 40;
\}}
\end{semiverbatim} \small
\uncover<3->{
Cette tentative détournée pour modifier la valeur à l'adresse \Code{x} est
sanctionnée par le compilateur par un avertissement. Il est ainsi
possible de modifier une valeur protégée.}
\end{multicols}
\medskip

\uncover<4->{
Le qualificateur \Code{const} est avant tout une aide pour le
développeur~: il informe d'un comportement à adopter.}
\end{frame}

%%%%%%%%%%%%%%%%%%%%%%%%%%%%%%%%%%%%%%%%%%%%%%%%%%%%%%%%%%%%%%%%%%%%%%%%
\begin{frame}[fragile] \frametitle{Qualificateur {\tt const}}
Le qualificateur \Code{const} permet quelques subtilités~:
\begin{center}
    \Code{T *const ID}
\end{center}
déclare un paramètre \Code{ID}, \alert{pointeur fixe} sur une zone mémoire
de type \Code{T} \\ (il n'est pas possible de faire pointer \Code{ID} ailleurs).
\uncover<2->{De plus,
\begin{center}
    \Code{const T *const ID}
\end{center}
déclare un paramètre \Code{ID}, pointeur fixe sur une zone mémoire
de type \Code{T}, dont le contenu est protégé en écriture.}
\bigskip

\uncover<3->{
Voici un exemple qui illustre les quatre cas de figure~:}
\begin{multicols}{2}
\begin{semiverbatim}\footnotesize
\uncover<3->{void exemple_3(int *a, const int *b,
        int *const c,
        const int *const d) \{}

    \uncover<3->{int e;}

    \uncover<4->{a = &e; /* Autorise */
    *a = 3; /* Autorise */}
    \uncover<5->{
    b = &e; /* Autorise */
    *b = 3; /* Interdit */}
    \uncover<6->{
    c = &e; /* Interdit */
    *c = 3; /* Autorise */}
    \uncover<7->{
    d = &e; /* Interdit */
    *d = 3; /* Interdit */}
\uncover<3->{\}}
\end{semiverbatim}
\end{multicols}
\end{frame}

%%%%%%%%%%%%%%%%%%%%%%%%%%%%%%%%%%%%%%%%%%%%%%%%%%%%%%%%%%%%%%%%%%%%%%%%
\begin{frame}[fragile] \frametitle{Récapitulatif}
Dans cette sous-partie, nous avons vu~:

\begin{enumerate}
    \item la différence entre le passage par valeur et le passage par
    adresse des arguments d'une fonction~;
    \smallskip

    \item le qualificateur \Code{const} pour les adresses figurant en
    paramètre d'une fonction~;
    \smallskip

    \item les quatre moyens d'utiliser ce qualificateur.
\end{enumerate}
\end{frame}

%%%%%%%%%%%%%%%%%%%%%%%%%%%%%%%%%%%%%%%%%%%%%%%%%%%%%%%%%%%%%%%%%%%%%%%%
%%%%%%%%%%%%%%%%%%%%%%%%%%%%%%%%%%%%%%%%%%%%%%%%%%%%%%%%%%%%%%%%%%%%%%%%
\subsection{Allocation dynamique}

%%%%%%%%%%%%%%%%%%%%%%%%%%%%%%%%%%%%%%%%%%%%%%%%%%%%%%%%%%%%%%%%%%%%%%%%
\begin{frame} \frametitle{Données persistantes en mémoire}
Nous savons que toutes les variables locales à une fonction n'existent
plus après son appel.
\medskip

\uncover<2->{
Pour créer des \alert{données persistantes} dans une fonction,
il faut écrire dans la mémoire ailleurs que dans la pile.
\bigskip}

\uncover<3->{
On écrit pour cela dans le \alert{tas}. Cela se fait en deux temps~:}
\begin{enumerate}
    \uncover<4->{
    \item on demande au système d'\alert{allouer} (de réserver) une
    certaine portion du tas~;}
    \uncover<5->{
    \item on écrit ensuite dans cette zone en lui affectant la valeur
    souhaitée.}
\end{enumerate}
\bigskip

\uncover<6->{
Pour allouer une portion du tas, on utilise la fonction
\begin{center}
    \Code{void *malloc(size\_t size);}
\end{center}}

\uncover<7->{
Elle renvoie un \alert{pointeur de type générique} \Code{void *} sur une
donnée en mémoire de taille \Code{size} octets.}
\end{frame}

%%%%%%%%%%%%%%%%%%%%%%%%%%%%%%%%%%%%%%%%%%%%%%%%%%%%%%%%%%%%%%%%%%%%%%%%
\begin{frame} \frametitle{Utilisation de {\tt malloc}}
Pour \alert{allouer} une zone de la mémoire pouvant accueillir \Code{N}
valeurs d'un type \Code{T}, on utilise l'instruction
\begin{center}
    \Code{ptr = (T *) malloc(sizeof(T) * N);}
\end{center}
où \Code{ptr} est un pointeur pointant sur une zone de la mémoire de type
\Code{T}.
\bigskip

\uncover<2->{
Explications~:
\begin{itemize}
    \item \Code{(T *)} sert à préciser que la zone de la mémoire à allouer
    est de type \Code{T}. Cette précision est nécessaire car, par défaut,
    \Code{malloc} renvoie un pointeur sur une zone non typée (\Code{void *})~;
    \smallskip}

    \uncover<3->{
    \item l'argument \Code{sizeof(T) * N} permet de demander à allouer
    une zone mémoire de taille \Code{sizeof(T) * N} octets. Celle-ci
    pourra donc accueillir \Code{N} valeurs de type \Code{T}.}
\end{itemize}
\bigskip

\uncover<4->{
Après exécution de cette instruction, \Code{ptr} pointe sur le début
d'une zone de la mémoire de taille \Code{sizeof(T) * N} octets.}
\end{frame}

%%%%%%%%%%%%%%%%%%%%%%%%%%%%%%%%%%%%%%%%%%%%%%%%%%%%%%%%%%%%%%%%%%%%%%%%
\begin{frame}[fragile] \frametitle{Tests d'erreurs et {\tt malloc}}
C'est le \alert{système d'exploitation} qui, lors de l'appel à \Code{malloc}
se charge de fournir une adresse pour la zone mémoire à allouer.
\smallskip

\uncover<2->{
Il se peut que pour une raison ou une autre, il ne soit pas possible
d'allouer la zone mémoire demandée. Dans ce cas, \Code{malloc} renvoie
une valeur spéciale~: \Code{NULL}. Cette valeur vaut \Code{0} et est une
constante définie dans \Code{stdlib.h}.
\bigskip}

\uncover<3->{
Il est impératif de tester, pour toute allocation dynamique réalisée,
son bon déroulement. On procède de la manière suivante~:}
\begin{semiverbatim}\small\uncover<3->{
char *ptr;
ptr = (char *) malloc(sizeof(char) * 1);
if (ptr == NULL) exit(EXIT_FAILURE);}
\end{semiverbatim}
\uncover<4->{
De cette manière, si l'allocation s'est mal passée, on interrompt
immédiatement l'exécution.
\smallskip}

\uncover<5->{
\Code{void exit(int status)} est une fonction
de \Code{stdlib.h} qui permet de terminer l'exécution d'un programme.
\Code{EXIT\_FAILURE} est une constante de \Code{stdlib.h}  qui vaut \Code{1}.}
\end{frame}

%%%%%%%%%%%%%%%%%%%%%%%%%%%%%%%%%%%%%%%%%%%%%%%%%%%%%%%%%%%%%%%%%%%%%%%%
\begin{frame}[fragile] \frametitle{Libération de mémoire}
Pour \alert{désallouer} une zone de la mémoire, on utilise la fonction
\begin{center}
    \Code{void free(void *ptr);}
\end{center}
\smallskip

\uncover<2->{
On l'utilise de la manière suivante~:
\begin{center}
    \Code{free(ptr);} \\
    \Code{ptr = NULL;}
\end{center}
où \Code{ptr} est un pointeur. La 2\ieme{} ligne sert à ne plus conserver
l'accès sur la zone libéré (non indispensable mais peut éviter des erreurs).
\bigskip}

\begin{multicols}{2}
\begin{semiverbatim}\small\uncover<3->{
short *ptr;
ptr = (short *)
    malloc(sizeof(short) * 15);
free(ptr);
ptr = NULL}
\end{semiverbatim}
\uncover<3->{
La l. 2 alloue une zone de la mémoire pouvant accueillir \Code{15}
valeurs de type \Code{short}. En l. 3, cette zone
est libérée. Il est d'ores impossible d'y accéder.}
\end{multicols}

\uncover<4->{
{\bf Important}~: pour éviter les \alert{fuites mémoire}, il faut
désallouer toute zone de la mémoire qui n'est plus utilisée.}
\end{frame}

%%%%%%%%%%%%%%%%%%%%%%%%%%%%%%%%%%%%%%%%%%%%%%%%%%%%%%%%%%%%%%%%%%%%%%%%
\begin{frame}[fragile] \frametitle{La fonction {\tt calloc}}
La fonction
\begin{center}
    \Code{void *calloc(size\_t nmemb, size\_t size);}
\end{center}
alloue et renvoie un pointeur sur une zone de la mémoire de
\Code{nmemb * size} octets, tous \alert{initialisés} avec
des \Code{0}.
\bigskip

\uncover<2->{
P.ex.,}
\begin{semiverbatim}\small\uncover<2->{
long *ptr;
ptr = (long *) calloc(13, sizeof(long));}
\end{semiverbatim}
\uncover<2->{
alloue une zone de la mémoire pouvant accueillir \Code{13} valeurs de
type \Code{long}, initialisées à \Code{0}.}
\end{frame}

%%%%%%%%%%%%%%%%%%%%%%%%%%%%%%%%%%%%%%%%%%%%%%%%%%%%%%%%%%%%%%%%%%%%%%%%
\begin{frame}[fragile] \frametitle{Récapitulatif}
Dans cette sous-partie, nous avons vu~:

\begin{enumerate}
    \item la notion de donnée persistante en mémoire~;
    \smallskip

    \item le principe de l'allocation dynamique~;
    \smallskip

    \item le pointeur générique \Code{void *}~;
    \smallskip

    \item la fonction \Code{malloc}~;
    \smallskip

    \item la fonction \Code{calloc}~;
    \smallskip

    \item la fonction \Code{exit} et le test d'erreur de demande
    d'allocation~;
    \smallskip

    \item la fonction \Code{free}.
\end{enumerate}
\end{frame}

%%%%%%%%%%%%%%%%%%%%%%%%%%%%%%%%%%%%%%%%%%%%%%%%%%%%%%%%%%%%%%%%%%%%%%%%
%%%%%%%%%%%%%%%%%%%%%%%%%%%%%%%%%%%%%%%%%%%%%%%%%%%%%%%%%%%%%%%%%%%%%%%%
\subsection{Tableaux dynamiques}

%%%%%%%%%%%%%%%%%%%%%%%%%%%%%%%%%%%%%%%%%%%%%%%%%%%%%%%%%%%%%%%%%%%%%%%%
\begin{frame}[fragile] \frametitle{Tableaux dynamiques}
Un \alert{tableau dynamique} de \Code{N} valeurs de type \Code{T} est
un pointeur pointant vers une zone de la mémoire de \Code{sizeof(T) * N}
octets.
\medskip

\uncover<2->{
P.ex.,}
\begin{semiverbatim}\small\uncover<2->{
int *tab;
tab = (int *) malloc(sizeof(int) * 66);}
\end{semiverbatim}
\uncover<2->{
déclare un tableau dynamique de valeurs de type \Code{int} de taille
\Code{66}.}
\bigskip

\uncover<3->{
On lit et écrit dans un tableau dynamique de la même manière que dans
un tableau statique. En effet, \Code{tab} pointe vers la première case
du tableau, et pour tout \Code{0} $\leq$ \Code{i} $<$ \Code{66},
\Code{(tab + i)} pointe vers la case d'indice \Code{i}.}
\medskip

\uncover<4->{
La fonction \Code{free} vue précédemment permet de désallouer un tableau.
On utilise donc}
\begin{semiverbatim}\small\uncover<4->{
free(tab);}
\end{semiverbatim}
\uncover<4->{
pour libérer la zone mémoire occupée par \Code{tab} (c.-à-d. les \Code{66}
entiers situés à partir de l'adresse \Code{tab}).}
\end{frame}

%%%%%%%%%%%%%%%%%%%%%%%%%%%%%%%%%%%%%%%%%%%%%%%%%%%%%%%%%%%%%%%%%%%%%%%%
\begin{frame}[fragile]
    \frametitle{Création d'un tableau dynamique à deux dimensions}
Un \alert{tableau à deux dimensions} est un tableau dont chaque case est
elle-même un tableau. \uncover<2->{Un tableau dynamique à deux dimensions
est donc un pointeur sur un pointeur.} \uncover<3->{Ceci se généralise
dans le cas des tableaux à plus de deux dimensions.}
\medskip

\uncover<4->{
Les instructions}
\begin{semiverbatim}\small\uncover<4->{
int i, j, **tab;
tab = (int **) malloc(sizeof(int *) * 12);
if (tab == NULL) exit(EXIT_FAILURE);
for (i = 0 ; i < 12 ; ++i) \{
    tab[i] = (int *) malloc(sizeof(int) * 25);
    if (tab[i] == NULL) exit(EXIT_FAILURE);
    for (j = 0 ; j < 25 ; ++j)
        tab[i][j] = 0;
\}}
\end{semiverbatim}
\uncover<4->{
permettent de créer un tableau dynamique à deux dimensions de taille \\
\Code{12} $\times$ \Code{25} de valeurs de type \Code{int} initialisées
à \Code{0}.}

\uncover<5->{
\begin{center}
    \todo{Dessiner la mémoire au fur et à mesure de la création.}
\end{center}}
\end{frame}

%%%%%%%%%%%%%%%%%%%%%%%%%%%%%%%%%%%%%%%%%%%%%%%%%%%%%%%%%%%%%%%%%%%%%%%%
\begin{frame}[fragile]
    \frametitle{Destruction d'un tableau dynamique à deux dimensions}
Pour libérer l'espace occupé par un tableau à deux dimensions, on utilise
plusieurs fois la fonction \Code{free}.
\bigskip

\uncover<2->{
Les instructions}
\begin{semiverbatim}\small\uncover<2->{
for (i = 0 ; i < 12 ; ++i) \{
    free(tab[i]);
    tab[i] = NULL;
\}
free(tab);
tab = NULL;}
\end{semiverbatim}
\uncover<2->{
permettent de libérer l'espace mémoire occupé par un tableau \Code{tab}
à deux dimensions de taille \Code{12} $\times$ \Code{N} de valeurs
de type \Code{T} où \Code{N} est un entier strictement positif
quelconque et \Code{T} est un type.}

\uncover<3->{
\begin{center}
    \todo{Dessiner la mémoire au fur et à mesure de la destruction.}
\end{center}}
\end{frame}

%%%%%%%%%%%%%%%%%%%%%%%%%%%%%%%%%%%%%%%%%%%%%%%%%%%%%%%%%%%%%%%%%%%%%%%%
\begin{frame}[fragile] \frametitle{La fonction {\tt realloc}}
Il est possible de \alert{modifier la taille} d'un tableau dynamique via
la fonction
\begin{center}
    \Code{void *realloc(void *ptr, size\_t size);}
\end{center}

\uncover<2->{
Celle-ci renvoie un pointeur sur une zone de la mémoire de taille
\Code{size} octets. Le contenu de cette zone mémoire est le même
que celui de la zone pointée par \Code{ptr}.
\medskip}

\uncover<3->{
{\bf Attention}~: l'adresse de la zone de la mémoire réallouée peut
être différente de son adresse d'origine.} \uncover<4->{En effet,}
\begin{semiverbatim}\small\uncover<4->{
int *tab;
tab = (int *) malloc(sizeof(int) * 150);
printf("%p\\n", tab);
tab = (int *) realloc(tab, sizeof(int) * 250000);
printf("%p\\n", tab);}
\end{semiverbatim}
\uncover<4->{
affiche \\
\Sortie{0x1205010 \\ 0x7fb123f9d010}.}
\end{frame}

%%%%%%%%%%%%%%%%%%%%%%%%%%%%%%%%%%%%%%%%%%%%%%%%%%%%%%%%%%%%%%%%%%%%%%%%
\begin{frame}[fragile] \frametitle{Exemple d'utilisation de {\tt realloc}}
Construction d'une chaîne de caractères, caractère par caractère, dans
un tableau dynamique. La lecture s'arrête sur lecture de \Code{'!'}.
\medskip

\begin{semiverbatim}\footnotesize
\uncover<2->{char car, *chaine;}
\uncover<2->{int t_max, t_reelle;}
\uncover<3->{t_reelle = 0;}
\uncover<4->{t_max = 2;}
\uncover<5->{chaine = (char *) malloc(sizeof(char) * t_max); /* Allocation initiale. */}
\uncover<6->{scanf(" %c", &car);}
\uncover<7->{while (car != '!') \{}
    \uncover<8->{if (t_reelle + 1 >= t_max) \{}
        \uncover<9->{t_max *= 2;}
        \uncover<10->{chaine = (char *) realloc(chaine, t_max); /* Augmentation. */}
    \uncover<8->{\}}
    \uncover<11->{chaine[t_reelle] = car;}
    \uncover<12->{t_reelle += 1;}
    \uncover<13->{scanf(" %c", &car);}
\uncover<7->{\}}
\uncover<14->{chaine[t_reelle] = '\\0';}
\uncover<15->{chaine = (char *) realloc(chaine, t_reelle + 1); /* Diminution. */}
\uncover<16->{printf("%s\\n", chaine);}
\end{semiverbatim}
\end{frame}

%%%%%%%%%%%%%%%%%%%%%%%%%%%%%%%%%%%%%%%%%%%%%%%%%%%%%%%%%%%%%%%%%%%%%%%%
\begin{frame}[fragile] \frametitle{Récapitulatif}
Dans cette sous-partie, nous avons vu~:

\begin{enumerate}
    \item la notion de tableau dynamique~;
    \smallskip

    \item l'allocation de tableaux dynamiques de dimension un, deux et au
    delà~;
    \smallskip

    \item la libération de tableaux dynamiques de dimension un, deux et
    au delà~;
    \smallskip

    \item la fonction \Code{realloc} pour modifier la taille d'une
    zone de mémoire allouée.
\end{enumerate}
\end{frame}

% Auteur : Samuele Giraudo
% Création : aout 2014
% Modifications : oct. 2014

%%%%%%%%%%%%%%%%%%%%%%%%%%%%%%%%%%%%%%%%%%%%%%%%%%%%%%%%%%%%%%%%%%%%%%%%
%%%%%%%%%%%%%%%%%%%%%%%%%%%%%%%%%%%%%%%%%%%%%%%%%%%%%%%%%%%%%%%%%%%%%%%%
%%%%%%%%%%%%%%%%%%%%%%%%%%%%%%%%%%%%%%%%%%%%%%%%%%%%%%%%%%%%%%%%%%%%%%%%
\section{Types}

%%%%%%%%%%%%%%%%%%%%%%%%%%%%%%%%%%%%%%%%%%%%%%%%%%%%%%%%%%%%%%%%%%%%%%%%
%%%%%%%%%%%%%%%%%%%%%%%%%%%%%%%%%%%%%%%%%%%%%%%%%%%%%%%%%%%%%%%%%%%%%%%%
\subsection{Notion de type}

%%%%%%%%%%%%%%%%%%%%%%%%%%%%%%%%%%%%%%%%%%%%%%%%%%%%%%%%%%%%%%%%%%%%%%%%
\begin{frame}\frametitle{Types}
Un \alert{type} peut être vu comme un ensemble (fini ou infini) de valeurs.
\bigskip

\uncover<2->{
Dire qu'une variable \Code{x} est de type \Code{T} signifie que
la valeur de \Code{x} est dans \Code{T}.
\bigskip}

\uncover<3->{
Il existe deux sortes de types~:

\begin{enumerate}
    \item les \alert{types scalaires}, qui sont des types atomiques et
    définis à l'avance dans le langage~;
    \smallskip}

    \uncover<4->{
    \item les \alert{types composites}, qui sont des assemblages de
    types scalaires ou de types composites par le biais des constructions
    \Code{struct}, \Code{enum} ou tableau.}
\end{enumerate}
\end{frame}

%%%%%%%%%%%%%%%%%%%%%%%%%%%%%%%%%%%%%%%%%%%%%%%%%%%%%%%%%%%%%%%%%%%%%%%%
\begin{frame}[fragile] \frametitle{Type d'une variable}
Le type d'une variable indique comment {\bf interpréter} la zone
mémoire qui lui est attribuée ainsi que sa {\bf taille}.
\bigskip

\uncover<2->{
L'opérateur \Code{sizeof} permet de connaître la taille en octets d'un
type. On peut aussi l'appliquer à une valeur. P.ex., \Code{sizeof(int)}
et \Code{sizeof(33)} valent \Code{4}.
\bigskip}

\begin{multicols}{2}
\begin{semiverbatim}\footnotesize\uncover<3->{
int t;
char c;

printf("%d ", sizeof(int));
printf("%d ", sizeof(t));
printf("%d ", sizeof(31));

printf("%d ", sizeof(char));
printf("%d ", sizeof(c));
printf("%d\\n", sizeof('a'));}
\end{semiverbatim}
\uncover<3->{
Ceci affiche \Sortie{4 4 4 1 1 4}.
\medskip

L'expression \Code{sizeof('a')} vaut \Code{4}. En effet, même
si \Code{'a'} est un caractère, c'est avant tout un entier.
\smallskip

La conversion est implicite.}
\end{multicols}

\uncover<4->{
Le type d'une variable est \alert{attribué à sa déclaration} et ne peut
pas être modifié.}
\end{frame}

%%%%%%%%%%%%%%%%%%%%%%%%%%%%%%%%%%%%%%%%%%%%%%%%%%%%%%%%%%%%%%%%%%%%%%%%
%%%%%%%%%%%%%%%%%%%%%%%%%%%%%%%%%%%%%%%%%%%%%%%%%%%%%%%%%%%%%%%%%%%%%%%%
\subsection{Types scalaires}

%%%%%%%%%%%%%%%%%%%%%%%%%%%%%%%%%%%%%%%%%%%%%%%%%%%%%%%%%%%%%%%%%%%%%%%%
\begin{frame} \frametitle{Types entier}
On se place sur une machine $64$ bits.
\medskip

\begin{center}
    \begin{tabular}{c|c|c}
        Nom & Taille (octets) & Plage \\ \hline
        \Code{char} & $1$ & $-128$ à $127$ \\
        \Code{short} & $2$ & $-32768$ à $32767$ \\
        \Code{int} & $4$ & $-2^{31}$ à $2^{31} - 1$ \\
        \Code{long} & $8$ & $-2^{63}$ à $2^{63} - 1$
    \end{tabular}
\end{center}
\medskip

\uncover<2->{
Chacun de ces types peut être précédé de \Code{unsigned} pour faire en
sorte de ne représenter que des entiers positifs. On a ainsi les plages suivantes~:
\begin{center}
    \begin{tabular}{c|c}
        Nom & Plage \\ \hline
        \Code{unsigned char} & $0$ à $255$ \\
        \Code{unsigned short} & $0$ à $65535$ \\
        \Code{unsigned int} & $0$ à $2^{32} - 1$ \\
        \Code{unsigned long} & $0$ à $2^{64} - 1$
    \end{tabular}
\end{center}}
\end{frame}

%%%%%%%%%%%%%%%%%%%%%%%%%%%%%%%%%%%%%%%%%%%%%%%%%%%%%%%%%%%%%%%%%%%%%%%%
\begin{frame}[fragile] \frametitle{Entiers non signés}
Dans le cas où l'on a besoin de représenter uniquement des valeurs
entières positives, on utilisera les versions non signées des types
entiers.
\medskip

\uncover<2->{
Quelques avantages de ce procédé~:

\begin{enumerate}
    \item possibilité de représenter des entiers plus grands~;
    \smallskip

    \item gain de lisibilité du programme.
\end{enumerate}
\bigskip
\bigskip}

\uncover<3->{
{\bf Attention}~: les instructions}
\begin{semiverbatim}\uncover<3->{
unsigned int i;
for (i = 8 ; i >= 0 ; -\,\!-i) \{
    ...
\}}
\end{semiverbatim}
\uncover<3->{
produisent une boucle infinie. En effet, \Code{i} étant non signé, il
est toujours positif et donc la condition \Code{i >= 0} est toujours
vraie.}
\end{frame}

%%%%%%%%%%%%%%%%%%%%%%%%%%%%%%%%%%%%%%%%%%%%%%%%%%%%%%%%%%%%%%%%%%%%%%%%
\begin{frame} \frametitle{Constantes entières}
Il existe plusieurs manières d'exprimer des \alert{constantes entières}~:
\medskip

\begin{itemize}
    \item en base dix~: \Code{0}, \Code{29}, \Code{-322}, \dots
    \medskip

    \uncover<2->{
    \item en octal~: \Code{01}, \Code{0145}, \Code{-01234567}, \dots
    \medskip}

    \uncover<3->{
    \item en hexadécimal~: \Code{0x1}, \Code{0x5555FFFF},
    \Code{-0x98879AFA}, \dots
    \medskip}

    \uncover<4->{
    \item par un caractère~: \Code{'a'}, \Code{'9'}, \Code{'*'},
    \Code{'\textbackslash n'}, \dots}
\end{itemize}
\bigskip

\uncover<5->{
Un entier peut être représenté par un caractère car tout
caractère est représenté par son code ASCII (qui est un entier
compris entre \Code{0} et \Code{127}).
\bigskip
\bigskip}

\uncover<6->{
{\bf Attention}~: ne pas confondre les caractères chiffres avec les
entiers (l'entier \Code{'1'} vaut \Code{49} et non pas \Code{1}).}
\end{frame}

%%%%%%%%%%%%%%%%%%%%%%%%%%%%%%%%%%%%%%%%%%%%%%%%%%%%%%%%%%%%%%%%%%%%%%%%
\begin{frame} \frametitle{Types flottant}
On se place sur une machine $64$ bits.
\medskip

\begin{center}
    \begin{tabular}{c|c|c}
        Nom & Taille (octets) & Valeur absolue maximale \\ \hline
        \Code{float} & $4$ & $3.40282 \times 10^{38}$ \\
        \Code{double} & $8$ & $1.79769 \times 10^{308}$ \\
        \Code{long double} & $16$ & $1.18973 \times 10^{4932}$ \\
    \end{tabular}
\end{center}
\medskip

Le fichier d'en-tête \Code{float.h} contient des constantes donnant d'autres
renseignements sur les types flottant.
\end{frame}

%%%%%%%%%%%%%%%%%%%%%%%%%%%%%%%%%%%%%%%%%%%%%%%%%%%%%%%%%%%%%%%%%%%%%%%%
\begin{frame}[fragile] \frametitle{Danger des types flottant}
\begin{multicols}{2}
\begin{semiverbatim}
float x = 10000001.0;
printf("%f\\n", x);
\end{semiverbatim}
Ces instructions affichent, de manière attendue, \Sortie{10000001.000000}.
\end{multicols}
\medskip

\begin{multicols}{2}
\begin{semiverbatim}\uncover<2->{
float x = 100000001.0;
printf("%f\\n", x);}
\end{semiverbatim}
\uncover<2->{
En revanche, ces instructions affichent, de manière inattendue,
\Sortie{100000000.000000}.}
\end{multicols}
\bigskip

\uncover<3->{
Les nombres flottants sont représentés de manière \alert{approchée}.
\medskip

Comme ces exemples le montrent, même certains entiers, représentables
de manière exacte par des types entier, ne le sont pas par des types
flottant.}
\end{frame}

%%%%%%%%%%%%%%%%%%%%%%%%%%%%%%%%%%%%%%%%%%%%%%%%%%%%%%%%%%%%%%%%%%%%%%%%
\begin{frame} \frametitle{Danger des types flottant~: une solution partielle}
Les types flottant présentent divers désavantages par rapport aux types
entier~:
\begin{enumerate}
    \item \alert{représentation non exacte} des nombres~;
    \item opérations arithmétiques beaucoup \alert{moins efficaces}.
\end{enumerate}
\bigskip

\uncover<2->{
Pour ces raisons, {\bf il est recommandé de ne jamais utiliser de types
flottant}.
\bigskip}

\uncover<3->{
{\bf Solution partielle}~: on représente par l'entier $x \times 10^{k}$
tout nombre $x$ qui dispose de $k \geq 0$ chiffres (en base dix) après
la virgule, $k$ étant fixé.
\medskip}

\uncover<4->{
P.ex., si l'on a besoin de manipuler des nombres à $k := 2$ chiffres
après la virgule, les nombres $0.15$ et $331.9$ sont respectivement
représentés par les entiers $15$ et $33190$.}
\end{frame}

%%%%%%%%%%%%%%%%%%%%%%%%%%%%%%%%%%%%%%%%%%%%%%%%%%%%%%%%%%%%%%%%%%%%%%%%
\begin{frame}\frametitle{Opérations sur les types scalaires}
Les valeurs d'un type scalaire (entier ou flottant) forment un
\alert{ensemble totalement ordonné}~: étant donné deux valeurs, il est
toujours possible de les comparer. On utilise pour cela les
\alert{opérateurs relationnels}
\begin{center}
    \Code{==}, \Code{!=}, \Code{<=}, \Code{>=}, \Code{<}, \Code{>}.
\end{center}

\uncover<2->{
Il est possible de mélanger des comparaisons de valeurs de types entier
et de types flottant. Dans ce cas, les entiers sont convertis
implicitement en une valeur de type flottant avant d'effectuer
la comparaison.
\bigskip}

\uncover<3->{
Sur des variables de type scalaires sont définis les
\alert{opérateurs arithmétiques}
\begin{center}
    \Code{+}, \Code{-}, \Code{*}, \Code{/}, \Code{++}, \Code{-\,\!-}.
\end{center}
Les opérateurs \Code{++} et \Code{-\,\!-} servent à additionner ou à
retrancher de $1$ la valeur des variables sur lesquels ils sont appliqués.}
\end{frame}

%%%%%%%%%%%%%%%%%%%%%%%%%%%%%%%%%%%%%%%%%%%%%%%%%%%%%%%%%%%%%%%%%%%%%%%%
\begin{frame}[fragile] \frametitle{Récapitulatif}
Dans cette sous-partie, nous avons vu~:

\begin{enumerate}
    \item les types scalaires~;
    \smallskip

    \item les entiers non signés~;
    \smallskip

    \item les types flottant et leurs pièges~;
    \smallskip

    \item des opérateurs relationnels et arithmétiques sur les types
    scalaires.
\end{enumerate}
\end{frame}

%%%%%%%%%%%%%%%%%%%%%%%%%%%%%%%%%%%%%%%%%%%%%%%%%%%%%%%%%%%%%%%%%%%%%%%%
%%%%%%%%%%%%%%%%%%%%%%%%%%%%%%%%%%%%%%%%%%%%%%%%%%%%%%%%%%%%%%%%%%%%%%%%
\subsection{Types construits}

%%%%%%%%%%%%%%%%%%%%%%%%%%%%%%%%%%%%%%%%%%%%%%%%%%%%%%%%%%%%%%%%%%%%%%%%
\begin{frame}[fragile] \frametitle{Types structurés}
La syntaxe
\smallskip

\Code{typedef struct ALIAS $\lbrace$} \\
\qquad \Code{TYPE\_1 CHAMP\_1;} \\
\qquad \Code{TYPE\_2 CHAMP\_2;} \\
\qquad \Code{\dots} \\
\Code{$\rbrace$ NOM;}
\smallskip

permet de \alert{déclarer un type structuré} \Code{NOM}, constitué
des {\bf champs} \Code{CHAMP\_1}, \Code{CHAMP\_2}, \dots.
L'{\bf alias} \Code{ALIAS} est facultatif.
\bigskip

\uncover<2->{
C'est un {\bf amalgame} de types.
\bigskip}

\uncover<3->{
P.ex., \vspace{-.5em}}
\begin{multicols}{2}
\begin{semiverbatim}\uncover<3->{
typedef struct \{
    int x;
    int y;
\} Couple;}
\end{semiverbatim}
\uncover<3->{
déclare un type structuré \Code{Couple} qui permet de représenter des
couples d'entiers.}
\end{multicols}
\end{frame}

%%%%%%%%%%%%%%%%%%%%%%%%%%%%%%%%%%%%%%%%%%%%%%%%%%%%%%%%%%%%%%%%%%%%%%%%
\begin{frame}[fragile] \frametitle{Types structurés}
Si \Code{x} est une variable d'un type structuré \Code{T} contenant
le champ \Code{ch}, on {\bf accède} à ce champ par la syntaxe
\begin{center}
    \Code{x.ch}
\end{center}
\medskip

\uncover<2->{
Si \Code{adr\_x} est une adresse sur une variable de type \Code{T},
on accède à ce même champ par la syntaxe
\begin{center}
    \Code{adr\_x->ch}
\end{center}}
\uncover<3->{
Cette syntaxe est un raccourci pour
\begin{center}
    \Code{(*adr\_x).ch}
\end{center}
\bigskip}

\uncover<4->{
P.ex., les trois suites d'instructions suivantes sont équivalentes~:
\vspace{-.5em}}
\begin{multicols}{3}
\begin{semiverbatim}\uncover<4->{
Couple *c;
...
c->x = c->x + 1;}
\end{semiverbatim}

\begin{semiverbatim}\uncover<4->{
Couple *c;
...
*(c).x = c->x + 1;}
\end{semiverbatim}

\begin{semiverbatim}\uncover<4->{
Couple *c;
...
c->x = (*c).x + 1;}
\end{semiverbatim}
\end{multicols}
\end{frame}

%%%%%%%%%%%%%%%%%%%%%%%%%%%%%%%%%%%%%%%%%%%%%%%%%%%%%%%%%%%%%%%%%%%%%%%%
\begin{frame}[fragile] \frametitle{Opérations sur les types structurés}
Les {\bf opérateurs relationnels} ne sont pas définis sur les types
structurés.
\medskip

Il est donc impossible de tester si deux variables d'un même type
structuré sont égales au moyen de l'opérateur \Code{==}. Il faut
tester l'égalité de chacun des champs qui les constituent.
\bigskip

\uncover<2->{
En revanche, l'{\bf opérateur d'affectation} \Code{=} est compatible avec
les types structurés.}

\begin{multicols}{2}
\begin{semiverbatim}\footnotesize\uncover<3->{
typedef struct \{
    char nom[32];
    char prenom[32];
    int age;
\} Personne;
...
Personne p1, p2;
scanf(" %s", p1.nom);
scanf(" %s", p1.prenom);
p1.age = 30;
p2 = p1;}
\end{semiverbatim}
\uncover<3->{
L'affectation en dernière ligne fait en sorte que tous les champs de
\Code{p2} contiennent les mêmes valeurs que ceux de \Code{p1}.
\bigskip}

\uncover<4->{
Il y a recopie des tableaux statiques \Code{p1.nom} et
\Code{p1.prenom} dans \Code{p2.nom} et \Code{p2.prenom}.}
\end{multicols}
\end{frame}

%%%%%%%%%%%%%%%%%%%%%%%%%%%%%%%%%%%%%%%%%%%%%%%%%%%%%%%%%%%%%%%%%%%%%%%%
\begin{frame}[fragile] \frametitle{Types énumérés}
La syntaxe
\smallskip

\Code{typedef enum $\lbrace$} \\
\qquad \Code{ENU\_1,} \\
\qquad \Code{ENU\_2,} \\
\qquad \Code{\dots} \\
\Code{$\rbrace$ NOM;}
\smallskip

permet de \alert{déclarer un type énuméré} \Code{NOM}, constitué
des {\bf énumérateurs} \Code{ENU\_1}, \Code{ENU\_2}, \dots.
(Attention, on utilise des \Code{,} et non pas des \Code{;}.)
\bigskip

\uncover<2->{
Une valeur de ce type prend pour valeur exactement un des énumérateurs
qui le constituent.
\bigskip}

\uncover<3->{P.ex.,\vspace{-.5em}}
\begin{multicols}{2}
\begin{semiverbatim}\uncover<3->{
typedef enum \{
    FAUX,
    VRAI
\} Booleen;}
\end{semiverbatim}
\uncover<3->{
est un type qui permet de représenter des booléens.
\smallskip

Une valeur de type \Code{Booleen} est soit \Code{FAUX}, soit \Code{VRAI}.}
\end{multicols}
\end{frame}

%%%%%%%%%%%%%%%%%%%%%%%%%%%%%%%%%%%%%%%%%%%%%%%%%%%%%%%%%%%%%%%%%%%%%%%%
\begin{frame}[fragile] \frametitle{Types énumérés}
\begin{multicols}{2}
\begin{semiverbatim}\footnotesize
typedef enum \{
    LUNDI,    /* = 0 */
    MARDI,    /* = 1 */
    MERCREDI, /* = 2 */
    JEUDI,    /* = 3 */
    VENDREDI, /* = 4 */
    SAMEDI,   /* = 5 */
    DIMANCHE  /* = 6 */
\} Jour;
...
printf("%d\\n", MERCREDI);
\end{semiverbatim}
Les énumérateurs sont des {\bf expressions entières}. Leur valeur est
déterminée par leur ordre de déclaration dans le type.
\smallskip

Ces instructions affichent \Sortie{2}.
En effet, \Code{LUNDI} vaut \Code{0} car il est le 1\ier{} énumérateur
déclaré et les valeurs des suivants s'incrémentent selon leur
ordre de déclaration.
\end{multicols}

\begin{multicols}{2}
\begin{semiverbatim}\footnotesize\uncover<2->{
typedef enum \{
    LA = 0,
    SI = 2,
    DO,       /* = 3 */
    RE = 5,
    MI = 7,
    FA = 8,
    SOL = 10
\} Note;}
\end{semiverbatim}\small
\uncover<2->{
Il est possible de spécifier manuellement les valeurs
des énumérateurs avec la syntaxe \Code{ENU = VAL} où \Code{ENU}
est un énumérateur et \Code{VAL} une constante entière.
\smallskip

Si une valeur n'est pas spécifiée, elle est déduite de la précédente
en l'incrémentant.}
\end{multicols}
\end{frame}

%%%%%%%%%%%%%%%%%%%%%%%%%%%%%%%%%%%%%%%%%%%%%%%%%%%%%%%%%%%%%%%%%%%%%%%%
\begin{frame}[fragile] \frametitle{Types énumérés}
L'utilisation de branchement \Code{switch} est particulièrement
adaptée pour traiter une variable d'un type énuméré.
\medskip

\begin{multicols}{2}
\begin{semiverbatim}\small\uncover<2->{
Note note;

scanf(" %d", &note);

switch (note) \{
  case LA : printf("A"); break;
  case SI : printf("B"); break;
  case DO : printf("C"); break;
  case RE : printf("D"); break;
  case MI : printf("E"); break;
  case FA : printf("F"); break;
  case SOL : printf("G");break;
  default : printf(
    "%d non note", note);
\}}
\end{semiverbatim}
\vspace{1em}

\uncover<2->{
Ces instructions lisent une valeur entière sur l'entrée standard
et l'affichent (en notation internationale).
\smallskip}

\uncover<3->{
{\bf Remarque}~: une variable d'un type énuméré peut prendre comme
valeur n'importe quel entier. Ceci explique la présence de la
clause \Code{default}.
\smallskip}

\uncover<4->{
L'intérêt de l'utilisation des types énumérés est principalement
{\bf sémantique}~: un programme qui les utilise est plus facile
à lire et à maintenir.}
\end{multicols}
\end{frame}

%%%%%%%%%%%%%%%%%%%%%%%%%%%%%%%%%%%%%%%%%%%%%%%%%%%%%%%%%%%%%%%%%%%%%%%%
\begin{frame}[fragile] \frametitle{Opérations sur les types énumérés}
Contrairement aux types structurés, il est possible de comparer les
éléments d'un type énuméré au moyen des {\bf opérateurs relationnels}.
\smallskip

Ceci est une conséquence du fait que les énumérateurs sont des entiers.

\begin{multicols}{2}
\begin{semiverbatim}
printf("%d\\n", SOL == SOL);
printf("%d\\n", SOL == FA);
printf("%d\\n", SI <= RE);
\end{semiverbatim}
affiche \Sortie{1} \\
affiche \Sortie{0} \\
affiche \Sortie{1}
\end{multicols}
\medskip

\uncover<2->{
De même, l'{\bf opérateur d'affectation} \Code{=} est compatible
avec les types énumérés.
\bigskip}

\uncover<3->{
L'{\bf opérateur de taille} \Code{sizeof} renvoie \Code{4} sur les
valeurs d'un type énuméré. C'est la taille occupée par le type \Code{int}.}
\end{frame}

%%%%%%%%%%%%%%%%%%%%%%%%%%%%%%%%%%%%%%%%%%%%%%%%%%%%%%%%%%%%%%%%%%%%%%%%
\begin{frame}[fragile] \frametitle{Récapitulatif}
Dans cette sous-partie, nous avons vu~:

\begin{enumerate}
    \item les types structurés~;
    \smallskip

    \item le comportement des variables d'un type structuré face
    aux opérateurs relationnels et à l'opérateur d'affectation~;
    \smallskip

    \item les types énumérés~;
    \smallskip

    \item l'utilisation combinée des types énumérés avec \Code{switch}~;
    \smallskip

    \item le comportement des variables d'un type énuméré face
    aux opérateurs relationnels et à l'opérateur d'affectation.
\end{enumerate}
\end{frame}

% Auteur : Samuele Giraudo
% Création : jan. 2014, jan. 2015, fév 2015, déc. 2015, mars 2016

\tikzstyle{Bloc}=[rectangle,draw=Vert!100,fill=Vert!20,
    line width=1pt,minimum width=1cm,minimum height=0.75cm]
\tikzstyle{BlocM}=[Bloc,draw=Marron=!100,fill=Marron!10]

%%%%%%%%%%%%%%%%%%%%%%%%%%%%%%%%%%%%%%%%%%%%%%%%%%%%%%%%%%%%%%%%%%%%%%%%
%%%%%%%%%%%%%%%%%%%%%%%%%%%%%%%%%%%%%%%%%%%%%%%%%%%%%%%%%%%%%%%%%%%%%%%%
%%%%%%%%%%%%%%%%%%%%%%%%%%%%%%%%%%%%%%%%%%%%%%%%%%%%%%%%%%%%%%%%%%%%%%%%
\section{Types structurés}

%%%%%%%%%%%%%%%%%%%%%%%%%%%%%%%%%%%%%%%%%%%%%%%%%%%%%%%%%%%%%%%%%%%%%%%%
%%%%%%%%%%%%%%%%%%%%%%%%%%%%%%%%%%%%%%%%%%%%%%%%%%%%%%%%%%%%%%%%%%%%%%%%
\subsection{Déclaration et initialisation}

%%%%%%%%%%%%%%%%%%%%%%%%%%%%%%%%%%%%%%%%%%%%%%%%%%%%%%%%%%%%%%%%%%%%%%%%
\begin{frame}[fragile]
\frametitle{Déclaration de types structurés récursifs}
Il est possible de \alert{déclarer des types structurés récursifs} en
faisant usage de l'{\bf alias} et du mot clé \Code{struct}~:
\medskip

\begin{lstlisting}
typedef struct _Liste {
    int e;
    struct _Liste *s;
} Liste;
\end{lstlisting}
\medskip

Ceci fonctionne car la taille d'un pointeur vers une valeur de type
\Code{T} est connue et indépendante de la nature de \Code{T}.
\bigskip

Attention, la déclaration
\begin{lstlisting}
typedef struct _Liste {
    int e;
    struct _Liste s;
} Liste;
\end{lstlisting}
n'est pas valide car le {\bf champ récursif} n'est pas un {\bf pointeur}.
\medskip

Le système ne peut pas connaître pas la taille de ce champ.
\end{frame}

%%%%%%%%%%%%%%%%%%%%%%%%%%%%%%%%%%%%%%%%%%%%%%%%%%%%%%%%%%%%%%%%%%%%%%%%
\begin{frame}[fragile]
\frametitle{Déclaration de types structurés mutuellement récursifs}
Il est possible de \alert{déclarer des types structurés mutuellement récursifs}~:
\begin{lstlisting}
typedef struct _Flip {
    int a;
    int b;
    struct _Flop *suiv;
} Flip;

typedef struct _Flop {
    double x;
    struct _Flip *suiv;
} Flop;
\end{lstlisting}

\begin{center}
    \todo{Dessiner un exemple de variable de type \Code{Flip}.}
\end{center}
\end{frame}

%%%%%%%%%%%%%%%%%%%%%%%%%%%%%%%%%%%%%%%%%%%%%%%%%%%%%%%%%%%%%%%%%%%%%%%%
\begin{frame}[fragile]
\frametitle{Initialisation d'une variable d'un type structuré}
Il est possible d'\alert{initialiser les champs} d'une variable d'un type
structuré au moment de sa \alert{déclaration}.
\medskip

On utilise pour cela l'opérateur d'affectation \Code{=} avec comme valeur
droite les valeurs des champs à affecter dans des accolades et séparées
par des virgules.
\medskip

Par exemple,
\begin{multicols}{2}
\begin{lstlisting}
typedef struct {
    char c;
    int a;
    double b;
} Triplet;
...
Triplet tr = {'h', 55, 214.35};
\end{lstlisting}
\bigskip

Déclare, en l'initialisant, la variable \Code{tr}.
\begin{center}
    \todo{Dessiner le contenu de \Code{tr}.}
\end{center}
\bigskip
\bigskip
\end{multicols}
\end{frame}

%%%%%%%%%%%%%%%%%%%%%%%%%%%%%%%%%%%%%%%%%%%%%%%%%%%%%%%%%%%%%%%%%%%%%%%%
%%%%%%%%%%%%%%%%%%%%%%%%%%%%%%%%%%%%%%%%%%%%%%%%%%%%%%%%%%%%%%%%%%%%%%%%
\subsection{Dans les fonctions}

%%%%%%%%%%%%%%%%%%%%%%%%%%%%%%%%%%%%%%%%%%%%%%%%%%%%%%%%%%%%%%%%%%%%%%%%
\begin{frame}[fragile]
\frametitle{Renvoi d'une variable d'un type structuré}
Le code
\begin{multicols}{2}
\begin{lstlisting}
typedef struct {
    int x;
    int y;
} Couple;


Couple twist(Couple c) {
    Couple res;
    res.x = c.y;
    res.y = c.x;
    return res;
}
\end{lstlisting}
\end{multicols}
est correct (\Code{twist} renvoie le couple obtenu par échange des
coordonnées de celui passé en argument).
\smallskip

\Code{twist} \alert{renvoie} une variable d'un \alert{type structuré}.
\medskip

Cependant, il n'est pas efficace car, à chaque appel de fonction
\begin{center}
    \Code{d = twist(c);}
\end{center}
la variable \Code{res}, qui vit dans la pile, doit être recopiée.
\end{frame}

%%%%%%%%%%%%%%%%%%%%%%%%%%%%%%%%%%%%%%%%%%%%%%%%%%%%%%%%%%%%%%%%%%%%%%%%
\begin{frame}[fragile]
\frametitle{Paramètre variable d'un type structuré}
Le code
\begin{multicols}{2}
\begin{lstlisting}
typedef struct {
    int tab1[2048];
    int tab2[2048];
} DeuxTab;

int prem_egaux(DeuxTab x) {
    return x.tab1[0]
        == x.tab2[0];
}
\end{lstlisting}
\end{multicols}
est correct (\Code{prem\_egaux} teste si les premières cases des tableaux
sont égales).
\smallskip

\Code{prem\_egaux} est \alert{paramétrée} par une variable
d'un \alert{type structuré}.
\medskip

Cependant, il n'est pas efficace car à chaque appel de fonction
\begin{center}
    \Code{prem\_egaux(y);}
\end{center}
les champs de l'{\bf argument} \Code{y} sont recopiés dans le
{\bf paramètre} \Code{x}.
\end{frame}

%%%%%%%%%%%%%%%%%%%%%%%%%%%%%%%%%%%%%%%%%%%%%%%%%%%%%%%%%%%%%%%%%%%%%%%%
\begin{frame}[fragile]
\frametitle{Passage par adresse {\em vs} passage par valeur}
Soit une fonction \Code{fct} paramétrée par une variable \Code{x} d'un
type structuré~\Code{T}.
\medskip

Il est d'usage courant d'adopter la convention suivante~:
\smallskip

\begin{itemize}
    \item si les champs de \Code{x} doivent être modifiés par la fonction,
    alors on recourt à un {\bf passage par adresse}
    \begin{center}
        \Code{... fct(T *x, ...) $\lbrace$ ... $\rbrace$}
    \end{center}
    \smallskip

    \item si les champs de \Code{x} ne doivent pas être modifiés par
    la fonction, alors on recourt à un {\bf passage par valeur}
    \begin{center}
        \Code{... fct(T x, ...) $\lbrace$ ... $\rbrace$}
    \end{center}
\end{itemize}
\bigskip

\alert{Cette conception est erronée} car il est possible de \og modifier \fg\,
une variable d'un type structuré passée par valeur à une fonction.
\end{frame}

%%%%%%%%%%%%%%%%%%%%%%%%%%%%%%%%%%%%%%%%%%%%%%%%%%%%%%%%%%%%%%%%%%%%%%%%
\begin{frame}[fragile]
\frametitle{Passage par adresse {\em vs} passage par valeur}
Considérons en effet le code suivant~:
\begin{multicols}{2}
\begin{lstlisting}
tyepdef struct {
    int *tab;
    int n;
} Tab;

void init(Tab t, int k) {
    int i;
    for (i = 0 ; i < t.n ; ++i)
        t.tab[i] = k;
}
\end{lstlisting}
\end{multicols}
Chaque appel de fonction
\begin{center}
    \Code{init(s, r);}
\end{center}
provoque la recopie de trois valeurs (ce qui est encore acceptable) mais
\og modifie \fg\, les valeurs pointées par le champ \Code{tab} de \Code{s},
malgré le passage par valeur.
\bigskip

{\bf Conclusion}~: écrire des fonctions avec passage par valeur des
paramètres d'un type structuré ne présente que des désavantages.
\end{frame}

%%%%%%%%%%%%%%%%%%%%%%%%%%%%%%%%%%%%%%%%%%%%%%%%%%%%%%%%%%%%%%%%%%%%%%%%
\begin{frame}[fragile]
\frametitle{Variables d'un type structuré dans les fonctions}
En résumé, on adopte les deux règles suivantes~:
\smallskip

\begin{enumerate}
    \item une fonction ne renvoie jamais de valeur d'un type structuré~;
    \smallskip

    \item tous les paramètres d'un type structuré sont passés par adresse
    dans une fonction.
\end{enumerate}
\medskip

Ainsi, un prototype de fonction habituel est
\begin{center}
    \Code{int fct(T *x,
    \textcolor{Rouge}{E1 e1}, ..., \textcolor{Rouge}{EN en},
    \textcolor{Vert}{S1 *s1}, ..., \textcolor{Vert}{SM *sm});}
\end{center}
où
\begin{itemize}
    \item le type de retour est \Code{int} (renvoi d'un code d'erreur)~;
    \smallskip

    \item \Code{x} est l'adresse d'une variable d'un type structuré \Code{T}~;
    \smallskip

    \item \textcolor{Rouge}{\tt e1}, ..., \textcolor{Rouge}{\tt en}
    sont les entrées de la fonction (adresses ou non)~;
    \smallskip

    \item \textcolor{Vert}{\tt s1}, ..., \textcolor{Vert}{\tt sm}
    sont les sorties de la fonction (qui sont des adresses).
\end{itemize}
\end{frame}

%%%%%%%%%%%%%%%%%%%%%%%%%%%%%%%%%%%%%%%%%%%%%%%%%%%%%%%%%%%%%%%%%%%%%%%%
\begin{frame}[fragile]
\frametitle{Variables d'un type structuré dans les fonctions}
P.ex., voici le nécessaire pour calculer la somme pondérée de deux
points selon les conventions établies~:
\begin{lstlisting}
typdef struct {
    float x;
    float y;
} Point;

void somme_points(const Point *p1, const Point *p2,
        float coeff1, float coeff2,
        Point *res) {

    assert(p1 != NULL);
    assert(p2 != NULL);
    assert(res != NULL);

    res->x = coeff1 * p1->x + coeff2 * p2->x;
    res->y = coeff1 * p1->y + coeff2 * p2->y;
}
\end{lstlisting}
\end{frame}

%%%%%%%%%%%%%%%%%%%%%%%%%%%%%%%%%%%%%%%%%%%%%%%%%%%%%%%%%%%%%%%%%%%%%%%%
%%%%%%%%%%%%%%%%%%%%%%%%%%%%%%%%%%%%%%%%%%%%%%%%%%%%%%%%%%%%%%%%%%%%%%%%
\subsection{Affectation et comparaison}

%%%%%%%%%%%%%%%%%%%%%%%%%%%%%%%%%%%%%%%%%%%%%%%%%%%%%%%%%%%%%%%%%%%%%%%%
\begin{frame}[fragile]
\frametitle{Affectation de variables d'un type structuré}
Considérons le code
\begin{multicols}{2}
\begin{lstlisting}
typedef struct {
    int a;
    float f;
} X;
...
X v1, v2;
v1.a = 10;
v1.f = 3.14;
v2 = v1;
v2.a = 20;
\end{lstlisting}
\end{multicols}
\medskip

\begin{center}
    \todo{Dessiner le contenu de \Code{v1} et \Code{v2} aux l. 6, 8, 9 et 10.}
\end{center}
\medskip

{\bf Observation}~: l'affectation recopie les champs d'une variable
d'un type scalaire.
\end{frame}

%%%%%%%%%%%%%%%%%%%%%%%%%%%%%%%%%%%%%%%%%%%%%%%%%%%%%%%%%%%%%%%%%%%%%%%%
\begin{frame}[fragile]
\frametitle{Affectation de variables d'un type structuré}
Considérons le code
\begin{multicols}{2}
\begin{lstlisting}
typedef struct {
    X x;
    char t[3];
} Y;
...
Y v1, v2;
v1.x.a = 10;
v1.x.f = 3.14;
v1.t = {'a', 'b', 'c'};
v2 = v1;
v2.x.f = 1.8;
v2.t[0] = 'g';
\end{lstlisting}
\end{multicols}
\medskip

\begin{center}
    \todo{Dessiner le contenu de \Code{v1} et \Code{v2} aux l. 9, 10 et 12.}
\end{center}
\medskip

{\bf Observation}~: l'affectation recopie les champs d'une variable d'un
type structuré de manière récursive et les tableaux statiques.
\end{frame}

%%%%%%%%%%%%%%%%%%%%%%%%%%%%%%%%%%%%%%%%%%%%%%%%%%%%%%%%%%%%%%%%%%%%%%%%
\begin{frame}[fragile]
\frametitle{Affectation de variables d'un type structuré}
Considérons le code
\begin{multicols}{2}
\begin{lstlisting}
typedef struct {
    char *t;
    int n;
} T;
...
T v1, v2;
v1.t = (char *)
    malloc(sizeof(char) * 3);
v1.n = 3;
v1.t[0] = 'a';
v1.t[1] = 'b';
v1.t[2] = 'c';
v2 = v1;
v2.n = 2;
v2.t[0] = 'g';
\end{lstlisting}
\end{multicols}
\medskip

\begin{center}
    \todo{Dessiner le contenu de \Code{v1} et \Code{v2} aux l. 12, 13 et 15.}
\end{center}
\medskip

{\bf Observation}~: l'affectation ne recopie pas les tableaux dynamiques.
Seule l'adresse d'un tableau dynamique est recopiée. C'est une
\alert{copie de surface}.
\end{frame}

%%%%%%%%%%%%%%%%%%%%%%%%%%%%%%%%%%%%%%%%%%%%%%%%%%%%%%%%%%%%%%%%%%%%%%%%
\begin{frame}[fragile]
\frametitle{Affectation de variables d'un type structuré}
{\bf Règle générale}~: pour chaque déclaration d'un type structuré
\Code{X}, on définit une fonction de prototype
\begin{center}
    \Code{void copier\_X(const X *v1, X *v2);}
\end{center}
qui \alert{copie en profondeur} les champs de \Code{v1} dans les champs
de \Code{v2}.
\medskip

Par exemple, la définition du type \Code{T} précédent s'accompagne de la
définition de la fonction
\begin{lstlisting}
void copier_T(const T *v1, T *v2) {
    int i;
    assert(v1 != NULL);
    assert(v2 != NULL);
    v2->n = v1->n;
    v2->t = (char *) malloc(sizeof(char) * v1->n);
    if (v2->t == NULL) exit(EXIT_FAILURE);
    for (i = 0 ; i < v1->n ; ++i)
        v2->t[i] = v1->t[i];
}
\end{lstlisting}

Il est possible de munir cette fonction du mécanisme habituel de
gestion d'erreurs.
\end{frame}

%%%%%%%%%%%%%%%%%%%%%%%%%%%%%%%%%%%%%%%%%%%%%%%%%%%%%%%%%%%%%%%%%%%%%%%%
\begin{frame}[fragile]
\frametitle{Comparaison de variables d'un type structuré}
Considérons le code
\begin{multicols}{2}
\begin{lstlisting}
typedef struct {
    int a;
    int b;
} A;
...
A v1, v2;
...
if (v1 == v2) {...}
...
if (v1 != v2) {...}
\end{lstlisting}
\end{multicols}
\medskip

Ce code est incorrect (il ne compile pas).
\medskip

Le compilateur n'accepte pas la comparaison de variables d'un type
structuré.
\smallskip

\Sortie{\small invalid operands to binary == (have 'A' and 'A')}
\smallskip

\Sortie{\small invalid operands to binary != (have 'A' and 'A')}
\end{frame}

%%%%%%%%%%%%%%%%%%%%%%%%%%%%%%%%%%%%%%%%%%%%%%%%%%%%%%%%%%%%%%%%%%%%%%%%
\begin{frame}[fragile]
\frametitle{Comparaison de variables d'un type structuré}
{\bf Règle générale}~: pour chaque déclaration d'un type structuré
\Code{X}, on définit deux fonctions de prototypes
\begin{center}
    \Code{int sont\_ega\_X(const X *v1, const X *v2);} \smallskip

    \Code{int sont\_dif\_X(const X *v1, const X *v2);}
\end{center}
qui \alert{testent l'égalité} et \alert{l'inégalité} entre \Code{v1}
et \Code{v2}.
\medskip

Par exemple, la définition du type \Code{A} précédent s'accompagne de la
définition des fonctions
\begin{multicols}{2}
\begin{lstlisting}
int sont_ega_A(A *v1, A *v2) {
    assert(v1 != NULL);
    assert(v2 != NULL);
    return (v1->a == v2->a)
        && (v1->b == v2->b);
}
int sont_dif_A(A *v1, A *v2) {
    assert(v1 != NULL);
    assert(v2 != NULL);
    return !sont_ega_A(v1, v2);
}
\end{lstlisting}
\end{multicols}
\medskip

{\bf Attention}~: si \Code{X} est composé d'un champ qui est un type
structuré \Code{Y}, il faut appeler dans \Code{sont\_ega\_X} la
fonction de comparaison \Code{sont\_ega\_Y}.
\end{frame}

%%%%%%%%%%%%%%%%%%%%%%%%%%%%%%%%%%%%%%%%%%%%%%%%%%%%%%%%%%%%%%%%%%%%%%%%
\begin{frame}[fragile]
\frametitle{Destruction de variables d'un type structuré}
{\bf Règle générale}~: pour chaque déclaration d'un type structuré
\Code{X}, on définit une fonction de prototype
\begin{center}
    \Code{void detruire\_X(X *v);}
\end{center}
qui \alert{libère l'espace mémoire} adressé par \Code{v}.
\medskip

Par exemple, la déclaration du type \Code{B} suivant s'accompagne de la
définition de la fonction
\begin{multicols}{2}
\begin{lstlisting}
typedef struct {
    int *tab;
    int n;
} B;

void detruire_B(B *v) {
    assert(v != NULL);
    free(v->tab);
    *v = NULL;
}
\end{lstlisting}
\end{multicols}
\medskip

{\bf Attention}~: si \Code{X} est composé d'un champ qui est un type
structuré \Code{Y}, il faut appeler dans \Code{detruire\_X} la
fonction de comparaison \Code{detruire\_Y}.
\end{frame}

%%%%%%%%%%%%%%%%%%%%%%%%%%%%%%%%%%%%%%%%%%%%%%%%%%%%%%%%%%%%%%%%%%%%%%%%
\begin{frame}[fragile]
\frametitle{Résumé}
Voici en résumé la bonne marche à suivre lors de la manipulation de
types structurés~:
\smallskip

\begin{enumerate}
    \item on utilise l'{\bf alias} lors de la déclaration de types structurés
    {\bf récursifs} et/ou {\bf mutuellement récursifs}~;
    \bigskip

    \item on ne {\bf renvoie jamais} de valeur d'un type structuré~;
    \bigskip

    \item on passe les {\bf paramètres} d'un type structuré {\bf par adresse}~;
    \bigskip

    \item toute {\bf déclaration d'un type structuré} s'accompagne de la
    définition des quatre fonctions suivantes~:

    \begin{itemize} \normalsize
        \item une fonction de {\bf copie}~;
        \smallskip

        \item une fonction de {\bf test d'égalité}~;
        \smallskip

        \item une fonction de {\bf test d'inégalité}~;
        \smallskip

        \item une fonction de {\bf destruction}.
    \end{itemize}
\end{enumerate}
\end{frame}

%%%%%%%%%%%%%%%%%%%%%%%%%%%%%%%%%%%%%%%%%%%%%%%%%%%%%%%%%%%%%%%%%%%%%%%%
%%%%%%%%%%%%%%%%%%%%%%%%%%%%%%%%%%%%%%%%%%%%%%%%%%%%%%%%%%%%%%%%%%%%%%%%
\subsection{Alignement en mémoire}

%%%%%%%%%%%%%%%%%%%%%%%%%%%%%%%%%%%%%%%%%%%%%%%%%%%%%%%%%%%%%%%%%%%%%%%%
\begin{frame}[fragile]
\frametitle{Alignement en mémoire}
L'\alert{alignement en mémoire} d'une donnée est la façon dont celle-ci
est organisée dans la mémoire.
\bigskip

Par exemple, nous avons déjà vu que les {\bf tableaux} de taille
\Code{n} d'éléments d'un type \Code{T} sont organisés en un segment
contigu de \Code{sizeof(T) * n} octets.
\medskip

Ainsi, un tableau \Code{t} de \Code{3} éléments de type \Code{short}
est organisé en
\begin{center}
\scalebox{.6}{\begin{tikzpicture}
    \node[BlocM](7)at(-1,0){};
    \node[BlocM](8)at(-2,0){};
    \node[BlocM](9)at(-3,0){};
    \node[BlocM](10)at(6,0){};
    \node[BlocM](11)at(7,0){};
    \node at(-4,0){\Large $\dots$};
    \node at(8,0){\Large $\dots$};
    \node[Bloc](1)at(0,0){};
    \node[Bloc](2)at(1,0){};
    \node[Bloc](3)at(2,0){};
    \node[Bloc](4)at(3,0){};
    \node[Bloc](5)at(4,0){};
    \node[Bloc](6)at(5,0){};
    \draw(-0.5,0.75)edge[anchor=south,<->,line width=1.5pt]
        node{\small 1 octet}(0.5,0.75);
    \draw(-0.5,-0.75)edge[anchor=north,<->,line width=1.5pt]
        node{\small \Code{t[0]}}(1.5,-0.75);
    \draw(1.5,-0.75)edge[anchor=north,<->,line width=1.5pt]
        node{\small \Code{t[1]}}(3.5,-0.75);
    \draw(3.5,-0.75)edge[anchor=north,<->,line width=1.5pt]
        node{\small \Code{t[2]}}(5.5,-0.75);
\end{tikzpicture}}
\end{center}
\bigskip

On peut se poser de la même manière la question de l'{\bf alignement
mémoire} des variables d'un {\bf type structuré}.
\end{frame}

%%%%%%%%%%%%%%%%%%%%%%%%%%%%%%%%%%%%%%%%%%%%%%%%%%%%%%%%%%%%%%%%%%%%%%%%
\begin{frame}[fragile]
\frametitle{Alignement en mémoire des variables d'un type structuré}
Considérons les déclarations de types
\begin{multicols}{2}
\begin{lstlisting}
typedef struct {
    short x;
    short y;
    int z;
} A;
typedef struct {
    short x;
    int z;
    short y;
} B;
\end{lstlisting}
\end{multicols}
\medskip

\Code{A} et \Code{B} sont des types structurés composés des mêmes champs.
Il n'y a que l'ordre de leur déclaration qui diffère.
\medskip

Cependant,
\begin{lstlisting}
    printf("%lu %lu\n", sizeof(A), sizeof(B));
\end{lstlisting}
affiche \Sortie{8 12}.
\medskip

Le fait que les tailles des variables de type \Code{A} et \Code{B}
diffèrent est dû à leur {\bf alignement en mémoire}.
\end{frame}

%%%%%%%%%%%%%%%%%%%%%%%%%%%%%%%%%%%%%%%%%%%%%%%%%%%%%%%%%%%%%%%%%%%%%%%%
\begin{frame}[fragile]
\frametitle{Alignement en mémoire des variables d'un type structuré}
Soit \Code{a} une variable de type \Code{A}. Cette variable est organisée
en mémoire en
\begin{center}
\scalebox{.6}{\begin{tikzpicture}
    \node[BlocM](9)at(-1,0){};
    \node[BlocM](10)at(-2,0){};
    \node[BlocM](11)at(-3,0){};
    \node[BlocM](12)at(8,0){};
    \node[BlocM](13)at(9,0){};
    \node[BlocM](14)at(10,0){};
    \node[BlocM](15)at(11,0){};
    \node[BlocM](16)at(12,0){};
    \node at(-4,0){\Large $\dots$};
    \node at(13,0){\Large $\dots$};
    \node[Bloc,fill=Jaune!40,draw=Jaune!100](1)at(0,0){};
    \node[Bloc,fill=Jaune!40,draw=Jaune!100](2)at(1,0){};
    \node[Bloc,fill=Rouge!40,draw=Rouge!100](3)at(2,0){};
    \node[Bloc,fill=Rouge!40,draw=Rouge!100](4)at(3,0){};
    \node[Bloc,fill=Bleu!40,draw=Bleu!100](5)at(4,0){};
    \node[Bloc,fill=Bleu!40,draw=Bleu!100](6)at(5,0){};
    \node[Bloc,fill=Bleu!40,draw=Bleu!100](7)at(6,0){};
    \node[Bloc,fill=Bleu!40,draw=Bleu!100](8)at(7,0){};
    \draw(-0.5,0.75)edge[anchor=south,<->,line width=1.5pt]
        node{\small 1 octet}(0.5,0.75);
    \draw(-0.5,-0.75)edge[anchor=north,<->,line width=1.5pt]
        node{\small \Code{a.x}}(1.5,-0.75);
    \draw(1.5,-0.75)edge[anchor=north,<->,line width=1.5pt]
        node{\small \Code{a.y}}(3.5,-0.75);
    \draw(3.5,-0.75)edge[anchor=north,<->,line width=1.5pt]
        node{\small \Code{a.z}}(7.5,-0.75);
\end{tikzpicture}}
\end{center}
\bigskip

Soit \Code{b} une variable de type \Code{B}. Cette variable est organisée
en mémoire en
\begin{center}
\scalebox{.6}{\begin{tikzpicture}
    \node[BlocM](9)at(-1,0){};
    \node[BlocM](10)at(-2,0){};
    \node[BlocM](11)at(-3,0){};
    \node[BlocM](16)at(12,0){};
    \node at(-4,0){\Large $\dots$};
    \node at(13,0){\Large $\dots$};
    \node[Bloc,fill=Jaune!40,draw=Jaune!100](1)at(0,0){};
    \node[Bloc,fill=Jaune!40,draw=Jaune!100](2)at(1,0){};
    \node[Bloc,fill=Bleu!40,draw=Bleu!100](5)at(4,0){};
    \node[Bloc,fill=Bleu!40,draw=Bleu!100](6)at(5,0){};
    \node[Bloc,fill=Bleu!40,draw=Bleu!100](7)at(6,0){};
    \node[Bloc,fill=Bleu!40,draw=Bleu!100](8)at(7,0){};
    \node[Bloc,fill=Rouge!40,draw=Rouge!100](12)at(8,0){};
    \node[Bloc,fill=Rouge!40,draw=Rouge!100](13)at(9,0){};
    \node[BlocM,fill=Noir!50,draw=Noir](14)at(10,0){};
    \node[BlocM,fill=Noir!50,draw=Noir](15)at(11,0){};
    \node[BlocM,fill=Noir!50,draw=Noir](3)at(2,0){};
    \node[BlocM,fill=Noir!50,draw=Noir](4)at(3,0){};
    \draw(-0.5,0.75)edge[anchor=south,<->,line width=1.5pt]
        node{\small 1 octet}(0.5,0.75);
    \draw(-0.5,-0.75)edge[anchor=north,<->,line width=1.5pt]
        node{\small \Code{a.x}}(1.5,-0.75);
    \draw(3.5,-0.75)edge[anchor=north,<->,line width=1.5pt]
        node{\small \Code{a.z}}(7.5,-0.75);
    \draw(7.5,-0.75)edge[anchor=north,<->,line width=1.5pt]
        node{\small \Code{a.y}}(9.5,-0.75);
\end{tikzpicture}}
\end{center}
\medskip

Les octets en gris intervenant dans l'alignement mémoire de \Code{b}
sont des \alert{octets de complétion}.
\end{frame}

%%%%%%%%%%%%%%%%%%%%%%%%%%%%%%%%%%%%%%%%%%%%%%%%%%%%%%%%%%%%%%%%%%%%%%%%
\begin{frame}[fragile]
\frametitle{Octets de complétion}
Des octets de complétion sont introduits pour que chaque champ \Code{c}
d'une variable d'un type structuré \alert{commence à une adresse multiple d'un
entier} dépendant du type de \Code{c}.
\medskip

Dans notre exemple, en sachant que tout champ de type \Code{short}
(resp. \Code{int}) doit commencer à une adresse multiple de \Code{2}
(resp. \Code{4}), on explique l'alignement en mémoire de \Code{b}
précédent~:
\begin{center}
\scalebox{.6}{\begin{tikzpicture}
    \node[BlocM](9)at(-1,0){};
    \node[BlocM](10)at(-2,0){};
    \node[BlocM](11)at(-3,0){};
    \node[BlocM](16)at(12,0){};
    \node at(-4,0){\Large $\dots$};
    \node at(13,0){\Large $\dots$};
    \node[Bloc,fill=Jaune!40,draw=Jaune!100](1)at(0,0){};
    \node[Bloc,fill=Jaune!40,draw=Jaune!100](2)at(1,0){};
    \node[Bloc,fill=Bleu!40,draw=Bleu!100](5)at(4,0){};
    \node[Bloc,fill=Bleu!40,draw=Bleu!100](6)at(5,0){};
    \node[Bloc,fill=Bleu!40,draw=Bleu!100](7)at(6,0){};
    \node[Bloc,fill=Bleu!40,draw=Bleu!100](8)at(7,0){};
    \node[Bloc,fill=Rouge!40,draw=Rouge!100](12)at(8,0){};
    \node[Bloc,fill=Rouge!40,draw=Rouge!100](13)at(9,0){};
    \node[BlocM,fill=Noir!50,draw=Noir](14)at(10,0){};
    \node[BlocM,fill=Noir!50,draw=Noir](15)at(11,0){};
    \node[BlocM,fill=Noir!50,draw=Noir](3)at(2,0){};
    \node[BlocM,fill=Noir!50,draw=Noir](4)at(3,0){};
    \draw(-0.5,0.75)edge[anchor=south,<->,line width=1.5pt]
        node{\small 1 octet}(0.5,0.75);
    \draw(-0.5,-0.75)edge[anchor=north,<->,line width=1.5pt]
        node{\small \Code{a.x}}(1.5,-0.75);
    \draw(3.5,-0.75)edge[anchor=north,<->,line width=1.5pt]
        node{\small \Code{a.z}}(7.5,-0.75);
    \draw(7.5,-0.75)edge[anchor=north,<->,line width=1.5pt]
        node{\small \Code{a.y}}(9.5,-0.75);
    \node(m1)at(0,-2){adr. mult. de \Code{4}};
    \draw[->](m1)--(0,-1);
    \node(m2)at(2,-3){adr. non mult. de \Code{4}};
    \draw[->](m2)--(2,-1);
\end{tikzpicture}}
\end{center}

Les derniers octets de complétion sont introduits pour que les tableaux
de variables de type \Code{B} puissent être représentés en vérifiant
cet alignement en mémoire.
\end{frame}

%%%%%%%%%%%%%%%%%%%%%%%%%%%%%%%%%%%%%%%%%%%%%%%%%%%%%%%%%%%%%%%%%%%%%%%%
\begin{frame}[fragile]
\frametitle{Accès manuel aux champs}
Soit \Code{a} une variable de type \Code{A} initialisée par
\begin{lstlisting}
    A a = {1000, 2000, 3000};
\end{lstlisting}
Cette variable est organisée en mémoire en
\begin{center}
\scalebox{.6}{\begin{tikzpicture}
    \node[BlocM](9)at(-1,0){};
    \node[BlocM](10)at(-2,0){};
    \node[BlocM](11)at(-3,0){};
    \node[BlocM](12)at(8,0){};
    \node[BlocM](13)at(9,0){};
    \node[BlocM](14)at(10,0){};
    \node[BlocM](15)at(11,0){};
    \node[BlocM](16)at(12,0){};
    \node at(-4,0){\Large $\dots$};
    \node at(13,0){\Large $\dots$};
    \node[Bloc,fill=Jaune!40,draw=Jaune!100](1)at(0,0){};
    \node[Bloc,fill=Jaune!40,draw=Jaune!100](2)at(1,0){};
    \node[Bloc,fill=Rouge!40,draw=Rouge!100](3)at(2,0){};
    \node[Bloc,fill=Rouge!40,draw=Rouge!100](4)at(3,0){};
    \node[Bloc,fill=Bleu!40,draw=Bleu!100](5)at(4,0){};
    \node[Bloc,fill=Bleu!40,draw=Bleu!100](6)at(5,0){};
    \node[Bloc,fill=Bleu!40,draw=Bleu!100](7)at(6,0){};
    \node[Bloc,fill=Bleu!40,draw=Bleu!100](8)at(7,0){};
    \draw(-0.5,0.75)edge[anchor=south,<->,line width=1.5pt]
        node{\small 1 octet}(0.5,0.75);
    \draw(-0.5,-0.75)edge[anchor=north,<->,line width=1.5pt]
        node{\small \Code{a.x}}(1.5,-0.75);
    \draw(1.5,-0.75)edge[anchor=north,<->,line width=1.5pt]
        node{\small \Code{a.y}}(3.5,-0.75);
    \draw(3.5,-0.75)edge[anchor=north,<->,line width=1.5pt]
        node{\small \Code{a.z}}(7.5,-0.75);
\end{tikzpicture}}
\end{center}
\bigskip

On peut accéder aux champs de \Code{a} de la manière suivante~:
\begin{lstlisting}
short x, y;
int z;
void *p;
p = &a;
x = *((short *) p); /* Equivalent a x = a.x; */
p += 2;
y = *((short *) p); /* Equivalent a y = a.y; */
p += 2;
z = *((int *) p); /* Equivalent a z = a.z; */
\end{lstlisting}
\end{frame}

%%%%%%%%%%%%%%%%%%%%%%%%%%%%%%%%%%%%%%%%%%%%%%%%%%%%%%%%%%%%%%%%%%%%%%%%
\begin{frame}[fragile]
\frametitle{Accès manuel aux champs}
Soit \Code{b} une variable de type \Code{B} initialisée par
\begin{lstlisting}
    B b = {1000, 2000, 3000};
\end{lstlisting}
Cette variable est organisée en mémoire en
\begin{center}
\scalebox{.6}{\begin{tikzpicture}
    \node[BlocM](9)at(-1,0){};
    \node[BlocM](10)at(-2,0){};
    \node[BlocM](11)at(-3,0){};
    \node[BlocM](16)at(12,0){};
    \node at(-4,0){\Large $\dots$};
    \node at(13,0){\Large $\dots$};
    \node[Bloc,fill=Jaune!40,draw=Jaune!100](1)at(0,0){};
    \node[Bloc,fill=Jaune!40,draw=Jaune!100](2)at(1,0){};
    \node[Bloc,fill=Bleu!40,draw=Bleu!100](5)at(4,0){};
    \node[Bloc,fill=Bleu!40,draw=Bleu!100](6)at(5,0){};
    \node[Bloc,fill=Bleu!40,draw=Bleu!100](7)at(6,0){};
    \node[Bloc,fill=Bleu!40,draw=Bleu!100](8)at(7,0){};
    \node[Bloc,fill=Rouge!40,draw=Rouge!100](12)at(8,0){};
    \node[Bloc,fill=Rouge!40,draw=Rouge!100](13)at(9,0){};
    \node[BlocM,fill=Noir!50,draw=Noir](14)at(10,0){};
    \node[BlocM,fill=Noir!50,draw=Noir](15)at(11,0){};
    \node[BlocM,fill=Noir!50,draw=Noir](3)at(2,0){};
    \node[BlocM,fill=Noir!50,draw=Noir](4)at(3,0){};
    \draw(-0.5,0.75)edge[anchor=south,<->,line width=1.5pt]
        node{\small 1 octet}(0.5,0.75);
    \draw(-0.5,-0.75)edge[anchor=north,<->,line width=1.5pt]
        node{\small \Code{a.x}}(1.5,-0.75);
    \draw(3.5,-0.75)edge[anchor=north,<->,line width=1.5pt]
        node{\small \Code{a.z}}(7.5,-0.75);
    \draw(7.5,-0.75)edge[anchor=north,<->,line width=1.5pt]
        node{\small \Code{a.y}}(9.5,-0.75);
\end{tikzpicture}}
\end{center}
\bigskip

On peut accéder aux champs de \Code{b} de la manière suivante~:
\begin{lstlisting}
short x, y;
int z;
void *p;
p = &b;
x = *((short *) p); /* Equivalent a x = a.x; */
p += 4;
z = *((int *) p); /* Equivalent a z = a.z; */
p += 4;
y = *((short *) p); /* Equivalent a y = a.y; */
\end{lstlisting}
\end{frame}

%%%%%%%%%%%%%%%%%%%%%%%%%%%%%%%%%%%%%%%%%%%%%%%%%%%%%%%%%%%%%%%%%%%%%%%%
\begin{frame}[fragile]
\frametitle{L'option {\tt Wpadded}}
L'option du compilateur \Code{-Wpadded} permet d'obtenir un avertissement
sanctionnant la déclaration d'un type structuré nécessitant des octets
de complétion.
\medskip

Par exemple, avec le type structuré \Code{B} défini par
\begin{lstlisting}
typedef struct {
    short x;
    int z;
    short y;
} B;
\end{lstlisting}
on obtient l'avertissement
\begin{footnotesize}
\begin{verbatim}
Prog.c:3:9: warning: padding struct to align ‘z’ [-Wpadded]
     int z;
         ^
Prog.c:5:1: warning: padding struct size to alignment boundary [-Wpadded]
 } B;
 ^
\end{verbatim}
\end{footnotesize}
\end{frame}

%%%%%%%%%%%%%%%%%%%%%%%%%%%%%%%%%%%%%%%%%%%%%%%%%%%%%%%%%%%%%%%%%%%%%%%%
\begin{frame}[fragile]
\frametitle{Alignement en mémoire --- ce qu'il faut retenir}
L'alignement mémoire d'une variable d'un type structuré dépend de
beaucoup de paramètres, notamment~:
\smallskip

\begin{itemize}
    \item de l'architecture de la machine exécutant ou ayant compilé le
    programme~;
    \smallskip

    \item du compilateur ayant compilé le programme.
\end{itemize}
\bigskip
\bigskip

Il est donc important de savoir que le calcul de la {\bf taille} d'une
variable d'un {\bf type structuré} n'est pas immédiat et
{\bf dépend du contexte}.
\end{frame}

% Auteur : Samuele Giraudo
% Création : oct. 2013
% Modifications : aout 2014 oct. 2014, déc. 2015, mars 2016

%%%%%%%%%%%%%%%%%%%%%%%%%%%%%%%%%%%%%%%%%%%%%%%%%%%%%%%%%%%%%%%%%%%%%%%%
%%%%%%%%%%%%%%%%%%%%%%%%%%%%%%%%%%%%%%%%%%%%%%%%%%%%%%%%%%%%%%%%%%%%%%%%
%%%%%%%%%%%%%%%%%%%%%%%%%%%%%%%%%%%%%%%%%%%%%%%%%%%%%%%%%%%%%%%%%%%%%%%%
\section{Entrées et sorties}

%%%%%%%%%%%%%%%%%%%%%%%%%%%%%%%%%%%%%%%%%%%%%%%%%%%%%%%%%%%%%%%%%%%%%%%%
%%%%%%%%%%%%%%%%%%%%%%%%%%%%%%%%%%%%%%%%%%%%%%%%%%%%%%%%%%%%%%%%%%%%%%%%
\subsection{Sortie}

%%%%%%%%%%%%%%%%%%%%%%%%%%%%%%%%%%%%%%%%%%%%%%%%%%%%%%%%%%%%%%%%%%%%%%%%
\begin{frame}[fragile] \frametitle{Écriture formatée}
La fonction
\begin{center}
    \Code{int printf(char *format, V\_1, ..., V\_N);}
\end{center}
permet de réaliser une \alert{écriture formatée} sur la sortie standard
\Code{stdout}.
\medskip

\uncover<2->{
{\bf Rappel~:} cette fonction renvoie le nombre de caractères affichés.
Si une erreur a lieu, elle renvoie un entier négatif.
\medskip}

\begin{multicols}{2}
\begin{semiverbatim}\uncover<3->{
int num;
num = printf("%s+%d\\n",
    "test", 200);
printf("%d\\n", num);
\end{semiverbatim}}
\uncover<3->{
Ces instructions affichent \\
\Sortie{test+200 \\
9}}
\end{multicols}
\end{frame}

%%%%%%%%%%%%%%%%%%%%%%%%%%%%%%%%%%%%%%%%%%%%%%%%%%%%%%%%%%%%%%%%%%%%%%%%
\begin{frame} \frametitle{Indicateurs de conversion}
On utilise les \alert{indicateurs de conversion} suivants pour interpréter
chaque valeur à écrire (ou à lire, voir la partie suivante) de manière
adéquate~:
\begin{center}
    \begin{tabular}{c|c}
        Indicateur de conversion & Affichage \\ \hline
        \Code{d}, \Code{i} & Entier en base dix \\
        \Code{u} & Entier non signé en base dix \\
        \Code{x}, \Code{X} & Entier en hexadécimal \\
        \Code{c} & Caractère \\
        \Code{e}, \Code{E} & Flottant en notation scientifique \\
        \Code{f}, \Code{g} & Flottant \\
        \Code{s} & Chaîne de caractères \\
        \Code{p} & Pointeur
    \end{tabular}
\end{center}
\end{frame}

%%%%%%%%%%%%%%%%%%%%%%%%%%%%%%%%%%%%%%%%%%%%%%%%%%%%%%%%%%%%%%%%%%%%%%%%
\begin{frame} \frametitle{Caractères spéciaux}
Certains caractères ne s'affichent pas mais produisent un effet sur
la sortie. Ce sont des \alert{caractères spéciaux}.
\begin{center}
    \begin{tabular}{c|c}
        Caractère spécial & Rôle \\ \hline
        \Code{\textbackslash n} & Passage à la ligne suivante \\
        \Code{\textbackslash b} & Retour en arrière d'un caractère \\
        \Code{\textbackslash f} & Passage à la ligne suivante avec alinéa \\
        \Code{\textbackslash r} & Retour chariot \\
        \Code{\textbackslash t} & Tabulation horizontale \\
        \Code{\textbackslash v} & Tabulation verticale \\
    \end{tabular}
\end{center}
\end{frame}

%%%%%%%%%%%%%%%%%%%%%%%%%%%%%%%%%%%%%%%%%%%%%%%%%%%%%%%%%%%%%%%%%%%%%%%%
\begin{frame}[fragile] \frametitle{Caractères d'attribut}
Il est possible de faire suivre le \Code{\%} d'un indicateur de conversion
de \alert{caractères d'attribut} pour réaliser un formatage avancé.
\begin{center}\small
    \begin{tabular}{c|c}
        Caractère d'attribut & Rôle \\ \hline
        \Code{0N} & Affichage du nombre sur \Code{N} chiffres (ajout de \Code{0}) \\
        \Code{N} & Affichage du nombre sur \Code{N} chiffres (ajout d'espaces) \\
        \Code{+} & Force l'affichage du signe d'un nombre \\
        \Code{-} & Justifie à gauche un nombre (à droite par défaut) \\
    \end{tabular}
\end{center}
\bigskip

\uncover<2->{P.ex.,} \vspace{-.5em}
\begin{multicols}{2}
\begin{semiverbatim}\uncover<2->{
printf("%+5d\\n", 23);}
\end{semiverbatim}
    \uncover<2->{affiche
    \Sortie{\textvisiblespace \textvisiblespace +23}}
    \uncover<3->{et}
\begin{semiverbatim}\uncover<3->{
printf("%+-5d %d\\n", 23, 5);}
\end{semiverbatim}
    \uncover<3->{affiche
    \Sortie{+23\textvisiblespace \textvisiblespace \textvisiblespace 5}.}
\end{multicols}
\end{frame}

%%%%%%%%%%%%%%%%%%%%%%%%%%%%%%%%%%%%%%%%%%%%%%%%%%%%%%%%%%%%%%%%%%%%%%%%
\begin{frame}[fragile] \frametitle{Écriture caractère par caractère}
La fonction
\begin{center}
    \Code{int putchar(int c);}
\end{center}
permet d'\alert{afficher un caractère} sur la sortie standard.
\medskip

\uncover<2->{
Cette fonction renvoie le caractère écrit (converti en un \Code{int}).
Elle renvoie la constante \Code{EOF} (end-of-file) si une erreur a lieu.
\medskip}

\begin{multicols}{2}
\begin{semiverbatim}\small\uncover<3->{
int ret;
ret = putchar('a');
if (ret == EOF)
    exit(EXIT_FAILURE);}
\end{semiverbatim}
\uncover<3->{
Ces instructions affichent \Sortie{a}. Un test est réalisé pour
détecter une erreur éventuelle.}
\end{multicols}
\end{frame}

%%%%%%%%%%%%%%%%%%%%%%%%%%%%%%%%%%%%%%%%%%%%%%%%%%%%%%%%%%%%%%%%%%%%%%%%
%%%%%%%%%%%%%%%%%%%%%%%%%%%%%%%%%%%%%%%%%%%%%%%%%%%%%%%%%%%%%%%%%%%%%%%%
\subsection{Entrée}

%%%%%%%%%%%%%%%%%%%%%%%%%%%%%%%%%%%%%%%%%%%%%%%%%%%%%%%%%%%%%%%%%%%%%%%%
\begin{frame}[fragile] \frametitle{Lecture formatée}
La fonction
\begin{center}
    \Code{int scanf (char *format, PTR\_1, ..., PTR\_N);}
\end{center}
permet de réaliser une \alert{lecture formatée} sur l'entrée standard
\Code{stdin}.
\bigskip

\uncover<2->{
Cette fonction renvoie le nombre d'éléments lus correctement assignés.
\bigskip}

\begin{multicols}{2}
\begin{semiverbatim}\uncover<3->{
int num, ret;
char chaine[128];
chaine[0] = '\\0';
ret = scanf("%d W %s",
  &num, chaine);
printf("%d %d %s\\n",
  ret, num, chaine);}
\end{semiverbatim}
\small
\uncover<3->{
Ces instructions lisent sur l'entrée standard un entier, un caractère
\Code{'W'}, puis une chaîne de caractères.
\medskip}

\uncover<4->{
\begin{multicols}{2}
\begin{enumerate} \footnotesize
    \item \Code{25\textvisiblespace W\textvisiblespace abc\textbackslash n} \\
    $\to$ \Sortie{2 25 abc}}
    \uncover<5->{
    \item \Code{25\textvisiblespace W%
        \textvisiblespace\textvisiblespace abc\textbackslash n} \\
    $\to$ \Sortie{2 25 abc}}
    \uncover<6->{
    \item \Code{25Wabc\textbackslash n} \\
    $\to$ \Sortie{2 25 abc}}
    \bigskip

    \uncover<7->{
    \item \Code{25\textvisiblespace abc\textbackslash n} \\
    $\to$ \Sortie{1 25}}
    \uncover<8->{
    \item \Code{xy\textvisiblespace W\textvisiblespace abc\textbackslash n} \\
    $\to$ \Sortie{0 ?}}
    \uncover<9->{
    \item \Code{25\textvisiblespace w\textvisiblespace abc\textbackslash n} \\
    $\to$ \Sortie{1 25}}
\end{enumerate}
\end{multicols}
\end{multicols}
\end{frame}

%%%%%%%%%%%%%%%%%%%%%%%%%%%%%%%%%%%%%%%%%%%%%%%%%%%%%%%%%%%%%%%%%%%%%%%%
\begin{frame}[fragile] \frametitle{Exemple d'utilisation de {\tt scanf}}
Ce programme lit sur l'entrée standard une chaîne d'au plus sept
caractères (format \Code{\%Ns}), une espace, puis un entier.

\begin{semiverbatim}
\uncover<4->{#include <stdio.h>
#include <string.h>

int main() \{
    char prenom[8];
    int age, ret;}

    \uncover<2->{ret = scanf("%7s %d", prenom, &age);}
    \uncover<3->{while (ret != 2) \{
        printf("%lu\\n", strlen(prenom));
        ret = scanf("%7s %d", prenom, &age);
    \}}
    \uncover<4->{return 0;
\}}
\end{semiverbatim}
\end{frame}

%%%%%%%%%%%%%%%%%%%%%%%%%%%%%%%%%%%%%%%%%%%%%%%%%%%%%%%%%%%%%%%%%%%%%%%%
\begin{frame}[fragile] \frametitle{Lecture caractère par caractère}
La fonction
\begin{center}
    \Code{int getchar();}
\end{center}
permet de \alert{lire un caractère} sur l'entrée standard.
\medskip

\uncover<2->{
Cette fonction renvoie le caractère lu (converti en un \Code{int}).
Elle renvoie la constante \Code{EOF} (end-of-file) lorsque la fin
de fichier est détectée (touche {\tt Ctrl + D}).
\medskip}

\begin{multicols}{2}
\begin{semiverbatim}\small\uncover<3->{
char c;
while ((c = getchar()) != EOF)
    printf("%c", c);}
\end{semiverbatim}
\uncover<3->{
Ces instructions affichent chaque caractère lu en entrée, tant que
\Code{EOF} n'est pas rencontré.}
\end{multicols}
\end{frame}

%%%%%%%%%%%%%%%%%%%%%%%%%%%%%%%%%%%%%%%%%%%%%%%%%%%%%%%%%%%%%%%%%%%%%%%%
\begin{frame}[fragile] \frametitle{Le tampon d'entrée}
Considérons les instructions
\begin{semiverbatim}
char c;
while ((c = getchar()) != EOF)
    printf("%c", c);
\end{semiverbatim}
\bigskip

\uncover<2->{
Il est possible de saisir plusieurs caractères avant le
\Code{'\textbackslash n'} (touche {\tt Entrée}).
\bigskip}

\uncover<3->{
Avant d'être lus par le \Code{getchar} de la boucle, ils sont situés
temporairement dans le \alert{tampon d'entrée}.
\bigskip}

\uncover<4->{Le tampon d'entrée se comporte comme un tableau
de caractères.
\bigskip}

\uncover<5->{Chaque appel à \Code{getchar} considère le tampon d'entrée et~:
\begin{itemize}
    \item s'il est vide, on attend la saisie d'un caractère de l'entrée
    standard~;
    \smallskip

    \item s'il contient au moins un élément, il le considère et le
    supprime du tampon d'entrée.
\end{itemize}}
\end{frame}

%%%%%%%%%%%%%%%%%%%%%%%%%%%%%%%%%%%%%%%%%%%%%%%%%%%%%%%%%%%%%%%%%%%%%%%%
%%%%%%%%%%%%%%%%%%%%%%%%%%%%%%%%%%%%%%%%%%%%%%%%%%%%%%%%%%%%%%%%%%%%%%%%
\subsection{Fichiers}

%%%%%%%%%%%%%%%%%%%%%%%%%%%%%%%%%%%%%%%%%%%%%%%%%%%%%%%%%%%%%%%%%%%%%%%%
\begin{frame} \frametitle{Descripteurs de fichiers et tête de lecture/écriture}
Tout fichier est manipulé par un pointeur sur une variable de type
\Code{FILE}.
\medskip

C'est un type structuré déclaré dans \Code{stdio.h} qui contient diverses
informations nécessaires et relatives au fichier qu'il adresse.
\medskip

Toute variable de type \Code{FILE *} est un \alert{descripteur de fichier}.
\bigskip

\uncover<2->{%
La lecture écriture dans un fichier se fait par l'intermédiaire d'une
\alert{tête de lecture}.
\medskip

Celle-ci désigne un caractère du fichier considéré comme le
{\bf caractère observé} à un instant donné.
\medskip

Les fonctions de lecture/écriture ont pour effet (en particulier) de
mettre à jour cette tête de lecture en avançant dans son positionnement
dans le fichier.}
\end{frame}

%%%%%%%%%%%%%%%%%%%%%%%%%%%%%%%%%%%%%%%%%%%%%%%%%%%%%%%%%%%%%%%%%%%%%%%%
\begin{frame} \frametitle{Tête de lecture/écriture}
Il existe trois fonctions de \Code{stdio.h} pour
\alert{déplacer manuellement la tête de lecture} ou obtenir des
informations à son sujet~:

\begin{itemize}
    \item \Code{int ftell(FILE *f);} qui renvoie l'indice de la tête
    de lecture du fichier pointé par \Code{f}.
    \smallskip

    Cette fonction est sans effet de bord~;
    \medskip

    \uncover<2->{%
    \item \Code{int fseek(FILE *f, int decalage, int mode);} qui décale
    la tête de lecture du fichier pointé par \Code{f} de \Code{decalage}
    caractères selon \Code{mode}.
    \smallskip

    Ce paramètre explique à quoi le décalage est relatif
    (\Code{SEEK\_SET}~: début du fichier, \Code{SEEK\_CUR}~: position
    courante, \Code{SEEK\_END}~: fin du fichier).
    \smallskip

    Cette fonction renvoie \Code{0} si elle s'est bien exécutée et
    \Code{-1} sinon~;
    \medskip}

    \uncover<3->{%
    \item \Code{void rewind(FILE *f)}; place la tête de lecture au début
    du fichier. L'appel \Code{void rewind(f)}; est ainsi équivalent à
    \Code{int fseek(f, O, SEEK\_SET);}.}
\end{itemize}
\end{frame}

%%%%%%%%%%%%%%%%%%%%%%%%%%%%%%%%%%%%%%%%%%%%%%%%%%%%%%%%%%%%%%%%%%%%%%%%
\begin{frame} \frametitle{Ouverture de fichiers}
La fonction
\begin{center}
    \Code{FILE *fopen(const char *chemin, const char *mode);}
\end{center}
permet d'\alert{ouvrir un fichier}.
\medskip

\uncover<2->{%
Cette fonction renvoie un descripteur de fichier sur le fichier de chemin
\Code{chemin} (relatif par rapport à l'exécutable).
\medskip}

\uncover<3->{%
La tête de lecture est positionnée sur son 1\ier{} caractère.
\medskip}

\uncover<4->{%
Le paramètre \Code{mode} désigne le mode d'ouverture désiré.
\bigskip}

\uncover<5->{
{\bf Attention}~: \Code{fopen} renvoie \Code{NULL} si l'ouverture s'est
mal passée. Il faut donc toujours tester la valeur de retour de \Code{fopen}.
\medskip}
\end{frame}

%%%%%%%%%%%%%%%%%%%%%%%%%%%%%%%%%%%%%%%%%%%%%%%%%%%%%%%%%%%%%%%%%%%%%%%%
\begin{frame} \frametitle{Modes d'ouverture}
Il existe plusieurs \alert{modes d'ouverture}. Chacun répond à un besoin
particulier~:
\smallskip

\begin{itemize}
    \item \Code{"r"}~: lecture seule.
    \medskip

    \item \Code{"w"}~: écriture seule. Si le fichier n'existe pas, il
    est créé.
    \medskip

    \item \Code{"a"}~: écriture en ajout. Permet d'écrire dans le fichier
    en partant de la fin. Si le fichier n'existe pas, il est créé.
    \medskip

    \item \Code{"r+"}~: lecture et écriture.
    \medskip

    \item \Code{"w+"}~: lecture et écriture avec suppression préalable
    du contenu du fichier. Si le fichier n'existe pas, il est créé.
    \medskip

    \item \Code{"a+"}~: lecture et écriture en ajout. Permet de lire et
    d'écrire dans le fichier en partant de la fin. Si le fichier
    n'existe pas, il est créé.
\end{itemize}
\end{frame}

%%%%%%%%%%%%%%%%%%%%%%%%%%%%%%%%%%%%%%%%%%%%%%%%%%%%%%%%%%%%%%%%%%%%%%%%
\begin{frame} \frametitle{Fermeture de fichiers}
La fonction
\begin{center}
    \Code{int fclose(FILE *f);}
\end{center}
permet de \alert{fermer un fichier}.
\bigskip

\uncover<2->{
Cette fonction permet de mettre à jour le fichier pointé par \Code{f}
de sorte que toutes les modifications effectuées soient prisent en compte.
Le pointeur \Code{f} n'est plus alors utilisable pour accéder au fichier.
\bigskip}

\uncover<3->{
Cette fonction renvoie \Code{0} si la fermeture s'est bien déroulée et
\Code{EOF} dans le cas contraire.
\bigskip}

\uncover<4->{{\bf Attention}~: toute ouverture d'un fichier lors de
l'exécution d'un programme doit s'accompagner tôt ou tard de la
fermeture future du fichier en question.}
\end{frame}

%%%%%%%%%%%%%%%%%%%%%%%%%%%%%%%%%%%%%%%%%%%%%%%%%%%%%%%%%%%%%%%%%%%%%%%%
\begin{frame}[fragile] \frametitle{Écriture dans un fichier}
La fonction
\begin{center}
    \Code{int fprintf(FILE *f, char *format, V\_1, ..., V\_N);}
\end{center}
permet de réaliser une \alert{écriture formatée} dans le fichier pointé
par \Code{f}.
\medskip

\uncover<2->{
Cette fonction se comporte comme \Code{printf}, à la différence que
cette dernière travaille sur le fichier \Code{stdout}.}

\begin{multicols}{2}
\begin{semiverbatim}\small\uncover<3->{
FILE *f;
int ret;
f = fopen("fic.txt", "w");
if (f == NULL)
    exit(EXIT_FAILURE);
fprintf(f, "abc\\n");
ret = fclose(f);
if (ret == EOF)
    exit(EXIT_FAILURE);}
\end{semiverbatim}
\uncover<3->{
Ces instructions déclarent un pointeur sur le fichier \Code{fic.txt}
et écrivent \Code{"abc"} dans \Code{fic.txt}.}
\bigskip
\bigskip
\bigskip
\bigskip
\bigskip
\bigskip
\end{multicols}
\end{frame}

%%%%%%%%%%%%%%%%%%%%%%%%%%%%%%%%%%%%%%%%%%%%%%%%%%%%%%%%%%%%%%%%%%%%%%%%
\begin{frame} \frametitle{Lecture dans un fichier}
La fonction
\begin{center}
    \Code{int fscanf(FILE *f, char *format, V\_1, ..., V\_N);}
\end{center}
permet de réaliser une \alert{lecture formatée} depuis le fichier pointé
par \Code{f}.
\bigskip

\uncover<2->{
Cette fonction se comporte comme \Code{scanf}, à la différence que
cette dernière travaille sur le fichier \Code{stdin}.}
\end{frame}

%%%%%%%%%%%%%%%%%%%%%%%%%%%%%%%%%%%%%%%%%%%%%%%%%%%%%%%%%%%%%%%%%%%%%%%%
%%%%%%%%%%%%%%%%%%%%%%%%%%%%%%%%%%%%%%%%%%%%%%%%%%%%%%%%%%%%%%%%%%%%%%%%
\subsection{Fichiers binaires}

%%%%%%%%%%%%%%%%%%%%%%%%%%%%%%%%%%%%%%%%%%%%%%%%%%%%%%%%%%%%%%%%%%%%%%%%
\begin{frame} \frametitle{Ouverture de fichiers en mode binaire}
On utilise les modes d'ouverture habituels avec un \Code{b} en plus
pour signaler l'ouverture en binaire.
\smallskip

\begin{itemize}
    \item \Code{"rb"}~: lecture binaire seule.
    \medskip

    \item \Code{"wb"}~: écriture binaire seule. Si le fichier n'existe
    pas, il est créé.
    \medskip

    \item \Code{"ab"}~: écriture binaire en ajout. Permet d'écrire dans le fichier
    en partant de la fin. Si le fichier n'existe pas, il est créé.
    \medskip

    \item \Code{"rb+"}~: lecture binaire  et écriture binaire .
    \medskip

    \item \Code{"wb+"}~: lecture binaire et écriture binaire
    avec suppression préalable
    du contenu du fichier. Si le fichier n'existe pas, il est créé.
    \medskip

    \item \Code{"ab+"}~: lecture binaire et écriture binaire en ajout. Permet de lire et
    d'écrire dans le fichier en partant de la fin. Si le fichier
    n'existe pas, il est créé.
\end{itemize}
\end{frame}

%%%%%%%%%%%%%%%%%%%%%%%%%%%%%%%%%%%%%%%%%%%%%%%%%%%%%%%%%%%%%%%%%%%%%%%%
\begin{frame} \frametitle{Écriture dans un fichier binaire}
On utilise la fonction
\begin{center}
    \Code{int fwrite(void *ptr, int taille, int nb, FILE *f);}
\end{center}
pour \alert{écrire} dans un fichier pointé par \Code{f} ouvert
en \alert{mode binaire}.
\bigskip

\uncover<2->{
Cette fonction écrit les \Code{nb} valeurs de taille \Code{taille} octets
pointées par le pointeur \Code{ptr} dans le fichier pointé
par \Code{f}.
\bigskip}

\uncover<3->{
Cette fonction renvoie le nombre de valeurs écrites.}
\end{frame}

%%%%%%%%%%%%%%%%%%%%%%%%%%%%%%%%%%%%%%%%%%%%%%%%%%%%%%%%%%%%%%%%%%%%%%%%
\begin{frame}[fragile] \frametitle{Exemple d'utilisation de {\tt fwrite}}
Ce programme écrit en mode binaire dans le fichier \Code{fic} la valeur
d'une variable d'un type structuré.
\medskip

\begin{semiverbatim}\footnotesize
\uncover<3->{#include <stdio.h>}
\uncover<2->{
typedef struct \{
    unsigned char rouge;
    unsigned char bleu;
    unsigned char vert;
\} Couleur;}

\uncover<3->{int main() \{
    FILE *f;
    Couleur coul;
    coul.rouge = 120; coul.bleu = 200; coul.vert = 12;}

    \uncover<4->{f = fopen("fic", "wb");}
    \uncover<5->{fwrite(&coul, sizeof(Couleur), 1, f);}
    \uncover<6->{fclose(f);}

    \uncover<3->{return 0;
\}}
\end{semiverbatim}
\end{frame}

%%%%%%%%%%%%%%%%%%%%%%%%%%%%%%%%%%%%%%%%%%%%%%%%%%%%%%%%%%%%%%%%%%%%%%%%
\begin{frame} \frametitle{Lecture dans un fichier binaire}
On utilise la fonction
\begin{center}
    \Code{int fread(void *ptr, int taille, int nb, FILE *f);}
\end{center}
pour \alert{lire} dans un fichier pointé par \Code{f} ouvert
en \alert{mode binaire}.
\bigskip

\uncover<2->{
Cette fonction lit \Code{nb} valeurs de taille \Code{taille} octets
dans le fichier pointé par \Code{f} et les place à l'adresse pointée
par le pointeur \Code{ptr}.
\bigskip}

\uncover<3->{
Cette fonction renvoie le nombre de valeurs lues.}
\end{frame}

%%%%%%%%%%%%%%%%%%%%%%%%%%%%%%%%%%%%%%%%%%%%%%%%%%%%%%%%%%%%%%%%%%%%%%%%
\begin{frame}[fragile] \frametitle{Exemple d'utilisation de {\tt fread}}
Ce programme écrit en mode binaire dans le fichier \Code{fic}
un tableau d'entiers et le lit.
\medskip

\begin{semiverbatim}\footnotesize
\uncover<2->{#include <stdio.h>}
\uncover<2->{
int main() \{
    FILE *f;
    int i, tab_1[12], tab_2[12];

    for (i = 0 ; i < 12 ; ++i)
        tab_1[i] = i;}

    \uncover<3->{f = fopen("fic", "wb");}
    \uncover<4->{fwrite(tab_1, sizeof(int), 12, f);}
    \uncover<5->{fclose(f);}

    \uncover<6->{f = fopen("fic", "rb");}
    \uncover<7->{fread(tab_2, sizeof(int), 12, f);}
    \uncover<8->{fclose(f);}

    \uncover<2->{return 0;
\}}
\end{semiverbatim}
\end{frame}


%%%%%%%%%%%%%%%%%%%%%%%%%%%%%%%%%%%%%%%%%%%%%%%%%%%%%%%%%%%%%%%%%%%%%%%%
%%%%%%%%%%%%%%%%%%%%%%%%%%%%%%%%%%%%%%%%%%%%%%%%%%%%%%%%%%%%%%%%%%%%%%%%
%%%%%%%%%%%%%%%%%%%%%%%%%%%%%%%%%%%%%%%%%%%%%%%%%%%%%%%%%%%%%%%%%%%%%%%%
\begin{frame} \frametitle{}
\part{Axe~3}
\begin{center} \Large
    {\bf Axe~3}~: utiliser quelques techniques avancées
\end{center}

\begin{footnotesize}
    \tableofcontents[hideallsubsections,part=3]
\end{footnotesize}
\end{frame}

% Auteur : Samuele Giraudo
% Création : nov. 2013
% Modifications : juil. 2014 oct. 2014 nov. 2014, déc. 2015

%%%%%%%%%%%%%%%%%%%%%%%%%%%%%%%%%%%%%%%%%%%%%%%%%%%%%%%%%%%%%%%%%%%%%%%%
%%%%%%%%%%%%%%%%%%%%%%%%%%%%%%%%%%%%%%%%%%%%%%%%%%%%%%%%%%%%%%%%%%%%%%%%
%%%%%%%%%%%%%%%%%%%%%%%%%%%%%%%%%%%%%%%%%%%%%%%%%%%%%%%%%%%%%%%%%%%%%%%%
\section{Opérateurs}

%%%%%%%%%%%%%%%%%%%%%%%%%%%%%%%%%%%%%%%%%%%%%%%%%%%%%%%%%%%%%%%%%%%%%%%%
%%%%%%%%%%%%%%%%%%%%%%%%%%%%%%%%%%%%%%%%%%%%%%%%%%%%%%%%%%%%%%%%%%%%%%%%
\subsection{Généralités}

%%%%%%%%%%%%%%%%%%%%%%%%%%%%%%%%%%%%%%%%%%%%%%%%%%%%%%%%%%%%%%%%%%%%%%%%
\begin{frame}[fragile]
\frametitle{Caractéristiques d'un opérateur}
Un opérateur dispose des {\bf caractéristiques structurelles} suivantes~:
\smallskip

\begin{enumerate}
    \uncover<2->{
    \item son \alert{arité}, qui désigne le nombre d'opérandes sur
    lequel il agit~;
    \medskip}

    \uncover<3->{
    \item sa \alert{précédence}, qui permet de savoir, dans une expression,
    dans quel ordre appliquer les différents opérateurs qui la composent~;
    \medskip}

    \uncover<4->{
    \item son \alert{sens d'associativité}, qui permet de savoir, dans
    une expression, dans quel sens appliquer des mêmes opérateurs qui la
    composent.}
\end{enumerate}
\end{frame}

%%%%%%%%%%%%%%%%%%%%%%%%%%%%%%%%%%%%%%%%%%%%%%%%%%%%%%%%%%%%%%%%%%%%%%%%
\begin{frame}[fragile]
\frametitle{Précédence et associativité des opérateurs}
Considérons l'expression \Code{3 * 2 + 1}.
\smallskip

\uncover<2->{
Suivant les priorités relatives des opérateurs \Code{*} et \Code{+},
il y deux manières de l'évaluer~:
\smallskip}

\begin{enumerate}
    \uncover<3->{
    \item \Code{(3 * 2) + 1}, si \Code{*} est {\bf plus prioritaire}
    que \Code{+}~;
    \smallskip}

    \uncover<4->{
    \item \Code{3 * (2 + 1)}, si \Code{+} est {\bf plus prioritaire}
    que \Code{*}.}
\end{enumerate}
\bigskip
\bigskip

\uncover<5->{
Considérons l'expression \Code{4 - 3 - 2 - 1}.
\smallskip}

\uncover<6->{
Suivant le sens d'associativité de \Code{-}, il y a deux manières de
l'évaluer~:
\smallskip}

\begin{enumerate}
    \uncover<7->{
    \item \Code{((4 - 3) - 2) - 1}, si \Code{-} est {\bf associatif de
    gauche à droite}~;
    \smallskip}

    \uncover<8->{
    \item \Code{4 - (3 - (2 - 1))}, si \Code{-} est {\bf associatif de
    droite à gauche}.}
\end{enumerate}
\bigskip
\bigskip

\uncover<9->{
Tout ceci peut être rendu explicite par l'\alert{utilisation de parenthèses}.}
\end{frame}

%%%%%%%%%%%%%%%%%%%%%%%%%%%%%%%%%%%%%%%%%%%%%%%%%%%%%%%%%%%%%%%%%%%%%%%%
%%%%%%%%%%%%%%%%%%%%%%%%%%%%%%%%%%%%%%%%%%%%%%%%%%%%%%%%%%%%%%%%%%%%%%%%
\subsection{Opérateurs d'accès}

%%%%%%%%%%%%%%%%%%%%%%%%%%%%%%%%%%%%%%%%%%%%%%%%%%%%%%%%%%%%%%%%%%%%%%%%
\begin{frame}[fragile]
\frametitle{Opérateurs de gestion la mémoire}

\begin{center}
    \begin{tabular}{c|c|c|c|c}
        {\bf Op.} & {\bf Rôle} & {\bf Ari.} & {\bf Assoc.}
            & {\bf Opérandes} \\ \hline \hline
        \Code{\&} & référencement & 1 & -- & une variable \\ \hline
        \Code{*} & déréférencement & 1 & -- & un pointeur \\ \hline
        \Code{[\,]} & élément d'un tableau & 2 & $\longrightarrow$
            & un pointeur et un entier \\ \hline
        \multirow{2}{*}{\Code{.}} & \multirow{2}{*}{valeur d'un champ} &
            \multirow{2}{*}{2} & \multirow{2}{*}{$\longrightarrow$}
            & une var. d'un type struct. \\
            & & & & et un id. de champ \\ \hline
        \multirow{3}{*}{\Code{->}} & \multirow{3}{*}{valeur d'un champ} &
            \multirow{3}{*}{2} & \multirow{3}{*}{$\longrightarrow$}
            & une pointeur sur \\
            & & & & une var. d'un type struct. \\
            & & & & et un id. de champ
    \end{tabular}
\end{center}
\end{frame}

%%%%%%%%%%%%%%%%%%%%%%%%%%%%%%%%%%%%%%%%%%%%%%%%%%%%%%%%%%%%%%%%%%%%%%%%
%%%%%%%%%%%%%%%%%%%%%%%%%%%%%%%%%%%%%%%%%%%%%%%%%%%%%%%%%%%%%%%%%%%%%%%%
\subsection{Opérateurs de calcul}

%%%%%%%%%%%%%%%%%%%%%%%%%%%%%%%%%%%%%%%%%%%%%%%%%%%%%%%%%%%%%%%%%%%%%%%%
\begin{frame}[fragile]
\frametitle{Opérateurs arithmétiques}

\begin{center}
    \begin{tabular}{c|c|c|c|c}
        {\bf Op.} & {\bf Rôle} & {\bf Ari.} & {\bf Assoc.}
            & {\bf Opérandes} \\ \hline \hline
        \Code{+}, \Code{-}, \Code{*}, \Code{/} & opérations arith.
            & 2 & $\longrightarrow$ & deux val. numériques \\ \hline
        \Code{\%} & modulo & 2 & $\longrightarrow$ & deux entiers \\ \hline
        \Code{+}, \Code{-} & signe  & 1 & -- & une val. numérique \\ \hline
        \multirow{2}{*}{\Code{++}, \Code{-\,-}} &
            \multirow{2}{*}{incr./décr.} & \multirow{2}{*}{1} &
            \multirow{2}{*}{--} &
            une var. d'un \\
            & & & & type numérique
    \end{tabular}
\end{center}
\end{frame}

%%%%%%%%%%%%%%%%%%%%%%%%%%%%%%%%%%%%%%%%%%%%%%%%%%%%%%%%%%%%%%%%%%%%%%%%
\begin{frame}[fragile]
\frametitle{L'opérateur modulo}
L'opérateur \alert{modulo} \Code{\%} calcule le reste de la division
euclidienne de son premier opérande par son second.
\medskip

\uncover<2->{
D'un point de vue mathématique, si $a$ et $b$ sont deux entiers, on a
$a / b = b \times q + r$, où $0 \leq r \leq b - 1$ et $q$ est un entier.
$q$ est le {\bf quotient} et $r$ est le {\bf reste}, \alert{toujours positif}.
\bigskip}

\uncover<3->{
Cependant, \Code{\%} peut produire des valeurs négatives, dans le cas où
l'un des deux opérandes est négatif.
\bigskip}

\uncover<4->{
Solution pour un modulo qui respecte la définition mathématique~:}
\begin{semiverbatim}\small\uncover<4->{
int vrai_modulo(int a, int b) \{
    int r;
    r = a % b;
    if (r < 0)
        return r + b;
    return r;
\}}
\end{semiverbatim}
\end{frame}

%%%%%%%%%%%%%%%%%%%%%%%%%%%%%%%%%%%%%%%%%%%%%%%%%%%%%%%%%%%%%%%%%%%%%%%%
\begin{frame}[fragile]
\frametitle{Les opérateurs d'incrémentation et de décrémentation}
Les opérateurs \Code{++} et \Code{-\,-} existent en deux versions,
suivant qu'ils soient préfixes ou suffixes~:
\smallskip

\begin{enumerate}
    \uncover<2->{
    \item \Code{a++}, incrémente (de un) la valeur de la variable \Code{a}
    et est une expression dont la valeur est l'\alert{ancienne valeur}
    de \Code{a}~;
    \medskip}

    \uncover<3->{
    \item \Code{++a}, incrémente (de un) la valeur de la variable \Code{a}
    et est une expression dont la valeur est la \alert{nouvelle valeur}
    de \Code{a}.}
\end{enumerate}
\bigskip

\begin{multicols}{2}
\begin{semiverbatim}\uncover<4->{
int a = 5, b;
b = 3 + a++;}
\end{semiverbatim}
\uncover<4->{\Code{b} vaut \Code{8} et \Code{a} vaut \Code{6}.}

\begin{semiverbatim}\uncover<5->{
int a = 5, b;
b = 3 + ++a;}
\end{semiverbatim}
\uncover<5->{\Code{b} vaut \Code{9} et \Code{a} vaut \Code{6}.}
\end{multicols}
\end{frame}

%%%%%%%%%%%%%%%%%%%%%%%%%%%%%%%%%%%%%%%%%%%%%%%%%%%%%%%%%%%%%%%%%%%%%%%%
\begin{frame}[fragile]
\frametitle{Les opérateurs d'incrémentation et de décrémentation}
Attention au {\bf pièges d'utilisation} de ces opérateurs.
\bigskip

\uncover<2->{
P.ex., les instructions}
\begin{multicols}{3}
\begin{semiverbatim}\uncover<2->{
int a = 5, b;
b = a++ + ++a;}
\end{semiverbatim}
\begin{semiverbatim}\uncover<2->{
int a = 5;
a = a++;}
\end{semiverbatim}
\begin{semiverbatim}\uncover<2->{
int a = 5;
a = ++a;}
\end{semiverbatim}
\end{multicols}
\uncover<2->{ne sont pas évaluables (l'effet produit par les lignes 2
dépend du compilateur et de ses options).
\bigskip
\bigskip}

\uncover<3->{
{\bf Règle}~: pour éviter ce type de piège, on s'interdit de réaliser
plus d'une modification d'une même variable dans une même expression.}
\end{frame}

%%%%%%%%%%%%%%%%%%%%%%%%%%%%%%%%%%%%%%%%%%%%%%%%%%%%%%%%%%%%%%%%%%%%%%%%
\begin{frame}[fragile]
\frametitle{Opérateurs relationnels}

\begin{center}
    \begin{tabular}{c|c|c|c|c}
        {\bf Op.} & {\bf Rôle} & {\bf Ari.} & {\bf Assoc.}
            & {\bf Opérandes} \\ \hline \hline
        \Code{<}, \Code{>} & comparaison stricte & 2 & $\longrightarrow$
            & deux val. numériques \\ \hline
        \Code{<=}, \Code{>=} & comparaison large & 2 & $\longrightarrow$
            & deux val. numériques \\ \hline
        \Code{==} & égalité & 2 & $\longrightarrow$ & deux val. numériques \\ \hline
        \Code{!=} & différence & 2 & $\longrightarrow$ & deux val. numériques
    \end{tabular}
\end{center}
\medskip

\uncover<2->{
Toutes les expressions de la forme
\begin{center} \Code{v1 CMP v2}\end{center}
où \Code{v1} et \Code{v2} sont des valeurs numériques et \Code{CMP} est
un opérateur de comparaison produisent une valeur~:
\begin{itemize}
    \item \Code{1} si la comparaison est vraie~;
    \item \Code{0} sinon.
\end{itemize}}
\end{frame}

%%%%%%%%%%%%%%%%%%%%%%%%%%%%%%%%%%%%%%%%%%%%%%%%%%%%%%%%%%%%%%%%%%%%%%%%
\begin{frame}[fragile]
\frametitle{Opérateurs relationnels}
Un pointeur étant une adresse, et donc une valeur numérique,
il est possible de comparer deux pointeurs.
\medskip

\begin{multicols}{2}
\begin{semiverbatim}\footnotesize\uncover<2->{
char *ptr1, ptr2;
char c;
c = 'a';
ptr1 = &c;
ptr2 = &c;
if (ptr1 == ptr2)
    printf("ok1\\n");
if (&ptr1 == &ptr2)
    printf("ok2\\n");}
\end{semiverbatim}
\uncover<2->{Ceci affiche juste \Sortie{ok1}.
\smallskip

En effet, les deux pointeurs \Code{ptr1} et \Code{ptr2}
pointent vers le même emplacement en mémoire.
\smallskip

Le second test est faux car les adresses des variables \Code{ptr1} et
\Code{ptr2} sont différentes.}
\end{multicols}
\medskip

\begin{multicols}{2}
\begin{semiverbatim}\footnotesize\uncover<3->{
int t1[2], t2[2];
t1[0] = 1;
t1[2] = 2;
t2[0] = 1;
t2[2] = 2;
if (t1 == t2)
    printf("ok\\n");}
\end{semiverbatim}
\small
\uncover<3->{
Ceci compare les {\bf adresses} de \Code{t1} et \Code{t2} et non
pas les valeurs de leurs cases.
\smallskip

Rien n'est donc affiché car les tableaux \Code{t1} et \Code{t2} sont à
des adresses différentes.}
\end{multicols}
\end{frame}

%%%%%%%%%%%%%%%%%%%%%%%%%%%%%%%%%%%%%%%%%%%%%%%%%%%%%%%%%%%%%%%%%%%%%%%%
\begin{frame}[fragile]
\frametitle{Opérateurs logiques}

\begin{center}
    \begin{tabular}{c|c|c|c|c}
        {\bf Op.} & {\bf Rôle} & {\bf Ari.} & {\bf Assoc.}
            & {\bf Opérandes} \\ \hline \hline
        \Code{\&\&} & et logique & 2 & $\longrightarrow$
            & deux val. numériques \\ \hline
        \Code{||} & ou logique & 2 & $\longrightarrow$
            & deux val. numériques \\ \hline
        \Code{!} & non logique & 1 & --
            & une val. numérique
    \end{tabular}
\end{center}
\medskip

\uncover<2->{
Toutes les expressions formées d'opérateurs logiques produisent une
valeur, \Code{0} ou bien \Code{1}.
\medskip

Cette valeur est
\begin{itemize}
    \item \Code{1} si l'expression logique est vraie~;
    \item \Code{0} sinon.
\end{itemize}}
\end{frame}

%%%%%%%%%%%%%%%%%%%%%%%%%%%%%%%%%%%%%%%%%%%%%%%%%%%%%%%%%%%%%%%%%%%%%%%%
\begin{frame}[fragile]
\frametitle{Opérateurs bit à bit}

\begin{center}
    \begin{tabular}{c|c|c|c|c}
        {\bf Op.} & {\bf Rôle} & {\bf Ari.} & {\bf Assoc.}
            & {\bf Opérandes} \\ \hline \hline
        \Code{\&} & et bit à bit & 2 & $\longrightarrow$
            & deux val. entières \\ \hline
        \Code{|} & ou bit à bit & 2 & $\longrightarrow$
            & deux val. entières \\ \hline
        \Code{\textasciicircum} & xor bit à bit & 2 & $\longrightarrow$
            & deux val. entières \\ \hline
        \Code{$\sim$} & non bit à bit & 1 & --
            & une val. entière \\ \hline
        \Code{<\,<}, \Code{>\,>} & déc. g./d. bit à bit & 2
            & $\longrightarrow$ & deux val. entières
    \end{tabular}
\end{center}
\end{frame}

%%%%%%%%%%%%%%%%%%%%%%%%%%%%%%%%%%%%%%%%%%%%%%%%%%%%%%%%%%%%%%%%%%%%%%%%
\begin{frame}[fragile]
\frametitle{Et/ou/xor/non bit à bit}
Quelques exemples d'opérations bit à bit~:
\bigskip

\begin{multicols}{2}
\scalebox{.75}{
\begin{tabular}{c}
    \Code{x} \\
    \Code{y} \\
    \Code{x \& y}
\end{tabular}
\begin{tabular}{|c|c|c|c|c|c|c|c|} \hline
    \Code{0} & \Code{1} & \Code{1} & \Code{1}
        & \Code{0} & \Code{1} & \Code{0} & \Code{1} \\ \hline
    \Code{1} & \Code{0} & \Code{1} & \Code{0}
        & \Code{1} & \Code{1} & \Code{0} & \Code{0} \\ \hline
    \Code{0} & \Code{0} & \Code{1} & \Code{0}
        & \Code{0} & \Code{1} & \Code{0} & \Code{0} \\ \hline
\end{tabular}}
\bigskip
\bigskip

\uncover<2->{
\scalebox{.75}{
\begin{tabular}{c}
    \Code{x} \\
    \Code{y} \\
    \Code{x | y}
\end{tabular}
\begin{tabular}{|c|c|c|c|c|c|c|c|} \hline
    \Code{0} & \Code{1} & \Code{1} & \Code{1}
        & \Code{0} & \Code{1} & \Code{0} & \Code{1} \\ \hline
    \Code{1} & \Code{0} & \Code{1} & \Code{0}
        & \Code{1} & \Code{1} & \Code{0} & \Code{0} \\ \hline
    \Code{1} & \Code{1} & \Code{1} & \Code{1}
        & \Code{1} & \Code{1} & \Code{0} & \Code{1} \\ \hline
\end{tabular}}}

\uncover<3->{
\scalebox{.75}{
\begin{tabular}{c}
    \Code{x} \\
    \Code{y} \\
    \Code{x \textasciicircum\,\! y}
\end{tabular}
\begin{tabular}{|c|c|c|c|c|c|c|c|} \hline
    \Code{0} & \Code{1} & \Code{1} & \Code{1}
        & \Code{0} & \Code{1} & \Code{0} & \Code{1} \\ \hline
    \Code{1} & \Code{0} & \Code{1} & \Code{0}
        & \Code{1} & \Code{1} & \Code{0} & \Code{0} \\ \hline
    \Code{1} & \Code{1} & \Code{0} & \Code{1}
        & \Code{1} & \Code{0} & \Code{0} & \Code{1} \\ \hline
\end{tabular}}}
\bigskip
\bigskip

\uncover<4->{
\scalebox{.75}{
\begin{tabular}{c}
    \Code{x} \\
    \Code{$\sim$\!\,x}
\end{tabular}
\begin{tabular}{|c|c|c|c|c|c|c|c|} \hline
    \Code{0} & \Code{1} & \Code{1} & \Code{1}
        & \Code{0} & \Code{1} & \Code{0} & \Code{1} \\ \hline
    \Code{1} & \Code{0} & \Code{0} & \Code{0}
        & \Code{1} & \Code{0} & \Code{1} & \Code{0} \\ \hline
\end{tabular}}}
\end{multicols}
\end{frame}

%%%%%%%%%%%%%%%%%%%%%%%%%%%%%%%%%%%%%%%%%%%%%%%%%%%%%%%%%%%%%%%%%%%%%%%%
\begin{frame}[fragile]
\frametitle{Et/ou/xor/non bit à bit}

Si les deux opérandes n'ont pas la même taille (en nombre de bits), le
plus petit est complété à gauche par des
\begin{itemize}
    \item {\bf zéros} s'il est non signé ou bien positif~;
    \item {\bf uns} s'il est négatif et signé.
\end{itemize}
\medskip

\uncover<2->{
Le signe d'un entier signé est lu sur son bit de poids fort~:
\begin{itemize}
    \item \Code{0} s'il est positif~;
    \item \Code{1} s'il est négatif.
\end{itemize}
\bigskip}

\begin{minipage}{.25\textwidth}
\begin{semiverbatim}\small\uncover<3->{
short x = 5;
char y = 10;
x = x | y;}
\end{semiverbatim}
\end{minipage}
\begin{minipage}{.63\textwidth}
\uncover<3->{
\scalebox{.7}{
\begin{tabular}{c}
    \Code{x} \\
    \Code{y} \\
    \Code{x | y}
\end{tabular}
\begin{tabular}{|c|c|c|c|c|c|c|c|c|c|c|c|c|c|c|c|} \hline
    \Code{0} & \Code{0} & \Code{0} & \Code{0} & \Code{0} & \Code{0} &
    \Code{0} & \Code{0} & \Code{0} & \Code{0} & \Code{0} & \Code{0} &
    \Code{0} & \Code{1} & \Code{0} & \Code{1} \\ \hline
    {\bf \CRouge{\tt 0}} & {\bf \CRouge{\tt 0}} & {\bf \CRouge{\tt 0}} &
    {\bf \CRouge{\tt 0}} & {\bf \CRouge{\tt 0}} & {\bf \CRouge{\tt 0}} &
    {\bf \CRouge{\tt 0}} & {\bf \CRouge{\tt 0}} &
    \Code{0} & \Code{0} & \Code{0} & \Code{0} &
    \Code{1} & \Code{0} & \Code{1} & \Code{0} \\ \hline
    \Code{0} & \Code{0} & \Code{0} & \Code{0} & \Code{0} & \Code{0} &
    \Code{0} & \Code{0} & \Code{0} & \Code{0} & \Code{0} & \Code{0} &
    \Code{1} & \Code{1} & \Code{1} & \Code{1} \\ \hline
\end{tabular}}}
\end{minipage}
\bigskip

\begin{minipage}{.25\textwidth}
\begin{semiverbatim}\small\uncover<4->{
short x = 5;
char y = -10;
x = x | y;}
\end{semiverbatim}
\end{minipage}
\begin{minipage}{.63\textwidth}
\uncover<4->{
\scalebox{.7}{
\begin{tabular}{c}
    \Code{x} \\
    \Code{y} \\
    \Code{x | y}
\end{tabular}
\begin{tabular}{|c|c|c|c|c|c|c|c|c|c|c|c|c|c|c|c|} \hline
    \Code{0} & \Code{0} & \Code{0} & \Code{0} & \Code{0} & \Code{0} &
    \Code{0} & \Code{0} & \Code{0} & \Code{0} & \Code{0} & \Code{0} &
    \Code{0} & \Code{1} & \Code{0} & \Code{1} \\ \hline
    {\bf \CVert{\tt 1}} & {\bf \CVert{\tt 1}} & {\bf \CVert{\tt 1}} &
    {\bf \CVert{\tt 1}} & {\bf \CVert{\tt 1}} & {\bf \CVert{\tt 1}} &
    {\bf \CVert{\tt 1}} & {\bf \CVert{\tt 1}} &
    \Code{1} & \Code{1} & \Code{1} & \Code{1} &
    \Code{0} & \Code{1} & \Code{1} & \Code{0} \\ \hline
    \Code{1} & \Code{1} & \Code{1} & \Code{1} & \Code{1} & \Code{1} &
    \Code{1} & \Code{1} & \Code{1} & \Code{1} & \Code{1} & \Code{1} &
    \Code{0} & \Code{1} & \Code{1} & \Code{1} \\ \hline
\end{tabular}}}
\end{minipage}
\end{frame}

%%%%%%%%%%%%%%%%%%%%%%%%%%%%%%%%%%%%%%%%%%%%%%%%%%%%%%%%%%%%%%%%%%%%%%%%
\begin{frame}[fragile]
\frametitle{Décalage bit à bit}

Si \Code{x} est non signé (déclaré avec \Code{unsigned}),
\begin{multicols}{2}
\uncover<2->{
\scalebox{.7}{
\begin{tabular}{c}
    \Code{x} \\
    \Code{x <\,< 3}
\end{tabular}
\begin{tabular}{|c|c|c|c|c|c|c|c|} \hline
    \Code{0} & \Code{1} & \Code{1} & \Code{1}
        & \Code{0} & \Code{1} & \Code{0} & \Code{1} \\ \hline
    \Code{1} & \Code{0} & \Code{1} & \Code{0}
        & \Code{1} & \CRouge{\tt 0} & \CRouge{\tt 0} & \CRouge{\tt 0} \\ \hline
\end{tabular}}}

\uncover<3->{
\scalebox{.7}{
\begin{tabular}{c}
    \Code{x} \\
    \Code{x >\,> 3}
\end{tabular}
\begin{tabular}{|c|c|c|c|c|c|c|c|} \hline
    \Code{0} & \Code{1} & \Code{1} & \Code{1}
        & \Code{0} & \Code{1} & \Code{0} & \Code{1} \\ \hline
    \CRouge{\tt 0} & \CRouge{\tt 0} & \CRouge{\tt 0} & \Code{0}
        & \Code{1} & \Code{1} & \Code{1} & \Code{0} \\ \hline
\end{tabular}}}
\end{multicols}
\bigskip
\bigskip

\uncover<4->{
Si \Code{x} est signé (déclaré sans \Code{unsigned}),}
\begin{multicols}{2}
\uncover<5->{
\scalebox{.7}{
\begin{tabular}{c}
    \Code{x} \\
    \Code{x <\,< 3}
\end{tabular}
\begin{tabular}{|c|c|c|c|c|c|c|c|} \hline
    \Code{0} & \Code{1} & \Code{1} & \Code{1}
        & \Code{0} & \Code{1} & \Code{0} & \Code{1} \\ \hline
    \Code{1} & \Code{0} & \Code{1} & \Code{0}
        & \Code{1} & \CRouge{\tt 0} & \CRouge{\tt 0} & \CRouge{\tt 0} \\ \hline
\end{tabular}}}

\uncover<6->{
\scalebox{.7}{
\begin{tabular}{c}
    \Code{x} \\
    \Code{x >\,> 3}
\end{tabular}
\begin{tabular}{|c|c|c|c|c|c|c|c|} \hline
    \Code{0} & \Code{1} & \Code{1} & \Code{1}
        & \Code{0} & \Code{1} & \Code{0} & \Code{1} \\ \hline
    \CRouge{\tt 0} & \CRouge{\tt 0} & \CRouge{\tt 0} & \Code{0}
        & \Code{1} & \Code{1} & \Code{1} & \Code{0} \\ \hline
\end{tabular}}}
\end{multicols}
\medskip

\begin{multicols}{2}
\uncover<7->{
\scalebox{.7}{
\begin{tabular}{c}
    \Code{x} \\
    \Code{x <\,< 3}
\end{tabular}
\begin{tabular}{|c|c|c|c|c|c|c|c|} \hline
    \Code{1} & \Code{1} & \Code{1} & \Code{1}
        & \Code{0} & \Code{1} & \Code{0} & \Code{1} \\ \hline
    \Code{1} & \Code{0} & \Code{1} & \Code{0}
        & \Code{1} & \CRouge{\tt 0} & \CRouge{\tt 0} & \CRouge{\tt 0} \\ \hline
\end{tabular}}}

\uncover<8->{
\scalebox{.7}{
\begin{tabular}{c}
    \Code{x} \\
    \Code{x >\,> 3}
\end{tabular}
\begin{tabular}{|c|c|c|c|c|c|c|c|} \hline
    \Code{1} & \Code{1} & \Code{1} & \Code{1}
        & \Code{0} & \Code{1} & \Code{0} & \Code{1} \\ \hline
    \CVert{\tt 1} & \CVert{\tt 1} & \CVert{\tt 1} & \Code{1}
        & \Code{1} & \Code{1} & \Code{1} & \Code{0} \\ \hline
\end{tabular}}}
\end{multicols}
\end{frame}

%%%%%%%%%%%%%%%%%%%%%%%%%%%%%%%%%%%%%%%%%%%%%%%%%%%%%%%%%%%%%%%%%%%%%%%%
\begin{frame}[fragile]
\frametitle{Compter le nombre de bits à un}
{\bf But}~: écrire une fonction qui renvoie le nombre de bits à un
de son paramètre.
\bigskip

\uncover<2->{
On travaille sur des variables de \Code{64} bits. On considère pour cela
le type \Code{Mot64}
définit par}
\begin{semiverbatim}\small\uncover<2->{
typedef unsigned long long Mot64;}
\end{semiverbatim}
\bigskip
\bigskip

\uncover<3->{
{\bf $1\iere$ méthode}~: attraper le bit de poids faible et le pousser
à droite.}
\begin{semiverbatim}\small
\uncover<4->{int compter_un_1(Mot64 x) \{}
    \uncover<5->{int res, i;
    res = 0;}
    \uncover<6->{for (i = 0 ; i < 64 ; ++i) \{}
        \uncover<7->{if ((x & 1) == 1)
            res += 1;}
        \uncover<6->{x = x >\,> 1;
    \}}
    \uncover<5->{return res;}
\uncover<4->{\}}
\end{semiverbatim}
\end{frame}

%%%%%%%%%%%%%%%%%%%%%%%%%%%%%%%%%%%%%%%%%%%%%%%%%%%%%%%%%%%%%%%%%%%%%%%%
\begin{frame}[fragile]
\frametitle{Compter le nombre de bits à un}
{\bf $2\ieme$ méthode}~: on constate que pour tout entier \Code{x} non
nul (avec au moins un bit à un), l'expression \Code{x \& -x} est l'entier
qui contient un unique bit à un, le plus à droite de \Code{x}.
\medskip

\uncover<2->{
Ainsi, l'instruction
\begin{center}
    \Code{x = x \textasciicircum\,\! (x \& -x);}
\end{center}
transforme le bit à un le plus à droite de \Code{x} en un zéro.
\bigskip}

\uncover<3->{L'exploitation de cette idée donne}
\begin{semiverbatim}\small
\uncover<4->{int compter_un_2(Mot64 x) \{}
    \uncover<5->{int res;
    res = 0;}
    \uncover<6->{while (x != 0) \{
        x = x ^ (x & -x);
        res += 1;
    \}}
    \uncover<5->{return res;}
\uncover<4->{\}}
\end{semiverbatim}
\end{frame}

%%%%%%%%%%%%%%%%%%%%%%%%%%%%%%%%%%%%%%%%%%%%%%%%%%%%%%%%%%%%%%%%%%%%%%%%
\begin{frame}[fragile]
\frametitle{Compter le nombre de bits à un}
{\bf $3\ieme$ méthode}~: cette troisième méthode n'utilise pas de boucle.
\medskip

\uncover<2->{
On commence par construire un tableau qui associe à tout entier de huit
bits (un \Code{char}) le nombre de bits à un qu'il contient, en utilisant
au choix l'une des deux méthodes précédentes.}

\begin{semiverbatim}\small\uncover<3->{
int nombre_un[256];

void initialiser_nombre_un() \{
    int i;
    Mot64 x;
    for (i = 0 ; i < 256 ; ++i) \{
        x = (Mot64) i;
        nombre_un[i] = compter_un_2(x);
    \}
\}}
\end{semiverbatim}
\end{frame}

%%%%%%%%%%%%%%%%%%%%%%%%%%%%%%%%%%%%%%%%%%%%%%%%%%%%%%%%%%%%%%%%%%%%%%%%
\begin{frame}[fragile]
\frametitle{Compter le nombre de bits à un}
On isole l'octet d'indice \Code{i} d'une variable \Code{x} de type
\Code{Mot64} par l'expression
\begin{center}
    \Code{0xFF \& (x >\,> (8 * i))}
\end{center}
\medskip

\uncover<2->{
Comme le nombre de bits à un de \Code{x} est la somme du nombre de bits
à un de chacun des huit octets qui le constituent, on obtient}
\begin{semiverbatim}\small
\uncover<3->{int compter_un_3(Mot64 x) \{}
    \uncover<4->{int res;
    res = 0;}
    \uncover<5->{res += nombre_un[0xFF & x];}
    \uncover<6->{res += nombre_un[0xFF & (x >\,> 8)];}
    \uncover<7->{res += nombre_un[0xFF & (x >\,> 16)];}
    \uncover<8->{res += nombre_un[0xFF & (x >\,> 24)];}
    \uncover<9->{res += nombre_un[0xFF & (x >\,> 32)];}
    \uncover<10->{res += nombre_un[0xFF & (x >\,> 40)];}
    \uncover<11->{res += nombre_un[0xFF & (x >\,> 48)];}
    \uncover<12->{res += nombre_un[0xFF & (x >\,> 56)];}
    \uncover<4->{return res;}
\uncover<3->{\}}
\end{semiverbatim}
\end{frame}

%%%%%%%%%%%%%%%%%%%%%%%%%%%%%%%%%%%%%%%%%%%%%%%%%%%%%%%%%%%%%%%%%%%%%%%%
\begin{frame}[fragile]
\frametitle{Compter le nombre de bits à un}
{\bf $4\ieme$ méthode}~: on peut pousser plus loin l'idée précédente en
considérant des blocs de deux octets (au lieu d'un seul).

\begin{semiverbatim}\small\uncover<2->{
int nombre_un[65536]; /* 2^16 */

void initialiser_nombre_un() \{
    int i;
    Mot64 x;
    for (i = 0 ; i < 65536 ; ++i) \{
        x = (Mot64) i;
        nombre_un[i] = compter_un_2(x);
    \}
\}}
\end{semiverbatim}
\bigskip

\uncover<3->{
Taille mémoire occupée par le tableau \Code{nombre\_un}~:
\begin{equation*}
    2^{16} \times \Code{sizeof(int)} \:\mathrm{o}
        \uncover<4->{\; = \; 2^{18} \:\mathrm{o}}
        \uncover<5->{\; = \; \frac{2^{18}}{2^{10}} \:\mathrm{Kio}}
        \uncover<6->{\; = \; 256 \:\mathrm{Kio}.}
\end{equation*}}
\end{frame}

%%%%%%%%%%%%%%%%%%%%%%%%%%%%%%%%%%%%%%%%%%%%%%%%%%%%%%%%%%%%%%%%%%%%%%%%
\begin{frame}[fragile]
\frametitle{Compter le nombre de bits à un}
Ceci fournit la solution suivante.
\begin{semiverbatim}\small
int compter_un_4(Mot64 x) \{
    int res;
    res = 0;
    res += nombre_un[0xFFFF & x];
    res += nombre_un[0xFFFF & (x >\,> 16)];
    res += nombre_un[0xFFFF & (x >\,> 32)];
    res += nombre_un[0xFFFF & (x >\,> 48)];
    return res;
\}
\end{semiverbatim}
\bigskip

\uncover<2->{
Elle ne demande que quatre lectures dans le tableau \Code{nombre\_un},
au lieu des huit de la méthode précédente.}
\end{frame}

%%%%%%%%%%%%%%%%%%%%%%%%%%%%%%%%%%%%%%%%%%%%%%%%%%%%%%%%%%%%%%%%%%%%%%%%
\begin{frame}[fragile]
\frametitle{Compter le nombre de bits à un}
{\bf $5\ieme$ méthode}~: cette méthode est la plus compliquée. Elle se base
sur une compréhension très fine du fonctionnement des opérateurs
bit à bit.
\bigskip

\begin{semiverbatim}\small\uncover<2->{
int compter_un_5(Mot64 x) \{
    x = x - ((x >\,> 1) & 0x5555555555555555LLU);
    x = (x & 0x3333333333333333LLU) +
        ((x >\,> 2) & 0x3333333333333333LLU);
    x = (x + (x >\,> 4)) & 0x0F0F0F0F0F0F0F0FLLU;
    x = (x * 0x0101010101010101LLU) >\,> 56;
    return (int) x;
\}}
\end{semiverbatim}
\end{frame}

%%%%%%%%%%%%%%%%%%%%%%%%%%%%%%%%%%%%%%%%%%%%%%%%%%%%%%%%%%%%%%%%%%%%%%%%
\begin{frame}[fragile]
\frametitle{Mesure du temps d'exécution}
Mesurons maintenant les efficacités des cinq méthodes.
\medskip

\uncover<2->{
On utilise pour cela la fonction
\begin{center}
    \Code{clock\_t clock(void);}
\end{center}
de \Code{time.h}.
\bigskip}

\uncover<3->{
Schéma général pour mesurer le temps d'exécution d'une suite
d'instructions~:}
\begin{semiverbatim}\small
\uncover<5->{clock_t debut, fin;
double temps;}
\uncover<6->{debut = clock();}
\uncover<4->{...
/* Instructions */
...}
\uncover<7->{fin = clock();}
\uncover<8->{temps = (double) (fin - debut) / CLOCKS_PER_SEC;}
\uncover<9->{printf("%g s\\n", temps);}
\end{semiverbatim}
\end{frame}

%%%%%%%%%%%%%%%%%%%%%%%%%%%%%%%%%%%%%%%%%%%%%%%%%%%%%%%%%%%%%%%%%%%%%%%%
\begin{frame}[fragile]
\frametitle{Mesure du temps d'exécution}
On mesure le temps d'exécution des cinq méthodes en leur donnant
en entrée des nombres de $64$ bits générés de manière aléatoire.
\medskip

\uncover<2->{
Pour générer un nombre de $32$ bits de manière aléatoire, on utilise
la fonction
\begin{center}
    \Code{int rand(void);}
\end{center}
de \Code{stdlib.h}.
\medskip}

\uncover<3->{
Le nombre de $64$ bits est construit en générant aléatoirement ses quatre
octets de droite, puis ses quatre octets de gauche et en les associant avec
un ou bit à bit~:}
\begin{semiverbatim}\small
\uncover<4->{Mot64 mot64_alea() \{}
    \uncover<5->{Mot64 gauche, droite;}
    \uncover<6->{droite = (Mot64) rand();}
    \uncover<7->{gauche = ((Mot64) rand()) <\,< 32;}
    \uncover<8->{return gauche | droite;}
\uncover<4->{\}}
\end{semiverbatim}
\end{frame}

%%%%%%%%%%%%%%%%%%%%%%%%%%%%%%%%%%%%%%%%%%%%%%%%%%%%%%%%%%%%%%%%%%%%%%%%
\begin{frame}[fragile]
\frametitle{Mesure du temps d'exécution}
Voici les temps réalisés par chacune des cinq méthodes sur $84000000$
nombres de $64$ bits aléatoires~:
\begin{center}
    \begin{tabular}{c|c|c}
        {\bf Méthode} & {\bf Caractéristique} & {\bf Temps (s)} \\ \hline
        1 & Décalage droite & 32.9 \\
        2 & Suppression \Code{1} droite & 9.38 \\
        3 & Tableau \Code{256} & 2.23 \\
        4 & Tableau \Code{65536} & 1.93 \\
        5 & Compliquée & 1.82
    \end{tabular}
\end{center}
\medskip

\uncover<2->{
La $4\ieme$ méthode est plus de $16$ fois plus rapide que la $1\iere$.
Elle demande en revanche (tout comme la $3\ieme$) un \alert{pré-calcul}
et une occupation mémoire (par le tableau \Code{nombre\_un}).
\medskip}

\uncover<3->{
La $5\ieme$ méthode est la plus rapide et ne demande aucun pré-calcul.
Elle est en revanche difficile à comprendre et très difficile à
imaginer.}
\end{frame}

%%%%%%%%%%%%%%%%%%%%%%%%%%%%%%%%%%%%%%%%%%%%%%%%%%%%%%%%%%%%%%%%%%%%%%%%
%%%%%%%%%%%%%%%%%%%%%%%%%%%%%%%%%%%%%%%%%%%%%%%%%%%%%%%%%%%%%%%%%%%%%%%%
\subsection{Opérateurs d'affectation}

%%%%%%%%%%%%%%%%%%%%%%%%%%%%%%%%%%%%%%%%%%%%%%%%%%%%%%%%%%%%%%%%%%%%%%%%
\begin{frame}[fragile]
\frametitle{Opérateurs d'affectation}

\begin{center}
    \begin{tabular}{c|c|c|c|c}
        {\bf Op.} & {\bf Rôle} & {\bf Ari.} & {\bf Assoc.}
            & {\bf Opérandes} \\ \hline \hline
        \multirow{2}{*}{\Code{=}} & \multirow{2}{*}{affect.}
            & \multirow{2}{*}{2} & \multirow{2}{*}{$\longleftarrow$}
            & une var. \\
        & & & & et une val. \\ \hline
        \multirow{2}{*}{\Code{+=}, \Code{-=}, \Code{*=}, \Code{/=}, \Code{\%=}}
            & affect. compo.
            & \multirow{2}{*}{2} & $\longleftarrow$ & une var. num. \\
        & arith. & & & et une val. num \\ \hline
        \multirow{2}{*}{\Code{\&=}, \Code{|=}, \Code{\textasciicircum=},
        \Code{<\,<=}, \Code{>\,>=}}
        & affect. compo.
            & \multirow{2}{*}{2} & $\longleftarrow$ & une var. ent. \\
        & bit à bit & & & et une val. ent.
    \end{tabular}
\end{center}
\medskip

\uncover<2->{
Toute expression de la forme \Code{a X= b} est équivalente à
\Code{a = a X b}.}
\end{frame}

%%%%%%%%%%%%%%%%%%%%%%%%%%%%%%%%%%%%%%%%%%%%%%%%%%%%%%%%%%%%%%%%%%%%%%%%
\begin{frame}[fragile]
\frametitle{Opérateurs d'affectation}
Toutes les expressions d'affectation produisent une valeur qui est
la valeur qui vient d'être affectée.
\medskip

\uncover<2->{Par exemple, dans}
\begin{semiverbatim}\small\uncover<2->{
int a, b;
a = 2;
b = 5;
a *= b += 3;}
\end{semiverbatim}
\uncover<2->{
à cause de l'associativité des opérateurs d'affectation, la l. 4
s'interprète comme \Code{a *= (b += 3);}.
\smallskip

Ainsi, comme \Code{b += 3} produit la valeur \Code{8}, \Code{a} vaut
finalement \Code{16}.}
\end{frame}

%%%%%%%%%%%%%%%%%%%%%%%%%%%%%%%%%%%%%%%%%%%%%%%%%%%%%%%%%%%%%%%%%%%%%%%%
%%%%%%%%%%%%%%%%%%%%%%%%%%%%%%%%%%%%%%%%%%%%%%%%%%%%%%%%%%%%%%%%%%%%%%%%
\subsection{Autres opérateurs}

%%%%%%%%%%%%%%%%%%%%%%%%%%%%%%%%%%%%%%%%%%%%%%%%%%%%%%%%%%%%%%%%%%%%%%%%
\begin{frame}[fragile]
\frametitle{Autres opérateurs}

\begin{center}
    \begin{tabular}{c|c|c|c|c}
        {\bf Op.} & {\bf Rôle} & {\bf Ari.} & {\bf Assoc.}
            & {\bf Opérandes} \\ \hline \hline
        \Code{sizeof} & taille & 1 & -- & une var. ou un type \\ \hline
        \multirow{2}{*}{\Code{(T)}} & coercition & \multirow{2}{*}{1}
            & \multirow{2}{*}{--} & \multirow{2}{*}{une val.} \\
            & \Code{T} est un type & & & \\ \hline
        \Code{? :} & condition & 3 & -- & une val. num. et deux val. \\ \hline
        \Code{,}  & séquence & 2 & $\longrightarrow$ & deux val
    \end{tabular}
\end{center}
\end{frame}

%%%%%%%%%%%%%%%%%%%%%%%%%%%%%%%%%%%%%%%%%%%%%%%%%%%%%%%%%%%%%%%%%%%%%%%%
\begin{frame}[fragile]
\frametitle{L'opérateur de séquence}
Dans l'expression \Code{V1, V2}, où \Code{V1} et \Code{V2} sont
des valeurs, on commence par évaluer \Code{V1} puis ensuite \Code{V2}.
Cette expression produit la valeur \Code{V2}.
\bigskip

\uncover<2->{
L'opérateur \Code{,} est le plus souvent utilisé dans les {\bf champs des
boucles \Code{for}}.
\medskip}

\uncover<3->{P.ex.,}
\begin{semiverbatim}\small\uncover<3->{
int i, j, l;
...
for (i = 0, j = l - 1 ; i < j ; ++i, -\,-j) \{
...
\}}
\end{semiverbatim}
\uncover<3->{
permet d'obtenir une boucle \Code{for} avec {\bf deux compteurs}~: \Code{i}
croît et \Code{j} décroît dans l'intervalle allant de \Code{0} à
\Code{l - 1}.}
\end{frame}

% Auteur : Samuele Giraudo
% Création : déc. 2013
% Modifications : août 2014, fév. 2015, déc. 2015

%%%%%%%%%%%%%%%%%%%%%%%%%%%%%%%%%%%%%%%%%%%%%%%%%%%%%%%%%%%%%%%%%%%%%%%%
%%%%%%%%%%%%%%%%%%%%%%%%%%%%%%%%%%%%%%%%%%%%%%%%%%%%%%%%%%%%%%%%%%%%%%%%
%%%%%%%%%%%%%%%%%%%%%%%%%%%%%%%%%%%%%%%%%%%%%%%%%%%%%%%%%%%%%%%%%%%%%%%%
\section{Mémoïsation}

%%%%%%%%%%%%%%%%%%%%%%%%%%%%%%%%%%%%%%%%%%%%%%%%%%%%%%%%%%%%%%%%%%%%%%%%
%%%%%%%%%%%%%%%%%%%%%%%%%%%%%%%%%%%%%%%%%%%%%%%%%%%%%%%%%%%%%%%%%%%%%%%%
\subsection{Variables statiques}

%%%%%%%%%%%%%%%%%%%%%%%%%%%%%%%%%%%%%%%%%%%%%%%%%%%%%%%%%%%%%%%%%%%%%%%%
\begin{frame}[fragile]
\frametitle{Notion de variable statique}
Une \alert{variable statique} est une variable dont la durée de vie est
égale à celle de l'exécution du programme. Une telle variable est créée
et initialisée à la compilation.
\medskip

L'instruction
\begin{center}
    \Code{static T ID;}
\end{center}
déclare une variable statique \Code{ID} de type \Code{T}.
\bigskip
\bigskip
\bigskip

Elles sont les contreparties des {\bf variables automatiques} dont la
durée de vie s'étend à l'exécution du bloc dans lesquels elles se trouvent
et dont la création se fait à l'exécution le moment voulu.
\medskip

Les variables automatiques sont les variables habituelles (se déclarent
sans le mot clé \Code{static}).
\end{frame}

%%%%%%%%%%%%%%%%%%%%%%%%%%%%%%%%%%%%%%%%%%%%%%%%%%%%%%%%%%%%%%%%%%%%%%%%
\begin{frame}[fragile]
\frametitle{Persistance des variables statiques}
Une variable statique est \alert{rémanente}~: elle
{\bf conserve sa dernière valeur} jusqu'à une éventuelle modification.
\medskip

\begin{multicols}{2}
\begin{lstlisting}
void fct(int a) {
    static int s = 3;
    s = s + a;
    printf("%d\n", s);
}
...
fct(1);
fct(5);
fct(2);
\end{lstlisting}

\bigskip
\bigskip
\bigskip
\bigskip
Produit l'affichage \\
\Sortie{4} \\
\Sortie{9} \\
\Sortie{11}.
\end{multicols}

Ici, \Code{s} est une variable statique. La valeur \Code{3} (initialisation)
ne lui est attribuée qu'une fois, lors de la compilation.
\medskip

De plus, \Code{s} conserve sa valeur (qui peut être modifiée) au fil
de l'exécution, d'appel en appel à \Code{fct}.
\end{frame}

%%%%%%%%%%%%%%%%%%%%%%%%%%%%%%%%%%%%%%%%%%%%%%%%%%%%%%%%%%%%%%%%%%%%%%%%
\begin{frame}[fragile]
\frametitle{Géographie de la mémoire lors d'une exécution}
Lors de l'exécution d'un programme, la mémoire qui lui est consacrée
est divisée de la manière suivante.
\begin{multicols}{2}
\begin{center}
    \scalebox{.9}{\begin{tikzpicture}
    \node[Zone,fill=Rouge!30](1)at(0,0){Zone statique};
    \node[Zone,fill=Bleu!30](2)at(0,-1.5){Tas\phantom{q}};
    \node[Zone,fill=Vert!30](3)at(0,-3){\dots\phantom{Tq}};
    \node[Zone,fill=Marron!30](4)at(0,-4.5){Pile\phantom{q}};
    \node[Zone,fill=gray!30,minimum height=.5cm](5)at(0,-5.5){Autre\phantom{q}};
    \draw[->,line width=1.5pt](-2,-1.5)--(-2,-2.5);
    \draw[->,line width=1.5pt](-2,-4.5)--(-2,-3.5);
    \end{tikzpicture}}
\end{center}
Les zones de la mémoire réservées à la {\bf pile} et au {\bf tas} ont
des {\bf tailles variables durant l'exécution}.
\medskip

Le tas grandit vers le bas (vers les adresses hautes) et la pile grandit
vers le haut (vers les adresses basses).
\medskip

La \alert{zone statique}, qui contient les variables statiques et le
code du programme, est de \alert{taille constante durant l'exécution}.
\medskip

Cette taille est connue et calculée lors de la compilation.
\end{multicols}
\end{frame}

%%%%%%%%%%%%%%%%%%%%%%%%%%%%%%%%%%%%%%%%%%%%%%%%%%%%%%%%%%%%%%%%%%%%%%%%
\begin{frame}[fragile]
\frametitle{Compter le nombre d'appels à une fonction}
Étant donnée une fonction, on souhaite compter le nombre de fois qu'elle
a été appelée depuis le lancement du programme.
\medskip

\begin{multicols}{2}
\begin{lstlisting}
int fact(int n) {
    if (n <= 1) return 1;
    return n * fact(n - 1);
}
\end{lstlisting}
\bigskip
\bigskip
\bigskip
\bigskip
\bigskip

\begin{lstlisting}
int fact(int n) {

    /* Mecanisme de comptage */
    static int nb_app = 0;
    nb_app += 1;

    if (n <= 1) return 1;
    return n * fact(n - 1);
}
\end{lstlisting}
\end{multicols}

À chaque instant, la variable statique \Code{nb\_app} a pour valeur le
nombre de fois que \Code{fact} a été appelée.
\medskip

{\bf Problème}~: cette variable a pour portée lexicale le corps de la
fonction \Code{fact} et est invisible ailleurs. On ne peut donc pas
la lire depuis l'extérieur.
\end{frame}

%%%%%%%%%%%%%%%%%%%%%%%%%%%%%%%%%%%%%%%%%%%%%%%%%%%%%%%%%%%%%%%%%%%%%%%%
\begin{frame}[fragile]
\frametitle{Compter le nombre d'appels à une fonction}
Pour avoir accès à la valeur de \Code{nb\_app} hors du corps de \Code{fact},
on la transmet à l'aide d'un {\bf passage par adresse}.
\begin{lstlisting}
int fact(int n, int *nb) {

    /* Mecanisme de comptage */
    static int nb_app = 0;
    nb_app += 1;
    *nb = nb_app;

    if (n <= 1) return 1;
    return n * fact(n - 1, nb);
}
\end{lstlisting}
\bigskip

On l'utilise se la manière suivante~:
\begin{multicols}{2}
\begin{lstlisting}
int nb;
fact(6, &nb);
printf("%d\n", nb);
\end{lstlisting}
Ceci affiche \Sortie{6}.
\bigskip
\bigskip
\bigskip
\end{multicols}
\end{frame}

%%%%%%%%%%%%%%%%%%%%%%%%%%%%%%%%%%%%%%%%%%%%%%%%%%%%%%%%%%%%%%%%%%%%%%%%
%%%%%%%%%%%%%%%%%%%%%%%%%%%%%%%%%%%%%%%%%%%%%%%%%%%%%%%%%%%%%%%%%%%%%%%%
\subsection{Mémoïsation}

%%%%%%%%%%%%%%%%%%%%%%%%%%%%%%%%%%%%%%%%%%%%%%%%%%%%%%%%%%%%%%%%%%%%%%%%
\begin{frame}[fragile]
\frametitle{Principe de la mémoïsation}
Soit \Code{f} une fonction {\bf sans effet de bord} et
{\bf déterministe} (sa valeur de retour ne dépend que de ses arguments).
\bigskip
\bigskip

{\bf Observation}~: si l'on appelle \Code{f} deux fois avec des
mêmes arguments, elle renverra deux fois la même valeur.
\bigskip

{\bf Conséquence}~: comme il est inutile (et inefficace) de refaire deux
fois un même calcul, il suffit de mémoriser le résultat renvoyé par
\Code{f} lors du 1\ier{} appel et de le renvoyer sans calcul lors du
2\ieme{}.
\bigskip

{\bf Marche à suivre}~: on enregistre toutes les valeurs de retour de
\Code{f} au fur et à mesure de ses appels.
\end{frame}

%%%%%%%%%%%%%%%%%%%%%%%%%%%%%%%%%%%%%%%%%%%%%%%%%%%%%%%%%%%%%%%%%%%%%%%%
\begin{frame}[fragile]
\frametitle{Détails sur la mémoïsation}
Les résultats renvoyés par \Code{f} sont mémorisés dans une
{\bf table de hachage} \Code{mem} associant à chaque jeu d'arguments
avec lequel elle est appelée la valeur de retour de \Code{f}.
\bigskip
\bigskip

Il est nécessaire de disposer d'une {\bf fonction de hachage} \Code{h}
associant à chaque jeu d'arguments un entier.
\bigskip
\bigskip

Il ne doit y avoir qu'une seule table \Code{mem} pour \Code{f}, dont
la durée de vie est égale à l'exécution du programme et rémanente~:
\Code{mem} est ainsi une \alert{variable statique}, locale à \Code{f}.
\end{frame}

%%%%%%%%%%%%%%%%%%%%%%%%%%%%%%%%%%%%%%%%%%%%%%%%%%%%%%%%%%%%%%%%%%%%%%%%
\begin{frame}[fragile]
\frametitle{Procédé général de mémoïsation d'une fonction}
Soit \Code{f} une fonction à \Code{n} paramètres.
\bigskip

On appelle \Code{f'} la \alert{version mémoïsée} de \Code{f}, définie de la
manière suivante.
\medskip

Lors de l'appel à \Code{f'(a\_1, \dots, a\_n)}~:
\medskip

\begin{enumerate}[(1)]
    \item \label{item:lecture}
    si \Code{mem} contient une donnée associée à \Code{(a\_1, \dots, a\_n)},
    \begin{enumerate}[(a)] \normalsize
        \item renvoyer cette valeur~;
    \end{enumerate}
    \medskip

    \item \label{item:memorisation}
    sinon,
    \begin{enumerate}[(a)] \normalsize
        \item calculer la valeur de retour de \Code{f(a\_1, \dots, a\_n)}~;
        \item {\bf mémoriser} cette valeur dans \Code{mem}~;
        \item renvoyer cette valeur.
    \end{enumerate}
\end{enumerate}
\bigskip

\eqref{item:lecture} est l'étape de {\bf lecture} et \eqref{item:memorisation}
est l'étape de {\bf mémorisation}.
\end{frame}

%%%%%%%%%%%%%%%%%%%%%%%%%%%%%%%%%%%%%%%%%%%%%%%%%%%%%%%%%%%%%%%%%%%%%%%%
\begin{frame}[fragile]
\frametitle{Exemple de mémoïsation d'une fonction à un paramètre}
On souhaite mémoïser la fonction
\begin{lstlisting}
int fibo(int n) {
    if (n <= 1) return n;
    return fibo(n - 1) + fibo(n - 2);
}
\end{lstlisting}
\bigskip

Cette fonction est paramétrée par un entier et renvoie un entier.
La table \Code{mem} est ainsi un tableau d'entiers. La valeur de hachage
d'un argument \Code{n} est \Code{n}.
\medskip

On l'utilise de la manière suivante~:
\begin{enumerate}
    \item elle est de taille \Code{T\_MEM} où \Code{T\_MEM} est une constante
    suffisamment grande~;
    \smallskip

    \item pour tout $\Code{0} \leq \Code{n} < \Code{T\_MEM}$,
    la valeur de \Code{mem[n]} est
    \begin{itemize} \normalsize
        \item \Code{-1} si \Code{fibo(n)} n'a pas encore été appelée~;
        \smallskip

        \item la valeur renvoyée par \Code{fibo(n)} si \Code{fibo} a
        été appelée au moins une fois avec l'argument \Code{n}.
    \end{itemize}
\end{enumerate}

\end{frame}

%%%%%%%%%%%%%%%%%%%%%%%%%%%%%%%%%%%%%%%%%%%%%%%%%%%%%%%%%%%%%%%%%%%%%%%%
\begin{frame}[fragile]
\frametitle{Exemple de mémoïsation d'une fonction à un paramètre}
\begin{multicols}{2}
\begin{lstlisting}
int fibo(int n) {
    /* Table */
    static int mem[T_MEM];

    /* Booleen 1er appel */
    static int init = 1;

    int i, res;

    /* Init. 1er appel */
    if (init) {
        for (i=0;i<T_MEM;++i)
            mem[i] = -1;
        init = 0;
    }

    /* Lecture */
    if (mem[n] != -1)
        return mem[n];

    /* Calcul */
    if (n <= 1)
        res = n;
    else
        res = fibo(n - 1)
            + fibo(n - 2);

    /* Memorisation */
    mem[n] = res;

    return res;
}
\end{lstlisting}
\end{multicols}
\end{frame}

%%%%%%%%%%%%%%%%%%%%%%%%%%%%%%%%%%%%%%%%%%%%%%%%%%%%%%%%%%%%%%%%%%%%%%%%
\begin{frame}[fragile]
\frametitle{Mémoïsation de fonctions à plusieurs paramètres}
Soit \Code{f} une fonction à \Code{n} paramètres.
\bigskip

La table \Code{mem} devient un \alert{tableau de variables d'un type structuré}
\Code{E\_f} composé
\begin{enumerate}
    \item d'un champ pour chaque {\bf paramètre} de \Code{f}~;
    \item d'un champ pour le {\bf résultat} renvoyé par \Code{f} appelé
    avec les arguments spécifiés par les précédents champs~;
    \item d'un champ {\bf booléen} indiquant si le résultat de \Code{f}
    appelée avec les arguments spécifiés par les précédents champs a bien
    été calculé.
\end{enumerate}
\bigskip

Il faut de plus disposer d'une \alert{fonction de hachage} \Code{h\_f}
de même signature que \Code{f} et qui renvoie un entier positif.
\medskip

Elle permet d'attribuer à chaque jeu d'arguments une position dans la
table \Code{mem}.
\end{frame}

%%%%%%%%%%%%%%%%%%%%%%%%%%%%%%%%%%%%%%%%%%%%%%%%%%%%%%%%%%%%%%%%%%%%%%%%
\begin{frame}[fragile]
\frametitle{Exemple de mémoïsation d'une fonction à deux paramètres}
On souhaite mémoïser la fonction
\begin{lstlisting}
int puiss(int n, int p) {
    if (p == 0) return 1;
    return n * puiss(n, p - 1);
}
\end{lstlisting}
\medskip

Celle-ci est paramétrée par deux entiers. Le {\bf type structuré}
\Code{E\_puiss} est donc défini par
\begin{lstlisting}
typedef struct {
    int n; /* 1er parametre */
    int p; /* 2e parametre */
    int res; /* Resultat */
    int ok; /* Indique si le calcul a deja ete fait */
} E_puiss;
\end{lstlisting}
\medskip

On utilise la {\bf fonction de hachage} suivante~:
\begin{lstlisting}
int h_puiss(int n, int p) {
    return n ^ (p << 2);
}
\end{lstlisting}
\end{frame}

%%%%%%%%%%%%%%%%%%%%%%%%%%%%%%%%%%%%%%%%%%%%%%%%%%%%%%%%%%%%%%%%%%%%%%%%
\begin{frame}[fragile]
\frametitle{Exemple de mémoïsation d'une fonction à deux paramètres}
\begin{multicols}{2}
\begin{lstlisting}
int puiss(int n, int p) {
    /* Table */
    static E_puiss mem[T_MEM];

    /* Booleen 1er appel */
    static int init = 1;

    int i, res, h;

    /* Init. 1er appel */
    if (init) {
        for (i=0;i<T_MEM;++i)
            mem[i].ok = 0;
        init = 0;
    }

    /* Lecture */
    h = h_puiss(n, p) % T_MEM;
    if (mem[h].ok) {
        if (mem[h].n == n
            && mem[h].p == p)
            return mem[h].res;
    }

    /* Calcul */
    if (p == 0) res = 1;
    else
      res = n * puiss(n, p - 1);

    /* Memorisation */
    mem[h].n = n;
    mem[h].p = p;
    mem[h].res = res;
    mem[h].ok = 1;

    return res;
}
\end{lstlisting}
\end{multicols}
\end{frame}

% Auteur : Samuele Giraudo
% Création : mars 2014 jan 2015, déc. 2015

%%%%%%%%%%%%%%%%%%%%%%%%%%%%%%%%%%%%%%%%%%%%%%%%%%%%%%%%%%%%%%%%%%%%%%%%
%%%%%%%%%%%%%%%%%%%%%%%%%%%%%%%%%%%%%%%%%%%%%%%%%%%%%%%%%%%%%%%%%%%%%%%%
%%%%%%%%%%%%%%%%%%%%%%%%%%%%%%%%%%%%%%%%%%%%%%%%%%%%%%%%%%%%%%%%%%%%%%%%
\section{Génération aléatoire}

%%%%%%%%%%%%%%%%%%%%%%%%%%%%%%%%%%%%%%%%%%%%%%%%%%%%%%%%%%%%%%%%%%%%%%%%
%%%%%%%%%%%%%%%%%%%%%%%%%%%%%%%%%%%%%%%%%%%%%%%%%%%%%%%%%%%%%%%%%%%%%%%%
\subsection{Générateurs aléatoires}

%%%%%%%%%%%%%%%%%%%%%%%%%%%%%%%%%%%%%%%%%%%%%%%%%%%%%%%%%%%%%%%%%%%%%%%%
\begin{frame}[fragile]\frametitle{Générateurs aléatoires}
Un \alert{générateur aléatoire} sur un ensemble $E$ d'objets est une
fonction $g$ à valeur dans $E$ non déterministe. La valeur renvoyée
par $g$ n'est pas nécessairement la même d'un appel à un autre.
\bigskip

On cherche dans la mesure du possible à faire en sorte que, étant
donné un ensemble $E$ d'objets, tout objet de $E$ ait la même probabilité
d'être renvoyé par le générateur aléatoire $g$. On parle alors de
\alert{générateur aléatoire uniforme}.
\bigskip
\bigskip

Par exemple, si $E$ est l'ensemble des faces d'un dé à six faces, le
procédé $g$ qui consiste à lancer le dé et renvoyer la face visible constitue
un générateur aléatoire uniforme (et non uniforme si le dé est pipé).
\end{frame}

%%%%%%%%%%%%%%%%%%%%%%%%%%%%%%%%%%%%%%%%%%%%%%%%%%%%%%%%%%%%%%%%%%%%%%%%
\begin{frame}[fragile]\frametitle{Intérêt des générateurs aléatoires}
Les générateurs aléatoires offrent de nombreuses applications. Ils sont
utilisés pour~:
\smallskip

\begin{enumerate}
    \item les {\bf méthodes de Monte-Carlo} pour obtenir des solutions
    approchées à des problèmes~;
    \smallskip

    \item {\bf tester des algorithmes} en générant des entrées de manière
    aléatoire (permutations, listes, arbres, graphes, {\em etc.})~;
    \smallskip

    \item {\bf rompre le caractère déterministe} d'un programme en y
    incluant des éléments imprévisibles (dans les jeux par exemple)~;
    \smallskip

    \item générer des {\bf mots de passe} ou des {\bf clés de chiffrement}.
\end{enumerate}
\end{frame}

%%%%%%%%%%%%%%%%%%%%%%%%%%%%%%%%%%%%%%%%%%%%%%%%%%%%%%%%%%%%%%%%%%%%%%%%
\begin{frame}[fragile]
\frametitle{Universalité des générateurs aléatoires d'entiers}
Un \alert{générateur aléatoire d'entiers d'ordre $n$} est un générateur
aléatoire sur l'ensemble $\{0, 1, 2, \dots, n - 1\}$.
\bigskip
\bigskip

Pour construire un générateur aléatoire sur un ensemble $E$ d'objets,
il suffit en pratique d'avoir un générateur aléatoire d'entiers d'ordre
$\# E$.
\medskip

En effet, il suffit de placer les objets (ou mieux~: les adresses des
objets) de $E$ dans un tableau
\begin{center}
    \begin{tabular}{|c|c|c|c|} \hline
        $e_0$ & $e_1$ & $\dots$ & $e_{n - 1}$ \\ \hline
    \end{tabular}\,,
\end{center}
d'appeler $g$ et de renvoyer l'objet du tableau figurant à l'indice
spécifié par l'entier généré par $g$.
\bigskip
\bigskip

Ainsi, pour faire de la génération aléatoire d'objets, il est suffisant
(en 1\iere{} approximation) de
{\bf savoir construire des générateurs aléatoires d'entiers} d'un ordre
suffisamment grand.
\end{frame}

%%%%%%%%%%%%%%%%%%%%%%%%%%%%%%%%%%%%%%%%%%%%%%%%%%%%%%%%%%%%%%%%%%%%%%%%
\begin{frame}[fragile]
\frametitle{Réduire l'ordre d'un générateur aléatoire d'entiers}
L'opérateur \og {\bf modulo} \fg{} offre un moyen très simple pour
{\bf réduire l'ordre} d'un générateur aléatoire d'entiers.
\medskip

En effet, supposons que l'on dispose d'un générateur aléatoire d'entiers
$g$ d'ordre $n$. On souhaite obtenir un générateur aléatoire d'entiers
$h$ d'ordre $k$ avec $1 \leq k \leq n$. On pose pour cela
\begin{equation*} h := g \mod k. \end{equation*}
\medskip

Il est clair que les entiers générés par $h$ sont dans l'ensemble
$\{0, \dots, k - 1\}$. On a ainsi réduit l'ordre du générateur $g$.
\bigskip
\bigskip

On dit que $h$ est la \alert{réduction} à l'ordre $k$ de $g$.
\end{frame}

%%%%%%%%%%%%%%%%%%%%%%%%%%%%%%%%%%%%%%%%%%%%%%%%%%%%%%%%%%%%%%%%%%%%%%%%
\begin{frame}[fragile]
\frametitle{Réduction de l'ordre et uniformité}
Supposons que $g$ soit un générateur aléatoire uniforme d'ordre $n := 4$
et considérons la réduction $h$ à l'ordre $k := 3$ de $g$.
\medskip

On a la situation suivante~:
\begin{center}
    \begin{tikzpicture}
        \node at(-1,0){$g :$};
        \node(0)at(0,0){$0$};
        \node(1)at(1,0){$1$};
        \node(2)at(2,0){$2$};
        \node(3)at(3,0){$3$};
        \node at(-1,-1){$h :$};
        \node(00)at(.5,-1){$0$};
        \node(11)at(1.5,-1){$1$};
        \node(22)at(2.5,-1){$2$};
        \draw[->](0)--(00);
        \draw[->](1)--(11);
        \draw[->](2)--(22);
        \draw(3)edge[->,out=-120,in=45]node{}(00);
    \end{tikzpicture}
\end{center}
Il y a ainsi deux manières de générer $0$ par $h$ (probabilité de $\frac{1}{2}$)
alors qu'il n'y en a qu'une seule pour $1$ et $2$ (probabilités resp.
de $\frac{1}{4}$).
\bigskip

Ceci montre que le générateur aléatoire $h$ \alert{perd la propriété d'uniformité}.
\end{frame}

%%%%%%%%%%%%%%%%%%%%%%%%%%%%%%%%%%%%%%%%%%%%%%%%%%%%%%%%%%%%%%%%%%%%%%%%
\begin{frame}[fragile]
\frametitle{Mesure de la non uniformité de la réduction}
Soient $g$ un générateur aléatoire uniforme d'ordre $n$ et $h$ la
réduction de $g$ à l'ordre $k$, où $k \leq n$.
\medskip

Notons $\mathrm{gen}_h(i)$ le nombre de manières de générer l'entier
$i \leq k - 1$ par $h$. Alors, on voit facilement que
\begin{equation*}
    \mathrm{gen}_h(i) =
    \begin{cases}
        \lfloor \frac{n}{k} \rfloor + 1 &
            \mbox{si } i \in \{0, \dots, (n \mod k) -1\} \\
        \lfloor \frac{n}{k} \rfloor &
            \mbox{sinon}.
    \end{cases}
\end{equation*}

La probabilité $\mathbb{P}_h(i)$ de générer l'entier $i \leq k - 1$ par
$h$ vérifie
\begin{equation*}
    \mathbb{P}_h(i) = \frac{\mathrm{gen}_h(i)}{n}.
\end{equation*}

Les entiers $i$ de l'ensemble $\{0, \dots, k - 1\}$ ont ainsi des
probabilités différentes d'être générés. Ils se divisent en deux classes~:
\begin{enumerate}
    \item {\bf les petits}, de $0$ à $(n \mod k) - 1$, avec une probabilité
    plus grande~;
    \item {\bf les grands}, de $(n \mod k)$ à $k - 1$, avec une probabilité
    plus petite.
\end{enumerate}
\end{frame}

%%%%%%%%%%%%%%%%%%%%%%%%%%%%%%%%%%%%%%%%%%%%%%%%%%%%%%%%%%%%%%%%%%%%%%%%
\begin{frame}[fragile]
\frametitle{Mesure de la non uniformité de la réduction}
Comparons numériquement les probabilités de génération des petits et des
grands nombres pour différentes valeurs de $n$ et de $k$~:
\begin{center}
    \footnotesize
    \begin{tabular}{c|c||c|c|c}
        $n$ & $k$ & Prob. des petits & Prob. des grands & Rapport \\ \hline \hline
        $4$ & $3$ & $0.5$ & $0.25$ & $2$ \\
        $400$ & $300$ & $0.005$ & $0.0025$ & $2$ \\
        $400$ & $200$ & $0.005$ & $0.005$ & $1$ \\
        $400$ & $101$ & $0.01$ & $0.0075$ & $\simeq1.3333$ \\
        $400$ & $99$ & $0.0125$ & $0.01$ & $1.25$ \\
        $500$ & $99$ & $0.012$ & $0.01$ & $1.2$ \\
        $5000$ & $99$ & $0.0102$ & $0.01$ & $1.02$ \\
        $50000$ & $99$ & $0.01012$ & $0.0101$ & $\simeq1.00198$
    \end{tabular}
\end{center}

On observe que plus $k$ est {\bf petit} par rapport à $n$, plus $h$ se
rapproche d'un générateur {\bf uniforme}. Lorsque $k$ est un diviseur
de $n$, l'uniformité est immédiate.
\medskip

En pratique, la réduction est considérée comme préservant l'uniformité,
à condition de partir d'un générateur aléatoire d'entiers uniforme $g$
d'ordre le  plus grand possible.
\end{frame}

%%%%%%%%%%%%%%%%%%%%%%%%%%%%%%%%%%%%%%%%%%%%%%%%%%%%%%%%%%%%%%%%%%%%%%%%
%%%%%%%%%%%%%%%%%%%%%%%%%%%%%%%%%%%%%%%%%%%%%%%%%%%%%%%%%%%%%%%%%%%%%%%%
\subsection{Nombres pseudo-aléatoires}

%%%%%%%%%%%%%%%%%%%%%%%%%%%%%%%%%%%%%%%%%%%%%%%%%%%%%%%%%%%%%%%%%%%%%%%%
\begin{frame}[fragile]\frametitle{Machines et aléatoire}
Un ordinateur est par essence une \alert{machine déterministe}~: le
résultat d'un programme est toujours le même s'il est exécuté deux fois
dans les mêmes conditions.
\bigskip

Il est ainsi impossible d'écrire un programme qui implante un générateur
aléatoire d'entiers.
\bigskip

La seule chose qu'il est possible de faire est de programmer une fonction
qui {\bf semble} le plus possible se comporter comme un générateur aléatoire.
On parle alors de \alert{générateur pseudo-aléatoire}.
\bigskip

Les entiers qu'un générateur pseudo-aléatoire génère lorsqu'on l'appelle
plusieurs fois de suite forme une suite. On appelle cette suite une
suite d'\alert{entiers pseudo-aléatoires}.
\end{frame}

%%%%%%%%%%%%%%%%%%%%%%%%%%%%%%%%%%%%%%%%%%%%%%%%%%%%%%%%%%%%%%%%%%%%%%%%
\begin{frame}[fragile]\frametitle{Principe de fonctionnement}
Un générateur pseudo-aléatoire $g$ d'ordre $n$ fonctionne de la
manière suivante~:
\smallskip

\begin{itemize}
    \item il dispose d'une \alert{graine} $g_1, g_2, \dots, g_\ell$,
    qui est une suite d'entiers~;
    \smallskip

    \item lorsqu'on l'appelle pour la $r\ieme$ fois, il renvoie un
    entier $g_{\ell + r}$ calculé à partir des entiers
    $g_1, \dots, g_\ell, \dots, g_{\ell + r - 1}$ précédemment calculés,
    selon une \alert{règle} spécifiée.
\end{itemize}
\medskip

Ainsi, $g$ produit une suite d'entiers
\begin{equation*}
    g_{\ell + 1}, g_{\ell + 2}, g_{\ell + 3}, \dots
\end{equation*}
à partir de la graine dans laquelle figurent les {\bf données initiales}
et de la règle qui explique comment générer le {\bf prochain entier} de
la suite en se basant sur les {\bf entiers précédemment générés}.
\end{frame}

%%%%%%%%%%%%%%%%%%%%%%%%%%%%%%%%%%%%%%%%%%%%%%%%%%%%%%%%%%%%%%%%%%%%%%%%
\begin{frame}[fragile]\frametitle{Qualités d'un générateur pseudo-aléatoire}
Un bon générateur pseudo-aléatoire produit une suite d'entiers qui
semble aléatoire.
\medskip

Il est possible de mesurer la qualité d'un générateur aléatoire conformément
à plusieurs critères~:
\smallskip

\begin{enumerate}
    \item {\bf uniformité} de la génération~;
    \smallskip

    \item ses {\bf périodes} (longueurs des cycles de génération)~;
    \smallskip

    \item {\bf sensibilité à la graine} (existence de mauvaises graines)~;
    \smallskip

    \item {\bf corrélation} entre un entier engendré et les précédents~;
    \smallskip

    \item {\bf analyse de la distribution} en plusieurs dimensions.
\end{enumerate}
\end{frame}

%%%%%%%%%%%%%%%%%%%%%%%%%%%%%%%%%%%%%%%%%%%%%%%%%%%%%%%%%%%%%%%%%%%%%%%%
\begin{frame}[fragile]\frametitle{Méthode du carré médian}
Le générateur pseudo-aléatoire $g$ utilisant la
\alert{méthode du carré médian} fonctionne de la manière suivante.
\medskip

La graine de ce générateur est un entier compris entre $0$ et $9999$.
Si $g_{i - 1}$ est l'entier précédemment généré (ou bien la graine), le
prochain entier généré est
\begin{equation*}
    g_i := \left(\left\lfloor \frac{g_{i - 1}}{10} \right\rfloor \mod 100\right)^2.
\end{equation*}

On obtient p.ex. les suites
\begin{center} \footnotesize
    \begin{tabular}{c|c}
        Graine & Suite \\ \hline
        1234 & 529, 2704, 4900, 8100, 100, 100, ... \\
        3733 & 5329, 1024, 4, 0, 0, ... \\
        9999 & 9801, 6400, 1600, 3600, 3600, ...
    \end{tabular}
\end{center}
\medskip

C'est un {\bf mauvais générateur pseudo-aléatoire}. Les périodes sont
bien trop courtes et il y a des mauvaises graines (comme $0$).
\end{frame}

%%%%%%%%%%%%%%%%%%%%%%%%%%%%%%%%%%%%%%%%%%%%%%%%%%%%%%%%%%%%%%%%%%%%%%%%
\begin{frame}[fragile]\frametitle{Méthode additive}
Le générateur pseudo-aléatoire $g$ utilisant la \alert{méthode additive}
fonctionne de la manière suivante.
\medskip

On se fixe un ordre $n \geq 1$. La graine de ce générateur est
un triplet $(g_1, g_2, g_3)$ d'entiers compris entre $0$ et $n - 1$.
La génération de $g_i$ dépend des trois entiers précédemment
générés $g_{i - 1}$, $g_{i - 2}$ et $g_{i - 3}$. Il est défini par
\begin{equation*}
    g_i := (g_{i - 1} + g_{i - 2} + g_{i - 3}) \mod n.
\end{equation*}

On obtient p.ex. les suites
\begin{center} \scriptsize
    \begin{tabular}{c|c|c}
        $n$ & Graine & Suite \\ \hline
        211 & $(1, 1, 1)$ & 3, 5, 9, 17, 31, 57, 105, 193, 144, 20, 146,
            99, 54, 88, 30, 172, 79 \\
        3099 & $(1, 1, 1)$ & 3, 5, 9, 17, 31, 57, 105, 193, 355, 653, 1201,
            2209, 964, 1275, 1349 \\
        1347 & $(600, 31, 1)$ & 632, 1263, 1148, 349, 66, 216, 631, 913,
            413, 610, 589, 265, 117
    \end{tabular}
\end{center}
\medskip

Ce générateur pseudo-aléatoire est bien meilleur que le précédent mais
échoue à de nombreux tests statistiques.
\end{frame}

%%%%%%%%%%%%%%%%%%%%%%%%%%%%%%%%%%%%%%%%%%%%%%%%%%%%%%%%%%%%%%%%%%%%%%%%
\begin{frame}[fragile]\frametitle{Générateurs congruentiels linéaires}
Un générateur pseudo-aléatoire $g$ \alert{congruentiel linéaire} fonctionne
de la manière suivante.
\medskip

On se fixe un ordre $n \geq 1$ ainsi que deux entiers positifs $a$ et $b$.
La graine de ce générateur est un entier compris entre $0$ et $n - 1$.
La génération de $g_i$ dépend de l'entier précédemment généré $g_{i - 1}$
et est défini par
\begin{equation*}
    g_i := (a \times g_{i - 1} + b) \mod n.
\end{equation*}

On obtient p.ex. les suites
\begin{center} \scriptsize
    \begin{tabular}{c|c|c|c|c}
        $n$ & $a$ & $b$ & Graine & Suite \\ \hline
        128 & 3 & 1 & 1 & 4, 13, 40, 121, 108, 69, 80, 113, 84, 125,
            120, 105, 60 \\
        128 & 7 & 1 & 1 & 8, 57, 16, 113, 24, 41, 32, 97, 40, 25,
        48, 81, 56, 9, 64 \\
        $2^{32}$ & 1664525 & 1013904223 & 1 &
            1015568748, 1586005467, 2165703038, 3027450565 \\
    \end{tabular}
\end{center}
\medskip

Sous réserve de choisir des bons paramètres $n$, $a$ et $b$, ce
générateur pseudo-aléatoire est plutôt bon. Il est très utilisé en
pratique.
\end{frame}

%%%%%%%%%%%%%%%%%%%%%%%%%%%%%%%%%%%%%%%%%%%%%%%%%%%%%%%%%%%%%%%%%%%%%%%%
%%%%%%%%%%%%%%%%%%%%%%%%%%%%%%%%%%%%%%%%%%%%%%%%%%%%%%%%%%%%%%%%%%%%%%%%
\subsection{Utilisation et implantation}

%%%%%%%%%%%%%%%%%%%%%%%%%%%%%%%%%%%%%%%%%%%%%%%%%%%%%%%%%%%%%%%%%%%%%%%%
\begin{frame}[fragile]\frametitle{Les fonctions {\tt rand} et {\tt srand}}
La fonction
\begin{center} \Code{int rand();} \end{center}
du module \Code{stdlib} implante un générateur pseudo-aléatoire d'ordre
\Code{RAND\_MAX + 1}. C'est un générateur congruentiel linéaire. À chaque
appel, il renvoie le prochain entier de la suite.
\bigskip
\bigskip

La graine de ce générateur peut-être affectée par la fonction
\begin{center} \Code{void srand(unsigned int g1);} \end{center}
Par défaut, la graine du générateur est \Code{1}.
\end{frame}

%%%%%%%%%%%%%%%%%%%%%%%%%%%%%%%%%%%%%%%%%%%%%%%%%%%%%%%%%%%%%%%%%%%%%%%%
\begin{frame}[fragile]
\frametitle{Modèle d'utilisation de {\tt rand} et de {\tt srand}}
\begin{multicols}{2}
Tout projet qui utilise la fonction \Code{rand} doit avoir une fonction
principale \Code{main} de la forme ci-contre.
\smallskip

L'appel à \Code{srand} doit être fait le plus tôt possible.

\begin{center}
\begin{minipage}[c]{.4\textwidth}
\begin{lstlisting}[frame=single,numbers=none,basicstyle=\scriptsize\tt]
/* Main.c */
...
#include <stdlib.h>
#include <time.h>
...
int main() {
    ...
    srand(time(NULL));
    ...
}
\end{lstlisting}
\end{minipage}
\end{center}
\end{multicols}

La fonction \Code{time\_t time (time\_t* timer);}
du module \Code{time}, lorsque appelée avec l'argument \Code{NULL}
renvoie le temps écoulé en secondes depuis le 1\ier{} janvier 1970 à
minuit, UTC.
\medskip

Considérer cette valeur est donc un bon moyen d'initialiser la graine
du générateur.
\medskip

{\bf Attention} ({\em point très important})~: il est totalement erroné
d'initialiser dans un même projet deux fois la graine. Le seul appel à
\Code{srand} doit figurer dans \Code{main} et nulle part ailleurs.
\end{frame}

%%%%%%%%%%%%%%%%%%%%%%%%%%%%%%%%%%%%%%%%%%%%%%%%%%%%%%%%%%%%%%%%%%%%%%%%
\begin{frame}[fragile]\frametitle{Implantation avec variable globale}
Une implantation possible des fonctions \Code{rand} et \Code{srand}
s'appuie sur une {\bf variable globale} pour mémoriser la dernière valeur
générée.
\medskip

\begin{center}
\begin{minipage}[c]{.41\textwidth}
\begin{lstlisting}[frame=single,numbers=none,basicstyle=\scriptsize\tt]
/* RandPerso.h */
#ifndef __RAND_PERSO__
#define __RAND_PERSO__

    #define RAND_PERSO_ORDRE 32767
    #define RAND_PERSO_A 1024
    #define RAND_PERSO_B 1
    #define RAND_PERSO_GRAINE 1

    int rand_perso();
    void srand_perso(int graine);

#endif
\end{lstlisting}
\end{minipage}
\quad
\begin{minipage}[c]{.45\textwidth}
\begin{lstlisting}[frame=single,numbers=none,basicstyle=\scriptsize\tt]
/* RandPerso.c */
#include "RandPerso.h"

unsigned int valeur_rand =
    RAND_PERSO_GRAINE;

int rand_perso() {
    valeur_rand =
        (valeur_rand * RAND_PERSO_A
            + RAND_PERSO_B)
        % RAND_PERSO_ORDRE;
    return valeur_rand;
}

void srand_perso(int graine) {
    valeur_rand = graine;
}
\end{lstlisting}
\end{minipage}
\end{center}
\end{frame}

%%%%%%%%%%%%%%%%%%%%%%%%%%%%%%%%%%%%%%%%%%%%%%%%%%%%%%%%%%%%%%%%%%%%%%%%
\begin{frame}[fragile]\frametitle{Implantation avec variable statique}
Dans cette implantation, on utilise une {\bf variable statique} pour
mémoriser la dernière valeur générée.

\begin{center}
\begin{minipage}[c]{.41\textwidth}
\begin{lstlisting}[frame=single,numbers=none,basicstyle=\scriptsize\tt]
/* RandPerso.h */
#ifndef __RAND_PERSO__
#define __RAND_PERSO__

    #define RAND_PERSO_ORDRE 32767
    #define RAND_PERSO_A 1024
    #define RAND_PERSO_B 1
    #define RAND_PERSO_GRAINE 1

    int rand_perso();

#endif
\end{lstlisting}
\end{minipage}
\quad
\begin{minipage}[c]{.45\textwidth}
\begin{lstlisting}[frame=single,numbers=none,basicstyle=\scriptsize\tt]
/* RandPerso.c */
#include "RandPerso.h"

#include <time.h>

int rand_perso() {
    static int premier_appel = 1;
    static unsigned int valeur_rand;
    if (premier_appel) {
        valeur_rand = time(NULL);
        premier_appel = 0;
    }
    valeur_rand =
        (valeur_rand * RAND_PERSO_A
            + RAND_PERSO_B)
        % RAND_PERSO_ORDRE;
    return valeur_rand;
}
\end{lstlisting}
\end{minipage}
\end{center}

\begin{small}
L'initialisation ne se fait qu'au 1\ier{} appel et l'utilisateur n'a
pas à la faire explicitement. Avec cette méthode, il n'est pas possible
de fixer la graine.
\end{small}
\end{frame}

%%%%%%%%%%%%%%%%%%%%%%%%%%%%%%%%%%%%%%%%%%%%%%%%%%%%%%%%%%%%%%%%%%%%%%%%
\begin{frame}[fragile]\frametitle{Retrouver le déterminisme}
Pour pouvoir reproduire les résultats fournis par un programme utilisant
la fonction \Code{rand}, il est nécessaire de proposer un
\alert{mode de fonctionnement déterministe} dans lequel la graine a été
fixée.
\medskip

Pour cela, on met en place dans le module principal d'un projet et dans sa
fonction \Code{main} le mécanisme suivant~:
\begin{multicols}{2}
\begin{center}
\begin{minipage}[c]{.35\textwidth}
\begin{lstlisting}[frame=single,numbers=none,basicstyle=\scriptsize\tt]
/* Main.c */
...
#define DETERMINISTE
...
int main() {
    ...
#ifdef DETERMINISTE
    srand(0);
#else
    srand(time(NULL));
#endif
    ...
}
\end{lstlisting}
\end{minipage}
\end{center}
De cette manière, on peut imposer un comportement déterministe
({\bf reproductible}) à l'exécution du projet en gardant la définition
de la macro \Code{DETERMINISTE}.
\smallskip

Pour éviter ce comportement, il suffit de commenter cette macro-définition.
\end{multicols}

\begin{footnotesize}
Meilleure solution~: utiliser l'option \Code{-D} de \Code{gcc}. Il
suffit de supprimer la ligne \Code{\#define DETERMINISTE} et de compiler
avec \Code{-DDETERMINISTE} pour définir la macro en question.
\end{footnotesize}
\end{frame}

% Auteur : Samuele Giraudo
% Création : déc. 2013
% Modifications : août 2014, déc. 2015

%%%%%%%%%%%%%%%%%%%%%%%%%%%%%%%%%%%%%%%%%%%%%%%%%%%%%%%%%%%%%%%%%%%%%%%%
%%%%%%%%%%%%%%%%%%%%%%%%%%%%%%%%%%%%%%%%%%%%%%%%%%%%%%%%%%%%%%%%%%%%%%%%
%%%%%%%%%%%%%%%%%%%%%%%%%%%%%%%%%%%%%%%%%%%%%%%%%%%%%%%%%%%%%%%%%%%%%%%%
\section{Pointeurs de fonction}

%%%%%%%%%%%%%%%%%%%%%%%%%%%%%%%%%%%%%%%%%%%%%%%%%%%%%%%%%%%%%%%%%%%%%%%%
%%%%%%%%%%%%%%%%%%%%%%%%%%%%%%%%%%%%%%%%%%%%%%%%%%%%%%%%%%%%%%%%%%%%%%%%
\subsection{Principe}

%%%%%%%%%%%%%%%%%%%%%%%%%%%%%%%%%%%%%%%%%%%%%%%%%%%%%%%%%%%%%%%%%%%%%%%%
\begin{frame}[fragile]
\frametitle{Pointeurs de fonction}
En {\sf C}, on peut manipuler divers types d'objets~: des variables
d'un type scalaire, des tableaux, des variables d'un type structuré,
des adresses, {\em etc.}
\bigskip

\uncover<2->{
À l'inverse, les {\bf fonctions} n'entrent pas dans cette catégorie
d'objets directement manipulables.
\bigskip}

\uncover<3->{
Cependant, au même titre qu'une variable, toute fonction possède une
{\bf adresse} en mémoire. Il devient alors possible de réaliser des
opérations sur les fonctions au moyen de leur adresse.
\bigskip}

\uncover<4->{
On parle alors de \alert{pointeur de fonction}.}
\end{frame}

%%%%%%%%%%%%%%%%%%%%%%%%%%%%%%%%%%%%%%%%%%%%%%%%%%%%%%%%%%%%%%%%%%%%%%%%
\begin{frame}[fragile]
\frametitle{Adresse d'une fonction}
Si \Code{fct} est une fonction, la syntaxe
\begin{center} \Code{\&fct} \end{center}
permet d'accéder à l'\alert{adresse} de \Code{fct}.
\bigskip

\begin{multicols}{2}
\begin{semiverbatim}\small\uncover<2->{
#include <stdio.h>
int somme(int a, int b) \{
    return a + b;
\}
int produit(int a, int b) \{
    return a * b;
\}
int main() \{
    printf("%p\\n", &somme);
    printf("%p\\n", &produit);
    return 0;
\}}
\end{semiverbatim}
\uncover<2->{
Ce programme affiche \Sortie{0x40052d} et \Sortie{0x400541},
respectivement les adresses des fonctions \Code{somme} et \Code{produit}.
\bigskip}

\uncover<3->{
{\bf Note}~: aux lignes 9 et 10, il est possible de ne pas mentionner
les \Code{\&}. Le compilateur comprend implicitement qu'il s'agit de
pointeurs de fonction.}
\end{multicols}
\end{frame}

%%%%%%%%%%%%%%%%%%%%%%%%%%%%%%%%%%%%%%%%%%%%%%%%%%%%%%%%%%%%%%%%%%%%%%%%
\begin{frame}[fragile]
\frametitle{Le type pointeur de fonction}
La syntaxe \vspace{-.5em}
\begin{center} \Code{T (*FCT)(T1, \dots, TN);} \end{center} \vspace{-.5em}
où

\begin{itemize}
    \item \Code{T}, \Code{T1}, \dots, \Code{TN} sont des types~;
    \smallskip

    \item \Code{FCT} est un identificateur~;
\end{itemize}
permet de \alert{déclarer} un pointeur de fonction.
\smallskip

\uncover<2->{
Celui-ci a \Code{FCT} pour identificateur et peut être l'adresse d'une
fonction de type de retour \Code{T} et de signature \Code{(T1, \dots, TN)}.
\bigskip}

\begin{multicols}{2}
\begin{semiverbatim}\footnotesize\uncover<3->{
float moyenne(int a, int b) \{
    return (0.0 + a + b) / 2;
\}
...
/* Decl. d'un ptr de fonction */
float (*moy)(int, int);

/* Utilisation */
moy = &moyenne;
printf("%f\\n", moy(2, 3));}
\end{semiverbatim}
\uncover<3->{
Pour la même raison que dans l'exemple précédent, il est possible
à la ligne 9 de ne pas mentionner le~\Code{\&}.
\bigskip

Cependant, pour la clarté du code, nous prenons la {\bf convention de
mentionner tous les  \Code{\&}}}.
\end{multicols}
\end{frame}

%%%%%%%%%%%%%%%%%%%%%%%%%%%%%%%%%%%%%%%%%%%%%%%%%%%%%%%%%%%%%%%%%%%%%%%%
\begin{frame}[fragile]
\frametitle{Champs pointeurs de fonction}
Un \alert{champ d'un type structuré} peut être un pointeur sur une fonction.
\medskip

\uncover<2->{
Ceci déclare un type structuré sensé modéliser des suites d'entiers~:}
\begin{semiverbatim}\small\uncover<2->{
typedef struct \{
    int t;
    int (*t_suiv)(int);
\} Suite;}
\end{semiverbatim}
\uncover<2->{
\Code{t} contient le terme courant de la suite et \Code{t\_suiv}
est la fonction qui, étant donné un terme en entrée, calcule le terme
suivant.}

\begin{multicols}{2}
\begin{semiverbatim}\small\uncover<3->{
int mul_2(int t) \{
    return 2 * t;
\}
...
int i;
Suite s;
s.t = 1;
s.t_suiv = &mul_2;
for (i = 0 ; i < 6 ; ++i) \{
    printf("%d ", s.t);
    s.t = s.t_suiv(s.t);
\}}
\end{semiverbatim}
\end{multicols}
\uncover<3->{Ceci affiche \Sortie{1 2 4 8 16 32}.}
\end{frame}

%%%%%%%%%%%%%%%%%%%%%%%%%%%%%%%%%%%%%%%%%%%%%%%%%%%%%%%%%%%%%%%%%%%%%%%%
\begin{frame}[fragile]
\frametitle{Tableaux de pointeurs de fonction}
Il est possible de manipuler des \alert{tableaux de pointeurs de fonction}.
\medskip

\uncover<2->{
Pour cela, on procède en deux étapes~:
\begin{enumerate}
    \item on définit un type alias pour le pointeur de fonction. Syntaxe~:
    \begin{center}
        \Code{typedef T (*FCT)(T1, \dots, TN);}
    \end{center}
    \smallskip}

    \uncover<3->{
    \item on déclare ensuite le tableau de manière usuelle. Syntaxe~:
    \begin{center}
        \Code{FCT tab[M];}
    \end{center}}
\end{enumerate}
\bigskip

\uncover<4->{
La syntaxe plus directe
\begin{center}
    \Code{T (*tab[M])(T1, \dots, TN);}
\end{center}
existe mais rend le code plus difficile à lire. Elle déclare un
tableau \Code{tab} de pointeurs de fonction sans la déclaration de type
préalable de la $1\iere$ méthode.}
\end{frame}

%%%%%%%%%%%%%%%%%%%%%%%%%%%%%%%%%%%%%%%%%%%%%%%%%%%%%%%%%%%%%%%%%%%%%%%%
\begin{frame}[fragile]
\frametitle{Tableaux de pointeurs de fonction}

\begin{multicols}{2}
\begin{semiverbatim}\small
/* Alias pour ptr. de fct. */
typedef int (*opb)(int, int);

int add(int a, int b) \{
    return a + b;
\}
int mul(int a, int b) \{
    return a * b;
\}
...
opb tab[2];

tab[0] = &add;
tab[1] = &mul;
printf("%d %d\\n",
    tab[0](10, 20),
    tab[1](10, 20));
\end{semiverbatim}
\end{multicols}
Ceci affiche \Sortie{30 200}.
\bigskip

\uncover<2->{
Les tableaux de pointeurs de fonction peuvent être {\bf dynamiques}. La
ligne 10 peut être remplacée par}
\begin{semiverbatim}\small\uncover<2->{
opb *tab;
tab = (opb *) malloc(sizeof(opb) * 2);}
\end{semiverbatim}
\end{frame}

%%%%%%%%%%%%%%%%%%%%%%%%%%%%%%%%%%%%%%%%%%%%%%%%%%%%%%%%%%%%%%%%%%%%%%%%
%%%%%%%%%%%%%%%%%%%%%%%%%%%%%%%%%%%%%%%%%%%%%%%%%%%%%%%%%%%%%%%%%%%%%%%%
\subsection{En paramètre et en retour}

%%%%%%%%%%%%%%%%%%%%%%%%%%%%%%%%%%%%%%%%%%%%%%%%%%%%%%%%%%%%%%%%%%%%%%%%
\begin{frame}[fragile]
\frametitle{Pointeur de fonction en paramètre}
Une fonction peut être \alert{paramétrée par un pointeur de fonction}.
Un paramètre pointeur de fonction est spécifié avec la même syntaxe que
celle qui sert à le déclarer.
\bigskip

\uncover<2->{P.ex.,}
\begin{semiverbatim}\small\uncover<2->{
int appliquer(int n, int k, int (*f)(int)) \{
    int i;
    for (i = 0 ; i < k ; ++i)
        n = f(n);
    return n;
\}}
\end{semiverbatim}
\uncover<2->{
est une fonction est paramétrée par un pointeur de fonction acceptant un
entier et renvoyant un entier.}
\end{frame}

%%%%%%%%%%%%%%%%%%%%%%%%%%%%%%%%%%%%%%%%%%%%%%%%%%%%%%%%%%%%%%%%%%%%%%%%
\begin{frame}[fragile]
\frametitle{Pointeur de fonction en paramètre}
\begin{multicols}{2}
\begin{semiverbatim}\small
int add_1(int n) \{
    return n + 1;
\}

int mul_2(int n) \{
    return 2 * n;
\}

...
printf("%d\\n",
    appliquer(3, 4, &add_1));

printf("%d\\n",
    appliquer(3, 4, &mul_2));
\end{semiverbatim}
\end{multicols}

\uncover<2->{
Le 1\ier{} appel à \Code{appliquer} calcule
\begin{center}
    \Code{(((3 + 1) + 1) + 1) + 1}
\end{center}
et affiche donc \Sortie{7}.
\medskip}

\uncover<3->{
Le 2\ieme{} appel à \Code{appliquer} calcule
\begin{center}
    \Code{(((3 * 2) * 2) * 2) * 2}
\end{center}
et affiche donc \Sortie{48}.}
\end{frame}

%%%%%%%%%%%%%%%%%%%%%%%%%%%%%%%%%%%%%%%%%%%%%%%%%%%%%%%%%%%%%%%%%%%%%%%%
\begin{frame}[fragile]
\frametitle{Renvoi d'un pointeur de fonction}
Il est possible de définir des fonctions dont le \alert{type de retour}
est un \alert{pointeur de fonction}.
\medskip

\uncover<2->{
Pour cela, on procède en deux étapes~:
\begin{enumerate}
    \item on définit un type alias \Code{R} pour le pointeur de fonction
    que l'on souhaite renvoyer~;
    \smallskip}

    \uncover<3->{
    \item on définit la fonction souhaitée, dont le type de retour
    est \Code{R}.}
\end{enumerate}
\bigskip

\uncover<4->{
La syntaxe plus directe
\smallskip

\Code{R (*FCT(T1 ARG1, \dots, TN ARGN))(R1, \dots, RM) $\lbrace$ \\
    \qquad ... \\
$\rbrace$}
\smallskip

permet de définir directement une fonction \Code{FCT} de signature \\
\Code{(T1, \dots, TN)} renvoyant l'adresse d'une fonction de type de
retour \Code{R} et de signature \Code{(R1, \dots, RM)}. Cependant,
le code devient illisible.}
\end{frame}

%%%%%%%%%%%%%%%%%%%%%%%%%%%%%%%%%%%%%%%%%%%%%%%%%%%%%%%%%%%%%%%%%%%%%%%%
\begin{frame}[fragile]
\frametitle{Renvoi d'un pointeur de fonction}
Exemple~: opération aléatoire sur des entiers.

\begin{multicols}{2}
\begin{semiverbatim}\small
/* Definition des operations */
int add(int a, int b) \{
    return a + b;
\}
int sub(int a, int b) \{
    return a - b;
\}
int mod(int a, int b) \{
    return a % b;
\}
/* Type de retour */
typedef int (*opb)(int, int);

opb op_alea() \{
    opb tab[3];
    tab[0] = &add;
    tab[1] = &sub;
    tab[2] = &mod;
    return tab[rand() % 3];
\}
\end{semiverbatim}
\end{multicols}

\uncover<2->{
On peut utiliser \Code{op\_alea} de la manière suivante~:}
\begin{semiverbatim}\small\uncover<2->{
int n;
n = op_alea()(3, 4);}
\end{semiverbatim}
\uncover<2->{
Ceci affecte, de manière aléatoire, \Code{7}, \Code{-1} ou \Code{3}
à \Code{n}.}
\end{frame}

%%%%%%%%%%%%%%%%%%%%%%%%%%%%%%%%%%%%%%%%%%%%%%%%%%%%%%%%%%%%%%%%%%%%%%%%
%%%%%%%%%%%%%%%%%%%%%%%%%%%%%%%%%%%%%%%%%%%%%%%%%%%%%%%%%%%%%%%%%%%%%%%%
\subsection{Généricité}

%%%%%%%%%%%%%%%%%%%%%%%%%%%%%%%%%%%%%%%%%%%%%%%%%%%%%%%%%%%%%%%%%%%%%%%%
\begin{frame}[fragile]
\frametitle{Principe de généricité}
Une {\bf fonction} est dite \alert{générique} si elle peut accepter des
arguments qui ne sont pas seulement ceux d'un type bien précis.
\medskip

\uncover<2->{
Exemples~:
\begin{itemize}
    \item une fonction qui teste si deux valeurs sont égales~;
    \smallskip
    }
    \uncover<3->{
    \item une fonction qui affiche plusieurs fois une même valeur.}
\end{itemize}
\bigskip
\bigskip

\uncover<4->{
Une {\bf structure de donnée} est dite \alert{générique} si elle peut
représenter des données dont le type n'est pas fixé.
\medskip}

\uncover<5->{
Exemples~:
\begin{itemize}
    \item une liste dont les éléments sont d'un type non fixé~;
    \smallskip
    }
    \uncover<6->{
    \item un arbre binaire dont les éléments sont d'un type non fixé.}
\end{itemize}
\end{frame}

%%%%%%%%%%%%%%%%%%%%%%%%%%%%%%%%%%%%%%%%%%%%%%%%%%%%%%%%%%%%%%%%%%%%%%%%
\begin{frame}[fragile]
\frametitle{Le type {\tt void *}}
Pour manipuler une donnée dont le type n'est pas spécifié à l'avance, on
utilise son {\bf adresse}.
\smallskip

\uncover<2->{
Il s'agit donc d'une adresse dont on ne connaît pas le type~: c'est une
adresse de type \Code{void *}.
\bigskip}

\uncover<3->{
Le type \Code{void *} est appelé \alert{type pointeur générique}.}
\bigskip
\bigskip

\uncover<4->{
Pour convertir un pointeur générique \Code{ptr\_g} vers un pointeur d'un
type connu \Code{T}, on utilise l'{\bf opérateur de coercition}
\begin{center}
    \Code{(T *) ptr\_g}.
\end{center}
\smallskip}

\uncover<5->{
Avant de pouvoir interpréter (c.-à-d. déréférencer) la valeur située à
une adresse spécifiée par un pointeur générique, le convertir est
primordial.}
\end{frame}

%%%%%%%%%%%%%%%%%%%%%%%%%%%%%%%%%%%%%%%%%%%%%%%%%%%%%%%%%%%%%%%%%%%%%%%%
\begin{frame}[fragile]
\frametitle{Fonction générique de comparaison}
\begin{semiverbatim}\small
int ega(int nbo, void *x, void *y) \{
    \uncover<2->{char *xc, *yc;
    int i;}

    \uncover<3->{xc = (char *) x;
    yc = (char *) y;}
    \uncover<4->{for (i = 0 ; i < nbo ; ++i)}
        \uncover<5->{if (xc[i] != yc[i])
            return 0;}
    \uncover<6->{return 1;}
\}
\end{semiverbatim}
La fonction \Code{ega} est générique~: elle permet de tester l'égalité
entre deux variables dont le {\bf type n'est pas connu lors de l'écriture
de la fonction}.
\smallskip

\uncover<7->{
On l'utilise de la manière suivante~:}
\begin{semiverbatim}\small\uncover<7->{
ega(sizeof(T), &t1, &t2)}
\end{semiverbatim}
\uncover<7->{
pour comparer deux variables \Code{t1} et \Code{t2} de type \Code{T}.}
\end{frame}

%%%%%%%%%%%%%%%%%%%%%%%%%%%%%%%%%%%%%%%%%%%%%%%%%%%%%%%%%%%%%%%%%%%%%%%%
\begin{frame}[fragile]
\frametitle{Fonction générique d'affichage de tableau}
\begin{semiverbatim}\small
void aff_tab(void **tab, int n, void (*aff_elt)(void *)) \{
    \uncover<2->{int i;}

    \uncover<3->{for (i = 0 ; i < n ; ++i) \{}
        \uncover<4->{aff_elt(tab[i]);}
        \uncover<5->{printf(" ");}
    \uncover<3->{\}}
\}
\end{semiverbatim}
\medskip

La fonction \Code{aff\_tab} est générique~: elle permet d'afficher les
éléments d'un tableau dont le {\bf type n'est pas connu lors de l'écriture
de la fonction}.
\end{frame}

%%%%%%%%%%%%%%%%%%%%%%%%%%%%%%%%%%%%%%%%%%%%%%%%%%%%%%%%%%%%%%%%%%%%%%%%
\begin{frame}[fragile]
\frametitle{Fonction générique d'affichage de tableau}
On l'utilise la fonction \Code{aff\_tab} de la manière suivante.
\bigskip

Pour afficher un tableau \Code{tab} de taille \Code{13} de pointeurs
sur des {\bf entiers}~:

\begin{semiverbatim} \small
\uncover<2->{void aff_int(void *x) \{
    \uncover<3->{int e;}
    \uncover<4->{e = *((int *) x);}
    \uncover<5->{printf("%d", e);}
\}}

\uncover<6->{/* Version raccourcie. */
void aff_int(void *x) \{
    printf("%d", *((int *) x));
\}}
\uncover<7->{...
aff_tab((void **) tab, 13, &aff_int);}
\end{semiverbatim}
\end{frame}

%%%%%%%%%%%%%%%%%%%%%%%%%%%%%%%%%%%%%%%%%%%%%%%%%%%%%%%%%%%%%%%%%%%%%%%%
\begin{frame}[fragile]
\frametitle{Fonction générique d'affichage de tableau}
Pour afficher un tableau \Code{tab} de taille \Code{23} de pointeurs
sur des variables de {\bf type structuré} \Code{Date}~:
\begin{semiverbatim}\small
typedef struct \{
    int jour;
    int mois;
    int annee;
\} Date;
...
void aff_date(void *d) \{
    Date dd;
    dd = *((Date *) d);
    printf("%d-%d-%d", dd.jour, dd.mois, dd.annee);
\}
...
aff_tab((void **) tab, 23, &aff_date);
\end{semiverbatim}
\end{frame}

%%%%%%%%%%%%%%%%%%%%%%%%%%%%%%%%%%%%%%%%%%%%%%%%%%%%%%%%%%%%%%%%%%%%%%%%
\begin{frame}[fragile]
\frametitle{Listes génériques}
On souhaite définir une structure de donnée liste dont les types des
éléments ne sont pas fixés.
\smallskip

\uncover<2->{Pour cela, on utilise un pointeur générique pour le champ
qui contient l'élément de chaque cellule~:}
\begin{semiverbatim}\small\uncover<2->{
typedef struct _Cellule \{
    struct _Cellule *suiv;
    void *e;
\} Cellule;

typedef Cellule *Liste;}
\end{semiverbatim}
\bigskip

\uncover<3->{
Le type \Code{Liste} permet ainsi de représenter des listes génériques.
\smallskip}

\uncover<4->{
C'est une structure de donnée générique car le {\bf type des éléments}
que les futures listes pourront contenir {\bf n'est pas connu lors
de l'écriture de la fonction}.}
\end{frame}

%%%%%%%%%%%%%%%%%%%%%%%%%%%%%%%%%%%%%%%%%%%%%%%%%%%%%%%%%%%%%%%%%%%%%%%%
\begin{frame}[fragile]
\frametitle{Listes génériques}
La fonction
\begin{semiverbatim}\small
void aff_lst(Liste lst, void (*aff_elt)(void *)) \{
    Cellule *x;

    assert(lst != NULL);
    assert(aff_elt != NULL);

    for (x = lst ; x != NULL ; x = x->suiv) \{
        aff_elt(x->e);
        printf(" ");
    \}
\}
\end{semiverbatim}
est une fonction générique pour l'affichage des éléments d'une liste
générique.
\end{frame}

%%%%%%%%%%%%%%%%%%%%%%%%%%%%%%%%%%%%%%%%%%%%%%%%%%%%%%%%%%%%%%%%%%%%%%%%
\begin{frame}[fragile]
\frametitle{Listes génériques}
On l'utilise de la manière suivante (dans le cas ici d'une liste d'entiers).
\medskip

\begin{semiverbatim}\small
void aff_int(void *e) \{
    printf("%d", *((int *) e));
\}
...
Liste lst;
int a, b, c;
a = 3; b = 14; c = 414;

lst = (Cellule *) malloc(sizeof(Cellule));
lst->e = &a;
lst->suiv = (Cellule *) malloc(sizeof(Cellule));
lst->suiv->e = &b;
lst->suiv->suiv = (Cellule *) malloc(sizeof(Cellule));
lst->suiv->suiv->e = &c;
lst->suiv->suiv->suiv = NULL;
aff_lst(lst, &aff_int);
\end{semiverbatim}
\end{frame}

%%%%%%%%%%%%%%%%%%%%%%%%%%%%%%%%%%%%%%%%%%%%%%%%%%%%%%%%%%%%%%%%%%%%%%%%
\begin{frame}[fragile]
\frametitle{Listes génériques}
Beaucoup de fonctions sur les listes peuvent ainsi être rendues génériques.
Entre autres~:
\smallskip

\begin{small}
\begin{enumerate}
    \uncover<2->{
    \item \Code{void aff\_lst(Liste lst, void (*aff\_elt)(void *));}
    \smallskip}

    \uncover<3->{
    \item \Code{void *elt\_indice(Liste lst, int i);}
    \smallskip}

    \uncover<4->{
    \item \label{item:lst_est_triee}
    \Code{int est\_triee(Liste lst, int (*est\_inf)(void *, void *));}
    \smallskip}

    \uncover<5->{
    \item \label{item:lst_max}
    \Code{void *max(Liste lst, int (*est\_inf)(void *, void *));}}
\end{enumerate}
\end{small}
\medskip

\uncover<6->{
Les cas \ref{item:lst_est_triee} et \ref{item:lst_max} supposent que
les éléments représentés par les listes sont comparables au moyen
d'une fonction \Code{est\_inf} à fournir.}
\end{frame}

%%%%%%%%%%%%%%%%%%%%%%%%%%%%%%%%%%%%%%%%%%%%%%%%%%%%%%%%%%%%%%%%%%%%%%%%
%%%%%%%%%%%%%%%%%%%%%%%%%%%%%%%%%%%%%%%%%%%%%%%%%%%%%%%%%%%%%%%%%%%%%%%%
\subsection{Implantation de monoïdes}

%%%%%%%%%%%%%%%%%%%%%%%%%%%%%%%%%%%%%%%%%%%%%%%%%%%%%%%%%%%%%%%%%%%%%%%%
\begin{frame}[fragile]
\frametitle{Monoïdes}
En maths, un \alert{monoïde} est un triplet
$(\mathcal{M}, \bullet, \mathbf{1})$ où
\begin{enumerate}
    \uncover<2->{
    \item $\mathcal{M}$ est un ensemble~;
    \smallskip}

    \uncover<3->{
    \item $\bullet$ est une application
    $\bullet : \mathcal{M} \times \mathcal{M} \to \mathcal{M}$
    associative, c.-à-d., pour tous $x, y, z \in \mathcal{M}$, on a
    $(x \bullet y) \bullet z = x \bullet (y \bullet z)$~;
    \smallskip}

    \uncover<4->{
    \item $\mathbf{1} \in \mathcal{M}$ est un élément unitaire,
    c.-à-d., pour tout $x \in \mathcal{M}$, on a
    $x \bullet \mathbf{1} = x = \mathbf{1} \bullet x$.}
\end{enumerate}
\bigskip

\uncover<5->{
Exemples~:
\begin{itemize}
    \item $(\mathbb{N}, +, 0)$ est le {\bf monoïde additif} des entiers
    naturels~;
    \smallskip}

    \uncover<6->{
    \item $(\mathbb{N}, \times, 1)$ est le {\bf monoïde multiplicatif}
    des entiers naturels~;
    \smallskip}

    \uncover<7->{
    \item $(\{{\tt a}, {\tt b}\}^*, \cdot, \epsilon)$ est le {\bf monoïde
    libre} sur les lettres ${\tt a}$ et ${\tt b}$. Ses éléments sont
    les chaînes de caractères composées de ${\tt a}$ et de ${\tt b}$.
    Son produit $\cdot$ est la concaténation et son unité $\epsilon$
    est la chaîne vide.}
\end{itemize}
\end{frame}

%%%%%%%%%%%%%%%%%%%%%%%%%%%%%%%%%%%%%%%%%%%%%%%%%%%%%%%%%%%%%%%%%%%%%%%%
\begin{frame}[fragile]
\frametitle{Implantation de monoïdes}
On se donne les deux objectifs suivants~:
\medskip

\begin{enumerate}
    \uncover<2->{
    \item implanter une {\bf structure de donnée générique} pour
    \alert{représenter des monoïdes} dont la nature des éléments n'est
    pas connue à l'avance~;
    \bigskip}

    \uncover<3->{
    \item implanter une {\bf fonction générique} qui
    \alert{élève à la puissance} $n \geq 0$ un élément $x$ d'un monoïde.}
\end{enumerate}
\end{frame}

%%%%%%%%%%%%%%%%%%%%%%%%%%%%%%%%%%%%%%%%%%%%%%%%%%%%%%%%%%%%%%%%%%%%%%%%
\begin{frame}[fragile]
\frametitle{Structure de donnée générique de monoïde}
D'après la définition mathématique d'un monoïde, on aboutit à la définition
suivante~:
\begin{semiverbatim}
typedef struct \{
    void *unite;
    void *(*produit)(void *, void *);
\} Monoide;
\end{semiverbatim}
\medskip

\uncover<2->{
On utilise des pointeurs génériques \Code{void *} pour les éléments
du monoïde.
\medskip

Le champ \Code{unite} est un pointeur générique vers l'unité
du monoïde.
\medskip

Le champ \Code{produit} est un pointeur de fonction vers
une fonction qui réalise le produit du monoïde.}
\end{frame}

%%%%%%%%%%%%%%%%%%%%%%%%%%%%%%%%%%%%%%%%%%%%%%%%%%%%%%%%%%%%%%%%%%%%%%%%
\begin{frame}[fragile]
\frametitle{Définition du monoïde additif des entiers naturels}
On commence par définir une fonction qui réalise le produit de
$(\mathbb{N}, +, 0)$ et dont le type de retour et la signature respectent
ceux du champ \Code{produit}~:
\begin{semiverbatim}\small
void *add(void *a, void *b) \{
    \uncover<2->{int *res;}
    \uncover<3->{res = (int *) malloc(sizeof(int));}
    \uncover<4->{*res = *((int *) a) + *((int *) b);}
    \uncover<2->{return (void *) res;}
\}
\end{semiverbatim}
\medskip

\uncover<5->{
Ensuite, on créé le monoïde de la manière suivante~:}
\begin{semiverbatim}\small\uncover<5->{
Monoide m_add;
int zero = 0;
m_add.unite = &zero;
m_add.produit = &add;}
\end{semiverbatim}
\end{frame}

%%%%%%%%%%%%%%%%%%%%%%%%%%%%%%%%%%%%%%%%%%%%%%%%%%%%%%%%%%%%%%%%%%%%%%%%
\begin{frame}[fragile]
\frametitle{Définition du monoïde multiplicatif des entiers naturels}
On commence par définir une fonction qui réalise le produit de
$(\mathbb{N}, \times, 1)$ et dont le type de retour et la signature
respectent ceux du champ \Code{produit}~:
\begin{semiverbatim}\small
void *mul(void *a, void *b) \{
    \uncover<2->{int *res;}
    \uncover<3->{res = (int *) malloc(sizeof(int));}
    \uncover<4->{*res = *((int *) a) * *((int *) b);}
    \uncover<2->{return (void *) res;}
\}
\end{semiverbatim}
\medskip

\uncover<2->{
Ensuite, on créé le monoïde de la manière suivante~:}
\begin{semiverbatim}\small\uncover<2->{
Monoide m_mul;
int un = 1;
m_mul.unite = &un;
m_mul.produit = &mul;}
\end{semiverbatim}
\end{frame}

%%%%%%%%%%%%%%%%%%%%%%%%%%%%%%%%%%%%%%%%%%%%%%%%%%%%%%%%%%%%%%%%%%%%%%%%
\begin{frame}[fragile]
\frametitle{Définition du monoïde libre sur {\tt a} et {\tt b}}
On commence par définir une fonction qui réalise le produit de
$(\{{\tt a}, {\tt b}\}^*, \cdot, \epsilon)$ et dont le type de retour et
la signature respectent ceux du champ \Code{produit}~:
\begin{semiverbatim}\small
void *conc(void *u, void *v) \{
    \uncover<2->{char *res;}
    \uncover<3->{int i, lu, lv;}
    \uncover<4->{lu = strlen((char *) u);}
    \uncover<4->{lv = strlen((char *) v);}
    \uncover<5->{res = (char *) malloc(sizeof(char) * (1 + lu + lv));}
    \uncover<6->{for (i = 0 ; i < lu ; ++i) res[i] = ((char *) u)[i];}
    \uncover<6->{for (i = 0 ; i < lv ; ++i) res[lu + i] = ((char *) v)[i];}
    \uncover<7->{res[lu + lv] = '\\0';}
    \uncover<2->{return (void *) res;}
\}
\end{semiverbatim}
\medskip

\uncover<8->{
Ensuite, on créé le monoïde de la manière suivante~:}
\begin{multicols}{2}
\begin{semiverbatim}\small\uncover<8->{
Monoide m_lib;
char epsilon[1] = "";
m_lib.unite = &epsilon;
m_lib.produit = &conc;}
\end{semiverbatim}
\end{multicols}
\end{frame}

%%%%%%%%%%%%%%%%%%%%%%%%%%%%%%%%%%%%%%%%%%%%%%%%%%%%%%%%%%%%%%%%%%%%%%%%
\begin{frame}[fragile]
\frametitle{Fonction générique d'exponentiation}
\begin{semiverbatim}\small
void *puissance(Monoide m, void *x, int n) \{
    \uncover<2->{void *tmp;}

    \uncover<3->{if (n == 0)
        return m.unite;}
    \uncover<4->{tmp = puissance(m, x, n / 2);}
    \uncover<5->{if (n % 2 == 0)
        return m.produit(tmp, tmp);}
    \uncover<6->{return m.produit(x, m.produit(tmp, tmp));}
\}
\end{semiverbatim}

\uncover<7->{
Cette fonction renvoie un pointeur générique sur une variable contenant
le résultat de la valeur à l'adresse \Code{x} élevée à la puissance
\Code{n}, pour le produit spécifié par le monoïde \Code{m}.
\medskip}

\uncover<8->{
Celle fonction utilise l'algorithme dit d'{\bf exponentiation rapide}.}
\end{frame}

%%%%%%%%%%%%%%%%%%%%%%%%%%%%%%%%%%%%%%%%%%%%%%%%%%%%%%%%%%%%%%%%%%%%%%%%
\begin{frame}[fragile]
\frametitle{Application de la fonction d'exponentiation}
\begin{semiverbatim}\small
int res, val = 5;
res = *((int *) puissance(m_add, &val, 3));
printf("%d\\n", res);
\end{semiverbatim}
Ceci affiche $5^3$ dans le monoïde additif, c'est à dire \Sortie{15}.
\bigskip

\begin{semiverbatim}\small\uncover<2->{
int res, val = 5;
res = *((int *) puissance(m_mul, &val, 3));
printf("%d\\n", res);}
\end{semiverbatim}
\uncover<2->{
Ceci affiche $5^3$ dans le monoïde multiplicatif, c'est à dire \Sortie{125}.}
\bigskip

\begin{semiverbatim}\small\uncover<3->{
char *res, ch[3] = "ab";
res = (char *) puissance(m_lib, &ch, 4);
printf("%s\\n", res);}
\end{semiverbatim}
\uncover<3->{
Ceci affiche $({\tt ab})^4$ dans le monoïde libre, c'est à dire
\Sortie{abababab}.}
\end{frame}

%%%%%%%%%%%%%%%%%%%%%%%%%%%%%%%%%%%%%%%%%%%%%%%%%%%%%%%%%%%%%%%%%%%%%%%%
%\begin{frame}[fragile] \frametitle{Un problème et sa solution}
%Il y a un {\bf problème} majeur dans cette façon se représenter les
%monoïdes.
%\medskip
%
%Il consiste en ce que le produit du monoïde est de type \\
%\Code{void *(*produit)(void *, void *)} et renvoie une adresse.
%\smallskip
%
%Le problème tient au fait qu'une {\bf allocation dynamique} doit être
%faite, ce qui induit une {\bf fuite mémoire} (voir la fonction \Code{puissance}).
%\bigskip
%
%\end{frame}


\end{document}
