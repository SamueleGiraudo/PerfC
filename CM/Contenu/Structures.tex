% Auteur : Samuele Giraudo
% Création : jan. 2014, jan. 2015, fév 2015, déc. 2015, mars 2016

\tikzstyle{Bloc}=[rectangle,draw=Vert!100,fill=Vert!20,
    line width=1pt,minimum width=1cm,minimum height=0.75cm]
\tikzstyle{BlocM}=[Bloc,draw=Marron=!100,fill=Marron!10]

%%%%%%%%%%%%%%%%%%%%%%%%%%%%%%%%%%%%%%%%%%%%%%%%%%%%%%%%%%%%%%%%%%%%%%%%
%%%%%%%%%%%%%%%%%%%%%%%%%%%%%%%%%%%%%%%%%%%%%%%%%%%%%%%%%%%%%%%%%%%%%%%%
%%%%%%%%%%%%%%%%%%%%%%%%%%%%%%%%%%%%%%%%%%%%%%%%%%%%%%%%%%%%%%%%%%%%%%%%
\section{Types structurés}

%%%%%%%%%%%%%%%%%%%%%%%%%%%%%%%%%%%%%%%%%%%%%%%%%%%%%%%%%%%%%%%%%%%%%%%%
%%%%%%%%%%%%%%%%%%%%%%%%%%%%%%%%%%%%%%%%%%%%%%%%%%%%%%%%%%%%%%%%%%%%%%%%
\subsection{Déclaration et initialisation}

%%%%%%%%%%%%%%%%%%%%%%%%%%%%%%%%%%%%%%%%%%%%%%%%%%%%%%%%%%%%%%%%%%%%%%%%
\begin{frame}[fragile]
\frametitle{Déclaration de types structurés récursifs}
Il est possible de \alert{déclarer des types structurés récursifs} en
faisant usage de l'{\bf alias} et du mot clé \Code{struct}~:
\medskip

\begin{lstlisting}
typedef struct _Liste {
    int e;
    struct _Liste *s;
} Liste;
\end{lstlisting}
\medskip

Ceci fonctionne car la taille d'un pointeur vers une valeur de type
\Code{T} est connue et indépendante de la nature de \Code{T}.
\bigskip

Attention, la déclaration
\begin{lstlisting}
typedef struct _Liste {
    int e;
    struct _Liste s;
} Liste;
\end{lstlisting}
n'est pas valide car le {\bf champ récursif} n'est pas un {\bf pointeur}.
\medskip

Le système ne peut pas connaître pas la taille de ce champ.
\end{frame}

%%%%%%%%%%%%%%%%%%%%%%%%%%%%%%%%%%%%%%%%%%%%%%%%%%%%%%%%%%%%%%%%%%%%%%%%
\begin{frame}[fragile]
\frametitle{Déclaration de types structurés mutuellement récursifs}
Il est possible de \alert{déclarer des types structurés mutuellement récursifs}~:
\begin{lstlisting}
typedef struct _Flip {
    int a;
    int b;
    struct _Flop *suiv;
} Flip;

typedef struct _Flop {
    double x;
    struct _Flip *suiv;
} Flop;
\end{lstlisting}

\begin{center}
    \todo{Dessiner un exemple de variable de type \Code{Flip}.}
\end{center}
\end{frame}

%%%%%%%%%%%%%%%%%%%%%%%%%%%%%%%%%%%%%%%%%%%%%%%%%%%%%%%%%%%%%%%%%%%%%%%%
\begin{frame}[fragile]
\frametitle{Initialisation d'une variable d'un type structuré}
Il est possible d'\alert{initialiser les champs} d'une variable d'un type
structuré au moment de sa \alert{déclaration}.
\medskip

On utilise pour cela l'opérateur d'affectation \Code{=} avec comme valeur
droite les valeurs des champs à affecter dans des accolades et séparées
par des virgules.
\medskip

Par exemple,
\begin{multicols}{2}
\begin{lstlisting}
typedef struct {
    char c;
    int a;
    double b;
} Triplet;
...
Triplet tr = {'h', 55, 214.35};
\end{lstlisting}
\bigskip

Déclare, en l'initialisant, la variable \Code{tr}.
\begin{center}
    \todo{Dessiner le contenu de \Code{tr}.}
\end{center}
\bigskip
\bigskip
\end{multicols}
\end{frame}

%%%%%%%%%%%%%%%%%%%%%%%%%%%%%%%%%%%%%%%%%%%%%%%%%%%%%%%%%%%%%%%%%%%%%%%%
%%%%%%%%%%%%%%%%%%%%%%%%%%%%%%%%%%%%%%%%%%%%%%%%%%%%%%%%%%%%%%%%%%%%%%%%
\subsection{Dans les fonctions}

%%%%%%%%%%%%%%%%%%%%%%%%%%%%%%%%%%%%%%%%%%%%%%%%%%%%%%%%%%%%%%%%%%%%%%%%
\begin{frame}[fragile]
\frametitle{Renvoi d'une variable d'un type structuré}
Le code
\begin{multicols}{2}
\begin{lstlisting}
typedef struct {
    int x;
    int y;
} Couple;


Couple twist(Couple c) {
    Couple res;
    res.x = c.y;
    res.y = c.x;
    return res;
}
\end{lstlisting}
\end{multicols}
est correct (\Code{twist} renvoie le couple obtenu par échange des
coordonnées de celui passé en argument).
\smallskip

\Code{twist} \alert{renvoie} une variable d'un \alert{type structuré}.
\medskip

Cependant, il n'est pas efficace car, à chaque appel de fonction
\begin{center}
    \Code{d = twist(c);}
\end{center}
la variable \Code{res}, qui vit dans la pile, doit être recopiée.
\end{frame}

%%%%%%%%%%%%%%%%%%%%%%%%%%%%%%%%%%%%%%%%%%%%%%%%%%%%%%%%%%%%%%%%%%%%%%%%
\begin{frame}[fragile]
\frametitle{Paramètre variable d'un type structuré}
Le code
\begin{multicols}{2}
\begin{lstlisting}
typedef struct {
    int tab1[2048];
    int tab2[2048];
} DeuxTab;

int prem_egaux(DeuxTab x) {
    return x.tab1[0]
        == x.tab2[0];
}
\end{lstlisting}
\end{multicols}
est correct (\Code{prem\_egaux} teste si les premières cases des tableaux
sont égales).
\smallskip

\Code{prem\_egaux} est \alert{paramétrée} par une variable
d'un \alert{type structuré}.
\medskip

Cependant, il n'est pas efficace car à chaque appel de fonction
\begin{center}
    \Code{prem\_egaux(y);}
\end{center}
les champs de l'{\bf argument} \Code{y} sont recopiés dans le
{\bf paramètre} \Code{x}.
\end{frame}

%%%%%%%%%%%%%%%%%%%%%%%%%%%%%%%%%%%%%%%%%%%%%%%%%%%%%%%%%%%%%%%%%%%%%%%%
\begin{frame}[fragile]
\frametitle{Passage par adresse {\em vs} passage par valeur}
Soit une fonction \Code{fct} paramétrée par une variable \Code{x} d'un
type structuré~\Code{T}.
\medskip

Il est d'usage courant d'adopter la convention suivante~:
\smallskip

\begin{itemize}
    \item si les champs de \Code{x} doivent être modifiés par la fonction,
    alors on recourt à un {\bf passage par adresse}
    \begin{center}
        \Code{... fct(T *x, ...) $\lbrace$ ... $\rbrace$}
    \end{center}
    \smallskip

    \item si les champs de \Code{x} ne doivent pas être modifiés par
    la fonction, alors on recourt à un {\bf passage par valeur}
    \begin{center}
        \Code{... fct(T x, ...) $\lbrace$ ... $\rbrace$}
    \end{center}
\end{itemize}
\bigskip

\alert{Cette conception est erronée} car il est possible de \og modifier \fg\,
une variable d'un type structuré passée par valeur à une fonction.
\end{frame}

%%%%%%%%%%%%%%%%%%%%%%%%%%%%%%%%%%%%%%%%%%%%%%%%%%%%%%%%%%%%%%%%%%%%%%%%
\begin{frame}[fragile]
\frametitle{Passage par adresse {\em vs} passage par valeur}
Considérons en effet le code suivant~:
\begin{multicols}{2}
\begin{lstlisting}
tyepdef struct {
    int *tab;
    int n;
} Tab;

void init(Tab t, int k) {
    int i;
    for (i = 0 ; i < t.n ; ++i)
        t.tab[i] = k;
}
\end{lstlisting}
\end{multicols}
Chaque appel de fonction
\begin{center}
    \Code{init(s, r);}
\end{center}
provoque la recopie de trois valeurs (ce qui est encore acceptable) mais
\og modifie \fg\, les valeurs pointées par le champ \Code{tab} de \Code{s},
malgré le passage par valeur.
\bigskip

{\bf Conclusion}~: écrire des fonctions avec passage par valeur des
paramètres d'un type structuré ne présente que des désavantages.
\end{frame}

%%%%%%%%%%%%%%%%%%%%%%%%%%%%%%%%%%%%%%%%%%%%%%%%%%%%%%%%%%%%%%%%%%%%%%%%
\begin{frame}[fragile]
\frametitle{Variables d'un type structuré dans les fonctions}
En résumé, on adopte les deux règles suivantes~:
\smallskip

\begin{enumerate}
    \item une fonction ne renvoie jamais de valeur d'un type structuré~;
    \smallskip

    \item tous les paramètres d'un type structuré sont passés par adresse
    dans une fonction.
\end{enumerate}
\medskip

Ainsi, un prototype de fonction habituel est
\begin{center}
    \Code{int fct(T *x,
    \textcolor{Rouge}{E1 e1}, ..., \textcolor{Rouge}{EN en},
    \textcolor{Vert}{S1 *s1}, ..., \textcolor{Vert}{SM *sm});}
\end{center}
où
\begin{itemize}
    \item le type de retour est \Code{int} (renvoi d'un code d'erreur)~;
    \smallskip

    \item \Code{x} est l'adresse d'une variable d'un type structuré \Code{T}~;
    \smallskip

    \item \textcolor{Rouge}{\tt e1}, ..., \textcolor{Rouge}{\tt en}
    sont les entrées de la fonction (adresses ou non)~;
    \smallskip

    \item \textcolor{Vert}{\tt s1}, ..., \textcolor{Vert}{\tt sm}
    sont les sorties de la fonction (qui sont des adresses).
\end{itemize}
\end{frame}

%%%%%%%%%%%%%%%%%%%%%%%%%%%%%%%%%%%%%%%%%%%%%%%%%%%%%%%%%%%%%%%%%%%%%%%%
\begin{frame}[fragile]
\frametitle{Variables d'un type structuré dans les fonctions}
P.ex., voici le nécessaire pour calculer la somme pondérée de deux
points selon les conventions établies~:
\begin{lstlisting}
typdef struct {
    float x;
    float y;
} Point;

void somme_points(const Point *p1, const Point *p2,
        float coeff1, float coeff2,
        Point *res) {

    assert(p1 != NULL);
    assert(p2 != NULL);
    assert(res != NULL);

    res->x = coeff1 * p1->x + coeff2 * p2->x;
    res->y = coeff1 * p1->y + coeff2 * p2->y;
}
\end{lstlisting}
\end{frame}

%%%%%%%%%%%%%%%%%%%%%%%%%%%%%%%%%%%%%%%%%%%%%%%%%%%%%%%%%%%%%%%%%%%%%%%%
%%%%%%%%%%%%%%%%%%%%%%%%%%%%%%%%%%%%%%%%%%%%%%%%%%%%%%%%%%%%%%%%%%%%%%%%
\subsection{Affectation et comparaison}

%%%%%%%%%%%%%%%%%%%%%%%%%%%%%%%%%%%%%%%%%%%%%%%%%%%%%%%%%%%%%%%%%%%%%%%%
\begin{frame}[fragile]
\frametitle{Affectation de variables d'un type structuré}
Considérons le code
\begin{multicols}{2}
\begin{lstlisting}
typedef struct {
    int a;
    float f;
} X;
...
X v1, v2;
v1.a = 10;
v1.f = 3.14;
v2 = v1;
v2.a = 20;
\end{lstlisting}
\end{multicols}
\medskip

\begin{center}
    \todo{Dessiner le contenu de \Code{v1} et \Code{v2} aux l. 6, 8, 9 et 10.}
\end{center}
\medskip

{\bf Observation}~: l'affectation recopie les champs d'une variable
d'un type scalaire.
\end{frame}

%%%%%%%%%%%%%%%%%%%%%%%%%%%%%%%%%%%%%%%%%%%%%%%%%%%%%%%%%%%%%%%%%%%%%%%%
\begin{frame}[fragile]
\frametitle{Affectation de variables d'un type structuré}
Considérons le code
\begin{multicols}{2}
\begin{lstlisting}
typedef struct {
    X x;
    char t[3];
} Y;
...
Y v1, v2;
v1.x.a = 10;
v1.x.f = 3.14;
v1.t = {'a', 'b', 'c'};
v2 = v1;
v2.x.f = 1.8;
v2.t[0] = 'g';
\end{lstlisting}
\end{multicols}
\medskip

\begin{center}
    \todo{Dessiner le contenu de \Code{v1} et \Code{v2} aux l. 9, 10 et 12.}
\end{center}
\medskip

{\bf Observation}~: l'affectation recopie les champs d'une variable d'un
type structuré de manière récursive et les tableaux statiques.
\end{frame}

%%%%%%%%%%%%%%%%%%%%%%%%%%%%%%%%%%%%%%%%%%%%%%%%%%%%%%%%%%%%%%%%%%%%%%%%
\begin{frame}[fragile]
\frametitle{Affectation de variables d'un type structuré}
Considérons le code
\begin{multicols}{2}
\begin{lstlisting}
typedef struct {
    char *t;
    int n;
} T;
...
T v1, v2;
v1.t = (char *)
    malloc(sizeof(char) * 3);
v1.n = 3;
v1.t[0] = 'a';
v1.t[1] = 'b';
v1.t[2] = 'c';
v2 = v1;
v2.n = 2;
v2.t[0] = 'g';
\end{lstlisting}
\end{multicols}
\medskip

\begin{center}
    \todo{Dessiner le contenu de \Code{v1} et \Code{v2} aux l. 12, 13 et 15.}
\end{center}
\medskip

{\bf Observation}~: l'affectation ne recopie pas les tableaux dynamiques.
Seule l'adresse d'un tableau dynamique est recopiée. C'est une
\alert{copie de surface}.
\end{frame}

%%%%%%%%%%%%%%%%%%%%%%%%%%%%%%%%%%%%%%%%%%%%%%%%%%%%%%%%%%%%%%%%%%%%%%%%
\begin{frame}[fragile]
\frametitle{Affectation de variables d'un type structuré}
{\bf Règle générale}~: pour chaque déclaration d'un type structuré
\Code{X}, on définit une fonction de prototype
\begin{center}
    \Code{void copier\_X(X *v1, X *v2);}
\end{center}
qui \alert{copie en profondeur} les champs de \Code{v1} dans les champs
de \Code{v2}.
\medskip

Par exemple, la définition du type \Code{T} précédent s'accompagne de la
définition de la fonction
\begin{lstlisting}
void copier_T(T *v1, T *v2) {
    int i;
    assert(v1 != NULL);
    assert(v2 != NULL);
    v2->n = v1->n;
    v2->t = (char *) malloc(sizeof(char) * v1->n);
    if (v2->t == NULL) exit(EXIT_FAILURE);
    for (i = 0 ; i < v1->n ; ++i)
        v2->t[i] = v1->t[i];
}
\end{lstlisting}

Il est possible de munir cette fonction du mécanisme habituel de
gestion d'erreurs.
\end{frame}

%%%%%%%%%%%%%%%%%%%%%%%%%%%%%%%%%%%%%%%%%%%%%%%%%%%%%%%%%%%%%%%%%%%%%%%%
\begin{frame}[fragile]
\frametitle{Comparaison de variables d'un type structuré}
Considérons le code
\begin{multicols}{2}
\begin{lstlisting}
typedef struct {
    int a;
    int b;
} A;
...
A v1, v2;
...
if (v1 == v2) {...}
...
if (v1 != v2) {...}
\end{lstlisting}
\end{multicols}
\medskip

Ce code est incorrect (il ne compile pas).
\medskip

Le compilateur n'accepte pas la comparaison de variables d'un type
structuré.
\smallskip

\Sortie{\small invalid operands to binary == (have 'A' and 'A')}
\smallskip

\Sortie{\small invalid operands to binary != (have 'A' and 'A')}
\end{frame}

%%%%%%%%%%%%%%%%%%%%%%%%%%%%%%%%%%%%%%%%%%%%%%%%%%%%%%%%%%%%%%%%%%%%%%%%
\begin{frame}[fragile]
\frametitle{Comparaison de variables d'un type structuré}
{\bf Règle générale}~: pour chaque déclaration d'un type structuré
\Code{X}, on définit deux fonctions de prototypes
\begin{center}
    \Code{int sont\_ega\_X(X *v1, X *v2);} \smallskip

    \Code{int sont\_dif\_X(X *v1, X *v2);}
\end{center}
qui \alert{testent l'égalité} et \alert{l'inégalité} entre \Code{v1}
et \Code{v2}.
\medskip

Par exemple, la définition du type \Code{A} précédent s'accompagne de la
définition des fonctions
\begin{multicols}{2}
\begin{lstlisting}
int sont_ega_A(A *v1, A *v2) {
    assert(v1 != NULL);
    assert(v2 != NULL);
    return (v1->a == v2->a)
        && (v1->b == v2->b);
}
int sont_dif_A(A *v1, A *v2) {
    assert(v1 != NULL);
    assert(v2 != NULL);
    return !sont_ega_A(v1, v2);
}
\end{lstlisting}
\end{multicols}
\medskip

{\bf Attention}~: si \Code{X} est composé d'un champ qui est un type
structuré \Code{Y}, il faut appeler dans \Code{sont\_ega\_X} la
fonction de comparaison \Code{sont\_ega\_Y}.
\end{frame}

%%%%%%%%%%%%%%%%%%%%%%%%%%%%%%%%%%%%%%%%%%%%%%%%%%%%%%%%%%%%%%%%%%%%%%%%
\begin{frame}[fragile]
\frametitle{Destruction de variables d'un type structuré}
{\bf Règle générale}~: pour chaque déclaration d'un type structuré
\Code{X}, on définit une fonction de prototype
\begin{center}
    \Code{void detruire\_X(X *v);}
\end{center}
qui \alert{libère l'espace mémoire} adressé par \Code{v}.
\medskip

Par exemple, la déclaration du type \Code{B} suivant s'accompagne de la
définition de la fonction
\begin{multicols}{2}
\begin{lstlisting}
typedef struct {
    int *tab;
    int n;
} B;

void detruire_B(B *v) {
    assert(v != NULL);
    free(v->tab);
    *v = NULL;
}
\end{lstlisting}
\end{multicols}
\medskip

{\bf Attention}~: si \Code{X} est composé d'un champ qui est un type
structuré \Code{Y}, il faut appeler dans \Code{detruire\_X} la
fonction de comparaison \Code{detruire\_Y}.
\end{frame}

%%%%%%%%%%%%%%%%%%%%%%%%%%%%%%%%%%%%%%%%%%%%%%%%%%%%%%%%%%%%%%%%%%%%%%%%
\begin{frame}[fragile]
\frametitle{Résumé}
Voici en résumé la bonne marche à suivre lors de la manipulation de
types structurés~:
\smallskip

\begin{enumerate}
    \item on utilise l'{\bf alias} lors de la déclaration de types structurés
    {\bf récursifs} et/ou {\bf mutuellement récursifs}~;
    \bigskip

    \item on ne {\bf renvoie jamais} de valeur d'un type structuré~;
    \bigskip

    \item on passe les {\bf paramètres} d'un type structuré {\bf par adresse}~;
    \bigskip

    \item toute {\bf déclaration d'un type structuré} s'accompagne de la
    définition des quatre fonctions suivantes~:

    \begin{itemize} \normalsize
        \item une fonction de {\bf copie}~;
        \smallskip

        \item une fonction de {\bf test d'égalité}~;
        \smallskip

        \item une fonction de {\bf test d'inégalité}~;
        \smallskip

        \item une fonction de {\bf destruction}.
    \end{itemize}
\end{enumerate}
\end{frame}

%%%%%%%%%%%%%%%%%%%%%%%%%%%%%%%%%%%%%%%%%%%%%%%%%%%%%%%%%%%%%%%%%%%%%%%%
%%%%%%%%%%%%%%%%%%%%%%%%%%%%%%%%%%%%%%%%%%%%%%%%%%%%%%%%%%%%%%%%%%%%%%%%
\subsection{Alignement en mémoire}

%%%%%%%%%%%%%%%%%%%%%%%%%%%%%%%%%%%%%%%%%%%%%%%%%%%%%%%%%%%%%%%%%%%%%%%%
\begin{frame}[fragile]
\frametitle{Alignement en mémoire}
L'\alert{alignement en mémoire} d'une donnée est la façon dont celle-ci
est organisée dans la mémoire.
\bigskip

Par exemple, nous avons déjà vu que les {\bf tableaux} de taille
\Code{n} d'éléments d'un type \Code{T} sont organisés en un segment
contigu de \Code{sizeof(T) * n} octets.
\medskip

Ainsi, un tableau \Code{t} de \Code{3} éléments de type \Code{short}
est organisé en
\begin{center}
\scalebox{.6}{\begin{tikzpicture}
    \node[BlocM](7)at(-1,0){};
    \node[BlocM](8)at(-2,0){};
    \node[BlocM](9)at(-3,0){};
    \node[BlocM](10)at(6,0){};
    \node[BlocM](11)at(7,0){};
    \node at(-4,0){\Large $\dots$};
    \node at(8,0){\Large $\dots$};
    \node[Bloc](1)at(0,0){};
    \node[Bloc](2)at(1,0){};
    \node[Bloc](3)at(2,0){};
    \node[Bloc](4)at(3,0){};
    \node[Bloc](5)at(4,0){};
    \node[Bloc](6)at(5,0){};
    \draw(-0.5,0.75)edge[anchor=south,<->,line width=1.5pt]
        node{\small 1 octet}(0.5,0.75);
    \draw(-0.5,-0.75)edge[anchor=north,<->,line width=1.5pt]
        node{\small \Code{t[0]}}(1.5,-0.75);
    \draw(1.5,-0.75)edge[anchor=north,<->,line width=1.5pt]
        node{\small \Code{t[1]}}(3.5,-0.75);
    \draw(3.5,-0.75)edge[anchor=north,<->,line width=1.5pt]
        node{\small \Code{t[2]}}(5.5,-0.75);
\end{tikzpicture}}
\end{center}
\bigskip

On peut se poser de la même manière la question de l'{\bf alignement
mémoire} des variables d'un {\bf type structuré}.
\end{frame}

%%%%%%%%%%%%%%%%%%%%%%%%%%%%%%%%%%%%%%%%%%%%%%%%%%%%%%%%%%%%%%%%%%%%%%%%
\begin{frame}[fragile]
\frametitle{Alignement en mémoire des variables d'un type structuré}
Considérons les déclarations de types
\begin{multicols}{2}
\begin{lstlisting}
typedef struct {
    short x;
    short y;
    int z;
} A;
typedef struct {
    short x;
    int z;
    short y;
} B;
\end{lstlisting}
\end{multicols}
\medskip

\Code{A} et \Code{B} sont des types structurés composés des mêmes champs.
Il n'y a que l'ordre de leur déclaration qui diffère.
\medskip

Cependant,
\begin{lstlisting}
    printf("%lu %lu\n", sizeof(A), sizeof(B));
\end{lstlisting}
affiche \Sortie{8 12}.
\medskip

Le fait que les tailles des variables de type \Code{A} et \Code{B}
diffèrent est dû à leur {\bf alignement en mémoire}.
\end{frame}

%%%%%%%%%%%%%%%%%%%%%%%%%%%%%%%%%%%%%%%%%%%%%%%%%%%%%%%%%%%%%%%%%%%%%%%%
\begin{frame}[fragile]
\frametitle{Alignement en mémoire des variables d'un type structuré}
Soit \Code{a} une variable de type \Code{A}. Cette variable est organisée
en mémoire en
\begin{center}
\scalebox{.6}{\begin{tikzpicture}
    \node[BlocM](9)at(-1,0){};
    \node[BlocM](10)at(-2,0){};
    \node[BlocM](11)at(-3,0){};
    \node[BlocM](12)at(8,0){};
    \node[BlocM](13)at(9,0){};
    \node[BlocM](14)at(10,0){};
    \node[BlocM](15)at(11,0){};
    \node[BlocM](16)at(12,0){};
    \node at(-4,0){\Large $\dots$};
    \node at(13,0){\Large $\dots$};
    \node[Bloc,fill=Jaune!40,draw=Jaune!100](1)at(0,0){};
    \node[Bloc,fill=Jaune!40,draw=Jaune!100](2)at(1,0){};
    \node[Bloc,fill=Rouge!40,draw=Rouge!100](3)at(2,0){};
    \node[Bloc,fill=Rouge!40,draw=Rouge!100](4)at(3,0){};
    \node[Bloc,fill=Bleu!40,draw=Bleu!100](5)at(4,0){};
    \node[Bloc,fill=Bleu!40,draw=Bleu!100](6)at(5,0){};
    \node[Bloc,fill=Bleu!40,draw=Bleu!100](7)at(6,0){};
    \node[Bloc,fill=Bleu!40,draw=Bleu!100](8)at(7,0){};
    \draw(-0.5,0.75)edge[anchor=south,<->,line width=1.5pt]
        node{\small 1 octet}(0.5,0.75);
    \draw(-0.5,-0.75)edge[anchor=north,<->,line width=1.5pt]
        node{\small \Code{a.x}}(1.5,-0.75);
    \draw(1.5,-0.75)edge[anchor=north,<->,line width=1.5pt]
        node{\small \Code{a.y}}(3.5,-0.75);
    \draw(3.5,-0.75)edge[anchor=north,<->,line width=1.5pt]
        node{\small \Code{a.z}}(7.5,-0.75);
\end{tikzpicture}}
\end{center}
\bigskip

Soit \Code{b} une variable de type \Code{B}. Cette variable est organisée
en mémoire en
\begin{center}
\scalebox{.6}{\begin{tikzpicture}
    \node[BlocM](9)at(-1,0){};
    \node[BlocM](10)at(-2,0){};
    \node[BlocM](11)at(-3,0){};
    \node[BlocM](16)at(12,0){};
    \node at(-4,0){\Large $\dots$};
    \node at(13,0){\Large $\dots$};
    \node[Bloc,fill=Jaune!40,draw=Jaune!100](1)at(0,0){};
    \node[Bloc,fill=Jaune!40,draw=Jaune!100](2)at(1,0){};
    \node[Bloc,fill=Bleu!40,draw=Bleu!100](5)at(4,0){};
    \node[Bloc,fill=Bleu!40,draw=Bleu!100](6)at(5,0){};
    \node[Bloc,fill=Bleu!40,draw=Bleu!100](7)at(6,0){};
    \node[Bloc,fill=Bleu!40,draw=Bleu!100](8)at(7,0){};
    \node[Bloc,fill=Rouge!40,draw=Rouge!100](12)at(8,0){};
    \node[Bloc,fill=Rouge!40,draw=Rouge!100](13)at(9,0){};
    \node[BlocM,fill=Noir!50,draw=Noir](14)at(10,0){};
    \node[BlocM,fill=Noir!50,draw=Noir](15)at(11,0){};
    \node[BlocM,fill=Noir!50,draw=Noir](3)at(2,0){};
    \node[BlocM,fill=Noir!50,draw=Noir](4)at(3,0){};
    \draw(-0.5,0.75)edge[anchor=south,<->,line width=1.5pt]
        node{\small 1 octet}(0.5,0.75);
    \draw(-0.5,-0.75)edge[anchor=north,<->,line width=1.5pt]
        node{\small \Code{a.x}}(1.5,-0.75);
    \draw(3.5,-0.75)edge[anchor=north,<->,line width=1.5pt]
        node{\small \Code{a.z}}(7.5,-0.75);
    \draw(7.5,-0.75)edge[anchor=north,<->,line width=1.5pt]
        node{\small \Code{a.y}}(9.5,-0.75);
\end{tikzpicture}}
\end{center}
\medskip

Les octets en gris intervenant dans l'alignement mémoire de \Code{b}
sont des \alert{octets de complétion}.
\end{frame}

%%%%%%%%%%%%%%%%%%%%%%%%%%%%%%%%%%%%%%%%%%%%%%%%%%%%%%%%%%%%%%%%%%%%%%%%
\begin{frame}[fragile]
\frametitle{Octets de complétion}
Des octets de complétion sont introduits pour que chaque champ \Code{c}
d'une variable d'un type structuré \alert{commence à une adresse multiple d'un
entier} dépendant du type de \Code{c}.
\medskip

Dans notre exemple, en sachant que tout champ de type \Code{short}
(resp. \Code{int}) doit commencer à une adresse multiple de \Code{2}
(resp. \Code{4}), on explique l'alignement en mémoire de \Code{b}
précédent~:
\begin{center}
\scalebox{.6}{\begin{tikzpicture}
    \node[BlocM](9)at(-1,0){};
    \node[BlocM](10)at(-2,0){};
    \node[BlocM](11)at(-3,0){};
    \node[BlocM](16)at(12,0){};
    \node at(-4,0){\Large $\dots$};
    \node at(13,0){\Large $\dots$};
    \node[Bloc,fill=Jaune!40,draw=Jaune!100](1)at(0,0){};
    \node[Bloc,fill=Jaune!40,draw=Jaune!100](2)at(1,0){};
    \node[Bloc,fill=Bleu!40,draw=Bleu!100](5)at(4,0){};
    \node[Bloc,fill=Bleu!40,draw=Bleu!100](6)at(5,0){};
    \node[Bloc,fill=Bleu!40,draw=Bleu!100](7)at(6,0){};
    \node[Bloc,fill=Bleu!40,draw=Bleu!100](8)at(7,0){};
    \node[Bloc,fill=Rouge!40,draw=Rouge!100](12)at(8,0){};
    \node[Bloc,fill=Rouge!40,draw=Rouge!100](13)at(9,0){};
    \node[BlocM,fill=Noir!50,draw=Noir](14)at(10,0){};
    \node[BlocM,fill=Noir!50,draw=Noir](15)at(11,0){};
    \node[BlocM,fill=Noir!50,draw=Noir](3)at(2,0){};
    \node[BlocM,fill=Noir!50,draw=Noir](4)at(3,0){};
    \draw(-0.5,0.75)edge[anchor=south,<->,line width=1.5pt]
        node{\small 1 octet}(0.5,0.75);
    \draw(-0.5,-0.75)edge[anchor=north,<->,line width=1.5pt]
        node{\small \Code{a.x}}(1.5,-0.75);
    \draw(3.5,-0.75)edge[anchor=north,<->,line width=1.5pt]
        node{\small \Code{a.z}}(7.5,-0.75);
    \draw(7.5,-0.75)edge[anchor=north,<->,line width=1.5pt]
        node{\small \Code{a.y}}(9.5,-0.75);
    \node(m1)at(0,-2){adr. mult. de \Code{4}};
    \draw[->](m1)--(0,-1);
    \node(m2)at(2,-3){adr. non mult. de \Code{4}};
    \draw[->](m2)--(2,-1);
\end{tikzpicture}}
\end{center}

Les derniers octets de complétion sont introduits pour que les tableaux
de variables de type \Code{B} puissent être représentés en vérifiant
cet alignement en mémoire.
\end{frame}

%%%%%%%%%%%%%%%%%%%%%%%%%%%%%%%%%%%%%%%%%%%%%%%%%%%%%%%%%%%%%%%%%%%%%%%%
\begin{frame}[fragile]
\frametitle{Accès manuel aux champs}
Soit \Code{a} une variable de type \Code{A} initialisée par
\begin{lstlisting}
    A a = {1000, 2000, 3000};
\end{lstlisting}
Cette variable est organisée en mémoire en
\begin{center}
\scalebox{.6}{\begin{tikzpicture}
    \node[BlocM](9)at(-1,0){};
    \node[BlocM](10)at(-2,0){};
    \node[BlocM](11)at(-3,0){};
    \node[BlocM](12)at(8,0){};
    \node[BlocM](13)at(9,0){};
    \node[BlocM](14)at(10,0){};
    \node[BlocM](15)at(11,0){};
    \node[BlocM](16)at(12,0){};
    \node at(-4,0){\Large $\dots$};
    \node at(13,0){\Large $\dots$};
    \node[Bloc,fill=Jaune!40,draw=Jaune!100](1)at(0,0){};
    \node[Bloc,fill=Jaune!40,draw=Jaune!100](2)at(1,0){};
    \node[Bloc,fill=Rouge!40,draw=Rouge!100](3)at(2,0){};
    \node[Bloc,fill=Rouge!40,draw=Rouge!100](4)at(3,0){};
    \node[Bloc,fill=Bleu!40,draw=Bleu!100](5)at(4,0){};
    \node[Bloc,fill=Bleu!40,draw=Bleu!100](6)at(5,0){};
    \node[Bloc,fill=Bleu!40,draw=Bleu!100](7)at(6,0){};
    \node[Bloc,fill=Bleu!40,draw=Bleu!100](8)at(7,0){};
    \draw(-0.5,0.75)edge[anchor=south,<->,line width=1.5pt]
        node{\small 1 octet}(0.5,0.75);
    \draw(-0.5,-0.75)edge[anchor=north,<->,line width=1.5pt]
        node{\small \Code{a.x}}(1.5,-0.75);
    \draw(1.5,-0.75)edge[anchor=north,<->,line width=1.5pt]
        node{\small \Code{a.y}}(3.5,-0.75);
    \draw(3.5,-0.75)edge[anchor=north,<->,line width=1.5pt]
        node{\small \Code{a.z}}(7.5,-0.75);
\end{tikzpicture}}
\end{center}
\bigskip

On peut accéder aux champs de \Code{a} de la manière suivante~:
\begin{lstlisting}
short x, y;
int z;
void *p;
p = &a;
x = *((short *) p); /* Equivalent a x = a.x; */
p += 2;
y = *((short *) p); /* Equivalent a y = a.y; */
p += 2;
z = *((int *) p); /* Equivalent a z = a.z; */
\end{lstlisting}
\end{frame}

%%%%%%%%%%%%%%%%%%%%%%%%%%%%%%%%%%%%%%%%%%%%%%%%%%%%%%%%%%%%%%%%%%%%%%%%
\begin{frame}[fragile]
\frametitle{Accès manuel aux champs}
Soit \Code{b} une variable de type \Code{B} initialisée par
\begin{lstlisting}
    B b = {1000, 2000, 3000};
\end{lstlisting}
Cette variable est organisée en mémoire en
\begin{center}
\scalebox{.6}{\begin{tikzpicture}
    \node[BlocM](9)at(-1,0){};
    \node[BlocM](10)at(-2,0){};
    \node[BlocM](11)at(-3,0){};
    \node[BlocM](16)at(12,0){};
    \node at(-4,0){\Large $\dots$};
    \node at(13,0){\Large $\dots$};
    \node[Bloc,fill=Jaune!40,draw=Jaune!100](1)at(0,0){};
    \node[Bloc,fill=Jaune!40,draw=Jaune!100](2)at(1,0){};
    \node[Bloc,fill=Bleu!40,draw=Bleu!100](5)at(4,0){};
    \node[Bloc,fill=Bleu!40,draw=Bleu!100](6)at(5,0){};
    \node[Bloc,fill=Bleu!40,draw=Bleu!100](7)at(6,0){};
    \node[Bloc,fill=Bleu!40,draw=Bleu!100](8)at(7,0){};
    \node[Bloc,fill=Rouge!40,draw=Rouge!100](12)at(8,0){};
    \node[Bloc,fill=Rouge!40,draw=Rouge!100](13)at(9,0){};
    \node[BlocM,fill=Noir!50,draw=Noir](14)at(10,0){};
    \node[BlocM,fill=Noir!50,draw=Noir](15)at(11,0){};
    \node[BlocM,fill=Noir!50,draw=Noir](3)at(2,0){};
    \node[BlocM,fill=Noir!50,draw=Noir](4)at(3,0){};
    \draw(-0.5,0.75)edge[anchor=south,<->,line width=1.5pt]
        node{\small 1 octet}(0.5,0.75);
    \draw(-0.5,-0.75)edge[anchor=north,<->,line width=1.5pt]
        node{\small \Code{a.x}}(1.5,-0.75);
    \draw(3.5,-0.75)edge[anchor=north,<->,line width=1.5pt]
        node{\small \Code{a.z}}(7.5,-0.75);
    \draw(7.5,-0.75)edge[anchor=north,<->,line width=1.5pt]
        node{\small \Code{a.y}}(9.5,-0.75);
\end{tikzpicture}}
\end{center}
\bigskip

On peut accéder aux champs de \Code{b} de la manière suivante~:
\begin{lstlisting}
short x, y;
int z;
void *p;
p = &b;
x = *((short *) p); /* Equivalent a x = a.x; */
p += 4;
z = *((int *) p); /* Equivalent a z = a.z; */
p += 4;
y = *((short *) p); /* Equivalent a y = a.y; */
\end{lstlisting}
\end{frame}

%%%%%%%%%%%%%%%%%%%%%%%%%%%%%%%%%%%%%%%%%%%%%%%%%%%%%%%%%%%%%%%%%%%%%%%%
\begin{frame}[fragile]
\frametitle{L'option {\tt Wpadded}}
L'option du compilateur \Code{-Wpadded} permet d'obtenir un avertissement
sanctionnant la déclaration d'un type structuré nécessitant des octets
de complétion.
\medskip

Par exemple, avec le type structuré \Code{B} défini par
\begin{lstlisting}
typedef struct {
    short x;
    int z;
    short y;
} B;
\end{lstlisting}
on obtient l'avertissement
\begin{footnotesize}
\begin{verbatim}
Prog.c:3:9: warning: padding struct to align ‘z’ [-Wpadded]
     int z;
         ^
Prog.c:5:1: warning: padding struct size to alignment boundary [-Wpadded]
 } B;
 ^
\end{verbatim}
\end{footnotesize}
\end{frame}

%%%%%%%%%%%%%%%%%%%%%%%%%%%%%%%%%%%%%%%%%%%%%%%%%%%%%%%%%%%%%%%%%%%%%%%%
\begin{frame}[fragile]
\frametitle{Alignement en mémoire --- ce qu'il faut retenir}
L'alignement mémoire d'une variable d'un type structuré dépend de
beaucoup de paramètres, notamment~:
\smallskip

\begin{itemize}
    \item de l'architecture de la machine exécutant ou ayant compilé le
    programme~;
    \smallskip

    \item du compilateur ayant compilé le programme.
\end{itemize}
\bigskip
\bigskip

Il est donc important de savoir que le calcul de la {\bf taille} d'une
variable d'un {\bf type structuré} n'est pas immédiat et
{\bf dépend du contexte}.
\end{frame}
