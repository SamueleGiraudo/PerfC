% Auteur : Samuele Giraudo
% Création : aout 2014
% Modifications : oct. 2014, mars 2016

%%%%%%%%%%%%%%%%%%%%%%%%%%%%%%%%%%%%%%%%%%%%%%%%%%%%%%%%%%%%%%%%%%%%%%%%
%%%%%%%%%%%%%%%%%%%%%%%%%%%%%%%%%%%%%%%%%%%%%%%%%%%%%%%%%%%%%%%%%%%%%%%%
%%%%%%%%%%%%%%%%%%%%%%%%%%%%%%%%%%%%%%%%%%%%%%%%%%%%%%%%%%%%%%%%%%%%%%%%
\section{Types}

%%%%%%%%%%%%%%%%%%%%%%%%%%%%%%%%%%%%%%%%%%%%%%%%%%%%%%%%%%%%%%%%%%%%%%%%
%%%%%%%%%%%%%%%%%%%%%%%%%%%%%%%%%%%%%%%%%%%%%%%%%%%%%%%%%%%%%%%%%%%%%%%%
\subsection{Notion de type}

%%%%%%%%%%%%%%%%%%%%%%%%%%%%%%%%%%%%%%%%%%%%%%%%%%%%%%%%%%%%%%%%%%%%%%%%
\begin{frame}\frametitle{Types}
Un \alert{type} peut être vu comme un ensemble (fini ou infini) de valeurs.
\bigskip

\uncover<2->{
Dire qu'une variable \Code{x} est de type \Code{T} signifie que
la valeur de \Code{x} est dans \Code{T}.
\bigskip}

\uncover<3->{
Il existe deux sortes de types~:

\begin{enumerate}
    \item les \alert{types scalaires}, qui sont des types atomiques et
    définis à l'avance dans le langage~;
    \smallskip}

    \uncover<4->{
    \item les \alert{types composites}, qui sont des assemblages de
    types scalaires ou de types composites par le biais des constructions
    \Code{struct}, \Code{enum} ou tableau.}
\end{enumerate}
\end{frame}

%%%%%%%%%%%%%%%%%%%%%%%%%%%%%%%%%%%%%%%%%%%%%%%%%%%%%%%%%%%%%%%%%%%%%%%%
\begin{frame}[fragile] \frametitle{Type d'une variable}
Le type d'une variable indique comment {\bf interpréter} la zone
mémoire qui lui est attribuée ainsi que sa {\bf taille}.
\bigskip

\uncover<2->{
L'opérateur \Code{sizeof} permet de connaître la taille en octets d'un
type. On peut aussi l'appliquer à une valeur. P.ex., \Code{sizeof(int)}
et \Code{sizeof(33)} valent \Code{4}.
\bigskip}

\begin{multicols}{2}
\begin{semiverbatim}\footnotesize\uncover<3->{
int t;
char c;

printf("%d ", sizeof(int));
printf("%d ", sizeof(t));
printf("%d ", sizeof(31));

printf("%d ", sizeof(char));
printf("%d ", sizeof(c));
printf("%d\\n", sizeof('a'));}
\end{semiverbatim}
\uncover<3->{
Ceci affiche \Sortie{4 4 4 1 1 4}.
\medskip

L'expression \Code{sizeof('a')} vaut \Code{4}. En effet, même
si \Code{'a'} est un caractère, c'est avant tout un entier.
\smallskip

La conversion est implicite.}
\end{multicols}

\uncover<4->{
Le type d'une variable est \alert{attribué à sa déclaration} et ne peut
pas être modifié.}
\end{frame}

%%%%%%%%%%%%%%%%%%%%%%%%%%%%%%%%%%%%%%%%%%%%%%%%%%%%%%%%%%%%%%%%%%%%%%%%
%%%%%%%%%%%%%%%%%%%%%%%%%%%%%%%%%%%%%%%%%%%%%%%%%%%%%%%%%%%%%%%%%%%%%%%%
\subsection{Types scalaires}

%%%%%%%%%%%%%%%%%%%%%%%%%%%%%%%%%%%%%%%%%%%%%%%%%%%%%%%%%%%%%%%%%%%%%%%%
\begin{frame} \frametitle{Types entier}
On se place sur une machine $64$ bits.
\medskip

\begin{center}
    \begin{tabular}{c|c|c}
        Nom & Taille (octets) & Plage \\ \hline
        \Code{char} & $1$ & $-128$ à $127$ \\
        \Code{short} & $2$ & $-32768$ à $32767$ \\
        \Code{int} & $4$ & $-2^{31}$ à $2^{31} - 1$ \\
        \Code{long} & $8$ & $-2^{63}$ à $2^{63} - 1$
    \end{tabular}
\end{center}
\medskip

\uncover<2->{
Chacun de ces types peut être précédé de \Code{unsigned} pour faire en
sorte de ne représenter que des entiers positifs. On a ainsi les plages suivantes~:
\begin{center}
    \begin{tabular}{c|c}
        Nom & Plage \\ \hline
        \Code{unsigned char} & $0$ à $255$ \\
        \Code{unsigned short} & $0$ à $65535$ \\
        \Code{unsigned int} & $0$ à $2^{32} - 1$ \\
        \Code{unsigned long} & $0$ à $2^{64} - 1$
    \end{tabular}
\end{center}}
\end{frame}

%%%%%%%%%%%%%%%%%%%%%%%%%%%%%%%%%%%%%%%%%%%%%%%%%%%%%%%%%%%%%%%%%%%%%%%%
\begin{frame}[fragile] \frametitle{Entiers non signés}
Dans le cas où l'on a besoin de représenter uniquement des valeurs
entières positives, on utilisera les versions non signées des types
entiers.
\medskip

\uncover<2->{
Quelques avantages de ce procédé~:

\begin{enumerate}
    \item possibilité de représenter des entiers plus grands~;
    \smallskip

    \item gain de lisibilité du programme.
\end{enumerate}
\bigskip
\bigskip}

\uncover<3->{
{\bf Attention}~: les instructions}
\begin{semiverbatim}\uncover<3->{
unsigned int i;
for (i = 8 ; i >= 0 ; -\,\!-i) \{
    ...
\}}
\end{semiverbatim}
\uncover<3->{
produisent une boucle infinie. En effet, \Code{i} étant non signé, il
est toujours positif et donc la condition \Code{i >= 0} est toujours
vraie.}
\end{frame}

%%%%%%%%%%%%%%%%%%%%%%%%%%%%%%%%%%%%%%%%%%%%%%%%%%%%%%%%%%%%%%%%%%%%%%%%
\begin{frame} \frametitle{Constantes entières}
Il existe plusieurs manières d'exprimer des \alert{constantes entières}~:
\medskip

\begin{itemize}
    \item en base dix~: \Code{0}, \Code{29}, \Code{-322}, \dots
    \medskip

    \uncover<2->{
    \item en octal~: \Code{01}, \Code{0145}, \Code{-01234567}, \dots
    \medskip}

    \uncover<3->{
    \item en hexadécimal~: \Code{0x1}, \Code{0x5555FFFF},
    \Code{-0x98879AFA}, \dots
    \medskip}

    \uncover<4->{
    \item par un caractère~: \Code{'a'}, \Code{'9'}, \Code{'*'},
    \Code{'\textbackslash n'}, \dots}
\end{itemize}
\bigskip

\uncover<5->{
Un entier peut être représenté par un caractère car tout
caractère est représenté par son code ASCII (qui est un entier
compris entre \Code{0} et \Code{127}).
\bigskip
\bigskip}

\uncover<6->{
{\bf Attention}~: ne pas confondre les caractères chiffres avec les
entiers (l'entier \Code{'1'} vaut \Code{49} et non pas \Code{1}).}
\end{frame}

%%%%%%%%%%%%%%%%%%%%%%%%%%%%%%%%%%%%%%%%%%%%%%%%%%%%%%%%%%%%%%%%%%%%%%%%
\begin{frame} \frametitle{Types flottant}
On se place sur une machine $64$ bits.
\medskip

\begin{center}
    \begin{tabular}{c|c|c}
        Nom & Taille (octets) & Valeur absolue maximale \\ \hline
        \Code{float} & $4$ & $3.40282 \times 10^{38}$ \\
        \Code{double} & $8$ & $1.79769 \times 10^{308}$ \\
        \Code{long double} & $16$ & $1.18973 \times 10^{4932}$ \\
    \end{tabular}
\end{center}
\medskip

Le fichier d'en-tête \Code{float.h} contient des constantes donnant d'autres
renseignements sur les types flottant.
\end{frame}

%%%%%%%%%%%%%%%%%%%%%%%%%%%%%%%%%%%%%%%%%%%%%%%%%%%%%%%%%%%%%%%%%%%%%%%%
\begin{frame}[fragile] \frametitle{Danger des types flottant}
\begin{multicols}{2}
\begin{semiverbatim}
float x = 10000001.0;
printf("%f\\n", x);
\end{semiverbatim}
Ces instructions affichent, de manière attendue, \Sortie{10000001.000000}.
\end{multicols}
\medskip

\begin{multicols}{2}
\begin{semiverbatim}\uncover<2->{
float x = 100000001.0;
printf("%f\\n", x);}
\end{semiverbatim}
\uncover<2->{
En revanche, ces instructions affichent, de manière inattendue,
\Sortie{100000000.000000}.}
\end{multicols}
\bigskip

\uncover<3->{
Les nombres flottants sont représentés de manière \alert{approchée}.
\medskip

Comme ces exemples le montrent, même certains entiers, représentables
de manière exacte par des types entier, ne le sont pas par des types
flottant.}
\end{frame}

%%%%%%%%%%%%%%%%%%%%%%%%%%%%%%%%%%%%%%%%%%%%%%%%%%%%%%%%%%%%%%%%%%%%%%%%
\begin{frame} \frametitle{Danger des types flottant~: une solution partielle}
Les types flottant présentent divers désavantages par rapport aux types
entier~:
\begin{enumerate}
    \item \alert{représentation non exacte} des nombres~;
    \item opérations arithmétiques beaucoup \alert{moins efficaces}.
\end{enumerate}
\bigskip

\uncover<2->{
Pour ces raisons, {\bf il est recommandé de ne jamais utiliser de types
flottant}.
\bigskip}

\uncover<3->{
{\bf Solution partielle}~: on représente par l'entier $x \times 10^{k}$
tout nombre $x$ qui dispose de $k \geq 0$ chiffres (en base dix) après
la virgule, $k$ étant fixé.
\medskip}

\uncover<4->{
P.ex., si l'on a besoin de manipuler des nombres à $k := 2$ chiffres
après la virgule, les nombres $0.15$ et $331.9$ sont respectivement
représentés par les entiers $15$ et $33190$.}
\end{frame}

%%%%%%%%%%%%%%%%%%%%%%%%%%%%%%%%%%%%%%%%%%%%%%%%%%%%%%%%%%%%%%%%%%%%%%%%
\begin{frame}\frametitle{Opérations sur les types scalaires}
Les valeurs d'un type scalaire (entier ou flottant) forment un
\alert{ensemble totalement ordonné}~: étant donné deux valeurs, il est
toujours possible de les comparer. On utilise pour cela les
\alert{opérateurs relationnels}
\begin{center}
    \Code{==}, \Code{!=}, \Code{<=}, \Code{>=}, \Code{<}, \Code{>}.
\end{center}

\uncover<2->{
Il est possible de mélanger des comparaisons de valeurs de types entier
et de types flottant. Dans ce cas, les entiers sont convertis
implicitement en une valeur de type flottant avant d'effectuer
la comparaison.
\bigskip}

\uncover<3->{
Sur des variables de type scalaires sont définis les
\alert{opérateurs arithmétiques}
\begin{center}
    \Code{+}, \Code{-}, \Code{*}, \Code{/}, \Code{++}, \Code{-\,\!-}.
\end{center}
Les opérateurs \Code{++} et \Code{-\,\!-} servent à additionner ou à
retrancher de $1$ la valeur des variables sur lesquels ils sont appliqués.}
\end{frame}

%%%%%%%%%%%%%%%%%%%%%%%%%%%%%%%%%%%%%%%%%%%%%%%%%%%%%%%%%%%%%%%%%%%%%%%%
%%%%%%%%%%%%%%%%%%%%%%%%%%%%%%%%%%%%%%%%%%%%%%%%%%%%%%%%%%%%%%%%%%%%%%%%
\subsection{Types construits}

%%%%%%%%%%%%%%%%%%%%%%%%%%%%%%%%%%%%%%%%%%%%%%%%%%%%%%%%%%%%%%%%%%%%%%%%
\begin{frame}[fragile] \frametitle{Types structurés}
La syntaxe
\smallskip

\Code{typedef struct ALIAS $\lbrace$} \\
\qquad \Code{TYPE\_1 CHAMP\_1;} \\
\qquad \Code{TYPE\_2 CHAMP\_2;} \\
\qquad \Code{\dots} \\
\Code{$\rbrace$ NOM;}
\smallskip

permet de \alert{déclarer un type structuré} \Code{NOM}, constitué
des {\bf champs} \Code{CHAMP\_1}, \Code{CHAMP\_2}, \dots.
L'{\bf alias} \Code{ALIAS} est facultatif.
\bigskip

\uncover<2->{
C'est un {\bf amalgame} de types.
\bigskip}

\uncover<3->{
P.ex., \vspace{-.5em}}
\begin{multicols}{2}
\begin{semiverbatim}\uncover<3->{
typedef struct \{
    int x;
    int y;
\} Couple;}
\end{semiverbatim}
\uncover<3->{
déclare un type structuré \Code{Couple} qui permet de représenter des
couples d'entiers.}
\end{multicols}
\end{frame}

%%%%%%%%%%%%%%%%%%%%%%%%%%%%%%%%%%%%%%%%%%%%%%%%%%%%%%%%%%%%%%%%%%%%%%%%
\begin{frame}[fragile] \frametitle{Types structurés}
Si \Code{x} est une variable d'un type structuré \Code{T} contenant
le champ \Code{ch}, on {\bf accède} à ce champ par la syntaxe
\begin{center}
    \Code{x.ch}
\end{center}
\medskip

\uncover<2->{
Si \Code{adr\_x} est une adresse sur une variable de type \Code{T},
on accède à ce même champ par la syntaxe
\begin{center}
    \Code{adr\_x->ch}
\end{center}}
\uncover<3->{
Cette syntaxe est un raccourci pour
\begin{center}
    \Code{(*adr\_x).ch}
\end{center}
\bigskip}

\uncover<4->{
P.ex., les trois suites d'instructions suivantes sont équivalentes~:
\vspace{-.5em}}
\begin{multicols}{3}
\begin{semiverbatim}\uncover<4->{
Couple *c;
...
c->x = c->x + 1;}
\end{semiverbatim}

\begin{semiverbatim}\uncover<4->{
Couple *c;
...
*(c).x = c->x + 1;}
\end{semiverbatim}

\begin{semiverbatim}\uncover<4->{
Couple *c;
...
c->x = (*c).x + 1;}
\end{semiverbatim}
\end{multicols}
\end{frame}

%%%%%%%%%%%%%%%%%%%%%%%%%%%%%%%%%%%%%%%%%%%%%%%%%%%%%%%%%%%%%%%%%%%%%%%%
\begin{frame}[fragile] \frametitle{Opérations sur les types structurés}
Les {\bf opérateurs relationnels} ne sont pas définis sur les types
structurés.
\medskip

Il est donc impossible de tester si deux variables d'un même type
structuré sont égales au moyen de l'opérateur \Code{==}. Il faut
tester l'égalité de chacun des champs qui les constituent.
\bigskip

\uncover<2->{
En revanche, l'{\bf opérateur d'affectation} \Code{=} est compatible avec
les types structurés.}

\begin{multicols}{2}
\begin{semiverbatim}\footnotesize\uncover<3->{
typedef struct \{
    char nom[32];
    char prenom[32];
    int age;
\} Personne;
...
Personne p1, p2;
scanf(" %s", p1.nom);
scanf(" %s", p1.prenom);
p1.age = 30;
p2 = p1;}
\end{semiverbatim}
\uncover<3->{
L'affectation en dernière ligne fait en sorte que tous les champs de
\Code{p2} contiennent les mêmes valeurs que ceux de \Code{p1}.
\bigskip}

\uncover<4->{
Il y a recopie des tableaux statiques \Code{p1.nom} et
\Code{p1.prenom} dans \Code{p2.nom} et \Code{p2.prenom}.}
\end{multicols}
\end{frame}

%%%%%%%%%%%%%%%%%%%%%%%%%%%%%%%%%%%%%%%%%%%%%%%%%%%%%%%%%%%%%%%%%%%%%%%%
\begin{frame}[fragile] \frametitle{Types énumérés}
La syntaxe
\smallskip

\Code{typedef enum $\lbrace$} \\
\qquad \Code{ENU\_1,} \\
\qquad \Code{ENU\_2,} \\
\qquad \Code{\dots} \\
\Code{$\rbrace$ NOM;}
\smallskip

permet de \alert{déclarer un type énuméré} \Code{NOM}, constitué
des {\bf énumérateurs} \Code{ENU\_1}, \Code{ENU\_2}, \dots.
(Attention, on utilise des \Code{,} et non pas des \Code{;}.)
\bigskip

\uncover<2->{
Une valeur de ce type prend pour valeur exactement un des énumérateurs
qui le constituent.
\bigskip}

\uncover<3->{P.ex.,\vspace{-.5em}}
\begin{multicols}{2}
\begin{semiverbatim}\uncover<3->{
typedef enum \{
    FAUX,
    VRAI
\} Booleen;}
\end{semiverbatim}
\uncover<3->{
est un type qui permet de représenter des booléens.
\smallskip

Une valeur de type \Code{Booleen} est soit \Code{FAUX}, soit \Code{VRAI}.}
\end{multicols}
\end{frame}

%%%%%%%%%%%%%%%%%%%%%%%%%%%%%%%%%%%%%%%%%%%%%%%%%%%%%%%%%%%%%%%%%%%%%%%%
\begin{frame}[fragile] \frametitle{Types énumérés}
\begin{multicols}{2}
\begin{semiverbatim}\footnotesize
typedef enum \{
    LUNDI,    /* = 0 */
    MARDI,    /* = 1 */
    MERCREDI, /* = 2 */
    JEUDI,    /* = 3 */
    VENDREDI, /* = 4 */
    SAMEDI,   /* = 5 */
    DIMANCHE  /* = 6 */
\} Jour;
...
printf("%d\\n", MERCREDI);
\end{semiverbatim}
Les énumérateurs sont des {\bf expressions entières}. Leur valeur est
déterminée par leur ordre de déclaration dans le type.
\smallskip

Ces instructions affichent \Sortie{2}.
En effet, \Code{LUNDI} vaut \Code{0} car il est le 1\ier{} énumérateur
déclaré et les valeurs des suivants s'incrémentent selon leur
ordre de déclaration.
\end{multicols}

\begin{multicols}{2}
\begin{semiverbatim}\footnotesize\uncover<2->{
typedef enum \{
    LA = 0,
    SI = 2,
    DO,       /* = 3 */
    RE = 5,
    MI = 7,
    FA = 8,
    SOL = 10
\} Note;}
\end{semiverbatim}\small
\uncover<2->{
Il est possible de spécifier manuellement les valeurs
des énumérateurs avec la syntaxe \Code{ENU = VAL} où \Code{ENU}
est un énumérateur et \Code{VAL} une constante entière.
\smallskip

Si une valeur n'est pas spécifiée, elle est déduite de la précédente
en l'incrémentant.}
\end{multicols}
\end{frame}

%%%%%%%%%%%%%%%%%%%%%%%%%%%%%%%%%%%%%%%%%%%%%%%%%%%%%%%%%%%%%%%%%%%%%%%%
\begin{frame}[fragile] \frametitle{Types énumérés}
L'utilisation de branchement \Code{switch} est particulièrement
adaptée pour traiter une variable d'un type énuméré.
\medskip

\begin{multicols}{2}
\begin{semiverbatim}\small\uncover<2->{
Note note;

scanf(" %d", &note);

switch (note) \{
  case LA : printf("A"); break;
  case SI : printf("B"); break;
  case DO : printf("C"); break;
  case RE : printf("D"); break;
  case MI : printf("E"); break;
  case FA : printf("F"); break;
  case SOL : printf("G");break;
  default : printf(
    "%d non note", note);
\}}
\end{semiverbatim}
\vspace{1em}

\uncover<2->{
Ces instructions lisent une valeur entière sur l'entrée standard
et l'affichent (en notation internationale).
\smallskip}

\uncover<3->{
{\bf Remarque}~: une variable d'un type énuméré peut prendre comme
valeur n'importe quel entier. Ceci explique la présence de la
clause \Code{default}.
\smallskip}

\uncover<4->{
L'intérêt de l'utilisation des types énumérés est principalement
{\bf sémantique}~: un programme qui les utilise est plus facile
à lire et à maintenir.}
\end{multicols}
\end{frame}

%%%%%%%%%%%%%%%%%%%%%%%%%%%%%%%%%%%%%%%%%%%%%%%%%%%%%%%%%%%%%%%%%%%%%%%%
\begin{frame}[fragile] \frametitle{Opérations sur les types énumérés}
Contrairement aux types structurés, il est possible de comparer les
éléments d'un type énuméré au moyen des {\bf opérateurs relationnels}.
\smallskip

Ceci est une conséquence du fait que les énumérateurs sont des entiers.

\begin{multicols}{2}
\begin{semiverbatim}
printf("%d\\n", SOL == SOL);
printf("%d\\n", SOL == FA);
printf("%d\\n", SI <= RE);
\end{semiverbatim}
affiche \Sortie{1} \\
affiche \Sortie{0} \\
affiche \Sortie{1}
\end{multicols}
\medskip

\uncover<2->{
De même, l'{\bf opérateur d'affectation} \Code{=} est compatible
avec les types énumérés.
\bigskip}

\uncover<3->{
L'{\bf opérateur de taille} \Code{sizeof} renvoie \Code{4} sur les
valeurs d'un type énuméré. C'est la taille occupée par le type \Code{int}.}
\end{frame}
