% Auteur : Samuele Giraudo
% Création : nov. 2013
% Modifications : juil. 2014 oct. 2014 nov. 2014, déc. 2015, mars 2016

%%%%%%%%%%%%%%%%%%%%%%%%%%%%%%%%%%%%%%%%%%%%%%%%%%%%%%%%%%%%%%%%%%%%%%%%
%%%%%%%%%%%%%%%%%%%%%%%%%%%%%%%%%%%%%%%%%%%%%%%%%%%%%%%%%%%%%%%%%%%%%%%%
%%%%%%%%%%%%%%%%%%%%%%%%%%%%%%%%%%%%%%%%%%%%%%%%%%%%%%%%%%%%%%%%%%%%%%%%
\section{Opérateurs}

%%%%%%%%%%%%%%%%%%%%%%%%%%%%%%%%%%%%%%%%%%%%%%%%%%%%%%%%%%%%%%%%%%%%%%%%
%%%%%%%%%%%%%%%%%%%%%%%%%%%%%%%%%%%%%%%%%%%%%%%%%%%%%%%%%%%%%%%%%%%%%%%%
\subsection{Généralités}

%%%%%%%%%%%%%%%%%%%%%%%%%%%%%%%%%%%%%%%%%%%%%%%%%%%%%%%%%%%%%%%%%%%%%%%%
\begin{frame}[fragile]
\frametitle{Caractéristiques d'un opérateur}
Un opérateur dispose des {\bf caractéristiques structurelles} suivantes~:
\smallskip

\begin{enumerate}
    \uncover<2->{
    \item son \alert{arité}, qui désigne le nombre d'opérandes sur
    lesquelles il agit~;
    \medskip}

    \uncover<3->{
    \item sa \alert{précédence}, qui permet de savoir, dans une expression,
    dans quel ordre appliquer les différents opérateurs qui la composent~;
    \medskip}

    \uncover<4->{
    \item son \alert{sens d'associativité}, qui permet de savoir, dans
    une expression, dans quel sens appliquer des mêmes opérateurs qui la
    composent.}
\end{enumerate}
\end{frame}

%%%%%%%%%%%%%%%%%%%%%%%%%%%%%%%%%%%%%%%%%%%%%%%%%%%%%%%%%%%%%%%%%%%%%%%%
\begin{frame}[fragile]
\frametitle{Précédence et associativité des opérateurs}
Considérons l'expression \Code{3 * 2 + 1}.
\smallskip

\uncover<2->{
Suivant les priorités relatives des opérateurs \Code{*} et \Code{+},
il y deux manières de l'évaluer~:
\smallskip}

\begin{enumerate}
    \uncover<3->{
    \item \Code{(3 * 2) + 1}, si \Code{*} est {\bf plus prioritaire}
    que \Code{+}~;
    \smallskip}

    \uncover<4->{
    \item \Code{3 * (2 + 1)}, si \Code{+} est {\bf plus prioritaire}
    que \Code{*}.}
\end{enumerate}
\bigskip
\bigskip

\uncover<5->{
Considérons l'expression \Code{4 - 3 - 2 - 1}.
\smallskip}

\uncover<6->{
Suivant le sens d'associativité de \Code{-}, il y a deux manières de
l'évaluer~:
\smallskip}

\begin{enumerate}
    \uncover<7->{
    \item \Code{((4 - 3) - 2) - 1}, si \Code{-} est {\bf associatif de
    gauche à droite}~;
    \smallskip}

    \uncover<8->{
    \item \Code{4 - (3 - (2 - 1))}, si \Code{-} est {\bf associatif de
    droite à gauche}.}
\end{enumerate}
\bigskip
\bigskip

\uncover<9->{
Tout ceci peut être rendu explicite par l'\alert{utilisation de parenthèses}.}
\end{frame}

%%%%%%%%%%%%%%%%%%%%%%%%%%%%%%%%%%%%%%%%%%%%%%%%%%%%%%%%%%%%%%%%%%%%%%%%
%%%%%%%%%%%%%%%%%%%%%%%%%%%%%%%%%%%%%%%%%%%%%%%%%%%%%%%%%%%%%%%%%%%%%%%%
\subsection{Opérateurs d'accès}

%%%%%%%%%%%%%%%%%%%%%%%%%%%%%%%%%%%%%%%%%%%%%%%%%%%%%%%%%%%%%%%%%%%%%%%%
\begin{frame}[fragile]
\frametitle{Opérateurs de gestion la mémoire}

\begin{center}
    \begin{tabular}{c|c|c|c|c}
        {\bf Op.} & {\bf Rôle} & {\bf Ari.} & {\bf Assoc.}
            & {\bf Opérandes} \\ \hline \hline
        \Code{\&} & référencement & 1 & -- & une variable \\ \hline
        \Code{*} & déréférencement & 1 & -- & un pointeur \\ \hline
        \Code{[\,]} & élément d'un tableau & 2 & $\longrightarrow$
            & un pointeur et un entier \\ \hline
        \multirow{2}{*}{\Code{.}} & \multirow{2}{*}{valeur d'un champ} &
            \multirow{2}{*}{2} & \multirow{2}{*}{$\longrightarrow$}
            & une var. d'un type struct. \\
            & & & & et un id. de champ \\ \hline
        \multirow{3}{*}{\Code{->}} & \multirow{3}{*}{valeur d'un champ} &
            \multirow{3}{*}{2} & \multirow{3}{*}{$\longrightarrow$}
            & une pointeur sur \\
            & & & & une var. d'un type struct. \\
            & & & & et un id. de champ
    \end{tabular}
\end{center}
\end{frame}

%%%%%%%%%%%%%%%%%%%%%%%%%%%%%%%%%%%%%%%%%%%%%%%%%%%%%%%%%%%%%%%%%%%%%%%%
%%%%%%%%%%%%%%%%%%%%%%%%%%%%%%%%%%%%%%%%%%%%%%%%%%%%%%%%%%%%%%%%%%%%%%%%
\subsection{Opérateurs de calcul}

%%%%%%%%%%%%%%%%%%%%%%%%%%%%%%%%%%%%%%%%%%%%%%%%%%%%%%%%%%%%%%%%%%%%%%%%
\begin{frame}[fragile]
\frametitle{Opérateurs arithmétiques}

\begin{center}
    \begin{tabular}{c|c|c|c|c}
        {\bf Op.} & {\bf Rôle} & {\bf Ari.} & {\bf Assoc.}
            & {\bf Opérandes} \\ \hline \hline
        \Code{+}, \Code{-}, \Code{*}, \Code{/} & opérations arith.
            & 2 & $\longrightarrow$ & deux val. numériques \\ \hline
        \Code{\%} & modulo & 2 & $\longrightarrow$ & deux entiers \\ \hline
        \Code{+}, \Code{-} & signe  & 1 & -- & une val. numérique \\ \hline
        \multirow{2}{*}{\Code{++}, \Code{-\,-}} &
            \multirow{2}{*}{incr./décr.} & \multirow{2}{*}{1} &
            \multirow{2}{*}{--} &
            une var. d'un \\
            & & & & type numérique
    \end{tabular}
\end{center}
\end{frame}

%%%%%%%%%%%%%%%%%%%%%%%%%%%%%%%%%%%%%%%%%%%%%%%%%%%%%%%%%%%%%%%%%%%%%%%%
\begin{frame}[fragile]
\frametitle{L'opérateur modulo}
L'opérateur \alert{modulo} \Code{\%} calcule le reste de la division
euclidienne de son premier opérande par son second.
\medskip

\uncover<2->{
D'un point de vue mathématique, si $a$ et $b$ sont deux entiers, on a
$a = b \times q + r$, où $0 \leq r \leq b - 1$ et $q$ est un entier.
$q$ est le {\bf quotient} et $r$ est le {\bf reste}, \alert{toujours positif}.
\bigskip}

\uncover<3->{
Cependant, \Code{\%} peut produire des valeurs négatives, dans le cas où
l'un des deux opérandes est négatif.
\bigskip}

\uncover<4->{
Solution pour un modulo qui respecte la définition mathématique~:}
\begin{semiverbatim}\small\uncover<4->{
int vrai_modulo(int a, int b) \{
    int r;
    r = a % b;
    if (r < 0)
        return r + b;
    return r;
\}}
\end{semiverbatim}
\end{frame}

%%%%%%%%%%%%%%%%%%%%%%%%%%%%%%%%%%%%%%%%%%%%%%%%%%%%%%%%%%%%%%%%%%%%%%%%
\begin{frame}[fragile]
\frametitle{Les opérateurs d'incrémentation et de décrémentation}
Les opérateurs \Code{++} et \Code{-\,-} existent en deux versions,
suivant qu'ils soient préfixes ou suffixes~:
\smallskip

\begin{enumerate}
    \uncover<2->{
    \item \Code{a++}, incrémente (de un) la valeur de la variable \Code{a}
    et est une expression dont la valeur est l'\alert{ancienne valeur}
    de \Code{a}~;
    \medskip}

    \uncover<3->{
    \item \Code{++a}, incrémente (de un) la valeur de la variable \Code{a}
    et est une expression dont la valeur est la \alert{nouvelle valeur}
    de \Code{a}.}
\end{enumerate}
\bigskip

\begin{multicols}{2}
\begin{semiverbatim}\uncover<4->{
int a = 5, b;
b = 3 + a++;}
\end{semiverbatim}
\uncover<4->{\Code{b} vaut \Code{8} et \Code{a} vaut \Code{6}.}

\begin{semiverbatim}\uncover<5->{
int a = 5, b;
b = 3 + ++a;}
\end{semiverbatim}
\uncover<5->{\Code{b} vaut \Code{9} et \Code{a} vaut \Code{6}.}
\end{multicols}
\end{frame}

%%%%%%%%%%%%%%%%%%%%%%%%%%%%%%%%%%%%%%%%%%%%%%%%%%%%%%%%%%%%%%%%%%%%%%%%
\begin{frame}[fragile]
\frametitle{Les opérateurs d'incrémentation et de décrémentation}
Attention au {\bf pièges d'utilisation} de ces opérateurs.
\bigskip

\uncover<2->{
P.ex., les instructions}
\begin{multicols}{3}
\begin{semiverbatim}\uncover<2->{
int a = 5, b;
b = a++ + ++a;}
\end{semiverbatim}
\begin{semiverbatim}\uncover<2->{
int a = 5;
a = a++;}
\end{semiverbatim}
\begin{semiverbatim}\uncover<2->{
int a = 5;
a = ++a;}
\end{semiverbatim}
\end{multicols}
\uncover<2->{ne sont pas évaluables (l'effet produit par les lignes 2
dépend du compilateur et de ses options).
\bigskip
\bigskip}

\uncover<3->{
{\bf Règle}~: pour éviter ce type de piège, on s'interdit de réaliser
plus d'une modification d'une même variable dans une même expression.}
\end{frame}

%%%%%%%%%%%%%%%%%%%%%%%%%%%%%%%%%%%%%%%%%%%%%%%%%%%%%%%%%%%%%%%%%%%%%%%%
\begin{frame}[fragile]
\frametitle{Opérateurs relationnels}

\begin{center}
    \begin{tabular}{c|c|c|c|c}
        {\bf Op.} & {\bf Rôle} & {\bf Ari.} & {\bf Assoc.}
            & {\bf Opérandes} \\ \hline \hline
        \Code{<}, \Code{>} & comparaison stricte & 2 & $\longrightarrow$
            & deux val. numériques \\ \hline
        \Code{<=}, \Code{>=} & comparaison large & 2 & $\longrightarrow$
            & deux val. numériques \\ \hline
        \Code{==} & égalité & 2 & $\longrightarrow$ & deux val. numériques \\ \hline
        \Code{!=} & différence & 2 & $\longrightarrow$ & deux val. numériques
    \end{tabular}
\end{center}
\medskip

\uncover<2->{
Toutes les expressions de la forme
\begin{center} \Code{v1 CMP v2}\end{center}
où \Code{v1} et \Code{v2} sont des valeurs numériques et \Code{CMP} est
un opérateur de comparaison produisent une valeur~:
\begin{itemize}
    \item \Code{1} si la comparaison est vraie~;
    \item \Code{0} sinon.
\end{itemize}}
\end{frame}

%%%%%%%%%%%%%%%%%%%%%%%%%%%%%%%%%%%%%%%%%%%%%%%%%%%%%%%%%%%%%%%%%%%%%%%%
\begin{frame}[fragile]
\frametitle{Opérateurs relationnels}
Un pointeur étant une adresse, et donc une valeur numérique,
il est possible de comparer deux pointeurs.
\medskip

\begin{multicols}{2}
\begin{semiverbatim}\footnotesize\uncover<2->{
char *ptr1, *ptr2;
char c;
c = 'a';
ptr1 = &c;
ptr2 = &c;
if (ptr1 == ptr2)
    printf("ok1\\n");
if (&ptr1 == &ptr2)
    printf("ok2\\n");}
\end{semiverbatim}
\uncover<2->{Ceci affiche juste \Sortie{ok1}.
\smallskip

En effet, les deux pointeurs \Code{ptr1} et \Code{ptr2}
pointent vers le même emplacement en mémoire.
\smallskip

Le second test est faux car les adresses des variables \Code{ptr1} et
\Code{ptr2} sont différentes.}
\end{multicols}
\medskip

\begin{multicols}{2}
\begin{semiverbatim}\footnotesize\uncover<3->{
int t1[2], t2[2];
t1[0] = 1;
t1[1] = 2;
t2[0] = 1;
t2[1] = 2;
if (t1 == t2)
    printf("ok\\n");}
\end{semiverbatim}
\small
\uncover<3->{
Ceci compare les {\bf adresses} de \Code{t1} et \Code{t2} et non
pas les valeurs de leurs cases.
\smallskip

Rien n'est donc affiché car les tableaux \Code{t1} et \Code{t2} sont à
des adresses différentes.}
\end{multicols}
\end{frame}

%%%%%%%%%%%%%%%%%%%%%%%%%%%%%%%%%%%%%%%%%%%%%%%%%%%%%%%%%%%%%%%%%%%%%%%%
\begin{frame}[fragile]
\frametitle{Opérateurs logiques}

\begin{center}
    \begin{tabular}{c|c|c|c|c}
        {\bf Op.} & {\bf Rôle} & {\bf Ari.} & {\bf Assoc.}
            & {\bf Opérandes} \\ \hline \hline
        \Code{\&\&} & et logique & 2 & $\longrightarrow$
            & deux val. numériques \\ \hline
        \Code{||} & ou logique & 2 & $\longrightarrow$
            & deux val. numériques \\ \hline
        \Code{!} & non logique & 1 & --
            & une val. numérique
    \end{tabular}
\end{center}
\medskip

\uncover<2->{
Toutes les expressions formées d'opérateurs logiques produisent une
valeur, \Code{0} ou bien \Code{1}.
\medskip

Cette valeur est
\begin{itemize}
    \item \Code{1} si l'expression logique est vraie~;
    \item \Code{0} sinon.
\end{itemize}}
\end{frame}

%%%%%%%%%%%%%%%%%%%%%%%%%%%%%%%%%%%%%%%%%%%%%%%%%%%%%%%%%%%%%%%%%%%%%%%%
\begin{frame}[fragile]
\frametitle{Opérateurs bit à bit}

\begin{center}
    \begin{tabular}{c|c|c|c|c}
        {\bf Op.} & {\bf Rôle} & {\bf Ari.} & {\bf Assoc.}
            & {\bf Opérandes} \\ \hline \hline
        \Code{\&} & et bit à bit & 2 & $\longrightarrow$
            & deux val. entières \\ \hline
        \Code{|} & ou bit à bit & 2 & $\longrightarrow$
            & deux val. entières \\ \hline
        \Code{\textasciicircum} & xor bit à bit & 2 & $\longrightarrow$
            & deux val. entières \\ \hline
        \Code{$\sim$} & non bit à bit & 1 & --
            & une val. entière \\ \hline
        \Code{<\,<}, \Code{>\,>} & déc. g./d. bit à bit & 2
            & $\longrightarrow$ & deux val. entières
    \end{tabular}
\end{center}
\end{frame}

%%%%%%%%%%%%%%%%%%%%%%%%%%%%%%%%%%%%%%%%%%%%%%%%%%%%%%%%%%%%%%%%%%%%%%%%
\begin{frame}[fragile]
\frametitle{Et/ou/xor/non bit à bit}
Quelques exemples d'opérations bit à bit~:
\bigskip

\begin{multicols}{2}
\scalebox{.75}{
\begin{tabular}{c}
    \Code{x} \\
    \Code{y} \\
    \Code{x \& y}
\end{tabular}
\begin{tabular}{|c|c|c|c|c|c|c|c|} \hline
    \Code{0} & \Code{1} & \Code{1} & \Code{1}
        & \Code{0} & \Code{1} & \Code{0} & \Code{1} \\ \hline
    \Code{1} & \Code{0} & \Code{1} & \Code{0}
        & \Code{1} & \Code{1} & \Code{0} & \Code{0} \\ \hline
    \Code{0} & \Code{0} & \Code{1} & \Code{0}
        & \Code{0} & \Code{1} & \Code{0} & \Code{0} \\ \hline
\end{tabular}}
\bigskip
\bigskip

\uncover<2->{
\scalebox{.75}{
\begin{tabular}{c}
    \Code{x} \\
    \Code{y} \\
    \Code{x | y}
\end{tabular}
\begin{tabular}{|c|c|c|c|c|c|c|c|} \hline
    \Code{0} & \Code{1} & \Code{1} & \Code{1}
        & \Code{0} & \Code{1} & \Code{0} & \Code{1} \\ \hline
    \Code{1} & \Code{0} & \Code{1} & \Code{0}
        & \Code{1} & \Code{1} & \Code{0} & \Code{0} \\ \hline
    \Code{1} & \Code{1} & \Code{1} & \Code{1}
        & \Code{1} & \Code{1} & \Code{0} & \Code{1} \\ \hline
\end{tabular}}}

\uncover<3->{
\scalebox{.75}{
\begin{tabular}{c}
    \Code{x} \\
    \Code{y} \\
    \Code{x \textasciicircum\,\! y}
\end{tabular}
\begin{tabular}{|c|c|c|c|c|c|c|c|} \hline
    \Code{0} & \Code{1} & \Code{1} & \Code{1}
        & \Code{0} & \Code{1} & \Code{0} & \Code{1} \\ \hline
    \Code{1} & \Code{0} & \Code{1} & \Code{0}
        & \Code{1} & \Code{1} & \Code{0} & \Code{0} \\ \hline
    \Code{1} & \Code{1} & \Code{0} & \Code{1}
        & \Code{1} & \Code{0} & \Code{0} & \Code{1} \\ \hline
\end{tabular}}}
\bigskip
\bigskip

\uncover<4->{
\scalebox{.75}{
\begin{tabular}{c}
    \Code{x} \\
    \Code{$\sim$\!\,x}
\end{tabular}
\begin{tabular}{|c|c|c|c|c|c|c|c|} \hline
    \Code{0} & \Code{1} & \Code{1} & \Code{1}
        & \Code{0} & \Code{1} & \Code{0} & \Code{1} \\ \hline
    \Code{1} & \Code{0} & \Code{0} & \Code{0}
        & \Code{1} & \Code{0} & \Code{1} & \Code{0} \\ \hline
\end{tabular}}}
\end{multicols}
\end{frame}

%%%%%%%%%%%%%%%%%%%%%%%%%%%%%%%%%%%%%%%%%%%%%%%%%%%%%%%%%%%%%%%%%%%%%%%%
\begin{frame}[fragile]
\frametitle{Et/ou/xor/non bit à bit}

Si les deux opérandes n'ont pas la même taille (en nombre de bits), le
plus petit est complété à gauche par des
\begin{itemize}
    \item {\bf zéros} s'il est non signé ou bien positif~;
    \item {\bf uns} s'il est négatif et signé.
\end{itemize}
\medskip

\uncover<2->{
Le signe d'un entier signé est lu sur son bit de poids fort~:
\begin{itemize}
    \item \Code{0} s'il est positif~;
    \item \Code{1} s'il est négatif.
\end{itemize}
\bigskip}

\begin{minipage}{.25\textwidth}
\begin{semiverbatim}\small\uncover<3->{
short x = 5;
char y = 10;
x = x | y;}
\end{semiverbatim}
\end{minipage}
\begin{minipage}{.63\textwidth}
\uncover<3->{
\scalebox{.7}{
\begin{tabular}{c}
    \Code{x} \\
    \Code{y} \\
    \Code{x | y}
\end{tabular}
\begin{tabular}{|c|c|c|c|c|c|c|c|c|c|c|c|c|c|c|c|} \hline
    \Code{0} & \Code{0} & \Code{0} & \Code{0} & \Code{0} & \Code{0} &
    \Code{0} & \Code{0} & \Code{0} & \Code{0} & \Code{0} & \Code{0} &
    \Code{0} & \Code{1} & \Code{0} & \Code{1} \\ \hline
    {\bf \CRouge{\tt 0}} & {\bf \CRouge{\tt 0}} & {\bf \CRouge{\tt 0}} &
    {\bf \CRouge{\tt 0}} & {\bf \CRouge{\tt 0}} & {\bf \CRouge{\tt 0}} &
    {\bf \CRouge{\tt 0}} & {\bf \CRouge{\tt 0}} &
    \Code{0} & \Code{0} & \Code{0} & \Code{0} &
    \Code{1} & \Code{0} & \Code{1} & \Code{0} \\ \hline
    \Code{0} & \Code{0} & \Code{0} & \Code{0} & \Code{0} & \Code{0} &
    \Code{0} & \Code{0} & \Code{0} & \Code{0} & \Code{0} & \Code{0} &
    \Code{1} & \Code{1} & \Code{1} & \Code{1} \\ \hline
\end{tabular}}}
\end{minipage}
\bigskip

\begin{minipage}{.25\textwidth}
\begin{semiverbatim}\small\uncover<4->{
short x = 5;
char y = -10;
x = x | y;}
\end{semiverbatim}
\end{minipage}
\begin{minipage}{.63\textwidth}
\uncover<4->{
\scalebox{.7}{
\begin{tabular}{c}
    \Code{x} \\
    \Code{y} \\
    \Code{x | y}
\end{tabular}
\begin{tabular}{|c|c|c|c|c|c|c|c|c|c|c|c|c|c|c|c|} \hline
    \Code{0} & \Code{0} & \Code{0} & \Code{0} & \Code{0} & \Code{0} &
    \Code{0} & \Code{0} & \Code{0} & \Code{0} & \Code{0} & \Code{0} &
    \Code{0} & \Code{1} & \Code{0} & \Code{1} \\ \hline
    {\bf \CVert{\tt 1}} & {\bf \CVert{\tt 1}} & {\bf \CVert{\tt 1}} &
    {\bf \CVert{\tt 1}} & {\bf \CVert{\tt 1}} & {\bf \CVert{\tt 1}} &
    {\bf \CVert{\tt 1}} & {\bf \CVert{\tt 1}} &
    \Code{1} & \Code{1} & \Code{1} & \Code{1} &
    \Code{0} & \Code{1} & \Code{1} & \Code{0} \\ \hline
    \Code{1} & \Code{1} & \Code{1} & \Code{1} & \Code{1} & \Code{1} &
    \Code{1} & \Code{1} & \Code{1} & \Code{1} & \Code{1} & \Code{1} &
    \Code{0} & \Code{1} & \Code{1} & \Code{1} \\ \hline
\end{tabular}}}
\end{minipage}
\end{frame}

%%%%%%%%%%%%%%%%%%%%%%%%%%%%%%%%%%%%%%%%%%%%%%%%%%%%%%%%%%%%%%%%%%%%%%%%
\begin{frame}[fragile]
\frametitle{Décalage bit à bit}

Si \Code{x} est non signé (déclaré avec \Code{unsigned}),
\begin{multicols}{2}
\uncover<2->{
\scalebox{.7}{
\begin{tabular}{c}
    \Code{x} \\
    \Code{x <\,< 3}
\end{tabular}
\begin{tabular}{|c|c|c|c|c|c|c|c|} \hline
    \Code{0} & \Code{1} & \Code{1} & \Code{1}
        & \Code{0} & \Code{1} & \Code{0} & \Code{1} \\ \hline
    \Code{1} & \Code{0} & \Code{1} & \Code{0}
        & \Code{1} & \CRouge{\tt 0} & \CRouge{\tt 0} & \CRouge{\tt 0} \\ \hline
\end{tabular}}}

\uncover<3->{
\scalebox{.7}{
\begin{tabular}{c}
    \Code{x} \\
    \Code{x >\,> 3}
\end{tabular}
\begin{tabular}{|c|c|c|c|c|c|c|c|} \hline
    \Code{0} & \Code{1} & \Code{1} & \Code{1}
        & \Code{0} & \Code{1} & \Code{0} & \Code{1} \\ \hline
    \CRouge{\tt 0} & \CRouge{\tt 0} & \CRouge{\tt 0} & \Code{0}
        & \Code{1} & \Code{1} & \Code{1} & \Code{0} \\ \hline
\end{tabular}}}
\end{multicols}
\bigskip
\bigskip

\uncover<4->{
Si \Code{x} est signé (déclaré sans \Code{unsigned}),}
\begin{multicols}{2}
\uncover<5->{
\scalebox{.7}{
\begin{tabular}{c}
    \Code{x} \\
    \Code{x <\,< 3}
\end{tabular}
\begin{tabular}{|c|c|c|c|c|c|c|c|} \hline
    \Code{0} & \Code{1} & \Code{1} & \Code{1}
        & \Code{0} & \Code{1} & \Code{0} & \Code{1} \\ \hline
    \Code{1} & \Code{0} & \Code{1} & \Code{0}
        & \Code{1} & \CRouge{\tt 0} & \CRouge{\tt 0} & \CRouge{\tt 0} \\ \hline
\end{tabular}}}

\uncover<6->{
\scalebox{.7}{
\begin{tabular}{c}
    \Code{x} \\
    \Code{x >\,> 3}
\end{tabular}
\begin{tabular}{|c|c|c|c|c|c|c|c|} \hline
    \Code{0} & \Code{1} & \Code{1} & \Code{1}
        & \Code{0} & \Code{1} & \Code{0} & \Code{1} \\ \hline
    \CRouge{\tt 0} & \CRouge{\tt 0} & \CRouge{\tt 0} & \Code{0}
        & \Code{1} & \Code{1} & \Code{1} & \Code{0} \\ \hline
\end{tabular}}}
\end{multicols}
\medskip

\begin{multicols}{2}
\uncover<7->{
\scalebox{.7}{
\begin{tabular}{c}
    \Code{x} \\
    \Code{x <\,< 3}
\end{tabular}
\begin{tabular}{|c|c|c|c|c|c|c|c|} \hline
    \Code{1} & \Code{1} & \Code{1} & \Code{1}
        & \Code{0} & \Code{1} & \Code{0} & \Code{1} \\ \hline
    \Code{1} & \Code{0} & \Code{1} & \Code{0}
        & \Code{1} & \CRouge{\tt 0} & \CRouge{\tt 0} & \CRouge{\tt 0} \\ \hline
\end{tabular}}}

\uncover<8->{
\scalebox{.7}{
\begin{tabular}{c}
    \Code{x} \\
    \Code{x >\,> 3}
\end{tabular}
\begin{tabular}{|c|c|c|c|c|c|c|c|} \hline
    \Code{1} & \Code{1} & \Code{1} & \Code{1}
        & \Code{0} & \Code{1} & \Code{0} & \Code{1} \\ \hline
    \CVert{\tt 1} & \CVert{\tt 1} & \CVert{\tt 1} & \Code{1}
        & \Code{1} & \Code{1} & \Code{1} & \Code{0} \\ \hline
\end{tabular}}}
\end{multicols}
\end{frame}


%%%%%%%%%%%%%%%%%%%%%%%%%%%%%%%%%%%%%%%%%%%%%%%%%%%%%%%%%%%%%%%%%%%%%%%%
\begin{frame} \frametitle{Exemple --- ensembles finis}
Les opérateurs bit à bit sont adaptés pour représenter des ensembles
finis et réaliser des opérations ensemblistes de manière simple et
efficace.
\bigskip

\uncover<2->{%
Soit $E$ un ensemble à $32$ éléments.
\smallskip

On considère que $E$ est muni d'une relation d'ordre totale de sorte
que ses éléments puissent être indexés de $0$ à $31$. Ainsi,
\begin{equation*}
    E = \{e_0, e_1, e_2, \dots, e_{31}\}.
\end{equation*}}

\uncover<3->{%
Cette indexation permet de représenter tout \alert{sous-ensemble} $S$ de
$E$ par un mot de $32$ bit dont le $i\ieme$ bit code la présence
(\Code{1}) ou l'absence (\Code{0}) de $e_i$ dans~$S$.
\medskip}

\uncover<4->{%
P.ex., l'entier dont l'écriture binaire est
\begin{equation*}
    {\tt 00010000 10000000 00000011 00000101}
\end{equation*}
code le sous ensemble $\{e_0, e_2, e_8, e_9, e_{23}, e_{28}\}$ de $E$.}
\end{frame}

%%%%%%%%%%%%%%%%%%%%%%%%%%%%%%%%%%%%%%%%%%%%%%%%%%%%%%%%%%%%%%%%%%%%%%%%
\begin{frame}[fragile] \frametitle{Exemple --- ensembles finis}
Ceci mène à la déclaration du type
\begin{lstlisting}
typedef unsigned int SousEnsemble;
\end{lstlisting}
\medskip

Pour tester si $e_i$ appartient à $S$, il suffit de réaliser un et bit
à bit entre l'entier représentant $S$ et le mot binaire constitué d'un
unique \Code{1} en $i\ieme{}$ position.
\smallskip

Cette expression vaut \Code{0} si $e_i \notin S$ et une valeur non nulle
sinon.
\medskip

Ainsi,
\begin{lstlisting}
int appartient_e_i(SousEnsemble s, int i) {
    assert(0 <= i);
    assert(i <= 31);

    return (1 << i) & s;
}
\end{lstlisting}

\end{frame}

%%%%%%%%%%%%%%%%%%%%%%%%%%%%%%%%%%%%%%%%%%%%%%%%%%%%%%%%%%%%%%%%%%%%%%%%
\begin{frame}[fragile] \frametitle{Exemple --- ensembles finis}
Pour réaliser l'union de deux sous-ensembles $S_1$ et $S_2$ de $E$,
il suffit de réaliser un ou bit à bit des deux entiers représentant
$S_1$ et $S_2$. En effet, pour tout $i$, $e_i \in S_1 \cup S_2$ si
$e_i \in S_1$ ou $e_i \in S_2$.
\medskip

Ainsi,
\begin{lstlisting}
SousEnsemble union(SousEnsemble s_1, SousEnsemble s_2) {
    return s_1 | s_2;
}
\end{lstlisting}
\bigskip

Pour réaliser l'intersection de deux sous-ensembles $S_1$ et $S_2$ de $E$,
il suffit de réaliser un et bit à bit des deux entiers représentant
$S_1$ et $S_2$. En effet, pour tout $i$, $e_i \in S_1 \cap S_2$ si
$e_i \in S_1$ et $e_i \in S_2$.
\medskip

Ainsi,
\begin{lstlisting}
SousEnsemble intersection(SousEnsemble s_1, SousEnsemble s_2) {
    return s_1 & s_2;
}
\end{lstlisting}
\end{frame}

%%%%%%%%%%%%%%%%%%%%%%%%%%%%%%%%%%%%%%%%%%%%%%%%%%%%%%%%%%%%%%%%%%%%%%%%
\begin{frame}[fragile]
\frametitle{Exemple --- Compter le nombre de bits à un}
{\bf But}~: écrire une fonction qui renvoie le nombre de bits à un
de son paramètre.
\bigskip

\uncover<2->{
On travaille sur des variables de \Code{64} bits. On considère pour cela
le type \Code{Mot64}
définit par}
\begin{semiverbatim}\small\uncover<2->{
typedef unsigned long long Mot64;}
\end{semiverbatim}
\bigskip
\bigskip

\uncover<3->{
{\bf $1\iere$ méthode}~: attraper le bit de poids faible et le pousser
à droite.}
\begin{semiverbatim}\small
\uncover<4->{int compter_un_1(Mot64 x) \{}
    \uncover<5->{int res, i;
    res = 0;}
    \uncover<6->{for (i = 0 ; i < 64 ; ++i) \{}
        \uncover<7->{if ((x & 1) == 1)
            res += 1;}
        \uncover<6->{x = x >\,> 1;
    \}}
    \uncover<5->{return res;}
\uncover<4->{\}}
\end{semiverbatim}
\end{frame}

%%%%%%%%%%%%%%%%%%%%%%%%%%%%%%%%%%%%%%%%%%%%%%%%%%%%%%%%%%%%%%%%%%%%%%%%
\begin{frame}[fragile]
\frametitle{Exemple --- Compter le nombre de bits à un}
{\bf $2\ieme$ méthode}~: on constate que pour tout entier \Code{x} non
nul (avec au moins un bit à un), l'expression \Code{x \& -x} est l'entier
qui contient un unique bit à un, le plus à droite de \Code{x}.
\medskip

\uncover<2->{
Ainsi, l'instruction
\begin{center}
    \Code{x = x \textasciicircum\,\! (x \& -x);}
\end{center}
transforme le bit à un le plus à droite de \Code{x} en un zéro.
\bigskip}

\uncover<3->{L'exploitation de cette idée donne}
\begin{semiverbatim}\small
\uncover<4->{int compter_un_2(Mot64 x) \{}
    \uncover<5->{int res;
    res = 0;}
    \uncover<6->{while (x != 0) \{
        x = x ^ (x & -x);
        res += 1;
    \}}
    \uncover<5->{return res;}
\uncover<4->{\}}
\end{semiverbatim}
\end{frame}

%%%%%%%%%%%%%%%%%%%%%%%%%%%%%%%%%%%%%%%%%%%%%%%%%%%%%%%%%%%%%%%%%%%%%%%%
\begin{frame}[fragile]
\frametitle{Exemple --- Compter le nombre de bits à un}
{\bf $3\ieme$ méthode}~: cette troisième méthode n'utilise pas de boucle.
\medskip

\uncover<2->{
On commence par construire un tableau qui associe à tout entier de huit
bits (un \Code{char}) le nombre de bits à un qu'il contient, en utilisant
au choix l'une des deux méthodes précédentes.}

\begin{semiverbatim}\small\uncover<3->{
int nombre_un[256];

void initialiser_nombre_un() \{
    int i;
    Mot64 x;
    for (i = 0 ; i < 256 ; ++i) \{
        x = (Mot64) i;
        nombre_un[i] = compter_un_2(x);
    \}
\}}
\end{semiverbatim}
\end{frame}

%%%%%%%%%%%%%%%%%%%%%%%%%%%%%%%%%%%%%%%%%%%%%%%%%%%%%%%%%%%%%%%%%%%%%%%%
\begin{frame}[fragile]
\frametitle{Exemple --- Compter le nombre de bits à un}
On isole l'octet d'indice \Code{i} d'une variable \Code{x} de type
\Code{Mot64} par l'expression
\begin{center}
    \Code{0xFF \& (x >\,> (8 * i))}
\end{center}
\medskip

\uncover<2->{
Comme le nombre de bits à un de \Code{x} est la somme du nombre de bits
à un de chacun des huit octets qui le constituent, on obtient}
\begin{semiverbatim}\small
\uncover<3->{int compter_un_3(Mot64 x) \{}
    \uncover<4->{int res;
    res = 0;}
    \uncover<5->{res += nombre_un[0xFF & x];}
    \uncover<6->{res += nombre_un[0xFF & (x >\,> 8)];}
    \uncover<7->{res += nombre_un[0xFF & (x >\,> 16)];}
    \uncover<8->{res += nombre_un[0xFF & (x >\,> 24)];}
    \uncover<9->{res += nombre_un[0xFF & (x >\,> 32)];}
    \uncover<10->{res += nombre_un[0xFF & (x >\,> 40)];}
    \uncover<11->{res += nombre_un[0xFF & (x >\,> 48)];}
    \uncover<12->{res += nombre_un[0xFF & (x >\,> 56)];}
    \uncover<4->{return res;}
\uncover<3->{\}}
\end{semiverbatim}
\end{frame}

%%%%%%%%%%%%%%%%%%%%%%%%%%%%%%%%%%%%%%%%%%%%%%%%%%%%%%%%%%%%%%%%%%%%%%%%
\begin{frame}[fragile]
\frametitle{Exemple --- Compter le nombre de bits à un}
{\bf $4\ieme$ méthode}~: on peut pousser plus loin l'idée précédente en
considérant des blocs de deux octets (au lieu d'un seul).

\begin{semiverbatim}\small\uncover<2->{
int nombre_un[65536]; /* 2^16 */

void initialiser_nombre_un() \{
    int i;
    Mot64 x;
    for (i = 0 ; i < 65536 ; ++i) \{
        x = (Mot64) i;
        nombre_un[i] = compter_un_2(x);
    \}
\}}
\end{semiverbatim}
\bigskip

\uncover<3->{
Taille mémoire occupée par le tableau \Code{nombre\_un}~:
\begin{equation*}
    2^{16} \times \Code{sizeof(int)} \:\mathrm{o}
        \uncover<4->{\; = \; 2^{18} \:\mathrm{o}}
        \uncover<5->{\; = \; \frac{2^{18}}{2^{10}} \:\mathrm{Kio}}
        \uncover<6->{\; = \; 256 \:\mathrm{Kio}.}
\end{equation*}}
\end{frame}

%%%%%%%%%%%%%%%%%%%%%%%%%%%%%%%%%%%%%%%%%%%%%%%%%%%%%%%%%%%%%%%%%%%%%%%%
\begin{frame}[fragile]
\frametitle{Exemple --- Compter le nombre de bits à un}
Ceci fournit la solution suivante.
\begin{semiverbatim}\small
int compter_un_4(Mot64 x) \{
    int res;
    res = 0;
    res += nombre_un[0xFFFF & x];
    res += nombre_un[0xFFFF & (x >\,> 16)];
    res += nombre_un[0xFFFF & (x >\,> 32)];
    res += nombre_un[0xFFFF & (x >\,> 48)];
    return res;
\}
\end{semiverbatim}
\bigskip

\uncover<2->{
Elle ne demande que quatre lectures dans le tableau \Code{nombre\_un},
au lieu des huit de la méthode précédente.}
\end{frame}

%%%%%%%%%%%%%%%%%%%%%%%%%%%%%%%%%%%%%%%%%%%%%%%%%%%%%%%%%%%%%%%%%%%%%%%%
\begin{frame}[fragile]
\frametitle{Exemple --- Compter le nombre de bits à un}
{\bf $5\ieme$ méthode}~: cette méthode est la plus compliquée. Elle se base
sur une compréhension très fine du fonctionnement des opérateurs
bit à bit.
\bigskip

\begin{semiverbatim}\small\uncover<2->{
int compter_un_5(Mot64 x) \{
    x = x - ((x >\,> 1) & 0x5555555555555555LLU);
    x = (x & 0x3333333333333333LLU) +
        ((x >\,> 2) & 0x3333333333333333LLU);
    x = (x + (x >\,> 4)) & 0x0F0F0F0F0F0F0F0FLLU;
    x = (x * 0x0101010101010101LLU) >\,> 56;
    return (int) x;
\}}
\end{semiverbatim}
\end{frame}

%%%%%%%%%%%%%%%%%%%%%%%%%%%%%%%%%%%%%%%%%%%%%%%%%%%%%%%%%%%%%%%%%%%%%%%%
\begin{frame}[fragile]
\frametitle{Mesure du temps d'exécution}
Mesurons maintenant les efficacités des cinq méthodes.
\medskip

\uncover<2->{
On utilise pour cela la fonction
\begin{center}
    \Code{clock\_t clock(void);}
\end{center}
de \Code{time.h}.
\bigskip}

\uncover<3->{
Schéma général pour mesurer le temps d'exécution d'une suite
d'instructions~:}
\begin{semiverbatim}\small
\uncover<5->{clock_t debut, fin;
double temps;}
\uncover<6->{debut = clock();}
\uncover<4->{...
/* Instructions */
...}
\uncover<7->{fin = clock();}
\uncover<8->{temps = (double) (fin - debut) / CLOCKS_PER_SEC;}
\uncover<9->{printf("%g s\\n", temps);}
\end{semiverbatim}
\end{frame}

%%%%%%%%%%%%%%%%%%%%%%%%%%%%%%%%%%%%%%%%%%%%%%%%%%%%%%%%%%%%%%%%%%%%%%%%
\begin{frame}[fragile]
\frametitle{Mesure du temps d'exécution}
On mesure le temps d'exécution des cinq méthodes en leur donnant
en entrée des nombres de $64$ bits générés de manière aléatoire.
\medskip

\uncover<2->{
Pour générer un nombre de $32$ bits de manière aléatoire, on utilise
la fonction
\begin{center}
    \Code{int rand(void);}
\end{center}
de \Code{stdlib.h}.
\medskip}

\uncover<3->{
Le nombre de $64$ bits est construit en générant aléatoirement ses quatre
octets de droite, puis ses quatre octets de gauche et en les associant avec
un ou bit à bit~:}
\begin{semiverbatim}\small
\uncover<4->{Mot64 mot64_alea() \{}
    \uncover<5->{Mot64 gauche, droite;}
    \uncover<6->{droite = (Mot64) rand();}
    \uncover<7->{gauche = ((Mot64) rand()) <\,< 32;}
    \uncover<8->{return gauche | droite;}
\uncover<4->{\}}
\end{semiverbatim}
\end{frame}

%%%%%%%%%%%%%%%%%%%%%%%%%%%%%%%%%%%%%%%%%%%%%%%%%%%%%%%%%%%%%%%%%%%%%%%%
\begin{frame}[fragile]
\frametitle{Mesure du temps d'exécution}
Voici les temps réalisés par chacune des cinq méthodes sur $84000000$
nombres de $64$ bits aléatoires~:
\begin{center}
    \begin{tabular}{c|c|c}
        {\bf Méthode} & {\bf Caractéristique} & {\bf Temps (s)} \\ \hline
        1 & Décalage droite & 32.9 \\
        2 & Suppression \Code{1} droite & 9.38 \\
        3 & Tableau \Code{256} & 2.23 \\
        4 & Tableau \Code{65536} & 1.93 \\
        5 & Compliquée & 1.82
    \end{tabular}
\end{center}
\medskip

\uncover<2->{
La $4\ieme$ méthode est plus de $16$ fois plus rapide que la $1\iere$.
Elle demande en revanche (tout comme la $3\ieme$) un \alert{pré-calcul}
et une occupation mémoire (par le tableau \Code{nombre\_un}).
\medskip}

\uncover<3->{
La $5\ieme$ méthode est la plus rapide et ne demande aucun pré-calcul.
Elle est en revanche difficile à comprendre et très difficile à
imaginer.}
\end{frame}

%%%%%%%%%%%%%%%%%%%%%%%%%%%%%%%%%%%%%%%%%%%%%%%%%%%%%%%%%%%%%%%%%%%%%%%%
%%%%%%%%%%%%%%%%%%%%%%%%%%%%%%%%%%%%%%%%%%%%%%%%%%%%%%%%%%%%%%%%%%%%%%%%
\subsection{Opérateurs d'affectation}

%%%%%%%%%%%%%%%%%%%%%%%%%%%%%%%%%%%%%%%%%%%%%%%%%%%%%%%%%%%%%%%%%%%%%%%%
\begin{frame}[fragile]
\frametitle{Opérateurs d'affectation}

\begin{center}
    \begin{tabular}{c|c|c|c|c}
        {\bf Op.} & {\bf Rôle} & {\bf Ari.} & {\bf Assoc.}
            & {\bf Opérandes} \\ \hline \hline
        \multirow{2}{*}{\Code{=}} & \multirow{2}{*}{affect.}
            & \multirow{2}{*}{2} & \multirow{2}{*}{$\longleftarrow$}
            & une var. \\
        & & & & et une val. \\ \hline
        \multirow{2}{*}{\Code{+=}, \Code{-=}, \Code{*=}, \Code{/=}, \Code{\%=}}
            & affect. compo.
            & \multirow{2}{*}{2} & $\longleftarrow$ & une var. num. \\
        & arith. & & & et une val. num \\ \hline
        \multirow{2}{*}{\Code{\&=}, \Code{|=}, \Code{\textasciicircum=},
        \Code{<\,<=}, \Code{>\,>=}}
        & affect. compo.
            & \multirow{2}{*}{2} & $\longleftarrow$ & une var. ent. \\
        & bit à bit & & & et une val. ent.
    \end{tabular}
\end{center}
\medskip

\uncover<2->{
Toute expression de la forme \Code{a X= b} est équivalente à
\Code{a = a X b}.}
\end{frame}

%%%%%%%%%%%%%%%%%%%%%%%%%%%%%%%%%%%%%%%%%%%%%%%%%%%%%%%%%%%%%%%%%%%%%%%%
\begin{frame}[fragile]
\frametitle{Opérateurs d'affectation}
Toutes les expressions d'affectation produisent une valeur qui est
la valeur qui vient d'être affectée.
\medskip

\uncover<2->{Par exemple, dans}
\begin{semiverbatim}\small\uncover<2->{
int a, b;
a = 2;
b = 5;
a *= b += 3;}
\end{semiverbatim}
\uncover<2->{
à cause de l'associativité des opérateurs d'affectation, la l. 4
s'interprète comme \Code{a *= (b += 3);}.
\smallskip

Ainsi, comme \Code{b += 3} produit la valeur \Code{8}, \Code{a} vaut
finalement \Code{16}.}
\end{frame}

%%%%%%%%%%%%%%%%%%%%%%%%%%%%%%%%%%%%%%%%%%%%%%%%%%%%%%%%%%%%%%%%%%%%%%%%
%%%%%%%%%%%%%%%%%%%%%%%%%%%%%%%%%%%%%%%%%%%%%%%%%%%%%%%%%%%%%%%%%%%%%%%%
\subsection{Autres opérateurs}

%%%%%%%%%%%%%%%%%%%%%%%%%%%%%%%%%%%%%%%%%%%%%%%%%%%%%%%%%%%%%%%%%%%%%%%%
\begin{frame}[fragile]
\frametitle{Autres opérateurs}

\begin{center}
    \begin{tabular}{c|c|c|c|c}
        {\bf Op.} & {\bf Rôle} & {\bf Ari.} & {\bf Assoc.}
            & {\bf Opérandes} \\ \hline \hline
        \Code{sizeof} & taille & 1 & -- & une var. ou un type \\ \hline
        \multirow{2}{*}{\Code{(T)}} & coercition & \multirow{2}{*}{1}
            & \multirow{2}{*}{--} & \multirow{2}{*}{une val.} \\
            & \Code{T} est un type & & & \\ \hline
        \Code{? :} & condition & 3 & -- & une val. num. et deux val. \\ \hline
        \Code{,}  & séquence & 2 & $\longrightarrow$ & deux val
    \end{tabular}
\end{center}
\end{frame}

%%%%%%%%%%%%%%%%%%%%%%%%%%%%%%%%%%%%%%%%%%%%%%%%%%%%%%%%%%%%%%%%%%%%%%%%
\begin{frame}[fragile]
\frametitle{L'opérateur de séquence}
Dans l'expression \Code{V1, V2}, où \Code{V1} et \Code{V2} sont
des valeurs, on commence par évaluer \Code{V1} puis ensuite \Code{V2}.
Cette expression produit la valeur \Code{V2}.
\bigskip

\uncover<2->{
L'opérateur \Code{,} est le plus souvent utilisé dans les {\bf champs des
boucles \Code{for}}.
\medskip}

\uncover<3->{P.ex.,}
\begin{semiverbatim}\small\uncover<3->{
int i, j, l;
...
for (i = 0, j = l - 1 ; i < j ; ++i, -\,-j) \{
...
\}}
\end{semiverbatim}
\uncover<3->{
permet d'obtenir une boucle \Code{for} avec {\bf deux compteurs}~: \Code{i}
croît et \Code{j} décroît dans l'intervalle allant de \Code{0} à
\Code{l - 1}.}
\end{frame}
