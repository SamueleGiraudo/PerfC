% Auteur : Samuele Giraudo
% Création : déc. 2015
% Modifications : déc. 2015

%%%%%%%%%%%%%%%%%%%%%%%%%%%%%%%%%%%%%%%%%%%%%%%%%%%%%%%%%%%%%%%%%%%%%%%%
%%%%%%%%%%%%%%%%%%%%%%%%%%%%%%%%%%%%%%%%%%%%%%%%%%%%%%%%%%%%%%%%%%%%%%%%
%%%%%%%%%%%%%%%%%%%%%%%%%%%%%%%%%%%%%%%%%%%%%%%%%%%%%%%%%%%%%%%%%%%%%%%%
\section{Habitudes}

%%%%%%%%%%%%%%%%%%%%%%%%%%%%%%%%%%%%%%%%%%%%%%%%%%%%%%%%%%%%%%%%%%%%%%%%
%%%%%%%%%%%%%%%%%%%%%%%%%%%%%%%%%%%%%%%%%%%%%%%%%%%%%%%%%%%%%%%%%%%%%%%%
\subsection{Mise en page}

%%%%%%%%%%%%%%%%%%%%%%%%%%%%%%%%%%%%%%%%%%%%%%%%%%%%%%%%%%%%%%%%%%%%%%%%
\begin{frame} \frametitle{Mise en page d'un programme}
Pour écrire un programme de valeur, il faut soigner les points suivants~:
\medskip

\begin{enumerate}
    \item l'indentation~;
    \medskip
    
    \item l'organisation des espaces autour des caractères~;
    \medskip
    
    \item le choix des identificateurs~;
    \medskip
    
    \item la documentation.
\end{enumerate}
\end{frame}

%%%%%%%%%%%%%%%%%%%%%%%%%%%%%%%%%%%%%%%%%%%%%%%%%%%%%%%%%%%%%%%%%%%%%%%%
\begin{frame}[fragile] \frametitle{Indentation}
L'\alert{indentation} consiste à disposer des caractères blancs au début 
de certaines lignes d'un programme. 
\medskip

Contrairement au {\sf Python}, celle-ci ne modifie pas le comportement 
d'un programme.
\medskip

L'objectif est d'augmenter sa lisibilité.
\medskip

Règle~: on place {\bf quatre espaces} (pas de tabulation) avant chaque 
instruction d'un bloc et on incrémente cet espacement de quatre en 
quatre en fonction de la profondeur des blocs.
\medskip

\begin{multicols}{2}
\begin{lstlisting}[showspaces=true]
/* Correct. */
a = 8;
if (b >= 0) {
    printf("%d\n", a);
    a = 0;
    for (i = 0 ; i <= b ; ++i)
        a /= 2;
}
\end{lstlisting}

\begin{lstlisting}[showspaces=true]
/* Incorrect. */
a = 8;
if (b >= 0) {
  printf("%d\n", a);
  a = 0;
  for (i = 0 ; i <= b ; ++i)
      a /= 2;
}
\end{lstlisting}
\end{multicols}
\end{frame}

%%%%%%%%%%%%%%%%%%%%%%%%%%%%%%%%%%%%%%%%%%%%%%%%%%%%%%%%%%%%%%%%%%%%%%%%
\begin{frame}[fragile] \frametitle{Organisation des blocs}
Un \alert{bloc} des une suite d'instructions délimitée par des accolades.
Il se trouve attaché à une instruction de branchement ou de boucle. Il
peut aussi être indépendant.
\medskip

On {\bf revient à la ligne} après une {\bf accolade ouvrante}.
\medskip

L'{\bf accolade fermante} se trouve horizontalement au {\bf niveau du 
début} de la ligne qui contient l'accolade ouvrante.

\begin{multicols}{2}
\begin{lstlisting}
/* Correct. */
if (valeur >= 1) {
    valeur -= 1;
}
\end{lstlisting}
\bigskip

\begin{lstlisting}
/* Incorrect. */
if (valeur >= 1) 
{
    valeur -= 1;
}
\end{lstlisting}
\end{multicols}

\begin{multicols}{2}
\begin{lstlisting}
/* Correct. */
valeur = 1;
{
    int a;
    valeur = 10;
}
\end{lstlisting}

\begin{lstlisting}
/* Incorrect. */
valeur = 1; {
    int a;
    valeur = 10;
}
\end{lstlisting}

\end{multicols}
\end{frame}

%%%%%%%%%%%%%%%%%%%%%%%%%%%%%%%%%%%%%%%%%%%%%%%%%%%%%%%%%%%%%%%%%%%%%%%%
\begin{frame}[fragile] \frametitle{Organisation des espaces}
On place une {\bf espace avant et après} chaque opérateur.
\begin{multicols}{2}
\begin{lstlisting}[showspaces=true]
/* Correct. */
a = b * 2 + 5;
\end{lstlisting}

\begin{lstlisting}[showspaces=true]
/* Incorrect. */
a = b*2 + 5;
\end{lstlisting}
\end{multicols}
\medskip

On utilise les règles habituelles de {\bf typographie} pour l'usage des 
{\bf virgules}.
\begin{multicols}{2}
\begin{lstlisting}[showspaces=true]
/* Correct. */
f(a, b, c, 16);
\end{lstlisting}

\begin{lstlisting}[showspaces=true]
/* Incorrect. */
f(a,b,c,16);
\end{lstlisting}
\end{multicols}
\medskip

On ne place {\bf pas d'espace} après une {\bf parenthèse ouvrante} ou 
avant une {\bf parenthèse fermante}.
\begin{multicols}{2}
\begin{lstlisting}[showspaces=true]
/* Correct. */
a = (f(a, 3) + a) * 2; 
\end{lstlisting}

\begin{lstlisting}[showspaces=true]
/* Incorrect. */
a = (f( a, 3) + a ) * 2;
\end{lstlisting}
\end{multicols}
\end{frame}

%%%%%%%%%%%%%%%%%%%%%%%%%%%%%%%%%%%%%%%%%%%%%%%%%%%%%%%%%%%%%%%%%%%%%%%%
\begin{frame}[fragile] \frametitle{Choix des identificateurs}
Les \alert{identificateurs} doivent à la fois {\bf renseigner sur le rôle} 
des entités auxquelles ils appartiennent (variables, paramètres, 
fonctions, modules, {\em etc.}) et être {\bf concis}. 
\begin{lstlisting}
/* Identificateur non explicite. */
v
\end{lstlisting}

\begin{lstlisting}
/* Identificateur trop long. */
valeur_choisie_pour_le_nombre_parties
\end{lstlisting}

\begin{lstlisting}
/* Identificateur acceptable. */
nb_parties
\end{lstlisting}
\medskip

On fixe la langue au {\bf français} pour leur construction.
\end{frame}

%%%%%%%%%%%%%%%%%%%%%%%%%%%%%%%%%%%%%%%%%%%%%%%%%%%%%%%%%%%%%%%%%%%%%%%%
\begin{frame}[fragile] \frametitle{Choix des identificateurs}
Les seuls identificateurs d'une lettre autorisés sont \Code{i},
\Code{j}, \Code{k} pour les indices de boucles.
\medskip

Les majuscules sont interdites dans les identificateurs. Dans un
identificateur composé de plusieurs mots, ces derniers sont séparés par 
des sous-tirets.

\begin{multicols}{2}
\begin{lstlisting}
/* Correct. */
nb_parties
\end{lstlisting}

\begin{lstlisting}
/* Incorrect. */
nbParties
\end{lstlisting}
\end{multicols}

Exception~: les identificateurs de constantes (définies par une 
instruction pré-processeur) sont écrits exclusivement en majuscules.
\begin{lstlisting}
/* Correct. */

#define TAILLE_MAX 1024
#define DEBUG
\end{lstlisting}
\end{frame}

%%%%%%%%%%%%%%%%%%%%%%%%%%%%%%%%%%%%%%%%%%%%%%%%%%%%%%%%%%%%%%%%%%%%%%%%
\begin{frame}[fragile] \frametitle{Documentation}
Un programme est documenté par des \alert{commentaires}. Ce sont des 
{\bf phrases}, placées entre \Code{/*} et \Code{*/}.
\medskip

On documente chaque \alert{fichier de programme}, au tout début, par

\begin{itemize}
    \item les prénoms et noms des auteurs~;
    \smallskip
    
    \item la date de création (au format jour-mois-année) ~;
    \smallskip
    
    \item la date de modification (au format précédent).
\end{itemize}

\begin{lstlisting}
/* Auteur : S. Giraudo
 * Creation : 21-12-2015
 * Modification : 21-12-2015 */
\end{lstlisting}
\bigskip

On commentera le moins possible les instructions.
\smallskip

Il faut éviter les commentaires inutiles.
\begin{lstlisting}
/* Incorrect. */
/* Affiche la valeur de `a`. */
printf("%d\n", a); 
\end{lstlisting}
\end{frame}

%%%%%%%%%%%%%%%%%%%%%%%%%%%%%%%%%%%%%%%%%%%%%%%%%%%%%%%%%%%%%%%%%%%%%%%%
\begin{frame}[fragile] \frametitle{Documentation des fonctions}
On documente la plupart des \alert{fonctions} par des commentaires 
situés avant leur déclaration (ou leur définition).
\medskip 

Un commentaire de fonction explique

\begin{itemize}
    \item le {\bf rôle} de chaque {\bf paramètre}~;
    \smallskip
    
    \item ce que {\bf renvoie} la fonction~; 
    \smallskip
     
    \item l'{\bf effet} produit par la fonction.
\end{itemize}

\begin{multicols}{2}
\begin{lstlisting}
/* Correct. */
/* Renvoie le plus grand entier 
 * parmi `a` et `b`. */
int max(int a, int b) {
    if (a >= b) 
        return a;
    return b;
}
\end{lstlisting}

\begin{lstlisting}
/* Incorrect. */
/* Calcule le maximum de deux 
 * entiers */
int max(int a, int b) {
    if (a >= b) 
        return a;
    return b;
}
\end{lstlisting}
\end{multicols}
\end{frame}

%%%%%%%%%%%%%%%%%%%%%%%%%%%%%%%%%%%%%%%%%%%%%%%%%%%%%%%%%%%%%%%%%%%%%%%%
\begin{frame}[fragile] \frametitle{Récapitulatif}
Dans cette sous-partie, nous avons vu~:

\begin{enumerate}
    \item des règles pour la mise en page des programmes~;
    \smallskip

    \item des règles pour la documentation des programmes.
\end{enumerate}
\end{frame}

%%%%%%%%%%%%%%%%%%%%%%%%%%%%%%%%%%%%%%%%%%%%%%%%%%%%%%%%%%%%%%%%%%%%%%%%
%%%%%%%%%%%%%%%%%%%%%%%%%%%%%%%%%%%%%%%%%%%%%%%%%%%%%%%%%%%%%%%%%%%%%%%%
\subsection{Gestion d'erreurs}

%%%%%%%%%%%%%%%%%%%%%%%%%%%%%%%%%%%%%%%%%%%%%%%%%%%%%%%%%%%%%%%%%%%%%%%%
\begin{frame} \frametitle{Mécanisme de gestion d'erreurs}
On écrira la plupart des fonctions selon le schéma suivant~:

\begin{itemize}
    \item le type de retour est \Code{int} et la valeur de retour, le 
    \alert{code d'erreur}, renseigne si l'exécution de la fonction s'est 
    bien déroulée~;
    \smallskip
    
    \item la (ou les) valeur(s) \og renvoyée(s) \fg{} par la fonction 
    se fait par un (des) passage(s) par adresse.
\end{itemize}
\bigskip

Schématiquement, une telle fonction admet ainsi le prototype
\begin{center}
    \Code{int FCT(T1 E1, \dots, TN EN, U1 S1, \dots, UK SK);}
\end{center}
dit \alert{prototype standard} où

\begin{itemize}
    \item les \Code{Ei} sont les paramètres d'entrée~;
    \smallskip
    
    \item les \Code{Ti} sont des types (potentiellement des adresses)~;
    \smallskip
    
    \item les \Code{Sj} sont les paramètres de sortie~;
    \smallskip
    
    \item les \Code{Uj} sont des types (potentiellement des adresses).
\end{itemize}
\end{frame}

%%%%%%%%%%%%%%%%%%%%%%%%%%%%%%%%%%%%%%%%%%%%%%%%%%%%%%%%%%%%%%%%%%%%%%%%
\begin{frame} \frametitle{Mécanisme de gestion d'erreurs}
Le \alert{code d'erreur} d'une fonction ayant un prototype standard 
suit la spécification suivante~:

\begin{itemize}
    \item une valeur nulle signifie que l'exécution de la fonction a 
    échoué~;
    \smallskip
    
    \item une valeur positive signifie que l'exécution de la fonction 
    s'est bien déroulée et apporte une information supplémentaire 
    exploitable~;
    \smallskip
    
    \item une valeur négative renseigne sur une erreur particulière.
\end{itemize}
\bigskip
\bigskip

{\bf Attention}~: il y a des fonctions de la librairie standard qui ne 
suivent pas cette spécification.
\smallskip

De notre côté, nous allons la suivre à la lettre dans les fonctions que 
nous écrirons.
\end{frame}

%%%%%%%%%%%%%%%%%%%%%%%%%%%%%%%%%%%%%%%%%%%%%%%%%%%%%%%%%%%%%%%%%%%%%%%%
\begin{frame}[fragile] \frametitle{Exemple 1 de gestion d'erreur}
Considérons la fonction
\begin{lstlisting}
int division(float x, float y, float *res) {
    if (y == 0)
        return 0;
    *res = x / y;
    return 1;
}
\end{lstlisting}

Les entrées sont les flottants \Code{x} et \Code{y}. La sortie est 
\Code{res}~; c'est une adresse qui pointera sur le résultat de la 
division de \Code{x} par \Code{y}.
\smallskip

La valeur de retour est un \alert{code d'erreur}~: il vaut \Code{0}
lorsque la division ne peut pas être calculée (\Code{y} nul) et vaut
\Code{1} sinon.
\smallskip

On remarque que l'on ne modifie pas \Code{*res} lorsque le calcul ne 
peut pas être réalisé.
\end{frame}

%%%%%%%%%%%%%%%%%%%%%%%%%%%%%%%%%%%%%%%%%%%%%%%%%%%%%%%%%%%%%%%%%%%%%%%%
\begin{frame}[fragile] \frametitle{Exemple 2 de gestion d'erreur}
\footnotesize
Considérons la fonction
\begin{lstlisting}
int nb_min_maj(char *chaine, int *res_min, int *res_maj)
    int i;
    *res_min = 0;
    *res_maj = 0;
    i = 0;
    while (chaine[i] != '\0') {
        if ('a' <= chaine[i] <= 'z')
            *res_min += 1;
        else if ('A' <= chaine[i] <= 'Z')
            *res_maj += 1;
        else
            return 0;
        i += 1;
    }
    return i;
}
\end{lstlisting}

L'entrée est la chaîne de caractères \Code{chaine}. Les sorties sont 
\Code{res\_min} et \Code{res\_maj} ; ces adresses pointeront
sur le nombre de minuscules et de majuscules dans \Code{chaine}.
\smallskip

La valeur de retour est un \alert{code d'erreur}~: il vaut \Code{0}
si un caractère non alphabétique apparaît dans \Code{chaine} et vaut 
la longueur de \Code{chaine} sinon.
\end{frame}

%%%%%%%%%%%%%%%%%%%%%%%%%%%%%%%%%%%%%%%%%%%%%%%%%%%%%%%%%%%%%%%%%%%%%%%%
\begin{frame} \frametitle{Fonctions classiques à gestion d'erreurs}
La librairie standard du {\sf C} contient beaucoup de fonctions à
gestion d'erreurs. Par exemple~:

\begin{itemize} \small
    \item \Code{printf} renvoie le nombre de caractères écrits (sans 
    compter \Code{'\textbackslash 0'})~;
    \smallskip
    
    \item \Code{scanf} renvoie le nombre d'affectations réalisées lors 
    de la lecture. La valeur \Code{EOF} est renvoyée si une erreur de 
    lecture a lieu~;
    \smallskip
    
    \item \Code{malloc} renvoie un pointeur vers la zone de la mémoire 
    allouée. Lorsque l'allocation échoue, la valeur \Code{NULL} est 
    renvoyée~;
    \smallskip
    
    \item \Code{fopen} renvoie un pointeur sur le fichier ouvert. 
    Lorsque l'ouverture échoue,  la valeur \Code{NULL} est renvoyée~;
    \smallskip
    
    \item \Code{fclose} renvoie \Code{0} si la fermeture du fichier 
    s'est bien déroulée (attention à ce cas particulier). Lorsque 
    la fermeture échoue, la valeur \Code{EOF} est renvoyée.
\end{itemize}

{\bf Remarque}~: certaines de ces fonctions ont une gestion d'erreurs
encore plus sophistiquée et modifient des variables globales 
comme \Code{errno} (de l'en-tête \Code{errno.h}) pour renseigner 
précisément sur l'erreur survenue.
\end{frame}

%%%%%%%%%%%%%%%%%%%%%%%%%%%%%%%%%%%%%%%%%%%%%%%%%%%%%%%%%%%%%%%%%%%%%%%%
\begin{frame}[fragile] \frametitle{Emploi des fonctions à gestion d'erreurs}
On combine l'appel d'une fonction à gestion d'erreurs avec un test pour 
traiter l'erreur éventuelle.
\begin{lstlisting}
if (division(8, a, &b) == 0) {
    /* Traitement de l'erreur lors
     * de la division par zero. */
}
/* Instructions suivantes. */
\end{lstlisting}

\begin{lstlisting}
if (nb_min_maj("UnDeuxTrois", &a, &b) == 0) {
    /* Traitement de l'erreur lorsque
     * la chaine de caracteres contient des 
     * caracteres non alphabetiques. */
}
/* Instructions suivantes. */
\end{lstlisting}
\end{frame}

%%%%%%%%%%%%%%%%%%%%%%%%%%%%%%%%%%%%%%%%%%%%%%%%%%%%%%%%%%%%%%%%%%%%%%%%
\begin{frame}[fragile] \frametitle{Emploi des fonctions à gestion d'erreurs}
Dans certains cas où une erreur survient, celle-ci peut être 
irrécupérable. Il faut donc interrompre l'exécution du programme. On
utilise pour cela la fonction
\begin{lstlisting}
    void exit(int status);
\end{lstlisting}
de \Code{stdlib.h}, appelée avec l'argument \Code{EXIT\_FAILURE}.

\begin{lstlisting}
/* Allocation dynamique. */
tab = (int *) malloc(sizeof(int) * 1024);

/* Verification de son succes. */
if (NULL == tab)
    /* Sur son echec, on interromp 
     * l'execution immediatement. */
    exit(EXIT_FAILURE);
/* Instructions suivantes. */
\end{lstlisting}
\end{frame}

%%%%%%%%%%%%%%%%%%%%%%%%%%%%%%%%%%%%%%%%%%%%%%%%%%%%%%%%%%%%%%%%%%%%%%%%
\begin{frame}[fragile] \frametitle{Récapitulatif}
Dans cette sous-partie, nous avons vu~:

\begin{enumerate}
    \item la notion de prototype standard pour les fonctions~;
    \smallskip

    \item un mécanisme pour renseigner les erreurs dans les fonctions.
\end{enumerate}
\end{frame}

%%%%%%%%%%%%%%%%%%%%%%%%%%%%%%%%%%%%%%%%%%%%%%%%%%%%%%%%%%%%%%%%%%%%%%%%
%%%%%%%%%%%%%%%%%%%%%%%%%%%%%%%%%%%%%%%%%%%%%%%%%%%%%%%%%%%%%%%%%%%%%%%%
\subsection{Assertions d'entrée}

%%%%%%%%%%%%%%%%%%%%%%%%%%%%%%%%%%%%%%%%%%%%%%%%%%%%%%%%%%%%%%%%%%%%%%%%
\begin{frame}[fragile] \frametitle{Assertions d'entrée}
Lors d'un appel à une fonction, certains arguments peuvent être dans un 
état incohérent.
\medskip

Au lieu de gérer ces cas de figure par l'usage de codes d'erreur, il est
possible de tester l'état des arguments.
\medskip

Une \alert{pré-assertion} (ou assertion d'entrée) est un test réalisé 
dans une fonction pour vérifier si elle appelée avec des arguments 
adéquats.
\bigskip

On utilise la fonction 
\begin{lstlisting}
void assert(int a);
\end{lstlisting} 
du fichier d’en-tête \Code{assert.h}. Elle fonction de la manière 
suivante~:

\begin{itemize}
    \item lorsque l'assertion \Code{a} est fausse, l’exécution du 
    programme est interrompue et diverses informations utiles sont 
    affichées~;
    \smallskip
    
    \item lorsque \Code{a} est vraie, l'exécution continue.
\end{itemize}
\end{frame}

%%%%%%%%%%%%%%%%%%%%%%%%%%%%%%%%%%%%%%%%%%%%%%%%%%%%%%%%%%%%%%%%%%%%%%%%
\begin{frame}[fragile] 
    \frametitle{Exemple 1 de fonction avec pré-assertions}
Considérons la fonction
\begin{lstlisting}
void afficher_tab(int tab[], int nb) {
    int i;
    assert(tab != NULL);
    assert(nb >= 0);
    for (i = 1 ; i <= nb ; ++i)
        printf("%c\n", tab[i]);
}
\end{lstlisting}

Elle possède deux pré-assertions~:

\begin{enumerate}
    \item la première teste si le tableau \Code{tab} est bien un pointeur
    valide (différent de \Code{NULL})~;
    \smallskip
    
    \item la seconde teste si la taille \Code{nb} donnée est bien 
    positive.
\end{enumerate}
\end{frame}

%%%%%%%%%%%%%%%%%%%%%%%%%%%%%%%%%%%%%%%%%%%%%%%%%%%%%%%%%%%%%%%%%%%%%%%%
\begin{frame}[fragile] \frametitle{Conception de pré-assertions}
Il est important de munir ses fonctions de pré-assertions les plus 
précises et complètes possibles. Quelques règles~:

\begin{itemize}
    \item la condition testée ne doit dépendre que des arguments d'une 
    fonction (elle ne dépend pas de données apprises à l'exécution)~;
    \smallskip
    
    \item la condition testée doit la plus atomique possible.
    \begin{multicols}{2}
        \begin{lstlisting}
/* Correct. */
assert(nb >= 0);
assert(nb <= 1024);
        \end{lstlisting}
  
        \begin{lstlisting}
/* Incorrect. */
assert(0 <= nb <= 1024);
        \end{lstlisting}
    \end{multicols}
    \smallskip
    
    \item Pour les concevoir, il faut imaginer les pires cas possibles
    à capturer qui peuvent survenir (p.ex., pointeurs nuls, quantités
    négatives, chaînes de caractères vides, {\em etc.}).
    \smallskip
    
    \item Elles sont situées juste après les déclarations de variables
    dans le corps d'une fonction.
\end{itemize}
\end{frame}

%%%%%%%%%%%%%%%%%%%%%%%%%%%%%%%%%%%%%%%%%%%%%%%%%%%%%%%%%%%%%%%%%%%%%%%%
\begin{frame}[fragile]
    \frametitle{Redondance nécessaire des pré-assertions}
Considérons les fonctions 
\begin{multicols}{2}
\begin{lstlisting}
int div(int a, int b) {
    assert(b != 0);
    return a / b;
}

\end{lstlisting}
\begin{lstlisting}
int somme_div(int a, int b) {
    return div(a, b)
        + div(b, a + 1);
}
\end{lstlisting}
\end{multicols}
{\bf Raisonnement}~: il n'y a pas de pré-assertion dans \Code{somme\_div}
mais cela n'est pas grave car les cas problématiques sont capturés par
\Code{div} qui contient une pré-assertion.
\medskip

Ceci est une {\bf fausse bonne idée}~: chaque fonction doit faire ses
propres pré-assertions. Toute erreur doit être capturée le plus en amont
possible.
\medskip

\begin{multicols}{2}
La bonne version de \Code{somme\_div} consiste à capturer les
mauvaises valeurs possibles de ses arguments de la manière suivante~:
\bigskip

\begin{lstlisting}
int somme_div(int a, int b) {
    assert(b != 0);
    assert(a + 1 != 0);
    return div(a, b)
        + div(b, a + 1);
}
\end{lstlisting}
\end{multicols}
\end{frame}

%%%%%%%%%%%%%%%%%%%%%%%%%%%%%%%%%%%%%%%%%%%%%%%%%%%%%%%%%%%%%%%%%%%%%%%%
\begin{frame}[fragile] \frametitle{Exemple 2 de fonction avec pré-assertion}
La fonction \Code{nb\_min\_maj}, à code d'erreur, doit être pourvue
de pré-assertions~:
\begin{lstlisting}
int nb_min_maj(char *chaine, int *res_min, int *res_maj)
    int i;
    
    assert(chaine != NULL);
    assert(res_min != NULL);
    assert(res_maj != NULL);
    
    /* Suite inchangee. */
}
\end{lstlisting}

On observe que le mécanisme de gestion d'erreurs par valeur de retour
teste des comportement incohérents complexes qui se déroulent à 
l'exécution, tandis que le mécanisme de pré-assertion permet de capturer 
des erreurs évidentes.
\end{frame}

%%%%%%%%%%%%%%%%%%%%%%%%%%%%%%%%%%%%%%%%%%%%%%%%%%%%%%%%%%%%%%%%%%%%%%%%
\begin{frame}[fragile] 
    \frametitle{Les pré-assertions pour corriger un programme}
\begin{center} \todo{À faire.} \end{center}
\end{frame}

%%%%%%%%%%%%%%%%%%%%%%%%%%%%%%%%%%%%%%%%%%%%%%%%%%%%%%%%%%%%%%%%%%%%%%%%
\begin{frame}[fragile] \frametitle{Récapitulatif}
Dans cette sous-partie, nous avons vu~:

\begin{enumerate}
    \item la notion de pré-assertions dans les fonctions~;
    \smallskip

    \item l'utilisation des pré-assertions pour corriger efficacement
    un programme.
\end{enumerate}
\end{frame}

