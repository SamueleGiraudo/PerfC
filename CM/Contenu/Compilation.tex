% Auteur : Samuele Giraudo
% Création : fév. 2014, jan 2015, mars 2015, déc. 2015, fév. 2016

\tikzstyle{Module}=[rectangle,draw=Violet!100,fill=Violet!20,
    line width=1pt,font=\scriptsize\tt]
\tikzstyle{FObj}=[rectangle,draw=Rouge!100,fill=Rouge!20,
    line width=1pt,font=\scriptsize\tt]
\tikzstyle{Bib}=[rectangle,draw=Vert!100,fill=Vert!20,
    line width=1pt,font=\scriptsize\tt]
\tikzstyle{Exec}=[rectangle,draw=Bleu!100,fill=Bleu!20,
    line width=1pt,font=\scriptsize\tt]
\tikzstyle{Fleche}=[->,draw=Rouge,line width=1.5pt]
\tikzstyle{Sommet}=[circle,draw=Marron!100,fill=Marron!10,line width=1.5pt]

%%%%%%%%%%%%%%%%%%%%%%%%%%%%%%%%%%%%%%%%%%%%%%%%%%%%%%%%%%%%%%%%%%%%%%%%
%%%%%%%%%%%%%%%%%%%%%%%%%%%%%%%%%%%%%%%%%%%%%%%%%%%%%%%%%%%%%%%%%%%%%%%%
%%%%%%%%%%%%%%%%%%%%%%%%%%%%%%%%%%%%%%%%%%%%%%%%%%%%%%%%%%%%%%%%%%%%%%%%
\section{Compilation}

%%%%%%%%%%%%%%%%%%%%%%%%%%%%%%%%%%%%%%%%%%%%%%%%%%%%%%%%%%%%%%%%%%%%%%%%
%%%%%%%%%%%%%%%%%%%%%%%%%%%%%%%%%%%%%%%%%%%%%%%%%%%%%%%%%%%%%%%%%%%%%%%%
\subsection{Étapes de compilation}

%%%%%%%%%%%%%%%%%%%%%%%%%%%%%%%%%%%%%%%%%%%%%%%%%%%%%%%%%%%%%%%%%%%%%%%%
\begin{frame}[fragile]
\frametitle{Compilation d'un projet d'un fichier}
La \alert{compilation} d'un projet constitué d'un \alert{unique fichier}
\Code{Fichier.c} contenant la fonction principale \Code{main} se fait
par la commande
\begin{center}\Code{gcc Fichier.c}\end{center}
\medskip

Cette commande réalise à la suite les étapes suivantes~:
\begin{enumerate}
    \item traitement préliminaire par le {\bf pré-processeur}~;
    \smallskip

    \item compilation en {\bf langage assembleur}~;
    \smallskip

    \item traduction du langage assembleur en {\bf langage machine}~;
    \smallskip

    \item {\bf édition des liens}.
\end{enumerate}
\medskip

Elle permet d'obtenir un fichier {\bf exécutable}.
\end{frame}

%%%%%%%%%%%%%%%%%%%%%%%%%%%%%%%%%%%%%%%%%%%%%%%%%%%%%%%%%%%%%%%%%%%%%%%%
\begin{frame}[fragile]
\frametitle{Compilation d'un projet d'un fichier}
\begin{center}
    \begin{tikzpicture}
        \node(c)at(0,0){\Code{Fichier.c}};
        \node(i)at(0,-3){\Code{Fichier.i}};
        \node(s)at(4,-3){\Code{Fichier.s}};
        \node(o)at(8,-3){\Code{Fichier.o}};
        \node(a)at(8,-6){\Code{a.out}};
        \draw[->](c)edge[anchor=west,font=\scriptsize] node{pré-processeur}(i);
        \draw[->](i)edge[anchor=north,font=\scriptsize] node{assembleur}(s);
        \draw[->](s)edge[anchor=north,font=\scriptsize] node{langage machine}(o);
        \draw[->](o)edge[anchor=east,font=\scriptsize] node{exécutable}(a);
    \end{tikzpicture}
\end{center}
\end{frame}

%%%%%%%%%%%%%%%%%%%%%%%%%%%%%%%%%%%%%%%%%%%%%%%%%%%%%%%%%%%%%%%%%%%%%%%%
\begin{frame}[fragile]
\frametitle{Pré-processeur}
Le \alert{pré-processeur} réalise un pré-traitement du fichier source
pour le rendre traduisible en langage machine.
\medskip

Il procède en
\begin{enumerate}
    \item supprimant les commentaires~;
    \smallskip

    \item incluant les fichiers d'en-tête
    (copie/colle les fichiers \Code{.h} inclus)~;
    \smallskip

    \item traitant les définitions de symboles par un
    mécanisme de substitution (\Code{\#define})~;
    \smallskip

    \item traitant les macro-instructions de contrôle de compilation
    (\Code{\#ifndef}, \Code{\#endif}, {\em etc.}).
\end{enumerate}
\medskip

Il est possible de récupérer le fichier d'extension \Code{.i} ainsi
obtenu par la commande
\begin{center} \Code{gcc -E Fichier.c >\,> Fichier.i} \end{center}
\end{frame}

%%%%%%%%%%%%%%%%%%%%%%%%%%%%%%%%%%%%%%%%%%%%%%%%%%%%%%%%%%%%%%%%%%%%%%%%
\begin{frame}[fragile]
\frametitle{Compilation en assembleur}
Après avoir été traité par le pré-processeur, le fichier \Code{Fichier.i}
est \alert{traduit en assembleur}.
\medskip

Il est possible de récupérer le fichier d'extension \Code{.s} ainsi
obtenu par la commande
\begin{center} \Code{gcc -S Fichier.c} \end{center}
\bigskip

L'assembleur est un langage très proche de la machine. Il peut se traduire
assez facilement en un langage directement exécutable par le procésseur.
\medskip

Il existe plusieurs langages d'assemblage différents~: au moins un par
architecture.
\end{frame}

%%%%%%%%%%%%%%%%%%%%%%%%%%%%%%%%%%%%%%%%%%%%%%%%%%%%%%%%%%%%%%%%%%%%%%%%
\begin{frame}[fragile]
\frametitle{Compilation en assembleur}
Par exemple, avec le fichier \Code{Fichier.c} suivant~:

\begin{minipage}[c]{.4\textwidth}
\begin{lstlisting}[frame=single,numbers=none,basicstyle=\ttfamily\scriptsize]
/* Fichier.c */
#include <stdio.h>

int main() {
    printf("Bonjour");
    return 0;
}
\end{lstlisting}
\end{minipage}

on obtient le fichier assembleur \Code{Fichier.s} suivant~:
\begin{multicols}{3}
\begin{lstlisting}[language={[x86masm]Assembler},numbers=none,
    basicstyle=\ttfamily\scriptsize]
    .file "Fichier.c"
    .section .rodata
.LC0:
    .string	"Bonjour"
    .text
    .globl main
    .type main, @function
main:
.LFB0:
    .cfi_startproc
    pushq %rbp
    .cfi_def_cfa_offset 16
    .cfi_offset 6, -16
    movq %rsp, %rbp
    .cfi_def_cfa_register 6
    movl $.LC0, %edi
    movl $0, %eax
    call printf
    movl $0, %eax
    popq %rbp
    .cfi_def_cfa 7, 8
    ret
    .cfi_endproc
.LFE0:
    .size main, .-main
    .ident "GCC: (Ubuntu/Linaro 4.8.1-10ubuntu9) 4.8.1"
    .section .note.GNU-stack,"",@progbits

\end{lstlisting}
\end{multicols}
\begin{math} \end{math}
\end{frame}

%%%%%%%%%%%%%%%%%%%%%%%%%%%%%%%%%%%%%%%%%%%%%%%%%%%%%%%%%%%%%%%%%%%%%%%%
\begin{frame}[fragile]
\frametitle{Traduction en langage machine}
Le code assembleur \Code{Fichier.s} est traduit en \alert{langage machine}.
\medskip

On obtient ce fichier d'extension \Code{.o} par la commande
\begin{center}\Code{gcc -c Fichier.c}\end{center}
\bigskip

Ce fichier s'appelle {\bf fichier objet}. Il est illisible pour un humain
mais peut cependant être affiché au moyen de la commande
\begin{center}\Code{od -x Fichier.o} ou bien \Code{od -a Fichier.o} \end{center}
\medskip

Le langage machine est directement compris par le processeur qui peut
de ce fait exécuter directement les instructions qu'il contient.
\end{frame}

%%%%%%%%%%%%%%%%%%%%%%%%%%%%%%%%%%%%%%%%%%%%%%%%%%%%%%%%%%%%%%%%%%%%%%%%
\begin{frame}[fragile]
\frametitle{Traduction en langage machine}
Par exemple, avec le programme précédent, le contenu de \Code{Fichier.o}
est
\medskip

{\tt \footnotesize
\setlength{\tabcolsep}{.07cm}
\begin{tabular}{ccccccccccccccccc}
0000000 &del &E &L &F &stx &soh &soh &nul &nul &nul &nul &nul &nul &nul &nul &nul \\
0000020 &soh &nul &> &nul &soh &nul &nul &nul &nul &nul &nul &nul &nul &nul &nul &nul \\
0000040 &nul &nul &nul &nul &nul &nul &nul &nul &0 &soh &nul &nul &nul &nul &nul &nul \\
0000060 &nul &nul &nul &nul &@ &nul &nul &nul &nul &nul &@ &nul &cr &nul &nl &nul \\
0000100 &U &H &ht &e &? &nul &nul &nul &nul &8 &nul &nul &nul &nul &h &nul \\
0000120 &nul &nul &nul &8 &nul &nul &nul &nul &] &C &B &o &n &j &o &u \\
0000140 &r &nul &nul &G &C &C &: &sp &( &U &b &u &n &t &u &/ \\
0000160 &L &i &n &a &r &o &sp &4 &. &8 &. &1 &- &1 &0 &u \\
0000200 &b &u &n &t &u &9 &) &sp &4 &. &8 &. &1 &nul &nul &nul \\
0000220 &dc4 &nul &nul &nul &nul &nul &nul &nul &soh &z &R &nul &soh &x &dle &soh \\
0000240 &esc &ff &bel &bs &dle &soh &nul &nul &fs &nul &nul &nul &fs &nul &nul &nul \\
0000260 &nul &nul &nul &nul &sub &nul &nul &nul &nul &A &so &dle &ack &stx &C &cr \\
0000300 &ack &U &ff &bel &bs &nul &nul &nul &nul &. &s &y &m &t &a &b \\
0000320 &nul &. &s &t &r &t &a &b &nul &. &s &h &s &t &r &t \\
0000340 &a &b &nul &. &r &e &l &a &. &t &e &x &t &nul &. &d \\
0000360 &a &t &a &nul &. &b &s &s &nul &. &r &o &d &a &t &a \\
\end{tabular}}
\end{frame}

%%%%%%%%%%%%%%%%%%%%%%%%%%%%%%%%%%%%%%%%%%%%%%%%%%%%%%%%%%%%%%%%%%%%%%%%
\begin{frame}[fragile]
\frametitle{Traduction en langage machine}
{\tt \footnotesize
\setlength{\tabcolsep}{.07cm}
\begin{tabular}{ccccccccccccccccc}
0000400 &nul &. &c &o &m &m &e &n &t &nul &. &n &o &t &e &. \\
0000420 &G &N &U &- &s &t &a &c &k &nul &. &r &e &l &a &. \\
0000440 &e &h &\_ &f &r &a &m &e &nul &nul &nul &nul &nul &nul &nul &nul \\
0000460 &nul &nul &nul &nul &nul &nul &nul &nul &nul &nul &nul &nul &nul &nul &nul &nul \\
* \\
0000560 &sp &nul &nul &nul &soh &nul &nul &nul &ack &nul &nul &nul &nul &nul &nul &nul \\
0000600 &nul &nul &nul &nul &nul &nul &nul &nul &@ &nul &nul &nul &nul &nul &nul &nul \\
0000620 &sub &nul &nul &nul &nul &nul &nul &nul &nul &nul &nul &nul &nul &nul &nul &nul \\
0000640 &soh &nul &nul &nul &nul &nul &nul &nul &nul &nul &nul &nul &nul &nul &nul &nul \\
0000660 &esc &nul &nul &nul &eot &nul &nul &nul &nul &nul &nul &nul &nul &nul &nul &nul \\
0000700 &nul &nul &nul &nul &nul &nul &nul &nul &dle &enq &nul &nul &nul &nul &nul &nul \\
0000720 &0 &nul &nul &nul &nul &nul &nul &nul &vt &nul &nul &nul &soh &nul &nul &nul \\
0000740 &bs &nul &nul &nul &nul &nul &nul &nul &can &nul &nul &nul &nul &nul &nul &nul \\
0000760 &\& &nul &nul &nul &soh &nul &nul &nul &etx &nul &nul &nul &nul &nul &nul &nul \\
0001000 &nul &nul &nul &nul &nul &nul &nul &nul &Z &nul &nul &nul &nul &nul &nul &nul \\
0001020 &nul &nul &nul &nul &nul &nul &nul &nul &nul &nul &nul &nul &nul &nul &nul &nul \\
0001040 &soh &nul &nul &nul &nul &nul &nul &nul &nul &nul &nul &nul &nul &nul &nul &nul \\
\end{tabular}}
\end{frame}

%%%%%%%%%%%%%%%%%%%%%%%%%%%%%%%%%%%%%%%%%%%%%%%%%%%%%%%%%%%%%%%%%%%%%%%%
\begin{frame}[fragile]
\frametitle{Traduction en langage machine}
{\tt \footnotesize
\setlength{\tabcolsep}{.07cm}
\begin{tabular}{ccccccccccccccccc}
0001060 &, &nul &nul &nul &bs &nul &nul &nul &etx &nul &nul &nul &nul &nul &nul &nul \\
0001100 &nul &nul &nul &nul &nul &nul &nul &nul &Z &nul &nul &nul &nul &nul &nul &nul \\
0001120 &nul &nul &nul &nul &nul &nul &nul &nul &nul &nul &nul &nul &nul &nul &nul &nul \\
0001140 &soh &nul &nul &nul &nul &nul &nul &nul &nul &nul &nul &nul &nul &nul &nul &nul \\
0001160 &1 &nul &nul &nul &soh &nul &nul &nul &stx &nul &nul &nul &nul &nul &nul &nul \\
0001200 &nul &nul &nul &nul &nul &nul &nul &nul &Z &nul &nul &nul &nul &nul &nul &nul \\
0001220 &bs &nul &nul &nul &nul &nul &nul &nul &nul &nul &nul &nul &nul &nul &nul &nul \\
0001240 &soh &nul &nul &nul &nul &nul &nul &nul &nul &nul &nul &nul &nul &nul &nul &nul \\
0001260 &9 &nul &nul &nul &soh &nul &nul &nul &0 &nul &nul &nul &nul &nul &nul &nul \\
0001300 &nul &nul &nul &nul &nul &nul &nul &nul &b &nul &nul &nul &nul &nul &nul &nul \\
0001320 &, &nul &nul &nul &nul &nul &nul &nul &nul &nul &nul &nul &nul &nul &nul &nul \\
0001340 &soh &nul &nul &nul &nul &nul &nul &nul &soh &nul &nul &nul &nul &nul &nul &nul \\
0001360 &B &nul &nul &nul &soh &nul &nul &nul &nul &nul &nul &nul &nul &nul &nul &nul \\
0001400 &nul &nul &nul &nul &nul &nul &nul &nul &so &nul &nul &nul &nul &nul &nul &nul \\
0001420 &nul &nul &nul &nul &nul &nul &nul &nul &nul &nul &nul &nul &nul &nul &nul &nul \\
0001440 &soh &nul &nul &nul &nul &nul &nul &nul &nul &nul &nul &nul &nul &nul &nul &nul \\
0001460 &W &nul &nul &nul &soh &nul &nul &nul &stx &nul &nul &nul &nul &nul &nul &nul \\
\end{tabular}}
\end{frame}

%%%%%%%%%%%%%%%%%%%%%%%%%%%%%%%%%%%%%%%%%%%%%%%%%%%%%%%%%%%%%%%%%%%%%%%%
\begin{frame}[fragile]
\frametitle{Traduction en langage machine}
{\tt \footnotesize
\setlength{\tabcolsep}{.07cm}
\begin{tabular}{ccccccccccccccccc}
0001500 &nul &nul &nul &nul &nul &nul &nul &nul &dle &nul &nul &nul &nul &nul &nul &nul \\
0001520 &8 &nul &nul &nul &nul &nul &nul &nul &nul &nul &nul &nul &nul &nul &nul &nul \\
0001540 &bs &nul &nul &nul &nul &nul &nul &nul &nul &nul &nul &nul &nul &nul &nul &nul \\
0001560 &R &nul &nul &nul &eot &nul &nul &nul &nul &nul &nul &nul &nul &nul &nul &nul \\
0001600 &nul &nul &nul &nul &nul &nul &nul &nul &@ &enq &nul &nul &nul &nul &nul &nul \\
0001620 &can &nul &nul &nul &nul &nul &nul &nul &vt &nul &nul &nul &bs &nul &nul &nul \\
0001640 &bs &nul &nul &nul &nul &nul &nul &nul &can &nul &nul &nul &nul &nul &nul &nul \\
0001660 &dc1 &nul &nul &nul &etx &nul &nul &nul &nul &nul &nul &nul &nul &nul &nul &nul \\
0001700 &nul &nul &nul &nul &nul &nul &nul &nul &H &nul &nul &nul &nul &nul &nul &nul \\
0001720 &a &nul &nul &nul &nul &nul &nul &nul &nul &nul &nul &nul &nul &nul &nul &nul \\
0001740 &soh &nul &nul &nul &nul &nul &nul &nul &nul &nul &nul &nul &nul &nul &nul &nul \\
0001760 &soh &nul &nul &nul &stx &nul &nul &nul &nul &nul &nul &nul &nul &nul &nul &nul \\
0002000 &nul &nul &nul &nul &nul &nul &nul &nul &p &eot &nul &nul &nul &nul &nul &nul \\
0002020 &bs &soh &nul &nul &nul &nul &nul &nul &ff &nul &nul &nul &ht &nul &nul &nul \\
0002040 &bs &nul &nul &nul &nul &nul &nul &nul &can &nul &nul &nul &nul &nul &nul &nul \\
0002060 &ht &nul &nul &nul &etx &nul &nul &nul &nul &nul &nul &nul &nul &nul &nul &nul \\
0002100 &nul &nul &nul &nul &nul &nul &nul &nul &x &enq &nul &nul &nul &nul &nul &nul \\
\end{tabular}}
\end{frame}

%%%%%%%%%%%%%%%%%%%%%%%%%%%%%%%%%%%%%%%%%%%%%%%%%%%%%%%%%%%%%%%%%%%%%%%%
\begin{frame}[fragile]
\frametitle{Traduction en langage machine}
{\tt \footnotesize
\setlength{\tabcolsep}{.07cm}
\begin{tabular}{ccccccccccccccccc}
0002120 &etb &nul &nul &nul &nul &nul &nul &nul &nul &nul &nul &nul &nul &nul &nul &nul \\
0002140 &soh &nul &nul &nul &nul &nul &nul &nul &nul &nul &nul &nul &nul &nul &nul &nul \\
0002160 &nul &nul &nul &nul &nul &nul &nul &nul &nul &nul &nul &nul &nul &nul &nul &nul \\
0002200 &nul &nul &nul &nul &nul &nul &nul &nul &soh &nul &nul &nul &eot &nul &q &del \\
0002220 &nul &nul &nul &nul &nul &nul &nul &nul &nul &nul &nul &nul &nul &nul &nul &nul \\
0002240 &nul &nul &nul &nul &etx &nul &soh &nul &nul &nul &nul &nul &nul &nul &nul &nul \\
0002260 &nul &nul &nul &nul &nul &nul &nul &nul &nul &nul &nul &nul &etx &nul &etx &nul \\
0002300 &nul &nul &nul &nul &nul &nul &nul &nul &nul &nul &nul &nul &nul &nul &nul &nul \\
0002320 &nul &nul &nul &nul &etx &nul &eot &nul &nul &nul &nul &nul &nul &nul &nul &nul \\
0002340 &nul &nul &nul &nul &nul &nul &nul &nul &nul &nul &nul &nul &etx &nul &enq &nul \\
0002360 &nul &nul &nul &nul &nul &nul &nul &nul &nul &nul &nul &nul &nul &nul &nul &nul \\
0002400 &nul &nul &nul &nul &etx &nul &bel &nul &nul &nul &nul &nul &nul &nul &nul &nul \\
0002420 &nul &nul &nul &nul &nul &nul &nul &nul &nul &nul &nul &nul &etx &nul &bs &nul \\
0002440 &nul &nul &nul &nul &nul &nul &nul &nul &nul &nul &nul &nul &nul &nul &nul &nul \\
0002460 &nul &nul &nul &nul &etx &nul &ack &nul &nul &nul &nul &nul &nul &nul &nul &nul \\
0002500 &nul &nul &nul &nul &nul &nul &nul &nul &vt &nul &nul &nul &dc2 &nul &soh &nul \\
0002520 &nul &nul &nul &nul &nul &nul &nul &nul &sub &nul &nul &nul &nul &nul &nul &nul \\
\end{tabular}}
\end{frame}

%%%%%%%%%%%%%%%%%%%%%%%%%%%%%%%%%%%%%%%%%%%%%%%%%%%%%%%%%%%%%%%%%%%%%%%%
\begin{frame}[fragile]
\frametitle{Traduction en langage machine}
{\tt \footnotesize
\setlength{\tabcolsep}{.07cm}
\begin{tabular}{ccccccccccccccccc}
0002540 &dle &nul &nul &nul &dle &nul &nul &nul &nul &nul &nul &nul &nul &nul &nul &nul \\
0002560 &nul &nul &nul &nul &nul &nul &nul &nul &nul &F &i &c &h &i &e &r \\
0002600 &. &c &nul &m &a &i &n &nul &p &r &i &n &t &f &nul &nul \\
0002620 &enq &nul &nul &nul &nul &nul &nul &nul &nl &nul &nul &nul &enq &nul &nul &nul \\
0002640 &nul &nul &nul &nul &nul &nul &nul &nul &si &nul &nul &nul &nul &nul &nul &nul \\
0002660 &stx &nul &nul &nul &nl &nul &nul &nul &| &del &del &del &del &del &del &del \\
0002700 &sp &nul &nul &nul &nul &nul &nul &nul &stx &nul &nul &nul &stx &nul &nul &nul \\
0002720 &nul &nul &nul &nul &nul &nul &nul &nul \\
0002730 \\
\end{tabular}}
\end{frame}

%%%%%%%%%%%%%%%%%%%%%%%%%%%%%%%%%%%%%%%%%%%%%%%%%%%%%%%%%%%%%%%%%%%%%%%%
\begin{frame}[fragile]
\frametitle{Édition des liens}
L'\alert{édition des liens} réunit le fichier objet et le code propre
aux fonctions et types de la librairie standard utilisés (comme
\Code{printf}, \Code{scanf}, {\em etc.}) pour produire l'exécutable
complet.
\bigskip

C'est dans cette phase de la compilation que la
\alert{résolution des symboles} a lieu. C'est l'étape qui consiste à
associer aux identificateurs de fonctions leur implantation.
\end{frame}

%%%%%%%%%%%%%%%%%%%%%%%%%%%%%%%%%%%%%%%%%%%%%%%%%%%%%%%%%%%%%%%%%%%%%%%%
\begin{frame}[fragile]
\frametitle{Compilation d'un projet de plusieurs fichiers}
On suppose que l'on travaille sur un projet constitué de trois modules
\Code{A}, \Code{B} et \Code{C} et d'un fichier principal \Code{Main.c}
contenant la fonction \Code{main}.
\medskip

La compilation de ce projet se réalise au moyen des étapes suivantes~:
\begin{enumerate}
    \item obtenir les fichiers objets de chaque module~;
    \item obtenir le fichier objet de \Code{Main.c}~:
    \item lier les fichiers objets ainsi obtenus en un exécutable.
\end{enumerate}
\medskip

\begin{center}
    \begin{tikzpicture}[every text node part/.style={align=left}]
        \node[Module](A)at(0,0){A.h \\ A.c};
        \node[Module](B)at(0,-1){B.h \\ B.c};
        \node[Module](C)at(0,-2){C.h \\ C.c};
        \node[Module](Main)at(0,-3){Main.c};
        \node[FObj](Ao)at(3.5,0){A.o};
        \node[FObj](Bo)at(3.5,-1){B.o};
        \node[FObj](Co)at(3.5,-2){C.o};
        \node[FObj](Maino)at(3.5,-3){Main.o};
        \node[Exec](Ex)at(9,-1.5){a.out};
        \draw(A)edge[->,anchor=south,font=\scriptsize \tt] node{gcc -c A.c A.h}(Ao);
        \draw(B)edge[->,anchor=south,font=\scriptsize \tt] node{gcc -c B.c B.h}(Bo);
        \draw(C)edge[->,anchor=south,font=\scriptsize \tt] node{gcc -c C.c C.h}(Co);
        \draw(Main)edge[->,anchor=south,font=\scriptsize \tt] node{gcc -c Main.c}(Maino);
        \draw(4,-1.5)edge[->,anchor=south,font=\scriptsize \tt] node{gcc Main.o A.o B.o C.o}(Ex);
    \end{tikzpicture}
\end{center}
\end{frame}

%%%%%%%%%%%%%%%%%%%%%%%%%%%%%%%%%%%%%%%%%%%%%%%%%%%%%%%%%%%%%%%%%%%%%%%%
\begin{frame}[fragile]
\frametitle{Création des fichiers objets}
Pour compiler le projet, on commence par créer un fichier objet pour
chaque module \Code{M}. On utilise pour cela la commande
\begin{center} \Code{gcc -c M.c M.h} \end{center}
\medskip

Cette commande est équivalente à
\begin{center} \Code{gcc -c M.c} \end{center}
car le fichier source \Code{M.c} inclut le fichier d'en-tête \Code{M.h}.
\bigskip

On utilisera donc de préférence cette 2\ieme{} commande.
\bigskip
\bigskip

Chaque module est ainsi \alert{compilé séparément} et dans un
\alert{ordre quelconque}.
\end{frame}

%%%%%%%%%%%%%%%%%%%%%%%%%%%%%%%%%%%%%%%%%%%%%%%%%%%%%%%%%%%%%%%%%%%%%%%%
\begin{frame}[fragile]
\frametitle{Symboles non résolus}
Pour compiler un module \Code{A}, il n'est pas nécessaire que \Code{A}
ait connaissance des définitions des symboles qu'il utilise.
\medskip

Seules {\bf leurs déclarations sont suffisantes}. Celles-ci se trouvent
dans les fichiers d'en-tête inclus dans \Code{A}.
\bigskip

On dit qu'un \alert{symbole n'est pas résolu} à un stade donné de la
compilation si sa définition n'est pas encore connue.
\bigskip

\begin{minipage}[c]{.3\textwidth}
\begin{lstlisting}[frame=single,numbers=none]
/* Fichier.c */
int g(int x);

int f(int x) {
    return g(x);
}
\end{lstlisting}
\end{minipage}\qquad
\begin{minipage}[c]{.6\textwidth}
Par exemple, ce fichier permet de produire un fichier objet sur la commande
\Code{gcc -c Fichier.c} même si le symbole \Code{g} est non résolu pour
le moment.
\smallskip

Sa déclaration (dans le fichier lui-même ou dans un fichier inclus) est
cependant nécessaire.
\end{minipage}
\end{frame}

%%%%%%%%%%%%%%%%%%%%%%%%%%%%%%%%%%%%%%%%%%%%%%%%%%%%%%%%%%%%%%%%%%%%%%%%
\begin{frame}[fragile]
\frametitle{Résolution des symboles}
Lors de l'édition des liens, un exécutable est créé. On utilise pour
cela la commande
\begin{center} \Code{gcc Main.o A1.o  ... An.o} \end{center}
dans le cadre d'un projet constitué des modules \Code{A1.o}, \dots, \Code{An.o}
et du module principal \Code{Main.o}.
\bigskip

Cette étape \alert{lie à chaque symbole sa définition}.
\bigskip
\bigskip

Tous les symboles utilisés dans le projet doivent être résolus (sinon,
un message d'erreur est produit et l'exécutable ne peut pas être
construit).
\end{frame}

%%%%%%%%%%%%%%%%%%%%%%%%%%%%%%%%%%%%%%%%%%%%%%%%%%%%%%%%%%%%%%%%%%%%%%%%
\begin{frame}[fragile]
\frametitle{Résolution des symboles --- exemple}
\begin{minipage}[c]{.18\textwidth}
\begin{lstlisting}[frame=single,numbers=none,basicstyle=\scriptsize\tt]
/* A.h */
#ifndef __A__
#define __A__
  int f(int x);
#endif
\end{lstlisting}
\end{minipage}
\enspace
\begin{minipage}[c]{.18\textwidth}
\begin{lstlisting}[frame=single,numbers=none,basicstyle=\scriptsize\tt]
/* A.c */
#include "A.h"
int f(int x) {
  return x * x;
}
\end{lstlisting}
\end{minipage}
\enspace
\begin{minipage}[c]{.18\textwidth}
\begin{lstlisting}[frame=single,numbers=none,basicstyle=\scriptsize\tt]
/* B.h */
#ifndef __B__
#define __B__
  int g(int x);
#endif
\end{lstlisting}
\end{minipage}
\enspace
\begin{minipage}[c]{.18\textwidth}
\begin{lstlisting}[frame=single,numbers=none,basicstyle=\scriptsize\tt]
/* B.c */
#include "B.h"
#include "A.h"
int g(int x) {
  return f(x);
}
\end{lstlisting}
\end{minipage}
\enspace
\begin{minipage}[c]{.17\textwidth}
\begin{lstlisting}[frame=single,numbers=none,basicstyle=\scriptsize\tt]
/* Main.c */
#include "B.h"
int main() {
  g(5);
  return 0;
}
\end{lstlisting}
\end{minipage}

Pour compiler ce projet, on emploie les commandes
\medskip

\Code{gcc -c A.c} \\
\Code{gcc -c B.c} \\
\Code{gcc -c Main.c} \\
\Code{gcc Main.o A.o B.o}
\bigskip

L'ordre d'exécution des trois 1\ieres{} commandes n'a aucune incidence
sur le résultat produit.
\end{frame}

%%%%%%%%%%%%%%%%%%%%%%%%%%%%%%%%%%%%%%%%%%%%%%%%%%%%%%%%%%%%%%%%%%%%%%%%
\begin{frame}[fragile]
\frametitle{Résolution des symboles --- exemple}
\begin{minipage}[c]{.18\textwidth}
\begin{lstlisting}[frame=single,numbers=none,basicstyle=\scriptsize\tt]
/* A.h */
#ifndef __A__
#define __A__
  int f(int x);
#endif
\end{lstlisting}
\end{minipage}
\enspace
\begin{minipage}[c]{.18\textwidth}
\begin{lstlisting}[frame=single,numbers=none,basicstyle=\scriptsize\tt]
/* A.c */
#include "A.h"
int f(int x) {
  return x * x;
}
\end{lstlisting}
\end{minipage}
\enspace
\begin{minipage}[c]{.18\textwidth}
\begin{lstlisting}[frame=single,numbers=none,basicstyle=\scriptsize\tt]
/* B.h */
#ifndef __B__
#define __B__
  int g(int x);
#endif
\end{lstlisting}
\end{minipage}
\enspace
\begin{minipage}[c]{.18\textwidth}
\begin{lstlisting}[frame=single,numbers=none,basicstyle=\scriptsize\tt]
/* B.c */
#include "B.h"
#include "A.h"
int g(int x) {
  return f(x);
}
\end{lstlisting}
\end{minipage}
\enspace
\begin{minipage}[c]{.17\textwidth}
\begin{lstlisting}[frame=single,numbers=none,basicstyle=\scriptsize\tt]
/* Main.c */
#include "B.h"
int main() {
  g(5);
  return 0;
}
\end{lstlisting}
\end{minipage}

Lors de la création de \Code{B.o}, le compilateur {\bf ignore} ce que fait
le symbole \Code{f}. Il sait seulement (grâce à l'inclusion de \Code{A}
dans \Code{B}) que \Code{f} est un symbole de fonction paramétrée par un
entier et renvoyant un entier et peut donc {\bf vérifier la correspondance
des types}.
\medskip

C'est au moment de l'édition des liens que le compilateur va
{\bf chercher l'implantation} du symbole \Code{f} pour créer l'exécutable
de la bonne manière.
\bigskip

{\bf Observation importante}~: la compilation d'un projet à plusieurs
fichiers ne dépend pas de la manière dont ses modules sont inclus les
uns dans les autres. Le schéma de compilation est toujours le même.
\end{frame}

%%%%%%%%%%%%%%%%%%%%%%%%%%%%%%%%%%%%%%%%%%%%%%%%%%%%%%%%%%%%%%%%%%%%%%%%
%%%%%%%%%%%%%%%%%%%%%%%%%%%%%%%%%%%%%%%%%%%%%%%%%%%%%%%%%%%%%%%%%%%%%%%%
\subsection{Compilation séparée}

%%%%%%%%%%%%%%%%%%%%%%%%%%%%%%%%%%%%%%%%%%%%%%%%%%%%%%%%%%%%%%%%%%%%%%%%
\begin{frame}\frametitle{Compilation séparée --- intuition}
{\bf Fait}~: pour compiler un module \Code{A}, il n'est pas nécessaire
d'avoir les fichiers objets des autres modules du projet dont
\Code{A} ne dépend pas (de manière étendue ou non).
\bigskip

\uncover<2->{%
{\bf Conséquence}~: si un module \Code{B} est modifié, il n'est nécessaire
de recompiler que \Code{B} et l'ensemble des modules qui dépendent
(de manière étendue) à \Code{B}.
\bigskip
\bigskip}

\uncover<3->{%
Attention à ne pas oublier de recompiler le module principal \Code{Main}
si celui-ci dépend de manière étendue à des modules modifiés.}
\end{frame}

%%%%%%%%%%%%%%%%%%%%%%%%%%%%%%%%%%%%%%%%%%%%%%%%%%%%%%%%%%%%%%%%%%%%%%%%
\begin{frame}
\frametitle{Compilation séparée --- exemple introductif}
Considérons par exemple le projet suivant~:
\begin{multicols}{2}
\begin{center}
\scalebox{.8}{\begin{tikzpicture}
    \node[Sommet](A)at(0,0){\Code{A}};
    \node[Sommet](B)at(0,-1){\Code{B}};
    \node[Sommet](C)at(0,-2){\Code{C}};
    \node[Sommet](Main)at(2,-1){\Code{Main}};
    \draw[Fleche](Main)edge[bend right=20,dashed,draw=Vert]node{}(A);
    \draw[Fleche](Main)edge[bend left=0,dashed,draw=Vert](B);
    \draw[Fleche](Main)edge[bend left=20,dashed,draw=Vert]node{}(C);
    \draw[Fleche](B)--(A);
    \draw[Fleche](A)edge[bend right=60]node{}(C);
\end{tikzpicture}}
\end{center}
On le compile pour la 1\iere{} fois par \\
\begin{footnotesize}
    \Code{gcc -c Main.c} \\
    \Code{gcc -c A.c} \\
    \Code{gcc -c B.c} \\
    \Code{gcc -c C.c} \\
    \Code{gcc Main.o A.o B.o C.o}
\end{footnotesize}
\end{multicols}
\medskip

\uncover<2->{%
Si on modifie par la suite le module \Code{A}, il suffit d'exécuter les
commandes
\begin{multicols}{2}
    \Code{gcc -c A.c} \\
    \Code{gcc -c B.c} \\
    \Code{gcc -c Main.c} \\
    \Code{gcc Main.o A.o B.o C.o} \\
    pour mettre à jour l'exécutable du projet.
    \smallskip

    {\bf Note}~: les $3$ premières lignes commutent.
\end{multicols}
\medskip

Il est inutile de recompiler \Code{C} car il ne dépend pas (de manière
étendue) à~\Code{A}.}
\end{frame}

%%%%%%%%%%%%%%%%%%%%%%%%%%%%%%%%%%%%%%%%%%%%%%%%%%%%%%%%%%%%%%%%%%%%%%%%
\begin{frame}
\frametitle{Schéma opérationnel de la compilation d'un projet}
À l'appui de cette observation, la compilation d'un projet
s'organise de la manière suivante.
\medskip

\begin{enumerate}[(1)]
    \uncover<2->{%
    \item Pour chaque module \Code{A} du projet~:
    \smallskip}

    \begin{enumerate}[(a)] \normalsize
        \uncover<3->{%
        \item compiler \Code{A} si au moins l'une des conditions
        suivante est vérifiée~:
        \smallskip}

        \begin{itemize} \normalsize
            \uncover<4->{%
            \item \Code{A.o} n'existe pas~;
            \smallskip}

            \uncover<5->{%
            \item \Code{A.c} ou \Code{A.h} ont été modifiés {\bf après}
            \Code{A.o}~;
            \smallskip}

            \uncover<6->{%
            \item il existe un module \Code{B} dont \Code{A} dépend
            (de manière étendue) et tel que \Code{B.h} a été modifié
            {\bf après} \Code{A.o}~;}
        \end{itemize}
    \end{enumerate}
    \smallskip

    \uncover<7->{%
    \item si au moins un module du projet a été (re)compilé, reconstruire
    l'exécutable.}
\end{enumerate}
\bigskip

\uncover<8->{%
Nous allons utiliser l'utilitaire \alert{\Code{make}} et les fichiers
\alert{\Code{Makefile}} pour rendre cette procédure automatique.}
\end{frame}

%%%%%%%%%%%%%%%%%%%%%%%%%%%%%%%%%%%%%%%%%%%%%%%%%%%%%%%%%%%%%%%%%%%%%%%%
%%%%%%%%%%%%%%%%%%%%%%%%%%%%%%%%%%%%%%%%%%%%%%%%%%%%%%%%%%%%%%%%%%%%%%%%
\subsection{{\tt Makefile} simples}

%%%%%%%%%%%%%%%%%%%%%%%%%%%%%%%%%%%%%%%%%%%%%%%%%%%%%%%%%%%%%%%%%%%%%%%%
\begin{frame}\frametitle{Fichiers {\tt Makefile}}
L'utilitaire \Code{make} est paramétré par un fichier dont le nom est
imposé~:
\begin{center}
    \og \Code{Makefile} \fg{} ou \og \Code{makefile} \fg.
\end{center}
Il doit se trouver dans le répertoire de travail (là où se trouvent les
autres fichiers du projet ou au niveau des répertoires \Code{include}
et \Code{src}).
\medskip

Ce fichier fait partie intégrante du projet.
\bigskip

Tout fichier \Code{Makefile} est constitué de \alert{règles}.
Elles sont de la forme
\smallskip

\Code{CIBLE: DEPENDANCES} \\
\Code{$\rightarrow$ COMMANDE} \\
\Code{\vdots} \\
\Code{$\rightarrow$ COMMANDE}
\bigskip

Le symbole \og\Code{$\rightarrow$}\fg{} désigne une tabulation.
\end{frame}

%%%%%%%%%%%%%%%%%%%%%%%%%%%%%%%%%%%%%%%%%%%%%%%%%%%%%%%%%%%%%%%%%%%%%%%%
\begin{frame}[fragile]
\frametitle{Fichiers {\tt Makefile} et fonctionnement}
Considérons le projet et le \Code{Makefile} suivants.
\begin{multicols}{2}
\begin{center}
\scalebox{.8}{\begin{tikzpicture}
    \node[Sommet](A)at(0,0){\Code{A}};
    \node[Sommet](B)at(0,-2){\Code{B}};
    \node[Sommet](Main)at(2,-1){\Code{Main}};
    \draw[Fleche,dashed,draw=Vert](Main)edge[bend right=20]node{}(A);
    \draw[Fleche,dashed,draw=Vert](Main)edge[bend left=20]node{}(B);
\end{tikzpicture}}
\end{center}
\bigskip
\bigskip
\begin{lstlisting}[language=make]
Prog: A.o B.o Main.o
    gcc -o Prog Main.o A.o B.o
Main.o: Main.c A.h B.h
    gcc -c Main.c
A.o: A.c A.h
    gcc -c A.c
B.o: B.c B.h
    gcc -c B.c
\end{lstlisting}
\end{multicols}
\bigskip

Lorsque l'on exécute la commande \Code{make}, \Code{make} tente de
résoudre la 1\iere{} règle (l. 1). Pour cela, il doit résoudre ses
dépendances. Ensuite, il exécute les commandes de la règle si la cible
n'est pas à jour.
\medskip

Concrètement, pour pouvoir créer l'exécutable \Code{Prog}, il est
nécessaire que \Code{A.o}, \Code{B.o} et \Code{Main.o} soient à jour.
Une fois qu'ils le sont, il suffit de procéder à l'édition des liens
(l. 2).
\end{frame}

%%%%%%%%%%%%%%%%%%%%%%%%%%%%%%%%%%%%%%%%%%%%%%%%%%%%%%%%%%%%%%%%%%%%%%%%
\begin{frame}\frametitle{Fichiers {\tt Makefile} et fonctionnement}
Pour savoir si une cible est à jour, \Code{make} regarde les dépendances
de la règle et~:
\begin{itemize}
    \item si la dépendance est la cible d'une autre règle, \Code{make}
    procède {\bf récursivement} à sa résolution~;
    \smallskip

    \item si la dépendance est le nom d'un fichier, \Code{make} compare
    la {\bf date de dernière modification} de la cible par rapport à
    celle du fichier.
\end{itemize}
\smallskip

S'il y a au moins un fichier dans les dépendances avec une date supérieure
à celle de la cible, les commandes de la règle sont exécutées.
\bigskip

On peut imposer à \Code{make} de commencer par la résolution de la
règle dont la cible est \Code{CIBLE} par
\begin{center}
    \Code{make CIBLE}
\end{center}
\end{frame}

%%%%%%%%%%%%%%%%%%%%%%%%%%%%%%%%%%%%%%%%%%%%%%%%%%%%%%%%%%%%%%%%%%%%%%%%
\begin{frame}[fragile]\frametitle{Déclarations de types et dépendances}
Soient \Code{A} et \Code{B} deux modules tels que
\Code{B} dépend (de manière étendue) à \Code{A}, qu'un type \Code{T}
soit déclaré dans \Code{A} et que \Code{B} utilise ce type.
\bigskip

Toute modification de \Code{A.h} doit être suivie d'une nouvelle
compilation de \Code{B}. En effet, si la déclaration de \Code{T} a été
modifiée, \Code{B} doit être recompilé pour la prendre en compte.
\medskip

En conséquence, dans le \Code{Makefile} du projet doit figurer la règle
\begin{multicols}{2}
\begin{lstlisting}[language=make]
B.o: B.c B.h A.h
    gcc -c B.c
\end{lstlisting}
pour déclarer que la construction de \Code{B.o} {\bf dépend aussi}
de \Code{A.h}.
\end{multicols}
\bigskip

{\bf Attention}~: ceci ne s'applique pas aux modifications de l'implantation
des fonctions de \Code{A} dans \Code{A.c} (comme nous l'avons déjà vu).
La cible \Code{B.o} ne dépend ainsi pas de \Code{A.c}. Elle ne dépend
que des {\bf déclarations} du module \Code{A} (et donc de \Code{A.h}).
\end{frame}

%%%%%%%%%%%%%%%%%%%%%%%%%%%%%%%%%%%%%%%%%%%%%%%%%%%%%%%%%%%%%%%%%%%%%%%%
\begin{frame}[fragile]
\frametitle{Exemple complet de {\tt Makefile}}
\begin{multicols}{2}
\begin{center}
\scalebox{.4}{\begin{tikzpicture}[scale=.8]
    \node[Sommet,minimum size=2cm,font=\Large](Piece)at(-4,0)
        {\small \Code{Piece}};
    \node[Sommet,minimum size=2cm,font=\Large](Case)at(-4,-4)
        {\small \Code{Case}};
    \node[Sommet,minimum size=2cm,font=\Large](Position)at(0,-2)
        {\small \Code{Position}};
    \node[Sommet,minimum size=2cm,font=\Large](IA)at(4,0)
        {\small \Code{IA}};
    \node[Sommet,minimum size=2cm,font=\Large](IGraph)at(4,-4)
        {\small \Code{IGraph}};
    \node[Sommet,minimum size=2cm,draw=Vert,fill=Vert!20,font=\Large]
        (Main)at(7,-2){\small \Code{Main}};
    \draw[Fleche,dashed,draw=Vert](Main)edge[bend right=20]node{}(IA);
    \draw[Fleche,dashed,draw=Vert](Main)edge[bend left=20]node{}(IGraph);
    \draw[Fleche](IGraph)edge[bend left=20]node{}(Case);
    \draw[Fleche](IGraph)edge[bend left=20]node{}(Position);
    \draw[Fleche](IGraph)edge[bend right=40]node{}(Piece);
    \draw[Fleche](IA)edge[bend left=40]node{}(Case);
    \draw[Fleche](IA)edge[bend right=20]node{}(Position);
    \draw[Fleche](IA)edge[bend right=20]node{}(Piece);
    \draw[Fleche](Position)edge[bend right=20]node{}(Case);
    \draw[Fleche](Position)edge[bend left=20]node{}(Piece);
\end{tikzpicture}}
\end{center}
Le \Code{Makefile} du projet dont le graphe d'inclusions (étendues) est
donné ci-contre est
\end{multicols}

\begin{lstlisting}[language=make,basicstyle=\footnotesize\tt]
Echecs: Main.o Piece.o Case.o Position.o IA.o IGraph.o
    gcc -o Echecs Main.o Piece.o Case.o Position.o IA.o IGraph.o
Main.o: Main.c IA.h IGraph.h
    gcc -c Main.c
Piece.o: Piece.c Piece.h
    gcc -c Piece.c
Case.o: Case.c Case.h
    gcc -c Case.c
Position.o: Position.c Position.h Piece.h Case.h
    gcc -c Position.c
IA.o: IA.c IA.h Piece.h Case.h Position.h
    gcc -c IA.c
IGraph.o: IGraph.c IGraph.h Piece.h Case.h Position.h
    gcc -c IGraph.c
\end{lstlisting}
\end{frame}

%%%%%%%%%%%%%%%%%%%%%%%%%%%%%%%%%%%%%%%%%%%%%%%%%%%%%%%%%%%%%%%%%%%%%%%%
\begin{frame}[fragile]
\frametitle{Écriture de {\tt Makefile} simples --- résumé}
Le \Code{Makefile} d'un projet contenant des modules \Code{A1}, \dots,
\Code{An} et un module principal \Code{Main} est génériquement de la
forme
\begin{lstlisting}[language=make]
NOM: Main.o A1.o ... An.o
    gcc -o NOM Main.o A1.o ... An.o

Main.o: Main.c EXTRAmain
    gcc -c Main.c

A1.o: A1.c A1.h EXTRA1
    gcc -c A1.c

...

An.o: An.c An.h EXTRAn
    gcc -c An.c
\end{lstlisting}
où \Code{EXTRAmain} est la suite des noms des fichiers \Code{.h} que
\Code{Main.c} inclut et pour tout \Code{$1 \leq k \leq n$}, \Code{EXTRAk}
est la suite des noms des fichiers \Code{.h} dont le module \Code{Ak}
dépend (de manière étendue).
\end{frame}

%%%%%%%%%%%%%%%%%%%%%%%%%%%%%%%%%%%%%%%%%%%%%%%%%%%%%%%%%%%%%%%%%%%%%%%%
%%%%%%%%%%%%%%%%%%%%%%%%%%%%%%%%%%%%%%%%%%%%%%%%%%%%%%%%%%%%%%%%%%%%%%%%
\subsection{{\tt Makefile} avancés}

%%%%%%%%%%%%%%%%%%%%%%%%%%%%%%%%%%%%%%%%%%%%%%%%%%%%%%%%%%%%%%%%%%%%%%%%
\begin{frame}[fragile]
\frametitle{Variables dans les {\tt Makefile}}
Il est possible de \alert{définir des variables} dans un \Code{Makefile}
par
\begin{center}\Code{ID=VAL}\end{center}
Ceci définit une variable identifiée par \Code{ID}. Sa valeur est
la {\bf chaîne de caractères} \Code{VAL}.
\bigskip

On accède à la valeur d'une variable identifiée par \Code{ID} par
\begin{center}\Code{\$(ID)}\end{center}
Ceci substitue à l'occurrence de \Code{\$(ID)} la chaîne de caractères qui
lui a été attribuée lors de sa définition.
\end{frame}

%%%%%%%%%%%%%%%%%%%%%%%%%%%%%%%%%%%%%%%%%%%%%%%%%%%%%%%%%%%%%%%%%%%%%%%%
\begin{frame}[fragile]
\frametitle{Variables dans les {\tt Makefile} --- exemple}

Les variables permettent de factoriser les règles d'un \Code{Makefile}~:
\medskip

\begin{lstlisting}[language=make,basicstyle=\footnotesize\tt]
Main: Main.o A.o
    gcc -o Main Main.o A.o -ansi -pedantic -Wall

Main.o: Main.c
    gcc -c Main.c -ansi -pedantic -Wall

A.o : A.c A.h
    gcc -c A.c -ansi -pedantic -Wall
\end{lstlisting}
\medskip

s'écrit plus simplement par
\medskip

\begin{lstlisting}[language=make,basicstyle=\footnotesize\tt]
CFLAGS=-ansi -pedantic -Wall

Main: Main.o A.o
    gcc -o Main Main.o A.o $(CFLAGS)

Main.o: Main.c
    gcc -c Main.c $(CFLAGS)

A.o : A.c A.h
    gcc -c A.c $(CFLAGS)
\end{lstlisting}
\begin{math}\end{math}
\end{frame}

%%%%%%%%%%%%%%%%%%%%%%%%%%%%%%%%%%%%%%%%%%%%%%%%%%%%%%%%%%%%%%%%%%%%%%%%
\begin{frame}[fragile]
\frametitle{Variables dans les {\tt Makefile}}
On utilise en général les noms de variable suivants~:
\begin{itemize}
    \item \Code{CFLAGS} pour les options de compilation, p.ex.,
\begin{lstlisting}[language=make,numbers=none,basicstyle=\footnotesize\tt]
CFLAGS=-ansi -pedantic -Wall
\end{lstlisting}

    \item \Code{LDFLAGS} pour l'inclusion de bibliothèques, p.ex.,
\begin{lstlisting}[language=make,numbers=none,basicstyle=\footnotesize\tt]
LDFLAGS=-lm -lMLV
\end{lstlisting}

    \item \Code{CC} pour le compilateur utilisé, p.ex.,

\begin{minipage}[c]{.16\textwidth}
\begin{lstlisting}[language=make,numbers=none,basicstyle=\footnotesize\tt]
CC=gcc
\end{lstlisting}
\end{minipage}
ou bien \hspace{2.5em}
\begin{minipage}[c]{.16\textwidth}
\begin{lstlisting}[language=make,numbers=none,basicstyle=\footnotesize\tt]
CC=colorgcc
\end{lstlisting}
\end{minipage}

    \item \Code{OPT} pour les option d'optimisation de code

\begin{minipage}[c]{.16\textwidth}
\begin{lstlisting}[language=make,numbers=none,basicstyle=\footnotesize\tt]
OPT=-O1
\end{lstlisting}
\end{minipage}
ou bien \hspace{2.5em}
\begin{minipage}[c]{.16\textwidth}
\begin{lstlisting}[language=make,numbers=none,basicstyle=\footnotesize\tt]
OPT=-O2
\end{lstlisting}
\end{minipage}
ou encore \hspace{2.5em}
\begin{minipage}[c]{.16\textwidth}
\begin{lstlisting}[language=make,numbers=none,basicstyle=\footnotesize\tt]
OPT=-O3
\end{lstlisting}
\end{minipage}
\end{itemize}
\end{frame}

%%%%%%%%%%%%%%%%%%%%%%%%%%%%%%%%%%%%%%%%%%%%%%%%%%%%%%%%%%%%%%%%%%%%%%%%
\begin{frame}[fragile]
\frametitle{Règles courantes}
{\bf Observation}~: la plupart des règles des \Code{Makefile} sont sous
l'une de ces deux formes~:
\medskip

\begin{enumerate}
\item
\Code{M.o: M.c DEP2... DEPn \\
$\rightarrow$ gcc -c M.c}
\medskip

\item
\Code{EXEC: DEP1 ... DEPn \\
$\rightarrow$ gcc -o EXEC DEP1 ... DEPn}
\end{enumerate}
\bigskip

La 1\iere{} forme de règle a pour but de construire le fichier objet
d'un module \Code{M}. Dans ce cas,
\Code{DEP2}, \dots, \Code{DEPn} sont les \Code{.h} dont le module
\Code{M} dépend.
\bigskip

La 2\ieme{} forme de règle a pour but de construire l'exécutable \Code{EXEC}
du projet. Les dépendances \Code{DEP1}, \dots, \Code{DEPn} sont dans ce
cas les fichiers \Code{.o} du projet.
\end{frame}

%%%%%%%%%%%%%%%%%%%%%%%%%%%%%%%%%%%%%%%%%%%%%%%%%%%%%%%%%%%%%%%%%%%%%%%%
\begin{frame}[fragile]
\frametitle{Variables internes dans les {\tt Makefile}}
Il est possible de simplifier l'écriture de ces règles courantes au moyen
des \alert{variables internes}. Il y en a trois principales et deux
secondaires~:
\begin{center}
    \begin{tabular}{c|c}
        Symbole & Ce qu'il désigne \\ \hline \hline
        \Code{\$@} & Nom de la cible \\
        \Code{\$<} & Nom de la 1\iere{} dép. \\
        \Code{\$\textasciicircum} & Noms de toutes les dép. \\ \hline
        \Code{\$?} & Noms des dép. plus récentes que la cible \\
        \Code{\$*} & Nom de la cible sans extension
    \end{tabular}
\end{center}
\medskip

En les utilisant, les deux règles précédentes deviennent

\begin{minipage}[c]{.4\textwidth}
\begin{lstlisting}[language=make,numbers=none,basicstyle=\footnotesize\tt]
M.o: M.c DEP2 ... DEPn
    gcc -c M.c
\end{lstlisting}
\end{minipage}
\hspace{1em}
$\longrightarrow$
\hspace{2em}
\begin{minipage}[c]{.35\textwidth}
\begin{lstlisting}[language=make,numbers=none,basicstyle=\footnotesize\tt]
M.o: M.c DEP2 ... DEPn
    gcc -c $<
\end{lstlisting}
\begin{math}\end{math}
\end{minipage}

\begin{minipage}[c]{.4\textwidth}
\begin{lstlisting}[language=make,numbers=none,basicstyle=\footnotesize\tt]
EXEC: DEP1 ... DEPn
    gcc -o EXEC DEP1 ... DEPn
\end{lstlisting}
\end{minipage}
\hspace{1em}
$\longrightarrow$
\hspace{2em}
\begin{minipage}[c]{.35\textwidth}
\begin{lstlisting}[language=make,numbers=none,basicstyle=\footnotesize\tt]
EXEC: DEP1 ... DEPn
    gcc -o $@ $^
\end{lstlisting}
\begin{math}\end{math}
\end{minipage}
\end{frame}

%%%%%%%%%%%%%%%%%%%%%%%%%%%%%%%%%%%%%%%%%%%%%%%%%%%%%%%%%%%%%%%%%%%%%%%%
\begin{frame}[fragile]
\frametitle{Variables internes dans les {\tt Makefile} --- exemple}
\begin{multicols}{2}
\begin{center}
\scalebox{.35}{\begin{tikzpicture}[scale=.8]
    \node[Sommet,minimum size=2cm,font=\Large](Piece)at(-4,0)
        {\small \Code{Piece}};
    \node[Sommet,minimum size=2cm,font=\Large](Case)at(-4,-4)
        {\small \Code{Case}};
    \node[Sommet,minimum size=2cm,font=\Large](Position)at(0,-2)
        {\small \Code{Position}};
    \node[Sommet,minimum size=2cm,font=\Large](IA)at(4,0)
        {\small \Code{IA}};
    \node[Sommet,minimum size=2cm,font=\Large](IGraph)at(4,-4)
        {\small \Code{IGraph}};
    \node[Sommet,minimum size=2cm,draw=Vert,fill=Vert!20,font=\Large]
        (Main)at(7,-2){\small \Code{Main}};
    \draw[Fleche,dashed,draw=Vert](Main)edge[bend right=20]node{}(IA);
    \draw[Fleche,dashed,draw=Vert](Main)edge[bend left=20]node{}(IGraph);
    \draw[Fleche](IGraph)edge[bend left=20]node{}(Case);
    \draw[Fleche](IGraph)edge[bend left=20]node{}(Position);
    \draw[Fleche](IGraph)edge[bend right=40]node{}(Piece);
    \draw[Fleche](IA)edge[bend left=40]node{}(Case);
    \draw[Fleche](IA)edge[bend right=20]node{}(Position);
    \draw[Fleche](IA)edge[bend right=20]node{}(Piece);
    \draw[Fleche](Position)edge[bend right=20]node{}(Case);
    \draw[Fleche](Position)edge[bend left=20]node{}(Piece);
\end{tikzpicture}}
\end{center}
L'utilisation des variables et variables internes permet de simplifier
les \Code{Makefile}.
\end{multicols}

\begin{multicols}{2}
\begin{lstlisting}[language=make,basicstyle=\footnotesize\tt]
CC=colorgcc
CFLAGS=-ansi -pedantic -Wall
OBJ=Main.o Piece.o Case.o
    Position.o IA.o IGraph.o

Echecs: $(OBJ)
    $(CC) -o $@ $^ $(CFLAGS)


Main.o: Main.c IA.h IGraph.h
    $(CC) -c $< $(CFLAGS)

Piece.o: Piece.c Piece.h
    $(CC) -c $< $(CFLAGS)

Case.o: Case.c Case.h
    $(CC) -c $< $(CFLAGS)

Position.o: Position.c Position.h
            Piece.h Case.h
    $(CC) -c $< $(CFLAGS)

IA.o: IA.c IA.h
      Piece.h Case.h Position.h
    $(CC) -c $< $(CFLAGS)

IGraph.o: IGraph.c IGraph.h
          Piece.h Case.h Position.h
    $(CC) -c $< $(CFLAGS)
\end{lstlisting}
\begin{math}\end{math}
\end{multicols}
\end{frame}

%%%%%%%%%%%%%%%%%%%%%%%%%%%%%%%%%%%%%%%%%%%%%%%%%%%%%%%%%%%%%%%%%%%%%%%%
\begin{frame}[fragile]
\frametitle{Règles génériques}
Il est possible de simplifier encore d'avantage l'écriture des \Code{Makefile}
au moyen des \alert{règles génériques}.
\medskip

Ce sont des règles de la forme
\smallskip

\Code{\%.o: \%.c \\
$\rightarrow$ COMMANDE}
\smallskip

où \Code{COMMANDE} est une commande.
\medskip

Cette syntaxe simule une règle
\smallskip

\Code{M.o: M.c \\
$\rightarrow$ COMMANDE}
\smallskip

pour chaque fichier \Code{M.c} du projet.
\bigskip

{\bf Intérêt principal}~: l'unique règle
\begin{lstlisting}[language=make,numbers=none,basicstyle=\footnotesize\tt]
%.o: %.c
    gcc -c $<
\end{lstlisting}
\begin{math}\end{math}
permet de construire chaque fichier objet du projet.
\end{frame}

%%%%%%%%%%%%%%%%%%%%%%%%%%%%%%%%%%%%%%%%%%%%%%%%%%%%%%%%%%%%%%%%%%%%%%%%
\begin{frame}[fragile]
\frametitle{Règles génériques}
{\bf Attention}~: la règle
\begin{lstlisting}[language=make,numbers=none,basicstyle=\footnotesize\tt]
%.o: %.c
    gcc -c $<
\end{lstlisting}%
\begin{math}\end{math}%
ne prend pas en compte des dépendances des modules aux fichiers \Code{.h}
concernés.
\bigskip

Il faut les mentionner explicitement de la manière suivante~:
\smallskip

\Code{M.o: M.c M.h DEP1 ... DEPn}
\smallskip

pour chaque module \Code{M} du projet. \Code{DEP1}, \dots, \Code{DEPn}
sont les \Code{.h} dont le module \Code{M} dépend.
\end{frame}

%%%%%%%%%%%%%%%%%%%%%%%%%%%%%%%%%%%%%%%%%%%%%%%%%%%%%%%%%%%%%%%%%%%%%%%%
\begin{frame}[fragile]
\frametitle{Règles génériques --- exemple}
\begin{multicols}{2}
\begin{center}
\scalebox{.35}{\begin{tikzpicture}[scale=.8]
    \node[Sommet,minimum size=2cm,font=\Large](Piece)at(-4,0)
        {\small \Code{Piece}};
    \node[Sommet,minimum size=2cm,font=\Large](Case)at(-4,-4)
        {\small \Code{Case}};
    \node[Sommet,minimum size=2cm,font=\Large](Position)at(0,-2)
        {\small \Code{Position}};
    \node[Sommet,minimum size=2cm,font=\Large](IA)at(4,0)
        {\small \Code{IA}};
    \node[Sommet,minimum size=2cm,font=\Large](IGraph)at(4,-4)
        {\small \Code{IGraph}};
    \node[Sommet,minimum size=2cm,draw=Vert,fill=Vert!20,font=\Large]
        (Main)at(7,-2){\small \Code{Main}};
    \draw[Fleche,dashed,draw=Vert](Main)edge[bend right=20]node{}(IA);
    \draw[Fleche,dashed,draw=Vert](Main)edge[bend left=20]node{}(IGraph);
    \draw[Fleche](IGraph)edge[bend left=20]node{}(Case);
    \draw[Fleche](IGraph)edge[bend left=20]node{}(Position);
    \draw[Fleche](IGraph)edge[bend right=40]node{}(Piece);
    \draw[Fleche](IA)edge[bend left=40]node{}(Case);
    \draw[Fleche](IA)edge[bend right=20]node{}(Position);
    \draw[Fleche](IA)edge[bend right=20]node{}(Piece);
    \draw[Fleche](Position)edge[bend right=20]node{}(Case);
    \draw[Fleche](Position)edge[bend left=20]node{}(Piece);
\end{tikzpicture}}
\end{center}

L'utilisation des règles génériques permet de simplifier encore
les \Code{Makefile}.
\end{multicols}

\begin{multicols}{2}
\begin{lstlisting}[language=make,basicstyle=\footnotesize\tt]
CC=colorgcc
CFLAGS=-ansi -pedantic -Wall
OBJ=Main.o Piece.o Case.o
    Position.o IA.o IGraph.o

Echecs: $(OBJ)
    $(CC) -o $@ $^ $(CFLAGS)


Main.o: Main.c IA.h IGraph.h

Piece.o: Piece.c Piece.h

Case.o: Case.c Case.h

Position.o: Position.c Position.h
            Piece.h Case.h

IA.o: IA.c IA.h
      Piece.h Case.h Position.h

IGraph.o: IGraph.c IGraph.h
          Piece.h Case.h Position.h

%.o: %.c
    $(CC) -c $< $(CFLAGS)
\end{lstlisting}
\begin{math}\end{math}
\end{multicols}
\end{frame}

%%%%%%%%%%%%%%%%%%%%%%%%%%%%%%%%%%%%%%%%%%%%%%%%%%%%%%%%%%%%%%%%%%%%%%%%
\begin{frame}[fragile]
\frametitle{Règles de nettoyage}
{\bf Règle de nettoyage}~:
\begin{multicols}{2}
\begin{lstlisting}[language=make]
clean:
    rm -f *.o
\end{lstlisting}
Cette règle permet, lorsque l'on saisit la commande \Code{make clean},
de supprimer les fichiers \Code{.o} du répertoire courant.
\end{multicols}
\bigskip
\bigskip

{\bf Règle de nettoyage total}~:
\begin{multicols}{2}
\begin{lstlisting}[language=make]
mrproper: clean
    rm -f EXEC
\end{lstlisting}
Cette règle,
où \Code{EXEC} est le nom de l'exécutable du projet, permet de supprimer
tous les {\bf fichiers regénérables} (c.-à-d. les fichiers \Code{.o} et
l'exécutable) à partir des fichiers \Code{.c} et \Code{.h} du projet.
\end{multicols}
\end{frame}

%%%%%%%%%%%%%%%%%%%%%%%%%%%%%%%%%%%%%%%%%%%%%%%%%%%%%%%%%%%%%%%%%%%%%%%%
\begin{frame}[fragile]
\frametitle{Règles d'installation / désinstallation}

{\bf Règle d'installation}~:
\begin{multicols}{2}
\begin{lstlisting}[language=make]
install: EXEC
    mkdir ../bin
    mv EXEC ../bin/EXEC
    make mrproper
\end{lstlisting}
Cette règle permet de compiler le projet, de placer son exécutable
\Code{EXEC} dans un répertoire \Code{bin} et de supprimer les fichiers
regénérables.
\end{multicols}
\bigskip
\bigskip

{\bf Règle de désinstallation}~:
\begin{multicols}{2}
\begin{lstlisting}[language=make]
uninstall: mrproper
    rm -f ../bin/EXEC
    rm -rf ../bin
\end{lstlisting}
Cette règle permet de supprimer les fichiers regénérables,
l'exécutable du projet, ainsi que le répertoire \Code{bin} le contenant.
\end{multicols}
\end{frame}

%%%%%%%%%%%%%%%%%%%%%%%%%%%%%%%%%%%%%%%%%%%%%%%%%%%%%%%%%%%%%%%%%%%%%%%%
%%%%%%%%%%%%%%%%%%%%%%%%%%%%%%%%%%%%%%%%%%%%%%%%%%%%%%%%%%%%%%%%%%%%%%%%
\subsection{Bibliothèques}

%%%%%%%%%%%%%%%%%%%%%%%%%%%%%%%%%%%%%%%%%%%%%%%%%%%%%%%%%%%%%%%%%%%%%%%%
\begin{frame}\frametitle{Principes généraux}
Une \alert{bibliothèque} est un regroupement de modules offrant des
fonctionnalités allant vers un même objectif.
\bigskip

\uncover<2->{%
Par exemple, la bibliothèque graphique {\sf MLV} est un ensemble de
plusieurs modules (\Code{MLV\_audio}, \Code{MLV\_keyboard},
\Code{MLV\_image}, {\em etc.}) réunis dans le but d'offrir des fonctions
d'affichage graphique et de gérer les entrées/sorties
(son, clavier, souris, {\em etc.}).
\bigskip}

\uncover<3->{%
L'intérêt des bibliothèques est double~:
\begin{enumerate}
    \item il suffit de ne réaliser qu'{\bf une seule inclusion} pour
    bénéficier des fonctionnalités d'une bibliothèque, plutôt que
    plusieurs inclusions sans être sûr du fichier d'en-tête à inclure~;
    \smallskip}

    \uncover<4->{%
    \item sous réserve de savoir comment créer des bibliothèques, il est
    possible de {\bf partager et réutiliser} son propre code entre
    plusieurs de ses projets, sans avoir à le recompiler.}
\end{enumerate}
\end{frame}

%%%%%%%%%%%%%%%%%%%%%%%%%%%%%%%%%%%%%%%%%%%%%%%%%%%%%%%%%%%%%%%%%%%%%%%%
\begin{frame} \frametitle{Bibliothèques statiques}
Une \alert{bibliothèque statique} est un fichier d'extension \Code{.a}.
\bigskip

\uncover<2->{%
Lors de son utilisation, son code est inclus dans l'exécutable pendant
l'édition des liens.
\bigskip}

\begin{itemize}
    \uncover<3->{%
    \item {\bf Avantage}~: tout projet qui utilise une bibliothèque
    statique peut être exécuté sur une machine où la bibliothèque n'est
    pas installée.
    \medskip}

    \uncover<4->{%
    \item {\bf Inconvénient}~: l'exécutable est plus volumineux.}
\end{itemize}
\end{frame}

%%%%%%%%%%%%%%%%%%%%%%%%%%%%%%%%%%%%%%%%%%%%%%%%%%%%%%%%%%%%%%%%%%%%%%%%
\begin{frame} \frametitle{Bibliothèques dynamiques}
Une \alert{bibliothèque dynamique} est un fichier d'extension \Code{.so}.
\bigskip

\uncover<2->{%
Lors de son utilisation, son code n'est pas inclus dans l'exécutable.
C'est lors de l'exécution que les symboles provenant de la bibliothèque
sont résolus au moyen de l'éditeur de liens dynamique.
\bigskip}

\begin{itemize}
    \uncover<3->{%
    \item {\bf Avantages}~: l'exécutable est moins volumineux. Il n'y a
    pas de duplication du code de la bibliothèque sur un même système si
    plusieurs projets l'utilisent.
    \medskip}

    \uncover<4->{%
    \item {\bf Inconvénient}~: tout projet qui utilise une bibliothèque
    dynamique ne peut être exécuté que sur un système où cette
    dernière est installée.}
\end{itemize}
\end{frame}

%%%%%%%%%%%%%%%%%%%%%%%%%%%%%%%%%%%%%%%%%%%%%%%%%%%%%%%%%%%%%%%%%%%%%%%%
\begin{frame}
\frametitle{Compilation d'un projet utilisant des bibliothèques}
On suppose que l'on travaille sur un projet constitué de deux modules
\Code{A} et \Code{B} et d'un fichier principal \Code{Main.c}.
Ce projet utilise deux bibliothèques statiques~: \Code{libc.a} et
\Code{libMLV.a}.
\medskip

La compilation de ce projet se réalise de manière habituelle. La
différence porte sur l'étape d'{\bf édition des liens} dans laquelle il
est nécessaire de {\bf signaler les bibliothèques utilisées}.
\vspace{-1.5em}
\begin{center}
    \begin{tikzpicture}[every text node part/.style={align=left},yscale=.9]
        \node[Module](A)at(0,0){A.h \\ A.c};
        \node[Module](B)at(0,-1){B.h \\ B.c};
        \node[Module](Main)at(0,-2){Main.c};
        \node[FObj](Ao)at(3.5,0){A.o};
        \node[FObj](Bo)at(3.5,-1){B.o};
        \node[FObj](Maino)at(3.5,-2){Main.o};
        \node[Bib](libm)at(6,.7){libm.a};
        \node[Bib](libMLV)at(8,.7){libMLV.a};
        \node[Exec](Ex)at(10,-1){a.out};
        \node[font=\scriptsize \tt](edliens)at(7,-1)
            {gcc Main.o A.o B.o -lm -lMLV};
        \draw(A)edge[->,anchor=south,font=\scriptsize \tt] node
            {gcc -c A.c}(Ao);
        \draw(B)edge[->,anchor=south,font=\scriptsize \tt] node
            {gcc -c B.c}(Bo);
        \draw(Main)edge[->,anchor=south,font=\scriptsize \tt] node
            {gcc -c Main.c}(Maino);
        \draw[->](Ao)edge[bend left=20]node{}(edliens);
        \draw[->](Bo)--(edliens);
        \draw[->](Maino)edge[bend right=20]node{}(edliens);
        \draw[->](libm)edge[bend left=30]node{}(edliens);
        \draw[->](libMLV)edge[bend left=30]node{}(edliens);
        \draw[->](edliens)--(Ex);
    \end{tikzpicture}
\end{center}
\medskip

Pour utiliser des bibliothèques qui ne sont pas situées dans le répertoire
standard des bibliothèques, on précisera leur chemin \Code{chem} lors de
l'édition des liens au moyen de l'option \Code{-Lchem}.
\end{frame}

%%%%%%%%%%%%%%%%%%%%%%%%%%%%%%%%%%%%%%%%%%%%%%%%%%%%%%%%%%%%%%%%%%%%%%%%
\begin{frame}[fragile]
\frametitle{Compilation d'un projet utilisant des bibliothèques}
Le nom d'une bibliothèque commence par \Code{lib}. On signale l'utilisation
d'une bibliothèque à l'éditeur de liens par \Code{-lNOM} où \Code{libNOM}
est le nom de la bibliothèque.
%\medskip

\begin{multicols}{2}
\footnotesize
En supposant que le graphe d'inclusions (étendues) du projet précédent
est celui ci-contre, un \Code{Makefile} possible est
\begin{center}
\scalebox{.6}{\begin{tikzpicture}
    \node[Sommet](A)at(0,-.5){\Code{A}};
    \node[Sommet](B)at(0,-1.5){\Code{B}};
    \node[Sommet](Main)at(2,-1){\Code{Main}};
    \draw[Fleche](B)--(A);
    \draw[Fleche,dashed,draw=Vert](Main)edge[bend right=20]node{}(A);
    \draw[Fleche,dashed,draw=Vert](Main)edge[bend left=20]node{}(B);
\end{tikzpicture}}
\end{center}
\end{multicols}\vspace{-1.5em}
\begin{multicols}{2}
\begin{lstlisting}[language=make,basicstyle=\scriptsize\tt]
CC=colorgcc
CFLAGS=-ansi -pedantic -Wall
LDFLAGS=-lm -lMLV
OBJ=Main.o A.o B.o

Projet: $(OBJ)
    $(CC) -o $@ $^ $(CFLAGS) $(LDFLAGS)

Main.o: Main.c A.h B.h

A.o: A.c A.h

B.o: B.c B.h A.h

%.o: %.c
    $(CC) -c $^ $(CFLAGS)

clean:
    rm -f *.o

mrproper: clean
    rm -f Projet

install: Projet
    mkdir ../bin
    mv Projet ../bin/Projet
    make mrproper

uninstall: mrproper
    rm -f ../bin/Projet
    rm -rf ../bin
\end{lstlisting}
\begin{math}\end{math}
\end{multicols}
\end{frame}

%%%%%%%%%%%%%%%%%%%%%%%%%%%%%%%%%%%%%%%%%%%%%%%%%%%%%%%%%%%%%%%%%%%%%%%%
\begin{frame}[fragile]
\frametitle{La bibliothèque standard}
La \alert{bibliothèque standard} \Code{libc.a} (\Code{libc.so}) regroupe
les vingt-quatre modules
\begin{center}
    \begin{tabular}{cccc}
    \Code{assert}, &\quad \Code{complex}, &\quad \Code{ctype}, &\quad \Code{errno}, \\
    \Code{fenv}, &\quad \Code{float}, &\quad \Code{inttypes}, &\quad \Code{iso646}, \\
    \Code{limits}, &\quad \Code{locale}, &\quad \Code{math}, &\quad \Code{setjmp}, \\
    \Code{signal}, &\quad \Code{stdarg}, &\quad \Code{stdbool}, &\quad \Code{stddef}, \\
    \Code{stdint}, &\quad \Code{stdio}, &\quad \Code{stdlib}, &\quad \Code{string}, \\
    \Code{tgmath}, &\quad \Code{time}, &\quad \Code{wchar}, &\quad \Code{wctype}.
    \end{tabular}
\end{center}
\medskip

Cette bibliothèque est {\bf liée implicitement} lors de toute édition
des liens.
\medskip

Il est donc possible d'utiliser les modules de la bibliothèque standard
juste en les incluant (\Code{\#include <NOM.h>}).
\end{frame}

%%%%%%%%%%%%%%%%%%%%%%%%%%%%%%%%%%%%%%%%%%%%%%%%%%%%%%%%%%%%%%%%%%%%%%%%
\begin{frame}[fragile]
\frametitle{Création de bibliothèques statiques}
Pour \alert{créer une bibliothèque statique} \Code{X}, on suit les étapes
suivantes~:
\smallskip

\begin{small}
\begin{enumerate}
    \item {\bf écriture des modules} \Code{M1}, \dots, \Code{Mn} qui vont
    constituer la bibliothèque~;
    \medskip

    \item {\bf compilation des modules} et création des fichiers objets
    \Code{M1.o}, \dots, \Code{Mn.o}~;
    \medskip

    \item {\bf création de l'archive} \Code{libX.a} par
    \begin{center}\Code{ar r libX.a M1.o ... Mn.o}\end{center}
    \medskip

    \item {\bf génération de l'index} de la bibliothèque par
    \begin{center} \Code{ranlib libX.a} \end{center}
    \medskip

    \item (étape facultative) {\bf écriture d'un fichier d'en-tête global}
    \begin{center}
    \begin{minipage}[c]{.3\textwidth}
\begin{lstlisting}[frame=single,numbers=none,basicstyle=\scriptsize\tt]
/* X.h */
#ifndef __X__
#define __X__

#include "M1.h"
...
#include "Mn.h"

#endif
\end{lstlisting}
    \end{minipage}
    \end{center}
\end{enumerate}
\end{small}
\end{frame}


%%%%%%%%%%%%%%%%%%%%%%%%%%%%%%%%%%%%%%%%%%%%%%%%%%%%%%%%%%%%%%%%%%%%%%%%
\begin{frame}[fragile]
\frametitle{Création de bibliothèques statiques --- exemple}
On suppose que l'on a créé trois modules~:
\begin{enumerate}
    \item \Code{Liste} pour la représentation des listes chaînées~;
    \smallskip

    \item \Code{Arbre} pour la représentation des arbres binaires de
    recherche~;
    \smallskip

    \item \Code{Tri} pour l'implantation d'algorithmes de tris de tableaux.
\end{enumerate}
\bigskip
\bigskip

On souhaite regrouper ces modules en une bibliothèque nommée \Code{algo}.
\medskip

Celle-ci sera constituée de deux fichiers~;
\begin{enumerate}
    \item \Code{libalgo.a}, l'implantation de la bibliothèque~;
    \smallskip

    \item \Code{Algo.h}, l'en-tête de la bibliothèque.
\end{enumerate}
\end{frame}

%%%%%%%%%%%%%%%%%%%%%%%%%%%%%%%%%%%%%%%%%%%%%%%%%%%%%%%%%%%%%%%%%%%%%%%%
\begin{frame}[fragile]
\frametitle{Création de bibliothèques statiques --- exemple}
Pour créer de la bibliothèque \Code{algo}, on saisit les commandes
\smallskip

\begin{small}
\Code{gcc -c Liste.c ; gcc -c Arbre.c ; gcc -c Tri.c} \\
\Code{ar r libalgo.a Liste.o Arbre.o Tri.o} \\
\Code{ranlib libalgo.a}
\end{small}
\smallskip

et on écrit l'en-tête global
\begin{center}
\begin{minipage}[c]{.3\textwidth}
\begin{lstlisting}[frame=single,numbers=none,basicstyle=\scriptsize\tt]
/* Algo.h */
#ifndef __ALGO__
#define __ALGO__

#include "Liste.h"
#include "Arbre.h"
#include "Tri.h"

#endif
\end{lstlisting}
\end{minipage}
\end{center}
\medskip

Pour utiliser la bibliothèque \Code{algo} dans un fichier \Code{F} d'un
projet \Code{P}, il faut inclure dans \Code{F} son en-tête, réaliser
l'édition des liens de \Code{P} avec l'option \Code{-lalgo} et indiquer
son chemin \Code{chem} avec l'option \Code{-Lchem}.
\end{frame}

%%%%%%%%%%%%%%%%%%%%%%%%%%%%%%%%%%%%%%%%%%%%%%%%%%%%%%%%%%%%%%%%%%%%%%%%
\begin{frame}[fragile]
\frametitle{Index}
Il est possible de \alert{consulter l'index} d'une bibliothèque statique
\Code{LIB} par la commande
\begin{center}
\Code{nm -s libLIB.a}
\end{center}
\medskip

\begin{center}
\begin{minipage}[c]{.22\textwidth}
\begin{lstlisting}[frame=single,numbers=none,basicstyle=\scriptsize\tt]
/* A.h */
#ifndef __A__
#define __A__
    char h(int a);
#endif
\end{lstlisting}
\end{minipage}
\enspace
\begin{minipage}[c]{.23\textwidth}
\begin{lstlisting}[frame=single,numbers=none,basicstyle=\scriptsize\tt]
/* A.c */
#include "A.h"
char h(int a) {
    return a % 256;
}
\end{lstlisting}
\end{minipage}
\quad
\begin{minipage}[c]{.23\textwidth}
\begin{lstlisting}[frame=single,numbers=none,basicstyle=\scriptsize\tt]
/* B.h */
#ifndef __B__
#define __B__
    typedef int S;

    int f(S s);
    char g(int a);
#endif
\end{lstlisting}
\end{minipage}
\enspace
\begin{minipage}[c]{.22\textwidth}
\begin{lstlisting}[frame=single,numbers=none,basicstyle=\scriptsize\tt]
/* B.c */
#include "B.h"
int f(S s) {
    return s;
}
char g(int a) {
    return a % 64;
}
\end{lstlisting}
\end{minipage}
\end{center}

\begin{center}
\begin{minipage}[c]{.4\textwidth}
{\bf Création}~: \\
\begin{small}
\Code{gcc -c A.c ; gcc -c B.c} \\
\Code{ar r libAB.a A.o B.o} \\
\Code{ranlib libAB.a}
\end{small}
\medskip

{\bf Affichage}~: \\
\begin{small}
\Code{nm -s libAB.a}
\end{small}
\end{minipage}
\qquad
\begin{minipage}[c]{.2\textwidth}
\begin{lstlisting}[frame=single,numbers=none,basicstyle=\ttfamily\tiny,
showstringspaces=false]
Indexe de l'archive:
h in A.o
f in B.o
g in B.o

A.o:
0000000000000000 T h

B.o:
0000000000000000 T f
000000000000000c T g
\end{lstlisting}
\end{minipage}
\end{center}
\end{frame}
