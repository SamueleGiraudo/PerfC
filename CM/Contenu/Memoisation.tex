% Auteur : Samuele Giraudo
% Création : déc. 2013
% Modifications : août 2014, fév. 2015, déc. 2015

%%%%%%%%%%%%%%%%%%%%%%%%%%%%%%%%%%%%%%%%%%%%%%%%%%%%%%%%%%%%%%%%%%%%%%%%
%%%%%%%%%%%%%%%%%%%%%%%%%%%%%%%%%%%%%%%%%%%%%%%%%%%%%%%%%%%%%%%%%%%%%%%%
%%%%%%%%%%%%%%%%%%%%%%%%%%%%%%%%%%%%%%%%%%%%%%%%%%%%%%%%%%%%%%%%%%%%%%%%
\section{Mémoïsation}

%%%%%%%%%%%%%%%%%%%%%%%%%%%%%%%%%%%%%%%%%%%%%%%%%%%%%%%%%%%%%%%%%%%%%%%%
%%%%%%%%%%%%%%%%%%%%%%%%%%%%%%%%%%%%%%%%%%%%%%%%%%%%%%%%%%%%%%%%%%%%%%%%
\subsection{Variables statiques}

%%%%%%%%%%%%%%%%%%%%%%%%%%%%%%%%%%%%%%%%%%%%%%%%%%%%%%%%%%%%%%%%%%%%%%%%
\begin{frame}[fragile]
\frametitle{Notion de variable statique}
Une \alert{variable statique} est une variable dont la durée de vie est
égale à celle de l'exécution du programme. Une telle variable est créée
et initialisée à la compilation.
\medskip

L'instruction
\begin{center}
    \Code{static T ID;}
\end{center}
déclare une variable statique \Code{ID} de type \Code{T}.
\bigskip
\bigskip
\bigskip

Elles sont les contreparties des {\bf variables automatiques} dont la
durée de vie s'étend à l'exécution du bloc dans lesquels elles se trouvent
et dont la création se fait à l'exécution le moment voulu.
\medskip

Les variables automatiques sont les variables habituelles (se déclarent
sans le mot clé \Code{static}).
\end{frame}

%%%%%%%%%%%%%%%%%%%%%%%%%%%%%%%%%%%%%%%%%%%%%%%%%%%%%%%%%%%%%%%%%%%%%%%%
\begin{frame}[fragile]
\frametitle{Persistance des variables statiques}
Une variable statique est \alert{rémanente}~: elle
{\bf conserve sa dernière valeur} jusqu'à une éventuelle modification.
\medskip

\begin{multicols}{2}
\begin{lstlisting}
void fct(int a) {
    static int s = 3;
    s = s + a;
    printf("%d\n", s);
}
...
fct(1);
fct(5);
fct(2);
\end{lstlisting}

\bigskip
\bigskip
\bigskip
\bigskip
Produit l'affichage \\
\Sortie{4} \\
\Sortie{9} \\
\Sortie{11}.
\end{multicols}

Ici, \Code{s} est une variable statique. La valeur \Code{3} (initialisation)
ne lui est attribuée qu'une fois, lors de la compilation.
\medskip

De plus, \Code{s} conserve sa valeur (qui peut être modifiée) au fil
de l'exécution, d'appel en appel à \Code{fct}.
\end{frame}

%%%%%%%%%%%%%%%%%%%%%%%%%%%%%%%%%%%%%%%%%%%%%%%%%%%%%%%%%%%%%%%%%%%%%%%%
\begin{frame}[fragile]
\frametitle{Géographie de la mémoire lors d'une exécution}
Lors de l'exécution d'un programme, la mémoire qui lui est consacrée
est divisée de la manière suivante.
\begin{multicols}{2}
\begin{center}
    \scalebox{.9}{\begin{tikzpicture}
    \node[Zone,fill=Rouge!30](1)at(0,0){Zone statique};
    \node[Zone,fill=Bleu!30](2)at(0,-1.5){Tas\phantom{q}};
    \node[Zone,fill=Vert!30](3)at(0,-3){\dots\phantom{Tq}};
    \node[Zone,fill=Marron!30](4)at(0,-4.5){Pile\phantom{q}};
    \node[Zone,fill=gray!30,minimum height=.5cm](5)at(0,-5.5){Autre\phantom{q}};
    \draw[->,line width=1.5pt](-2,-1.5)--(-2,-2.5);
    \draw[->,line width=1.5pt](-2,-4.5)--(-2,-3.5);
    \end{tikzpicture}}
\end{center}
Les zones de la mémoire réservées à la {\bf pile} et au {\bf tas} ont
des {\bf tailles variables durant l'exécution}.
\medskip

Le tas grandit vers le bas (vers les adresses hautes) et la pile grandit
vers le haut (vers les adresses basses).
\medskip

La \alert{zone statique}, qui contient les variables statiques et le
code du programme, est de \alert{taille constante durant l'exécution}.
\medskip

Cette taille est connue et calculée lors de la compilation.
\end{multicols}
\end{frame}

%%%%%%%%%%%%%%%%%%%%%%%%%%%%%%%%%%%%%%%%%%%%%%%%%%%%%%%%%%%%%%%%%%%%%%%%
\begin{frame}[fragile]
\frametitle{Compter le nombre d'appels à une fonction}
Étant donnée une fonction, on souhaite compter le nombre de fois qu'elle
a été appelée depuis le lancement du programme.
\medskip

\begin{multicols}{2}
\begin{lstlisting}
int fact(int n) {
    if (n <= 1) return 1;
    return n * fact(n - 1);
}
\end{lstlisting}
\bigskip
\bigskip
\bigskip
\bigskip
\bigskip

\begin{lstlisting}
int fact(int n) {

    /* Mecanisme de comptage */
    static int nb_app = 0;
    nb_app += 1;

    if (n <= 1) return 1;
    return n * fact(n - 1);
}
\end{lstlisting}
\end{multicols}

À chaque instant, la variable statique \Code{nb\_app} a pour valeur le
nombre de fois que \Code{fact} a été appelée.
\medskip

{\bf Problème}~: cette variable a pour portée lexicale le corps de la
fonction \Code{fact} et est invisible ailleurs. On ne peut donc pas
la lire depuis l'extérieur.
\end{frame}

%%%%%%%%%%%%%%%%%%%%%%%%%%%%%%%%%%%%%%%%%%%%%%%%%%%%%%%%%%%%%%%%%%%%%%%%
\begin{frame}[fragile]
\frametitle{Compter le nombre d'appels à une fonction}
Pour avoir accès à la valeur de \Code{nb\_app} hors du corps de \Code{fact},
on la transmet à l'aide d'un {\bf passage par adresse}.
\begin{lstlisting}
int fact(int n, int *nb) {

    /* Mecanisme de comptage */
    static int nb_app = 0;
    nb_app += 1;
    *nb = nb_app;

    if (n <= 1) return 1;
    return n * fact(n - 1, nb);
}
\end{lstlisting}
\bigskip

On l'utilise se la manière suivante~:
\begin{multicols}{2}
\begin{lstlisting}
int nb;
fact(6, &nb);
printf("%d\n", nb);
\end{lstlisting}
Ceci affiche \Sortie{6}.
\bigskip
\bigskip
\bigskip
\end{multicols}
\end{frame}

%%%%%%%%%%%%%%%%%%%%%%%%%%%%%%%%%%%%%%%%%%%%%%%%%%%%%%%%%%%%%%%%%%%%%%%%
%%%%%%%%%%%%%%%%%%%%%%%%%%%%%%%%%%%%%%%%%%%%%%%%%%%%%%%%%%%%%%%%%%%%%%%%
\subsection{Mémoïsation}

%%%%%%%%%%%%%%%%%%%%%%%%%%%%%%%%%%%%%%%%%%%%%%%%%%%%%%%%%%%%%%%%%%%%%%%%
\begin{frame}[fragile]
\frametitle{Principe de la mémoïsation}
Soit \Code{f} une fonction {\bf sans effet de bord} et
{\bf déterministe} (sa valeur de retour ne dépend que de ses arguments).
\bigskip
\bigskip

{\bf Observation}~: si l'on appelle \Code{f} deux fois avec des
mêmes arguments, elle renverra deux fois la même valeur.
\bigskip

{\bf Conséquence}~: comme il est inutile (et inefficace) de refaire deux
fois un même calcul, il suffit de mémoriser le résultat renvoyé par
\Code{f} lors du 1\ier{} appel et de le renvoyer sans calcul lors du
2\ieme{}.
\bigskip

{\bf Marche à suivre}~: on enregistre toutes les valeurs de retour de
\Code{f} au fur et à mesure de ses appels.
\end{frame}

%%%%%%%%%%%%%%%%%%%%%%%%%%%%%%%%%%%%%%%%%%%%%%%%%%%%%%%%%%%%%%%%%%%%%%%%
\begin{frame}[fragile]
\frametitle{Détails sur la mémoïsation}
Les résultats renvoyés par \Code{f} sont mémorisés dans une
{\bf table de hachage} \Code{mem} associant à chaque jeu d'arguments
avec lequel elle est appelée la valeur de retour de \Code{f}.
\bigskip
\bigskip

Il est nécessaire de disposer d'une {\bf fonction de hachage} \Code{h}
associant à chaque jeu d'arguments un entier.
\bigskip
\bigskip

Il ne doit y avoir qu'une seule table \Code{mem} pour \Code{f}, dont
la durée de vie est égale à l'exécution du programme et rémanente~:
\Code{mem} est ainsi une \alert{variable statique}, locale à \Code{f}.
\end{frame}

%%%%%%%%%%%%%%%%%%%%%%%%%%%%%%%%%%%%%%%%%%%%%%%%%%%%%%%%%%%%%%%%%%%%%%%%
\begin{frame}[fragile]
\frametitle{Procédé général de mémoïsation d'une fonction}
Soit \Code{f} une fonction à \Code{n} paramètres.
\bigskip

On appelle \Code{f'} la \alert{version mémoïsée} de \Code{f}, définie de la
manière suivante.
\medskip

Lors de l'appel à \Code{f'(a\_1, \dots, a\_n)}~:
\medskip

\begin{enumerate}[(1)]
    \item \label{item:lecture}
    si \Code{mem} contient une donnée associée à \Code{(a\_1, \dots, a\_n)},
    \begin{enumerate}[(a)] \normalsize
        \item renvoyer cette valeur~;
    \end{enumerate}
    \medskip

    \item \label{item:memorisation}
    sinon,
    \begin{enumerate}[(a)] \normalsize
        \item calculer la valeur de retour de \Code{f(a\_1, \dots, a\_n)}~;
        \item {\bf mémoriser} cette valeur dans \Code{mem}~;
        \item renvoyer cette valeur.
    \end{enumerate}
\end{enumerate}
\bigskip

\eqref{item:lecture} est l'étape de {\bf lecture} et \eqref{item:memorisation}
est l'étape de {\bf mémorisation}.
\end{frame}

%%%%%%%%%%%%%%%%%%%%%%%%%%%%%%%%%%%%%%%%%%%%%%%%%%%%%%%%%%%%%%%%%%%%%%%%
\begin{frame}[fragile]
\frametitle{Exemple de mémoïsation d'une fonction à un paramètre}
On souhaite mémoïser la fonction
\begin{lstlisting}
int fibo(int n) {
    if (n <= 1) return n;
    return fibo(n - 1) + fibo(n - 2);
}
\end{lstlisting}
\bigskip

Cette fonction est paramétrée par un entier et renvoie un entier.
La table \Code{mem} est ainsi un tableau d'entiers. La valeur de hachage
d'un argument \Code{n} est \Code{n}.
\medskip

On l'utilise de la manière suivante~:
\begin{enumerate}
    \item elle est de taille \Code{T\_MEM} où \Code{T\_MEM} est une constante
    suffisamment grande~;
    \smallskip

    \item pour tout $\Code{0} \leq \Code{n} < \Code{T\_MEM}$,
    la valeur de \Code{mem[n]} est
    \begin{itemize} \normalsize
        \item \Code{-1} si \Code{fibo(n)} n'a pas encore été appelée~;
        \smallskip

        \item la valeur renvoyée par \Code{fibo(n)} si \Code{fibo} a
        été appelée au moins une fois avec l'argument \Code{n}.
    \end{itemize}
\end{enumerate}

\end{frame}

%%%%%%%%%%%%%%%%%%%%%%%%%%%%%%%%%%%%%%%%%%%%%%%%%%%%%%%%%%%%%%%%%%%%%%%%
\begin{frame}[fragile]
\frametitle{Exemple de mémoïsation d'une fonction à un paramètre}
\begin{multicols}{2}
\begin{lstlisting}
int fibo(int n) {
    /* Table */
    static int mem[T_MEM];

    /* Booleen 1er appel */
    static int init = 1;

    int i, res;

    /* Init. 1er appel */
    if (init) {
        for (i=0;i<T_MEM;++i)
            mem[i] = -1;
        init = 0;
    }

    /* Lecture */
    if (mem[n] != -1)
        return mem[n];

    /* Calcul */
    if (n <= 1)
        res = n;
    else
        res = fibo(n - 1)
            + fibo(n - 2);

    /* Memorisation */
    mem[n] = res;

    return res;
}
\end{lstlisting}
\end{multicols}
\end{frame}

%%%%%%%%%%%%%%%%%%%%%%%%%%%%%%%%%%%%%%%%%%%%%%%%%%%%%%%%%%%%%%%%%%%%%%%%
\begin{frame}[fragile]
\frametitle{Mémoïsation de fonctions à plusieurs paramètres}
Soit \Code{f} une fonction à \Code{n} paramètres.
\bigskip

La table \Code{mem} devient un \alert{tableau de variables d'un type structuré}
\Code{E\_f} composé
\begin{enumerate}
    \item d'un champ pour chaque {\bf paramètre} de \Code{f}~;
    \item d'un champ pour le {\bf résultat} renvoyé par \Code{f} appelé
    avec les arguments spécifiés par les précédents champs~;
    \item d'un champ {\bf booléen} indiquant si le résultat de \Code{f}
    appelée avec les arguments spécifiés par les précédents champs a bien
    été calculé.
\end{enumerate}
\bigskip

Il faut de plus disposer d'une \alert{fonction de hachage} \Code{h\_f}
de même signature que \Code{f} et qui renvoie un entier positif.
\medskip

Elle permet d'attribuer à chaque jeu d'arguments une position dans la
table \Code{mem}.
\end{frame}

%%%%%%%%%%%%%%%%%%%%%%%%%%%%%%%%%%%%%%%%%%%%%%%%%%%%%%%%%%%%%%%%%%%%%%%%
\begin{frame}[fragile]
\frametitle{Exemple de mémoïsation d'une fonction à deux paramètres}
On souhaite mémoïser la fonction
\begin{lstlisting}
int puiss(int n, int p) {
    if (p == 0) return 1;
    return n * puiss(n, p - 1);
}
\end{lstlisting}
\medskip

Celle-ci est paramétrée par deux entiers. Le {\bf type structuré}
\Code{E\_puiss} est donc défini par
\begin{lstlisting}
typedef struct {
    int n; /* 1er parametre */
    int p; /* 2e parametre */
    int res; /* Resultat */
    int ok; /* Indique si le calcul a deja ete fait */
} E_puiss;
\end{lstlisting}
\medskip

On utilise la {\bf fonction de hachage} suivante~:
\begin{lstlisting}
int h_puiss(int n, int p) {
    return n ^ (p << 2);
}
\end{lstlisting}
\end{frame}

%%%%%%%%%%%%%%%%%%%%%%%%%%%%%%%%%%%%%%%%%%%%%%%%%%%%%%%%%%%%%%%%%%%%%%%%
\begin{frame}[fragile]
\frametitle{Exemple de mémoïsation d'une fonction à deux paramètres}
\begin{multicols}{2}
\begin{lstlisting}
int puiss(int n, int p) {
    /* Table */
    static E_puiss mem[T_MEM];

    /* Booleen 1er appel */
    static int init = 1;

    int i, res, h;

    /* Init. 1er appel */
    if (init) {
        for (i=0;i<T_MEM;++i)
            mem[i].ok = 0;
        init = 0;
    }

    /* Lecture */
    h = h_puiss(n, p) % T_MEM;
    if (mem[h].ok) {
        if (mem[h].n == n
            && mem[h].p == p)
            return mem[h].res;
    }

    /* Calcul */
    if (p == 0) res = 1;
    else
      res = n * puiss(n, p - 1);

    /* Memorisation */
    mem[h].n = n;
    mem[h].p = p;
    mem[h].res = res;
    mem[h].ok = 1;

    return res;
}
\end{lstlisting}
\end{multicols}
\end{frame}
