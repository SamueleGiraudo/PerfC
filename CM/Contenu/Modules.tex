% Auteur : Samuele Giraudo
% Création : jan. 2014, jan. 2015, fév. 2015 , déc. 2015

\tikzstyle{Module}=[rectangle,draw=Violet!100,fill=Violet!20,
    line width=1pt,font=\scriptsize\tt]
\tikzstyle{ModuleC}=[Module,draw=Bleu!100,fill=Bleu!20]
\tikzstyle{ModuleH}=[Module,draw=Vert!100,fill=Vert!20]
\tikzstyle{Fleche}=[->,draw=Rouge,line width=1.5pt]
\tikzstyle{Sommet}=[circle,draw=Marron!100,fill=Marron!10,line width=1.5pt]

%%%%%%%%%%%%%%%%%%%%%%%%%%%%%%%%%%%%%%%%%%%%%%%%%%%%%%%%%%%%%%%%%%%%%%%%
%%%%%%%%%%%%%%%%%%%%%%%%%%%%%%%%%%%%%%%%%%%%%%%%%%%%%%%%%%%%%%%%%%%%%%%%
%%%%%%%%%%%%%%%%%%%%%%%%%%%%%%%%%%%%%%%%%%%%%%%%%%%%%%%%%%%%%%%%%%%%%%%%
\section{Modules}

%%%%%%%%%%%%%%%%%%%%%%%%%%%%%%%%%%%%%%%%%%%%%%%%%%%%%%%%%%%%%%%%%%%%%%%%
%%%%%%%%%%%%%%%%%%%%%%%%%%%%%%%%%%%%%%%%%%%%%%%%%%%%%%%%%%%%%%%%%%%%%%%%
\subsection{Notion de modularité}

%%%%%%%%%%%%%%%%%%%%%%%%%%%%%%%%%%%%%%%%%%%%%%%%%%%%%%%%%%%%%%%%%%%%%%%%
\begin{frame}[fragile]
\frametitle{Modules}
\alert{Modulariser} un projet signifie le découper de manière cohérente
en plusieurs parties plus petites.
\medskip

Un \alert{module} est un ensemble de données et d'instructions qui
permettent de gérer une partie bien ciblée d'un projet.
\bigskip

Il existe deux manières de concevoir un projet~:
\smallskip

\begin{enumerate}
    \item programmer dans un unique fichier contenant tout
    le code nécessaire~;
    \smallskip

    \item programmer dans divers fichiers qui fractionnent le projet
    en plusieurs sous-parties.
\end{enumerate}
\medskip

À partir de maintenant, on adoptera la $2\ieme$ manière.
\bigskip

Il reste à savoir comment {\bf découper un projet de manière cohérente}
et comment {\bf utiliser les outils offerts par le langage} pour gérer ce
découpage.
\end{frame}

%%%%%%%%%%%%%%%%%%%%%%%%%%%%%%%%%%%%%%%%%%%%%%%%%%%%%%%%%%%%%%%%%%%%%%%%
\begin{frame}[fragile]
\frametitle{Avantages offerts par la modularité}
La modularité, illustration du principe stratégique universel
\begin{center}
    \og {\em diviser pour régner} \fg,
\end{center}
offre les avantages suivants.
\medskip

\begin{enumerate}
    \item La {\bf lisibilité} du code est accrue, ainsi que la facilité
    de son {\bf entretien}.
    \medskip

    \item Permet de {\bf regrouper} les types et les fonctions selon leurs
    objectifs.
    \medskip

    \item Il devient possible de {\bf réutiliser} dans un nouveau projet
    un module créé dans un projet antérieur.
    \medskip

    \item Permet de {\bf cacher des fonctions} (notion de fonctions privées).
    \medskip

    \item Facilite le {\bf travail par équipe}.
    \medskip

    \item Permet de rendre la {\bf compilation localisée}
    (compilation module par module).
\end{enumerate}
\end{frame}

%%%%%%%%%%%%%%%%%%%%%%%%%%%%%%%%%%%%%%%%%%%%%%%%%%%%%%%%%%%%%%%%%%%%%%%%
%%%%%%%%%%%%%%%%%%%%%%%%%%%%%%%%%%%%%%%%%%%%%%%%%%%%%%%%%%%%%%%%%%%%%%%%
\subsection{Découpage d'un projet}

%%%%%%%%%%%%%%%%%%%%%%%%%%%%%%%%%%%%%%%%%%%%%%%%%%%%%%%%%%%%%%%%%%%%%%%%
\begin{frame}[fragile]
\frametitle{Spécification d'un projet}
Considérons le \alert{projet spécifié} de la manière suivante~:
\medskip

\begin{itemize}
    \item le but est de fournir un programme qui permet de décider si des
    formules logiques sans quantificateur sont valides ou contradictoires.
    \medskip

    \item La syntaxe d'une formule est la suivante~: on dispose
    du jeu de formules atomiques $a$, $b$, \dots, $z$ et on écrit les
    formules de manière infixe et totalement parenthésée.
    Par exemple, la formule $(A \to (B \vee \neg C)) \wedge A$ s'écrit
    \Code{((a IMP (b OU (NON c))) ET a)}.
    \medskip

    \item L'interaction utilisateur/programme se fait de la manière
    suivante~:
    \begin{enumerate} \normalsize
        \item l'utilisateur fournit un fichier en entrée contenant une
        formule par ligne~;
        \smallskip

        \item le programme produit un fichier en sortie contenant, ligne
        par ligne, la réponse \Sortie{erreur} si la formule correspondante
        est syntaxiquement erronée ou bien \Sortie{valide},
        \Sortie{contradictoire} ou \Sortie{rien} selon la nature de la
        formule.
    \end{enumerate}
\end{itemize}
\end{frame}

%%%%%%%%%%%%%%%%%%%%%%%%%%%%%%%%%%%%%%%%%%%%%%%%%%%%%%%%%%%%%%%%%%%%%%%%
\begin{frame}[fragile]
\frametitle{Découpage du projet}
Il y a deux parties bien distinctes dans ce projet~:
\smallskip

\begin{enumerate}
    \item la {\bf représentation des formules} et le test de
    validité/contradiction~;
    \smallskip

    \item la {\bf gestion syntaxique des formules} (lecture/écriture
    d'une formule dans un fichier).
\end{enumerate}

Ces deux parties dictent le découpage suivant~:
\begin{center}
\begin{tikzpicture}[every text node part/.style={align=left}]
    \node[Module](Formule)at(0,0){
        -- type formule \\
        -- test validité \\
        -- test contradiction \\[.5em]
        {\small \bf Formule}};
    %
    \node[Module](Parseur)at(5,0){
        -- test syntaxique \\
        \phantom{--} d'une formule \\
        -- chaîne vers formule \\
        -- formule vers chaîne \\[.5em]
        {\small \bf Parseur}};
    %
    \node[Module](Main)at(2.5,-3){
        -- lecture fichier de \\
        \phantom{--} formules et écriture \\
        \phantom{--} des résultats \\[.5em]
        {\small \bf Main}};
    %
    \draw[Fleche](Parseur)--(Formule);
    \draw[Fleche](Main)edge[bend left](Formule);
    \draw[Fleche](Main)edge[bend right](Parseur);
\end{tikzpicture}
\end{center}
Toute flèche \hspace{-1em}
\raisebox{-.4em}{
\begin{tikzpicture}
    \node(1)at(0,0){$A$};
    \node(2)at(1,0){$B$};
    \draw[Fleche](1)--(2);
\end{tikzpicture}}
signifie que le module $A$ \alert{dépend} du module $B$.
\end{frame}

%%%%%%%%%%%%%%%%%%%%%%%%%%%%%%%%%%%%%%%%%%%%%%%%%%%%%%%%%%%%%%%%%%%%%%%%
%%%%%%%%%%%%%%%%%%%%%%%%%%%%%%%%%%%%%%%%%%%%%%%%%%%%%%%%%%%%%%%%%%%%%%%%
\subsection{Fichiers source/d'en-tête}

%%%%%%%%%%%%%%%%%%%%%%%%%%%%%%%%%%%%%%%%%%%%%%%%%%%%%%%%%%%%%%%%%%%%%%%%
\begin{frame}[fragile]
\frametitle{Composition d'un module}
Un module est composé de deux fichiers~:
\begin{enumerate}
    \item un \alert{fichier d'en-tête} d'extension \Code{.h}~;
    \item un \alert{fichier source} d'extension \Code{.c}.
\end{enumerate}
\medskip

Les noms de ces deux fichiers sont les mêmes (extension mise à part).
\medskip

\begin{center}
\begin{tikzpicture}[every text node part/.style={align=left}]
    \node[ModuleH](Ah)at(0,0){
        -- ... \\
        -- ... \\[.5em]
        {\small \bf A.h}};
    %
    \node[ModuleC](Ac)at(1.5,0){
        -- ... \\
        -- ... \\[.5em]
        {\small \bf A.c}};
\end{tikzpicture}
\end{center}

Seul le module principal est constitué d'un seul fichier source \Code{Main.c}.
\begin{center}
\begin{tikzpicture}[every text node part/.style={align=left}]
    \node[ModuleC](Mainc)at(0,0){
        -- ... \\
        -- ... \\[.5em]
        {\small \bf Main.c}};
\end{tikzpicture}
\end{center}
\bigskip

Par exemple, le projet précédent est constitué des fichiers
\Code{Formule.h}, \Code{Formule.c}, \Code{Parseur.h}, \Code{Parseur.c}
et \Code{Main.c}.
\end{frame}

%%%%%%%%%%%%%%%%%%%%%%%%%%%%%%%%%%%%%%%%%%%%%%%%%%%%%%%%%%%%%%%%%%%%%%%%
\begin{frame}[fragile]
\frametitle{Fichiers d'en-tête}
Un fichier d'en-tête contient des \alert{déclarations} de types et
de fonctions (prototypes).
\bigskip

Les prototypes qui y figurent sont ceux des fonctions que l'on
souhaite rendre visibles (utilisables) par d'autres modules.
\bigskip

Par exemple, un en-tête possible du module \Code{Formule} est
\medskip

\begin{minipage}[c]{.4\textwidth}
\begin{lstlisting}[frame=single,numbers=none]
/* Formule.h */

typedef struct {
...
} Form;

int est_valide(Form *f);
int est_contra(Form *f);
\end{lstlisting}
\end{minipage}
\end{frame}

%%%%%%%%%%%%%%%%%%%%%%%%%%%%%%%%%%%%%%%%%%%%%%%%%%%%%%%%%%%%%%%%%%%%%%%%
\begin{frame}[fragile]
\frametitle{Fichiers d'en-tête}
Les fichiers d'en-tête sont ceux que le programmeur regarde en premier
pour connaître le \alert{rôle d'un type ou d'une fonction}.
\medskip

De ce fait, c'est dans les fichiers d'en-tête que l'{\bf on commente chaque
type et fonction} pour préciser leur rôle.
\medskip

Par exemple, l'en-tête du module \Code{Formule} devrait être de la forme
\medskip

\begin{minipage}[c]{.9\textwidth}
\begin{lstlisting}[frame=single,numbers=none]
/* Formule.h */

/* Representation des formules logiques sans quantificateur. */
typedef struct {
...
} Form;

/* Renvoie `1` si la formule pointee par `f` est valide et
 * `0` sinon. */
int est_valide(Form *f);
...
\end{lstlisting}
\end{minipage}
\end{frame}

%%%%%%%%%%%%%%%%%%%%%%%%%%%%%%%%%%%%%%%%%%%%%%%%%%%%%%%%%%%%%%%%%%%%%%%%
\begin{frame}[fragile]
\frametitle{Fichiers source}
Un fichier source contient une \alert{implantation} de l'en-tête du
module auquel il appartient.
\medskip

Il contient de ce fait les définitions de fonctions déclarées dans
l'en-tête.
\medskip

Par exemple, une implantation possible du module \Code{Formule} est
\medskip

\begin{minipage}[c]{.4\textwidth}
\begin{lstlisting}[frame=single,numbers=none]
/* Formule.c */

...
int est_valide(Form *f) {
    int i, j;
    assert(f != NULL);
    ...
}

int est_contra(Form *f) {
    ...
}
\end{lstlisting}
\end{minipage}
\end{frame}

%%%%%%%%%%%%%%%%%%%%%%%%%%%%%%%%%%%%%%%%%%%%%%%%%%%%%%%%%%%%%%%%%%%%%%%%
\begin{frame}[fragile]
\frametitle{Fichiers source}
Il faut impérativement que toutes les fonctions déclarées dans le fichier
d'en-tête du module soient \alert{définies} dans le fichier source
correspondant.
\bigskip
\bigskip

Il est en revanche possible de définir dans un fichier source des fonctions
qui ne sont pas déclarées dans l'en-tête correspondant.
\bigskip
\bigskip

Une fonction définie dans un fichier source mais pas dans l'en-tête
correspondant s'appelle \alert{fonction privée}.
\bigskip
\bigskip

L'intérêt des fonctions privées est d'être des {\bf fonctions outils}
dont le champ d'application est local au module dans lequel elles sont
définies. On ne souhaite pas les rendre utilisables en dehors.
\end{frame}

%%%%%%%%%%%%%%%%%%%%%%%%%%%%%%%%%%%%%%%%%%%%%%%%%%%%%%%%%%%%%%%%%%%%%%%%
\begin{frame}[fragile]
\frametitle{Fichiers source et fonctions privées}
Une fonction privée se définit avec le mot clé \Code{static} (à ne pas
confondre avec le \Code{static} pour la déclaration de variables).
\bigskip

Par exemple, on peut avoir besoin d'une fonction privée appartenant au
module \Code{Formule} qui permet de compter le nombre d'occurrences d'un
atome dans une formule. On la définit alors dans \Code{Formule.c} par
\begin{lstlisting}
static int nb_occ(Form *f, char atome) {
    ...
    assert(f != NULL);
    assert('a' <= atome && atome <= 'z');
    ...
}
\end{lstlisting}

La portée lexicale de cette fonction s'étend à tout ce qui suit
sa définition dans le fichier \Code{Formule.c}. Elle est invisible
ailleurs.
\end{frame}

%%%%%%%%%%%%%%%%%%%%%%%%%%%%%%%%%%%%%%%%%%%%%%%%%%%%%%%%%%%%%%%%%%%%%%%%
\begin{frame}[fragile]
\frametitle{Fichiers source et fonctions privées}
C'est un contresens que de définir une fonction dans un fichier source
sans le mot clé \Code{static} et sans l'avoir déclarée dans le fichier
d'en-tête.
\bigskip

C'est aussi un contresens que de définir dans un fichier source une
fonction déclarée dans le fichier d'en-tête avec \Code{static}.
\bigskip

Ainsi, pour résumer, toute fonction définie dans un fichier source est
\begin{enumerate}
    \item soit déclarée dans le fichier d'en-tête~;
    \item soit non déclarée dans le fichier d'en-tête mais définie
    par \Code{static}.
\end{enumerate}
\end{frame}

%%%%%%%%%%%%%%%%%%%%%%%%%%%%%%%%%%%%%%%%%%%%%%%%%%%%%%%%%%%%%%%%%%%%%%%%
\begin{frame}[fragile]
\frametitle{Allure d'un projet}
Il y a deux manières d'organiser un projet en termes de fichiers et de
répertoires~:
\medskip

\begin{enumerate}
    \item la 1\iere{} consiste à regrouper l'ensemble des fichiers
    d'en-tête et des fichiers sources dans un même répertoire. Il y figure
    donc un nombre impair de fichiers (le module principal et les
    paires en-tête/source pour chaque module)~;
    \bigskip

    \item la 2\ieme{} consiste à séparer les fichiers du projet
    en deux répertoires frères, \Code{include} et \Code{src}, le
    premier contenant les fichiers d'en-tête et l'autre, les fichiers
    sources et le module principal du projet.
\end{enumerate}

\end{frame}

%%%%%%%%%%%%%%%%%%%%%%%%%%%%%%%%%%%%%%%%%%%%%%%%%%%%%%%%%%%%%%%%%%%%%%%%
%%%%%%%%%%%%%%%%%%%%%%%%%%%%%%%%%%%%%%%%%%%%%%%%%%%%%%%%%%%%%%%%%%%%%%%%
\subsection{Création de modules}

%%%%%%%%%%%%%%%%%%%%%%%%%%%%%%%%%%%%%%%%%%%%%%%%%%%%%%%%%%%%%%%%%%%%%%%%
\begin{frame}[fragile]
\frametitle{Inclusion de modules}
Pour utiliser un module \Code{Module} dans un fichier \Code{F}, on doit
l'y \alert{inclure}.
\bigskip

On utilise pour cela dans \Code{F} la commande pré-processeur
\begin{center}
    \Code{\#include "Module.h"}
\end{center}
\bigskip

Celle-ci sera remplacée par le pré-processeur par le contenu de
\Code{Module.h}.
\bigskip

Cette commande peut se trouver
\smallskip

\begin{itemize}
    \item dans un fichier source pour bénéficier des fonctions définies et
    des types déclarés par le module~;
    \medskip

    \item dans un fichier d'en-tête pour bénéficier des types déclarés
    par le module.
\end{itemize}
\end{frame}

%%%%%%%%%%%%%%%%%%%%%%%%%%%%%%%%%%%%%%%%%%%%%%%%%%%%%%%%%%%%%%%%%%%%%%%%
\begin{frame}[fragile]
\frametitle{Inclusion de modules}
Par exemple, si \Code{A} est un module définissant une fonction
\Code{f}, pour utiliser \Code{f} dans un fichier \Code{B.c}
 situé dans le même répertoire que \Code{A.c} et \Code{A.h},
on écrit
\medskip

\begin{minipage}[c]{.3\textwidth}
\begin{lstlisting}[frame=single,numbers=none]
/* B.c */

#include "A.h"
...
\end{lstlisting}
\end{minipage}
\medskip

Il est possible d'inclure à la suite plusieurs modules dans un même fichier~:
\medskip

\begin{minipage}[c]{.4\textwidth}
\begin{lstlisting}[frame=single,numbers=none]
/* Fichier.c ou Fichier.h */

#include "A.h"
#include "B.h"
#include "C.h"
...
\end{lstlisting}
\end{minipage}
\qquad
\begin{minipage}[c]{.5\textwidth}
De cette manière, \Code{Fichier.c} ou \Code{Fichier.h} bénéficie de tout
ce qui est déclaré et défini dans les modules \Code{A}, \Code{B} et
\Code{C}.
\medskip

{\bf Attention}~: les modules inclus ne doivent pas déclarer/définir des
éléments d'un identificateur commun.
\end{minipage}
\end{frame}

%%%%%%%%%%%%%%%%%%%%%%%%%%%%%%%%%%%%%%%%%%%%%%%%%%%%%%%%%%%%%%%%%%%%%%%%
\begin{frame}[fragile]
\frametitle{Inclusion de modules}
Le fichier incluant n'a \alert{accès} qu'au fichier d'en-tête du module,
et donc qu'aux \alert{déclarations effectuées}.
\medskip

Le fichier incluant n'a pas besoin de connaître l'implantation du module.
\bigskip

Considérons par exemple le module \Code{Couple} défini par
\medskip

\begin{minipage}[c]{.38\textwidth}
\begin{lstlisting}[frame=single,numbers=none]
/* Couple.h */

typedef int Couple[2];

int est_zero(Couple c);
void afficher(Couple c);
\end{lstlisting}
\end{minipage}
\quad
\begin{minipage}[c]{.55\textwidth}
\begin{lstlisting}[frame=single,numbers=none]
/* Couple.c */
...
int est_zero(Couple c) {
  return (c[0] == 0) && (c[1]== 0);
}
void afficher(Couple c) {
    printf("(%d, %d)", c[0], c[1]);
}
\end{lstlisting}
\end{minipage}
\end{frame}

%%%%%%%%%%%%%%%%%%%%%%%%%%%%%%%%%%%%%%%%%%%%%%%%%%%%%%%%%%%%%%%%%%%%%%%%
\begin{frame}[fragile]
\frametitle{Inclusion de modules}
Supposons que l'on ait besoin d'inclure le module \Code{Couple} dans un
fichier \Code{Fichier.c}.
\medskip

Le pré-processeur aura donc l'effet suivant sur \Code{Fichier.c}~:
\medskip

\begin{minipage}[c]{.3\textwidth}
\begin{lstlisting}[frame=single,numbers=none]
/* Fichier.c avant
 * la passe du
 * pre-processeur */

#include "Couple.h"
...

if (!est_zero(c))
    afficher(c);
...
\end{lstlisting}
\end{minipage}
\qquad \qquad
\begin{minipage}[c]{.4\textwidth}
\begin{lstlisting}[frame=single,numbers=none]
/* Fichier.c apres la passe
 * du pre-processeur */

typedef int Couple[2];
int est_zero(Couple c);
void afficher(Couple c);
...

if (!est_zero(c))
    afficher(c);
...
\end{lstlisting}
\end{minipage}
\medskip

\Code{Fichier.c} a besoin uniquement de connaître les types de retour et
les signatures des fonctions qu'il invoque (connus à leur déclaration).
Il n'a à ce stade pas besoin de connaître les définitions de ces fonctions.
\end{frame}

%%%%%%%%%%%%%%%%%%%%%%%%%%%%%%%%%%%%%%%%%%%%%%%%%%%%%%%%%%%%%%%%%%%%%%%%
\begin{frame}[fragile]
\frametitle{Création complète d'un module}
Pour créer un module \Code{A}, il faut inclure son fichier d'en-tête
\Code{A.h} dans son fichier source \Code{A.c}.
\medskip

\begin{minipage}[c]{.3\textwidth}
\begin{lstlisting}[frame=single,numbers=none]
/* A.h */
...
\end{lstlisting}
\end{minipage}
\qquad \qquad
\begin{minipage}[c]{.3\textwidth}
\begin{lstlisting}[frame=single,numbers=none]
/* A.c */

#include "A.h"
...
\end{lstlisting}
\end{minipage}
\medskip

De cette manière,
\begin{itemize}
    \item d'une part, \Code{A.c} a accès aux types et
    aux prototypes de fonctions déclarés dans \Code{A.h}~;
    \smallskip

    \item d'autre part, cela permet d'implanter les fonctions déclarées
    dans \Code{A.h} sans contrainte d'ordre.
\end{itemize}
\end{frame}

%%%%%%%%%%%%%%%%%%%%%%%%%%%%%%%%%%%%%%%%%%%%%%%%%%%%%%%%%%%%%%%%%%%%%%%%
\begin{frame}[fragile]
\frametitle{Création complète d'un module}
Considérons la situation suivante~:
\bigskip

\begin{minipage}[c]{.22\textwidth}
\begin{lstlisting}[frame=single,numbers=none]
/* A.h */
...
#include "C.h"
...
\end{lstlisting}
\end{minipage}
\enspace
\begin{minipage}[c]{.22\textwidth}
\begin{lstlisting}[frame=single,numbers=none]
/* A.c */

#include "A.h"
...
\end{lstlisting}
\end{minipage}
\quad
\begin{minipage}[c]{.22\textwidth}
\begin{lstlisting}[frame=single,numbers=none]
/* B.h */
...
#include "C.h"
...
\end{lstlisting}
\end{minipage}
\enspace
\begin{minipage}[c]{.22\textwidth}
\begin{lstlisting}[frame=single,numbers=none]
/* B.c */

#include "B.h"
...
\end{lstlisting}
\end{minipage}
\medskip

\begin{minipage}[c]{.22\textwidth}
\begin{lstlisting}[frame=single,numbers=none]
/* C.h */
...
int f();
...
\end{lstlisting}
\end{minipage}
\enspace
\begin{minipage}[c]{.22\textwidth}
\begin{lstlisting}[frame=single,numbers=none]
/* C.c */

#include "C.h"
...
\end{lstlisting}
\end{minipage}
\qquad \qquad
\begin{minipage}[c]{.22\textwidth}
\begin{lstlisting}[frame=single,numbers=none]
/* D.c */

#include "A.h"
#include "B.h"
...
\end{lstlisting}
\end{minipage}
\end{frame}

%%%%%%%%%%%%%%%%%%%%%%%%%%%%%%%%%%%%%%%%%%%%%%%%%%%%%%%%%%%%%%%%%%%%%%%%
\begin{frame}[fragile]
\frametitle{Création complète d'un module}
Le pré-processeur transforme ces fichiers en
\bigskip

\begin{minipage}[c]{.22\textwidth}
\begin{lstlisting}[frame=single,numbers=none]
/* A.h */
...
int f();
...
\end{lstlisting}
\end{minipage}
\enspace
\begin{minipage}[c]{.22\textwidth}
\begin{lstlisting}[frame=single,numbers=none]
/* A.c */

/* Copie A.h */
...
\end{lstlisting}
\end{minipage}
\quad
\begin{minipage}[c]{.22\textwidth}
\begin{lstlisting}[frame=single,numbers=none]
/* B.h */
...
int f();
...
\end{lstlisting}
\end{minipage}
\enspace
\begin{minipage}[c]{.22\textwidth}
\begin{lstlisting}[frame=single,numbers=none]
/* B.c */

/* Copie B.h */
...
\end{lstlisting}
\end{minipage}
\medskip

\begin{minipage}[c]{.22\textwidth}
\begin{lstlisting}[frame=single,numbers=none]
/* C.h */
...
int f();
...
\end{lstlisting}
\end{minipage}
\enspace
\begin{minipage}[c]{.22\textwidth}
\begin{lstlisting}[frame=single,numbers=none]
/* C.c */

/* Copie C.h */
...
\end{lstlisting}
\end{minipage}
\qquad \qquad
\begin{minipage}[c]{.22\textwidth}
\begin{lstlisting}[frame=single,numbers=none]
/* D.c */

int f();
int f();
...
\end{lstlisting}
\end{minipage}
\medskip

{\bf Problème}~: le contenu de \Code{C.h} est copié deux fois dans
\Code{D.c}. Ceci n'est pas accepté par le compilateur car il y a
\alert{multiple déclaration} d'un même symbole (\Code{f} ici).
\end{frame}

%%%%%%%%%%%%%%%%%%%%%%%%%%%%%%%%%%%%%%%%%%%%%%%%%%%%%%%%%%%%%%%%%%%%%%%%
\begin{frame}[fragile]
\frametitle{Création complète d'un module}
La parade consiste à \alert{inclure} un fichier d'en-tête de
\alert{manière conditionnelle}~: on procède à l'inclusion que s'il n'a
pas déjà été inclus.
\medskip

On utilise pour cela les macro-instructions de contrôle de compilation
\Code{\#ifndef} et \Code{\#endif} ainsi que \Code{\#define}.
\medskip

Le schéma général est
\medskip

\begin{minipage}[c]{.49\textwidth}
\begin{lstlisting}[frame=single,numbers=none]
/* A.h */
#ifndef __A__
#define __A__

    /* Declaration de types */

    /* Declaration de fonctions */

#endif
\end{lstlisting}
\end{minipage}\quad
\begin{minipage}[c]{.45\textwidth}
    Ainsi, lors d'une inclusion de \Code{A.h}, le pré-processeur
    vérifie si la macro \Code{\_\_A\_\_} n'existe pas.
    \begin{itemize}
        \item Si elle n'existe pas, alors on la définit
        (\Code{\#define \_\_A\_\_}) et le contenu du module est pris
        en compte~;
        \item sinon, cela signifie que le contenu a déjà été pris en
        compte. Celui-ci n'est pas repris en compte une 2\ieme{} fois.
    \end{itemize}
\end{minipage}
\end{frame}

%%%%%%%%%%%%%%%%%%%%%%%%%%%%%%%%%%%%%%%%%%%%%%%%%%%%%%%%%%%%%%%%%%%%%%%%
\begin{frame}[fragile]
\frametitle{Squelette d'un module}
Pour résumer, tous les modules doivent avoir le squelette suivant~:
\begin{minipage}[c]{.42\textwidth}
\begin{lstlisting}[frame=single,numbers=none]
/* A.h */
#ifndef __A__
#define __A__

  /* Inclusions eventuelles
   * de modules */

  /* Definitions eventuelles
   * de macros */

  /* Declarations eventuelles
   * de types */

  /* Declarations eventuelles
   * de fonctions */

#endif
\end{lstlisting}
\end{minipage}
\qquad
\begin{minipage}[c]{.45\textwidth}
\begin{lstlisting}[frame=single,numbers=none]
/* A.c */
#include "A.h"

/* Inclusions eventuelles
 * de modules */

/* Definitions eventuelles
 * de fonctions privees */

/* Definitions de toutes les
 * fonctions declarees dans
 * le fichier d'en-tete */
\end{lstlisting}
\end{minipage}
\end{frame}

%%%%%%%%%%%%%%%%%%%%%%%%%%%%%%%%%%%%%%%%%%%%%%%%%%%%%%%%%%%%%%%%%%%%%%%%
\begin{frame}[fragile]
\frametitle{Exemple du module {\tt Parseur}}
\begin{minipage}[c]{.9\textwidth}
\begin{lstlisting}[frame=single,numbers=none,basicstyle=\ttfamily\scriptsize]
/* Parseur.h */
#ifndef __PARSEUR__
#define __PARSEUR__

#include "Formule.h"

    /* Convertit la chaine de caracteres `ch` sensee representer une formule
     * en une variable de type `Form`, qui va etre ecrite dans `f`.
     * Renvoie `1` si `ch` represente bien une formule et `0` sinon. */
    int chaine_vers_form(Form *f, char *ch);

#endif
\end{lstlisting}
\end{minipage}

\begin{minipage}[c]{.5\textwidth}
\begin{lstlisting}[frame=single,numbers=none,basicstyle=\ttfamily\scriptsize]
/* Parseur.c */
#include "Parseur.h"

#include <stdio.h>
#include <stdlib.h>
#include <assert.h>

int chaine_vers_form(Form *f, char *ch) {
    ...
    assert(f != NULL);
    ...
}
\end{lstlisting}
\end{minipage}
\end{frame}

%%%%%%%%%%%%%%%%%%%%%%%%%%%%%%%%%%%%%%%%%%%%%%%%%%%%%%%%%%%%%%%%%%%%%%%%
%%%%%%%%%%%%%%%%%%%%%%%%%%%%%%%%%%%%%%%%%%%%%%%%%%%%%%%%%%%%%%%%%%%%%%%%
\subsection{Graphes d'inclusions}

%%%%%%%%%%%%%%%%%%%%%%%%%%%%%%%%%%%%%%%%%%%%%%%%%%%%%%%%%%%%%%%%%%%%%%%%
\begin{frame}[fragile]
\frametitle{Graphes d'inclusions}
On rappelle qu'un module \Code{A} \alert{dépend} d'un module \Code{B} si
le {\bf fichier d'en-tête} de \Code{A} inclut \Code{B}.
\medskip

Pour visualiser l'allure d'un projet, on trace son \alert{graphe d'inclusions}.
On représente pour cela chacun des modules qui le composent dans des cercles
({\bf sommets}) et on trace des flèches ({\bf arcs}) de \Code{A} vers
\Code{B} pour tout module \Code{A} dépendant de \Code{B}.
\medskip

Par exemple, le graphe d'inclusions
\begin{multicols}{2}
\begin{center}
\scalebox{.8}{\begin{tikzpicture}
    \node[Sommet](A)at(0,0){\Code{A}};
    \node[Sommet](B)at(3,0){\Code{B}};
    \node[Sommet](C)at(0,-2){\Code{C}};
    \node[Sommet](D)at(3,-2){\Code{D}};
    \draw[Fleche](A)edge[bend left=20]node{}(B);
    \draw[Fleche](A)edge[bend right=20]node{}(D);
    \draw[Fleche](B)edge[bend left=20]node{}(D);
    \draw[Fleche](C)edge[bend right=20]node{}(D);
\end{tikzpicture}}
\end{center}
signifie que
\begin{itemize}
    \item \Code{A.h} inclut \Code{B.h} et \Code{D.h}~;
    \item \Code{B.h} inclut \Code{D.h}~;
    \item \Code{C.h} inclut \Code{D.h}~;
    \item \Code{D.h} n'inclut rien.
\end{itemize}
\end{multicols}
\medskip

On ne mentionne pas dans les graphes d'inclusions les inclusions aux
fichiers d'en-tête standards (\Code{stdio.h}, \Code{stdlib.h},
\Code{assert.h}, {\em etc.}).
\end{frame}

%%%%%%%%%%%%%%%%%%%%%%%%%%%%%%%%%%%%%%%%%%%%%%%%%%%%%%%%%%%%%%%%%%%%%%%%
\begin{frame}[fragile]
\frametitle{Graphe d'inclusions et fichier principal}
Le graphe d'inclusions d'un projet consistant à faire jouer l'ordinateur
aux échecs contre un humain peut être le suivant~:
\begin{center}
\scalebox{.75}{\begin{tikzpicture}
    \node[Sommet,minimum size=2cm](Piece)at(-4,0){\Code{Piece}};
    \node[Sommet,minimum size=2cm](Case)at(-4,-4){\Code{Case}};
    \node[Sommet,minimum size=2cm](Position)at(0,-2){\Code{Position}};
    \node[Sommet,minimum size=2cm](IA)at(4,0){\Code{IA}};
    \node[Sommet,minimum size=2cm](IGraph)at(4,-4){\Code{IGraph}};
    \node[Sommet,minimum size=2cm,draw=Vert,fill=Vert!20](Main)at(7,-2){\Code{Main}};
    \draw[Fleche](Main)edge[bend right=20]node{}(IA);
    \draw[Fleche](Main)edge[bend left=20]node{}(IGraph);
    \draw[Fleche](IGraph)edge[bend left=20]node{}(Case);
    \draw[Fleche](IGraph)edge[bend left=20]node{}(Position);
    \draw[Fleche](IGraph)edge[bend right=40]node{}(Piece);
    \draw[Fleche](IA)edge[bend left=40]node{}(Case);
    \draw[Fleche](IA)edge[bend right=20]node{}(Position);
    \draw[Fleche](IA)edge[bend right=20]node{}(Piece);
    \draw[Fleche](Position)edge[bend right=20]node{}(Case);
    \draw[Fleche](Position)edge[bend left=20]node{}(Piece);
\end{tikzpicture}}
\end{center}
\medskip

Tout projet contient un fichier source \Code{Main.c},
\alert{le fichier principal} du projet, où figure la fonction \Code{main}
(le {\bf point d'entrée} de l'exécution du programme). Celui-ci apparaît
dans le graphe d'inclusions.
\end{frame}

%%%%%%%%%%%%%%%%%%%%%%%%%%%%%%%%%%%%%%%%%%%%%%%%%%%%%%%%%%%%%%%%%%%%%%%%
\begin{frame}[fragile]
\frametitle{Inclusions circulaires}
Les graphes d'inclusions permettent d'avoir une vision globale de
l'architecture d'un projet.
\bigskip

Ils permettent aussi de mettre en évidence des problèmes de conception
et notamment les problèmes d'\alert{inclusion circulaire}. Ce type de
problème s'observe par la présence d'un {\bf cycle} dans le graphe
d'inclusions~:
\begin{center}
\scalebox{.8}{\begin{tikzpicture}
    \node[Sommet](A)at(0,0){\Code{A}};
    \node[Sommet](B)at(2,0){\Code{B}};
    \node[Sommet](C)at(0,-2){\Code{C}};
    \node[Sommet](D)at(2,-2){\Code{D}};
    \node[Sommet](E)at(4,-1){\Code{E}};
    \draw[Fleche,draw=Violet,line width=3pt](A)edge[bend left=20]node{}(B);
    \draw[Fleche,draw=Violet,line width=3pt](B)edge[bend left=20]node{}(D);
    \draw[Fleche,draw=Violet,line width=3pt](D)edge[bend left=20]node{}(C);
    \draw[Fleche,draw=Violet,line width=3pt](C)edge[bend left=20]node{}(A);
    \draw[Fleche](E)edge[bend right=20]node{}(B);
    \draw[Fleche](E)edge[bend left=20]node{}(D);
    \draw[Fleche](B)edge[bend left=20]node{}(C);
\end{tikzpicture}}
\end{center}
\medskip

{\bf Règle importante}~: il ne doit jamais y avoir de cycle dans le
graphe d'inclusions d'un projet. S'il y a un cycle, c'est que le projet
est mal découpé en modules.
\end{frame}

%%%%%%%%%%%%%%%%%%%%%%%%%%%%%%%%%%%%%%%%%%%%%%%%%%%%%%%%%%%%%%%%%%%%%%%%
\begin{frame}[fragile]
\frametitle{Limiter les inclusions}
La plupart des inclusions circulaires peuvent être évitées en
\alert{réduisant au maximum les inclusions} de modules dans les
\alert{fichiers d'en-tête} en les faisant plutôt si possible dans les
fichiers source.
\bigskip
\bigskip

Par exemple, supposons que l'on dispose d'un module \Code{Tri} qui
permet de trier des tableaux génériques (nous aborderons plus loin ce
concept de généricité). Il est de la forme~:
\medskip

\begin{minipage}[c]{.45\textwidth}
\begin{lstlisting}[frame=single,numbers=none,basicstyle=\ttfamily\scriptsize]
/* Tri.h */
#ifndef __TRI__
#define __TRI__

    void trier_tab(void **t, int n,
        int (*est_inf)(void *, void *));
    ...
#endif
\end{lstlisting}
\end{minipage}
\quad
\begin{minipage}[c]{.45\textwidth}
\begin{lstlisting}[frame=single,numbers=none,basicstyle=\ttfamily\scriptsize]
/* Tri.c */
#include "Tri.h"
...
void trier_tab(void **t, int n,
    int (*est_inf)(void *, void *)) {
    ...
}
...
\end{lstlisting}
\end{minipage}
\end{frame}

%%%%%%%%%%%%%%%%%%%%%%%%%%%%%%%%%%%%%%%%%%%%%%%%%%%%%%%%%%%%%%%%%%%%%%%%
\begin{frame}[fragile]
\frametitle{Limiter les inclusions}
On souhaite maintenant écrire un module \Code{TabInt} pour gérer des
tableaux d'entiers.
\medskip

On n'écrira pas

\begin{minipage}[c]{.55\textwidth}
\begin{lstlisting}[frame=single,numbers=none,basicstyle=\ttfamily\scriptsize]
/* TabInt.h */
#ifndef __TAB_INT__
#define __TAB_INT__

#include "Tri.h"

    typedef struct {int n; int *tab;} TabInt;
    int trier_tab_int(TabInt *t);
#endif
\end{lstlisting}
\end{minipage}
\quad
\begin{minipage}[c]{.35\textwidth}
\begin{lstlisting}[frame=single,numbers=none,basicstyle=\ttfamily\scriptsize]
/* TabInt.c */
#include "TabInt.h"

int trier_tab_int(TabInt *t) {
    /* Util. de `trier_tab` */
}
\end{lstlisting}
\end{minipage}

mais plutôt

\begin{minipage}[c]{.55\textwidth}
\begin{lstlisting}[frame=single,numbers=none,basicstyle=\ttfamily\scriptsize]
/* TabInt.h */
#ifndef __TAB_INT__
#define __TAB_INT__

    typedef struct {int n; int *tab;} TabInt;
    int trier_tab_int(TabInt *t);
#endif
\end{lstlisting}
\end{minipage}
\quad
\begin{minipage}[c]{.35\textwidth}
\begin{lstlisting}[frame=single,numbers=none,basicstyle=\ttfamily\scriptsize]
/* TabInt.c */
#include "TabInt.h"

#include "Tri.h"

int trier_tab_int(TabInt *t) {
    /* Util. de `trier_tab` */
}
\end{lstlisting}
\end{minipage}
\end{frame}
