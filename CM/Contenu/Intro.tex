% Auteur : Samuele Giraudo
% Création : déc. 2015
% Modifications : déc. 2015

%%%%%%%%%%%%%%%%%%%%%%%%%%%%%%%%%%%%%%%%%%%%%%%%%%%%%%%%%%%%%%%%%%%%%%%%
%%%%%%%%%%%%%%%%%%%%%%%%%%%%%%%%%%%%%%%%%%%%%%%%%%%%%%%%%%%%%%%%%%%%%%%%
%%%%%%%%%%%%%%%%%%%%%%%%%%%%%%%%%%%%%%%%%%%%%%%%%%%%%%%%%%%%%%%%%%%%%%%%
\section*{Introduction}

%%%%%%%%%%%%%%%%%%%%%%%%%%%%%%%%%%%%%%%%%%%%%%%%%%%%%%%%%%%%%%%%%%%%%%%%
\begin{frame} \frametitle{Introduction}
L'objectif de ce module est d'approfondir les concepts de base et 
d'approcher certains concepts plus avancés de la programmation en {\sf C}.
\bigskip

Celui-ci est organisé en trois axes.
\medskip

\begin{enumerate}
    \item {\bf Axe~1}~: \alert{écrire un projet maintenable}.
    \medskip
    
    Bonnes habitudes de programmation, documentation, pré-assertion,
    gestion des erreurs, programmation modulaire, compilation.
    \bigskip
    
    \item {\bf Axe~2}~: \alert{comprendre les mécanismes de base}.
    \medskip
    
    Pointeurs, allocation dynamique, types, types structurés, entrées/sorties.
    \bigskip
    
    \item {\bf Axe~3}~: \alert{utiliser quelques techniques avancées}.
    \medskip
    
    Opérateurs bit à bit, mémoïsation, génération aléatoire, pointeurs
    de fonction, généricité.
\end{enumerate}
\end{frame}

%%%%%%%%%%%%%%%%%%%%%%%%%%%%%%%%%%%%%%%%%%%%%%%%%%%%%%%%%%%%%%%%%%%%%%%%
\begin{frame} \frametitle{Contenu du cours}
\begin{multicols}{3}
    \begin{footnotesize}
        {\large \bf Axe~1.}
        \tableofcontents[hideallsubsections,part=1]
        \bigskip

        {\large \bf Axe~2.}
        \tableofcontents[hideallsubsections,part=2]
        \bigskip
        
        {\large \bf Axe~3.}
        \tableofcontents[hideallsubsections,part=3]
    \end{footnotesize}
\end{multicols}
\end{frame}

%%%%%%%%%%%%%%%%%%%%%%%%%%%%%%%%%%%%%%%%%%%%%%%%%%%%%%%%%%%%%%%%%%%%%%%%
\begin{frame} \frametitle{Pré-requis}
Ce cours demande les pré-requis suivants~:
\smallskip

\begin{itemize}
    \item des bases en {\bf programmation générale} (notion de programme, 
    d'instruction, de compilation)~;
    \medskip

    \item des bases en {\bf programmation impérative} (notion de variable,
    de structures de contrôle, de fonction)~;
    \medskip
    
    \item des bases en {\bf algorithmique} (manipulation de structures de 
    donnée élémentaires comme les tableaux, les chaînes de caractères,
    les listes)~;
    \medskip
    
    \item des bases en {\bf programmation en {\sf C}} (écriture de 
    fonctions, manipulation de tableaux, de chaînes de caractères, 
    connaissance des types numériques, des types structurés, 
    notion d'allocation dynamique, de pointeur).
\end{itemize}
\end{frame}

%%%%%%%%%%%%%%%%%%%%%%%%%%%%%%%%%%%%%%%%%%%%%%%%%%%%%%%%%%%%%%%%%%%%%%%%
\begin{frame} \frametitle{Quelques dates}
Le langage {\sf C} est apparu en 1972 dans les laboratoires Bell.
Il a été créé par \alert{D. Ritchie} et \alert{K. Thompson}
au même moment que l'apparition des systèmes {\sf UNIX}.
\medskip

\uncover<2->{
Il est influencé par le langage {\sf B} qui, contrairement au {\sf C},
ne possédait pas de système de type.
\medskip}

\uncover<3->{
Quelques dates sur l'évolution de la spécification du langage~:

\begin{itemize}
    \item 1978~: première description complète du langage ({\sf C} traditionnel)~;
    \smallskip

    \item 1989~:%
    {\only<4->{\bf} publication de la norme ANSI {\sf C} (ou {\sf C}89)}~;

    \smallskip

    \item 1995~: évolution du langage {\sf C}94/95~;
    \smallskip

    \item 1999~: évolution du langage {\sf C}99~;
    \smallskip

    \item 2011~: nouvelle version du standard {\sf C}11.
\end{itemize}}
\end{frame}

%%%%%%%%%%%%%%%%%%%%%%%%%%%%%%%%%%%%%%%%%%%%%%%%%%%%%%%%%%%%%%%%%%%%%%%%
\begin{frame} \frametitle{Quelques caractéristiques}
Le {\sf C} est un \alert{langage de bas niveau}~: il offre la possibilité
de se placer \og proche \fg{} de la machine. Il permet en effet de manipuler
finement des données en mémoire, des adresses et des suites de bits.
\bigskip

\uncover<2->{
Ce langage a été initialement pensé pour la
\alert{conception de systèmes d'exploitation}.
\medskip

Il est cependant suffisamment expressif pour s'adapter à un large
éventail d'utilisations.
\bigskip}

\uncover<3->{
Il se situe à la croisée des chemins du monde des langages de programmation~:
beaucoup de langages modernes sont traduits en {\sf C} pour être compilés.
\bigskip}

\uncover<4->{
L'un de ses compilateurs, {\tt gcc}, fruit de nombreuses optimisations,
fait que le {\sf C} est un langage d'une \alert{efficacité extrême}.}
\end{frame}

