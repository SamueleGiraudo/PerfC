% Auteur : Samuele Giraudo
% Création : déc. 2013
% Modifications : août 2014, déc. 2015

%%%%%%%%%%%%%%%%%%%%%%%%%%%%%%%%%%%%%%%%%%%%%%%%%%%%%%%%%%%%%%%%%%%%%%%%
%%%%%%%%%%%%%%%%%%%%%%%%%%%%%%%%%%%%%%%%%%%%%%%%%%%%%%%%%%%%%%%%%%%%%%%%
%%%%%%%%%%%%%%%%%%%%%%%%%%%%%%%%%%%%%%%%%%%%%%%%%%%%%%%%%%%%%%%%%%%%%%%%
\section{Pointeurs de fonction}

%%%%%%%%%%%%%%%%%%%%%%%%%%%%%%%%%%%%%%%%%%%%%%%%%%%%%%%%%%%%%%%%%%%%%%%%
%%%%%%%%%%%%%%%%%%%%%%%%%%%%%%%%%%%%%%%%%%%%%%%%%%%%%%%%%%%%%%%%%%%%%%%%
\subsection{Principe}

%%%%%%%%%%%%%%%%%%%%%%%%%%%%%%%%%%%%%%%%%%%%%%%%%%%%%%%%%%%%%%%%%%%%%%%%
\begin{frame}[fragile]
\frametitle{Pointeurs de fonction}
En {\sf C}, on peut manipuler divers types d'objets~: des variables
d'un type scalaire, des tableaux, des variables d'un type structuré,
des adresses, {\em etc.}
\bigskip

\uncover<2->{
À l'inverse, les {\bf fonctions} n'entrent pas dans cette catégorie
d'objets directement manipulables.
\bigskip}

\uncover<3->{
Cependant, au même titre qu'une variable, toute fonction possède une
{\bf adresse} en mémoire. Il devient alors possible de réaliser des
opérations sur les fonctions au moyen de leur adresse.
\bigskip}

\uncover<4->{
On parle alors de \alert{pointeur de fonction}.}
\end{frame}

%%%%%%%%%%%%%%%%%%%%%%%%%%%%%%%%%%%%%%%%%%%%%%%%%%%%%%%%%%%%%%%%%%%%%%%%
\begin{frame}[fragile]
\frametitle{Adresse d'une fonction}
Si \Code{fct} est une fonction, la syntaxe
\begin{center} \Code{\&fct} \end{center}
permet d'accéder à l'\alert{adresse} de \Code{fct}.
\bigskip

\begin{multicols}{2}
\begin{semiverbatim}\small\uncover<2->{
#include <stdio.h>
int somme(int a, int b) \{
    return a + b;
\}
int produit(int a, int b) \{
    return a * b;
\}
int main() \{
    printf("%p\\n", &somme);
    printf("%p\\n", &produit);
    return 0;
\}}
\end{semiverbatim}
\uncover<2->{
Ce programme affiche \Sortie{0x40052d} et \Sortie{0x400541},
respectivement les adresses des fonctions \Code{somme} et \Code{produit}.
\bigskip}

\uncover<3->{
{\bf Note}~: aux lignes 9 et 10, il est possible de ne pas mentionner
les \Code{\&}. Le compilateur comprend implicitement qu'il s'agit de
pointeurs de fonction.}
\end{multicols}
\end{frame}

%%%%%%%%%%%%%%%%%%%%%%%%%%%%%%%%%%%%%%%%%%%%%%%%%%%%%%%%%%%%%%%%%%%%%%%%
\begin{frame}[fragile]
\frametitle{Le type pointeur de fonction}
La syntaxe \vspace{-.5em}
\begin{center} \Code{T (*FCT)(T1, \dots, TN);} \end{center} \vspace{-.5em}
où

\begin{itemize}
    \item \Code{T}, \Code{T1}, \dots, \Code{TN} sont des types~;
    \smallskip

    \item \Code{FCT} est un identificateur~;
\end{itemize}
permet de \alert{déclarer} un pointeur de fonction.
\smallskip

\uncover<2->{
Celui-ci a \Code{FCT} pour identificateur et peut être l'adresse d'une
fonction de type de retour \Code{T} et de signature \Code{(T1, \dots, TN)}.
\bigskip}

\begin{multicols}{2}
\begin{semiverbatim}\footnotesize\uncover<3->{
float moyenne(int a, int b) \{
    return (0.0 + a + b) / 2;
\}
...
/* Decl. d'un ptr de fonction */
float (*moy)(int, int);

/* Utilisation */
moy = &moyenne;
printf("%f\\n", moy(2, 3));}
\end{semiverbatim}
\uncover<3->{
Pour la même raison que dans l'exemple précédent, il est possible
à la ligne 9 de ne pas mentionner le~\Code{\&}.
\bigskip

Cependant, pour la clarté du code, nous prenons la {\bf convention de
mentionner tous les  \Code{\&}}}.
\end{multicols}
\end{frame}

%%%%%%%%%%%%%%%%%%%%%%%%%%%%%%%%%%%%%%%%%%%%%%%%%%%%%%%%%%%%%%%%%%%%%%%%
\begin{frame}[fragile]
\frametitle{Champs pointeurs de fonction}
Un \alert{champ d'un type structuré} peut être un pointeur sur une fonction.
\medskip

\uncover<2->{
Ceci déclare un type structuré sensé modéliser des suites d'entiers~:}
\begin{semiverbatim}\small\uncover<2->{
typedef struct \{
    int t;
    int (*t_suiv)(int);
\} Suite;}
\end{semiverbatim}
\uncover<2->{
\Code{t} contient le terme courant de la suite et \Code{t\_suiv}
est la fonction qui, étant donné un terme en entrée, calcule le terme
suivant.}

\begin{multicols}{2}
\begin{semiverbatim}\small\uncover<3->{
int mul_2(int t) \{
    return 2 * t;
\}
...
int i;
Suite s;
s.t = 1;
s.t_suiv = &mul_2;
for (i = 0 ; i < 6 ; ++i) \{
    printf("%d ", s.t);
    s.t = s.t_suiv(s.t);
\}}
\end{semiverbatim}
\end{multicols}
\uncover<3->{Ceci affiche \Sortie{1 2 4 8 16 32}.}
\end{frame}

%%%%%%%%%%%%%%%%%%%%%%%%%%%%%%%%%%%%%%%%%%%%%%%%%%%%%%%%%%%%%%%%%%%%%%%%
\begin{frame}[fragile]
\frametitle{Tableaux de pointeurs de fonction}
Il est possible de manipuler des \alert{tableaux de pointeurs de fonction}.
\medskip

\uncover<2->{
Pour cela, on procède en deux étapes~:
\begin{enumerate}
    \item on définit un type alias pour le pointeur de fonction. Syntaxe~:
    \begin{center}
        \Code{typedef T (*FCT)(T1, \dots, TN);}
    \end{center}
    \smallskip}

    \uncover<3->{
    \item on déclare ensuite le tableau de manière usuelle. Syntaxe~:
    \begin{center}
        \Code{FCT tab[M];}
    \end{center}}
\end{enumerate}
\bigskip

\uncover<4->{
La syntaxe plus directe
\begin{center}
    \Code{T (*tab[M])(T1, \dots, TN);}
\end{center}
existe mais rend le code plus difficile à lire. Elle déclare un
tableau \Code{tab} de pointeurs de fonction sans la déclaration de type
préalable de la $1\iere$ méthode.}
\end{frame}

%%%%%%%%%%%%%%%%%%%%%%%%%%%%%%%%%%%%%%%%%%%%%%%%%%%%%%%%%%%%%%%%%%%%%%%%
\begin{frame}[fragile]
\frametitle{Tableaux de pointeurs de fonction}

\begin{multicols}{2}
\begin{semiverbatim}\small
/* Alias pour ptr. de fct. */
typedef int (*opb)(int, int);

int add(int a, int b) \{
    return a + b;
\}
int mul(int a, int b) \{
    return a * b;
\}
...
opb tab[2];

tab[0] = &add;
tab[1] = &mul;
printf("%d %d\\n",
    tab[0](10, 20),
    tab[1](10, 20));
\end{semiverbatim}
\end{multicols}
Ceci affiche \Sortie{30 200}.
\bigskip

\uncover<2->{
Les tableaux de pointeurs de fonction peuvent être {\bf dynamiques}. La
ligne 10 peut être remplacée par}
\begin{semiverbatim}\small\uncover<2->{
opb *tab;
tab = (opb *) malloc(sizeof(opb) * 2);}
\end{semiverbatim}
\end{frame}

%%%%%%%%%%%%%%%%%%%%%%%%%%%%%%%%%%%%%%%%%%%%%%%%%%%%%%%%%%%%%%%%%%%%%%%%
%%%%%%%%%%%%%%%%%%%%%%%%%%%%%%%%%%%%%%%%%%%%%%%%%%%%%%%%%%%%%%%%%%%%%%%%
\subsection{En paramètre et en retour}

%%%%%%%%%%%%%%%%%%%%%%%%%%%%%%%%%%%%%%%%%%%%%%%%%%%%%%%%%%%%%%%%%%%%%%%%
\begin{frame}[fragile]
\frametitle{Pointeur de fonction en paramètre}
Une fonction peut être \alert{paramétrée par un pointeur de fonction}.
Un paramètre pointeur de fonction est spécifié avec la même syntaxe que
celle qui sert à le déclarer.
\bigskip

\uncover<2->{P.ex.,}
\begin{semiverbatim}\small\uncover<2->{
int appliquer(int n, int k, int (*f)(int)) \{
    int i;
    for (i = 0 ; i < k ; ++i)
        n = f(n);
    return n;
\}}
\end{semiverbatim}
\uncover<2->{
est une fonction est paramétrée par un pointeur de fonction acceptant un
entier et renvoyant un entier.}
\end{frame}

%%%%%%%%%%%%%%%%%%%%%%%%%%%%%%%%%%%%%%%%%%%%%%%%%%%%%%%%%%%%%%%%%%%%%%%%
\begin{frame}[fragile]
\frametitle{Pointeur de fonction en paramètre}
\begin{multicols}{2}
\begin{semiverbatim}\small
int add_1(int n) \{
    return n + 1;
\}

int mul_2(int n) \{
    return 2 * n;
\}

...
printf("%d\\n",
    appliquer(3, 4, &add_1));

printf("%d\\n",
    appliquer(3, 4, &mul_2));
\end{semiverbatim}
\end{multicols}

\uncover<2->{
Le 1\ier{} appel à \Code{appliquer} calcule
\begin{center}
    \Code{(((3 + 1) + 1) + 1) + 1}
\end{center}
et affiche donc \Sortie{7}.
\medskip}

\uncover<3->{
Le 2\ieme{} appel à \Code{appliquer} calcule
\begin{center}
    \Code{(((3 * 2) * 2) * 2) * 2}
\end{center}
et affiche donc \Sortie{48}.}
\end{frame}

%%%%%%%%%%%%%%%%%%%%%%%%%%%%%%%%%%%%%%%%%%%%%%%%%%%%%%%%%%%%%%%%%%%%%%%%
\begin{frame}[fragile]
\frametitle{Renvoi d'un pointeur de fonction}
Il est possible de définir des fonctions dont le \alert{type de retour}
est un \alert{pointeur de fonction}.
\medskip

\uncover<2->{
Pour cela, on procède en deux étapes~:
\begin{enumerate}
    \item on définit un type alias \Code{R} pour le pointeur de fonction
    que l'on souhaite renvoyer~;
    \smallskip}

    \uncover<3->{
    \item on définit la fonction souhaitée, dont le type de retour
    est \Code{R}.}
\end{enumerate}
\bigskip

\uncover<4->{
La syntaxe plus directe
\smallskip

\Code{R (*FCT(T1 ARG1, \dots, TN ARGN))(R1, \dots, RM) $\lbrace$ \\
    \qquad ... \\
$\rbrace$}
\smallskip

permet de définir directement une fonction \Code{FCT} de signature \\
\Code{(T1, \dots, TN)} renvoyant l'adresse d'une fonction de type de
retour \Code{R} et de signature \Code{(R1, \dots, RM)}. Cependant,
le code devient illisible.}
\end{frame}

%%%%%%%%%%%%%%%%%%%%%%%%%%%%%%%%%%%%%%%%%%%%%%%%%%%%%%%%%%%%%%%%%%%%%%%%
\begin{frame}[fragile]
\frametitle{Renvoi d'un pointeur de fonction}
Exemple~: opération aléatoire sur des entiers.

\begin{multicols}{2}
\begin{semiverbatim}\small
/* Definition des operations */
int add(int a, int b) \{
    return a + b;
\}
int sub(int a, int b) \{
    return a - b;
\}
int mod(int a, int b) \{
    return a % b;
\}
/* Type de retour */
typedef int (*opb)(int, int);

opb op_alea() \{
    opb tab[3];
    tab[0] = &add;
    tab[1] = &sub;
    tab[2] = &mod;
    return tab[rand() % 3];
\}
\end{semiverbatim}
\end{multicols}

\uncover<2->{
On peut utiliser \Code{op\_alea} de la manière suivante~:}
\begin{semiverbatim}\small\uncover<2->{
int n;
n = op_alea()(3, 4);}
\end{semiverbatim}
\uncover<2->{
Ceci affecte, de manière aléatoire, \Code{7}, \Code{-1} ou \Code{3}
à \Code{n}.}
\end{frame}

%%%%%%%%%%%%%%%%%%%%%%%%%%%%%%%%%%%%%%%%%%%%%%%%%%%%%%%%%%%%%%%%%%%%%%%%
%%%%%%%%%%%%%%%%%%%%%%%%%%%%%%%%%%%%%%%%%%%%%%%%%%%%%%%%%%%%%%%%%%%%%%%%
\subsection{Généricité}

%%%%%%%%%%%%%%%%%%%%%%%%%%%%%%%%%%%%%%%%%%%%%%%%%%%%%%%%%%%%%%%%%%%%%%%%
\begin{frame}[fragile]
\frametitle{Principe de généricité}
Une {\bf fonction} est dite \alert{générique} si elle peut accepter des
arguments qui ne sont pas seulement ceux d'un type bien précis.
\medskip

\uncover<2->{
Exemples~:
\begin{itemize}
    \item une fonction qui teste si deux valeurs sont égales~;
    \smallskip
    }
    \uncover<3->{
    \item une fonction qui affiche plusieurs fois une même valeur.}
\end{itemize}
\bigskip
\bigskip

\uncover<4->{
Une {\bf structure de donnée} est dite \alert{générique} si elle peut
représenter des données dont le type n'est pas fixé.
\medskip}

\uncover<5->{
Exemples~:
\begin{itemize}
    \item une liste dont les éléments sont d'un type non fixé~;
    \smallskip
    }
    \uncover<6->{
    \item un arbre binaire dont les éléments sont d'un type non fixé.}
\end{itemize}
\end{frame}

%%%%%%%%%%%%%%%%%%%%%%%%%%%%%%%%%%%%%%%%%%%%%%%%%%%%%%%%%%%%%%%%%%%%%%%%
\begin{frame}[fragile]
\frametitle{Le type {\tt void *}}
Pour manipuler une donnée dont le type n'est pas spécifié à l'avance, on
utilise son {\bf adresse}.
\smallskip

\uncover<2->{
Il s'agit donc d'une adresse dont on ne connaît pas le type~: c'est une
adresse de type \Code{void *}.
\bigskip}

\uncover<3->{
Le type \Code{void *} est appelé \alert{type pointeur générique}.}
\bigskip
\bigskip

\uncover<4->{
Pour convertir un pointeur générique \Code{ptr\_g} vers un pointeur d'un
type connu \Code{T}, on utilise l'{\bf opérateur de coercition}
\begin{center}
    \Code{(T *) ptr\_g}.
\end{center}
\smallskip}

\uncover<5->{
Avant de pouvoir interpréter (c.-à-d. déréférencer) la valeur située à
une adresse spécifiée par un pointeur générique, le convertir est
primordial.}
\end{frame}

%%%%%%%%%%%%%%%%%%%%%%%%%%%%%%%%%%%%%%%%%%%%%%%%%%%%%%%%%%%%%%%%%%%%%%%%
\begin{frame}[fragile]
\frametitle{Fonction générique de comparaison}
\begin{semiverbatim}\small
int ega(int nbo, void *x, void *y) \{
    \uncover<2->{char *xc, *yc;
    int i;}

    \uncover<3->{xc = (char *) x;
    yc = (char *) y;}
    \uncover<4->{for (i = 0 ; i < nbo ; ++i)}
        \uncover<5->{if (xc[i] != yc[i])
            return 0;}
    \uncover<6->{return 1;}
\}
\end{semiverbatim}
La fonction \Code{ega} est générique~: elle permet de tester l'égalité
entre deux variables dont le {\bf type n'est pas connu lors de l'écriture
de la fonction}.
\smallskip

\uncover<7->{
On l'utilise de la manière suivante~:}
\begin{semiverbatim}\small\uncover<7->{
ega(sizeof(T), &t1, &t2)}
\end{semiverbatim}
\uncover<7->{
pour comparer deux variables \Code{t1} et \Code{t2} de type \Code{T}.}
\end{frame}

%%%%%%%%%%%%%%%%%%%%%%%%%%%%%%%%%%%%%%%%%%%%%%%%%%%%%%%%%%%%%%%%%%%%%%%%
\begin{frame}[fragile]
\frametitle{Fonction générique d'affichage de tableau}
\begin{semiverbatim}\small
void aff_tab(void **tab, int n, void (*aff_elt)(void *)) \{
    \uncover<2->{int i;}

    \uncover<3->{for (i = 0 ; i < n ; ++i) \{}
        \uncover<4->{aff_elt(tab[i]);}
        \uncover<5->{printf(" ");}
    \uncover<3->{\}}
\}
\end{semiverbatim}
\medskip

La fonction \Code{aff\_tab} est générique~: elle permet d'afficher les
éléments d'un tableau dont le {\bf type n'est pas connu lors de l'écriture
de la fonction}.
\end{frame}

%%%%%%%%%%%%%%%%%%%%%%%%%%%%%%%%%%%%%%%%%%%%%%%%%%%%%%%%%%%%%%%%%%%%%%%%
\begin{frame}[fragile]
\frametitle{Fonction générique d'affichage de tableau}
On l'utilise la fonction \Code{aff\_tab} de la manière suivante.
\bigskip

Pour afficher un tableau \Code{tab} de taille \Code{13} de pointeurs
sur des {\bf entiers}~:

\begin{semiverbatim} \small
\uncover<2->{void aff_int(void *x) \{
    \uncover<3->{int e;}
    \uncover<4->{e = *((int *) x);}
    \uncover<5->{printf("%d", e);}
\}}

\uncover<6->{/* Version raccourcie. */
void aff_int(void *x) \{
    printf("%d", *((int *) x));
\}}
\uncover<7->{...
aff_tab((void **) tab, 13, &aff_int);}
\end{semiverbatim}
\end{frame}

%%%%%%%%%%%%%%%%%%%%%%%%%%%%%%%%%%%%%%%%%%%%%%%%%%%%%%%%%%%%%%%%%%%%%%%%
\begin{frame}[fragile]
\frametitle{Fonction générique d'affichage de tableau}
Pour afficher un tableau \Code{tab} de taille \Code{23} de pointeurs
sur des variables de {\bf type structuré} \Code{Date}~:
\begin{semiverbatim}\small
typedef struct \{
    int jour;
    int mois;
    int annee;
\} Date;
...
void aff_date(void *d) \{
    Date dd;
    dd = *((Date *) d);
    printf("%d-%d-%d", dd.jour, dd.mois, dd.annee);
\}
...
aff_tab((void **) tab, 23, &aff_date);
\end{semiverbatim}
\end{frame}

%%%%%%%%%%%%%%%%%%%%%%%%%%%%%%%%%%%%%%%%%%%%%%%%%%%%%%%%%%%%%%%%%%%%%%%%
\begin{frame}[fragile]
\frametitle{Listes génériques}
On souhaite définir une structure de donnée liste dont les types des
éléments ne sont pas fixés.
\smallskip

\uncover<2->{Pour cela, on utilise un pointeur générique pour le champ
qui contient l'élément de chaque cellule~:}
\begin{semiverbatim}\small\uncover<2->{
typedef struct _Cellule \{
    struct _Cellule *suiv;
    void *e;
\} Cellule;

typedef Cellule *Liste;}
\end{semiverbatim}
\bigskip

\uncover<3->{
Le type \Code{Liste} permet ainsi de représenter des listes génériques.
\smallskip}

\uncover<4->{
C'est une structure de donnée générique car le {\bf type des éléments}
que les futures listes pourront contenir {\bf n'est pas connu lors
de l'écriture de la fonction}.}
\end{frame}

%%%%%%%%%%%%%%%%%%%%%%%%%%%%%%%%%%%%%%%%%%%%%%%%%%%%%%%%%%%%%%%%%%%%%%%%
\begin{frame}[fragile]
\frametitle{Listes génériques}
La fonction
\begin{semiverbatim}\small
void aff_lst(Liste lst, void (*aff_elt)(void *)) \{
    Cellule *x;

    assert(lst != NULL);
    assert(aff_elt != NULL);

    for (x = lst ; x != NULL ; x = x->suiv) \{
        aff_elt(x->e);
        printf(" ");
    \}
\}
\end{semiverbatim}
est une fonction générique pour l'affichage des éléments d'une liste
générique.
\end{frame}

%%%%%%%%%%%%%%%%%%%%%%%%%%%%%%%%%%%%%%%%%%%%%%%%%%%%%%%%%%%%%%%%%%%%%%%%
\begin{frame}[fragile]
\frametitle{Listes génériques}
On l'utilise de la manière suivante (dans le cas ici d'une liste d'entiers).
\medskip

\begin{semiverbatim}\small
void aff_int(void *e) \{
    printf("%d", *((int *) e));
\}
...
Liste lst;
int a, b, c;
a = 3; b = 14; c = 414;

lst = (Cellule *) malloc(sizeof(Cellule));
lst->e = &a;
lst->suiv = (Cellule *) malloc(sizeof(Cellule));
lst->suiv->e = &b;
lst->suiv->suiv = (Cellule *) malloc(sizeof(Cellule));
lst->suiv->suiv->e = &c;
lst->suiv->suiv->suiv = NULL;
aff_lst(lst, &aff_int);
\end{semiverbatim}
\end{frame}

%%%%%%%%%%%%%%%%%%%%%%%%%%%%%%%%%%%%%%%%%%%%%%%%%%%%%%%%%%%%%%%%%%%%%%%%
\begin{frame}[fragile]
\frametitle{Listes génériques}
Beaucoup de fonctions sur les listes peuvent ainsi être rendues génériques.
Entre autres~:
\smallskip

\begin{small}
\begin{enumerate}
    \uncover<2->{
    \item \Code{void aff\_lst(Liste lst, void (*aff\_elt)(void *));}
    \smallskip}

    \uncover<3->{
    \item \Code{void *elt\_indice(Liste lst, int i);}
    \smallskip}

    \uncover<4->{
    \item \label{item:lst_est_triee}
    \Code{int est\_triee(Liste lst, int (*est\_inf)(void *, void *));}
    \smallskip}

    \uncover<5->{
    \item \label{item:lst_max}
    \Code{void *max(Liste lst, int (*est\_inf)(void *, void *));}}
\end{enumerate}
\end{small}
\medskip

\uncover<6->{
Les cas \ref{item:lst_est_triee} et \ref{item:lst_max} supposent que
les éléments représentés par les listes sont comparables au moyen
d'une fonction \Code{est\_inf} à fournir.}
\end{frame}

%%%%%%%%%%%%%%%%%%%%%%%%%%%%%%%%%%%%%%%%%%%%%%%%%%%%%%%%%%%%%%%%%%%%%%%%
%%%%%%%%%%%%%%%%%%%%%%%%%%%%%%%%%%%%%%%%%%%%%%%%%%%%%%%%%%%%%%%%%%%%%%%%
\subsection{Implantation de monoïdes}

%%%%%%%%%%%%%%%%%%%%%%%%%%%%%%%%%%%%%%%%%%%%%%%%%%%%%%%%%%%%%%%%%%%%%%%%
\begin{frame}[fragile]
\frametitle{Monoïdes}
En maths, un \alert{monoïde} est un triplet
$(\mathcal{M}, \bullet, \mathbf{1})$ où
\begin{enumerate}
    \uncover<2->{
    \item $\mathcal{M}$ est un ensemble~;
    \smallskip}

    \uncover<3->{
    \item $\bullet$ est une application
    $\bullet : \mathcal{M} \times \mathcal{M} \to \mathcal{M}$
    associative, c.-à-d., pour tous $x, y, z \in \mathcal{M}$, on a
    $(x \bullet y) \bullet z = x \bullet (y \bullet z)$~;
    \smallskip}

    \uncover<4->{
    \item $\mathbf{1} \in \mathcal{M}$ est un élément unitaire,
    c.-à-d., pour tout $x \in \mathcal{M}$, on a
    $x \bullet \mathbf{1} = x = \mathbf{1} \bullet x$.}
\end{enumerate}
\bigskip

\uncover<5->{
Exemples~:
\begin{itemize}
    \item $(\mathbb{N}, +, 0)$ est le {\bf monoïde additif} des entiers
    naturels~;
    \smallskip}

    \uncover<6->{
    \item $(\mathbb{N}, \times, 1)$ est le {\bf monoïde multiplicatif}
    des entiers naturels~;
    \smallskip}

    \uncover<7->{
    \item $(\{{\tt a}, {\tt b}\}^*, \cdot, \epsilon)$ est le {\bf monoïde
    libre} sur les lettres ${\tt a}$ et ${\tt b}$. Ses éléments sont
    les chaînes de caractères composées de ${\tt a}$ et de ${\tt b}$.
    Son produit $\cdot$ est la concaténation et son unité $\epsilon$
    est la chaîne vide.}
\end{itemize}
\end{frame}

%%%%%%%%%%%%%%%%%%%%%%%%%%%%%%%%%%%%%%%%%%%%%%%%%%%%%%%%%%%%%%%%%%%%%%%%
\begin{frame}[fragile]
\frametitle{Implantation de monoïdes}
On se donne les deux objectifs suivants~:
\medskip

\begin{enumerate}
    \uncover<2->{
    \item implanter une {\bf structure de donnée générique} pour
    \alert{représenter des monoïdes} dont la nature des éléments n'est
    pas connue à l'avance~;
    \bigskip}

    \uncover<3->{
    \item implanter une {\bf fonction générique} qui
    \alert{élève à la puissance} $n \geq 0$ un élément $x$ d'un monoïde.}
\end{enumerate}
\end{frame}

%%%%%%%%%%%%%%%%%%%%%%%%%%%%%%%%%%%%%%%%%%%%%%%%%%%%%%%%%%%%%%%%%%%%%%%%
\begin{frame}[fragile]
\frametitle{Structure de donnée générique de monoïde}
D'après la définition mathématique d'un monoïde, on aboutit à la définition
suivante~:
\begin{semiverbatim}
typedef struct \{
    void *unite;
    void *(*produit)(void *, void *);
\} Monoide;
\end{semiverbatim}
\medskip

\uncover<2->{
On utilise des pointeurs génériques \Code{void *} pour les éléments
du monoïde.
\medskip

Le champ \Code{unite} est un pointeur générique vers l'unité
du monoïde.
\medskip

Le champ \Code{produit} est un pointeur de fonction vers
une fonction qui réalise le produit du monoïde.}
\end{frame}

%%%%%%%%%%%%%%%%%%%%%%%%%%%%%%%%%%%%%%%%%%%%%%%%%%%%%%%%%%%%%%%%%%%%%%%%
\begin{frame}[fragile]
\frametitle{Définition du monoïde additif des entiers naturels}
On commence par définir une fonction qui réalise le produit de
$(\mathbb{N}, +, 0)$ et dont le type de retour et la signature respectent
ceux du champ \Code{produit}~:
\begin{semiverbatim}\small
void *add(void *a, void *b) \{
    \uncover<2->{int *res;}
    \uncover<3->{res = (int *) malloc(sizeof(int));}
    \uncover<4->{*res = *((int *) a) + *((int *) b);}
    \uncover<2->{return (void *) res;}
\}
\end{semiverbatim}
\medskip

\uncover<5->{
Ensuite, on créé le monoïde de la manière suivante~:}
\begin{semiverbatim}\small\uncover<5->{
Monoide m_add;
int zero = 0;
m_add.unite = &zero;
m_add.produit = &add;}
\end{semiverbatim}
\end{frame}

%%%%%%%%%%%%%%%%%%%%%%%%%%%%%%%%%%%%%%%%%%%%%%%%%%%%%%%%%%%%%%%%%%%%%%%%
\begin{frame}[fragile]
\frametitle{Définition du monoïde multiplicatif des entiers naturels}
On commence par définir une fonction qui réalise le produit de
$(\mathbb{N}, \times, 1)$ et dont le type de retour et la signature
respectent ceux du champ \Code{produit}~:
\begin{semiverbatim}\small
void *mul(void *a, void *b) \{
    \uncover<2->{int *res;}
    \uncover<3->{res = (int *) malloc(sizeof(int));}
    \uncover<4->{*res = *((int *) a) * *((int *) b);}
    \uncover<2->{return (void *) res;}
\}
\end{semiverbatim}
\medskip

\uncover<2->{
Ensuite, on créé le monoïde de la manière suivante~:}
\begin{semiverbatim}\small\uncover<2->{
Monoide m_mul;
int un = 1;
m_mul.unite = &un;
m_mul.produit = &mul;}
\end{semiverbatim}
\end{frame}

%%%%%%%%%%%%%%%%%%%%%%%%%%%%%%%%%%%%%%%%%%%%%%%%%%%%%%%%%%%%%%%%%%%%%%%%
\begin{frame}[fragile]
\frametitle{Définition du monoïde libre sur {\tt a} et {\tt b}}
On commence par définir une fonction qui réalise le produit de
$(\{{\tt a}, {\tt b}\}^*, \cdot, \epsilon)$ et dont le type de retour et
la signature respectent ceux du champ \Code{produit}~:
\begin{semiverbatim}\small
void *conc(void *u, void *v) \{
    \uncover<2->{char *res;}
    \uncover<3->{int i, lu, lv;}
    \uncover<4->{lu = strlen((char *) u);}
    \uncover<4->{lv = strlen((char *) v);}
    \uncover<5->{res = (char *) malloc(sizeof(char) * (1 + lu + lv));}
    \uncover<6->{for (i = 0 ; i < lu ; ++i) res[i] = ((char *) u)[i];}
    \uncover<6->{for (i = 0 ; i < lv ; ++i) res[lu + i] = ((char *) v)[i];}
    \uncover<7->{res[lu + lv] = '\\0';}
    \uncover<2->{return (void *) res;}
\}
\end{semiverbatim}
\medskip

\uncover<8->{
Ensuite, on créé le monoïde de la manière suivante~:}
\begin{multicols}{2}
\begin{semiverbatim}\small\uncover<8->{
Monoide m_lib;
char epsilon[1] = "";
m_lib.unite = &epsilon;
m_lib.produit = &conc;}
\end{semiverbatim}
\end{multicols}
\end{frame}

%%%%%%%%%%%%%%%%%%%%%%%%%%%%%%%%%%%%%%%%%%%%%%%%%%%%%%%%%%%%%%%%%%%%%%%%
\begin{frame}[fragile]
\frametitle{Fonction générique d'exponentiation}
\begin{semiverbatim}\small
void *puissance(Monoide m, void *x, int n) \{
    \uncover<2->{void *tmp;}

    \uncover<3->{if (n == 0)
        return m.unite;}
    \uncover<4->{tmp = puissance(m, x, n / 2);}
    \uncover<5->{if (n % 2 == 0)
        return m.produit(tmp, tmp);}
    \uncover<6->{return m.produit(x, m.produit(tmp, tmp));}
\}
\end{semiverbatim}

\uncover<7->{
Cette fonction renvoie un pointeur générique sur une variable contenant
le résultat de la valeur à l'adresse \Code{x} élevée à la puissance
\Code{n}, pour le produit spécifié par le monoïde \Code{m}.
\medskip}

\uncover<8->{
Celle fonction utilise l'algorithme dit d'{\bf exponentiation rapide}.}
\end{frame}

%%%%%%%%%%%%%%%%%%%%%%%%%%%%%%%%%%%%%%%%%%%%%%%%%%%%%%%%%%%%%%%%%%%%%%%%
\begin{frame}[fragile]
\frametitle{Application de la fonction d'exponentiation}
\begin{semiverbatim}\small
int res, val = 5;
res = *((int *) puissance(m_add, &val, 3));
printf("%d\\n", res);
\end{semiverbatim}
Ceci affiche $5^3$ dans le monoïde additif, c'est à dire \Sortie{15}.
\bigskip

\begin{semiverbatim}\small\uncover<2->{
int res, val = 5;
res = *((int *) puissance(m_mul, &val, 3));
printf("%d\\n", res);}
\end{semiverbatim}
\uncover<2->{
Ceci affiche $5^3$ dans le monoïde multiplicatif, c'est à dire \Sortie{125}.}
\bigskip

\begin{semiverbatim}\small\uncover<3->{
char *res, ch[3] = "ab";
res = (char *) puissance(m_lib, &ch, 4);
printf("%s\\n", res);}
\end{semiverbatim}
\uncover<3->{
Ceci affiche $({\tt ab})^4$ dans le monoïde libre, c'est à dire
\Sortie{abababab}.}
\end{frame}

%%%%%%%%%%%%%%%%%%%%%%%%%%%%%%%%%%%%%%%%%%%%%%%%%%%%%%%%%%%%%%%%%%%%%%%%
%\begin{frame}[fragile] \frametitle{Un problème et sa solution}
%Il y a un {\bf problème} majeur dans cette façon se représenter les
%monoïdes.
%\medskip
%
%Il consiste en ce que le produit du monoïde est de type \\
%\Code{void *(*produit)(void *, void *)} et renvoie une adresse.
%\smallskip
%
%Le problème tient au fait qu'une {\bf allocation dynamique} doit être
%faite, ce qui induit une {\bf fuite mémoire} (voir la fonction \Code{puissance}).
%\bigskip
%
%\end{frame}
