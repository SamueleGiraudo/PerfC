% Auteur : Samuele Giraudo
% Création : oct. 2013
% Modifications : aout 2014 oct. 2014, déc. 2015, mars 2016

%%%%%%%%%%%%%%%%%%%%%%%%%%%%%%%%%%%%%%%%%%%%%%%%%%%%%%%%%%%%%%%%%%%%%%%%
%%%%%%%%%%%%%%%%%%%%%%%%%%%%%%%%%%%%%%%%%%%%%%%%%%%%%%%%%%%%%%%%%%%%%%%%
%%%%%%%%%%%%%%%%%%%%%%%%%%%%%%%%%%%%%%%%%%%%%%%%%%%%%%%%%%%%%%%%%%%%%%%%
\section{Entrées et sorties}

%%%%%%%%%%%%%%%%%%%%%%%%%%%%%%%%%%%%%%%%%%%%%%%%%%%%%%%%%%%%%%%%%%%%%%%%
%%%%%%%%%%%%%%%%%%%%%%%%%%%%%%%%%%%%%%%%%%%%%%%%%%%%%%%%%%%%%%%%%%%%%%%%
\subsection{Sortie}

%%%%%%%%%%%%%%%%%%%%%%%%%%%%%%%%%%%%%%%%%%%%%%%%%%%%%%%%%%%%%%%%%%%%%%%%
\begin{frame}[fragile] \frametitle{Écriture formatée}
La fonction
\begin{center}
    \Code{int printf(char *format, V\_1, ..., V\_N);}
\end{center}
permet de réaliser une \alert{écriture formatée} sur la sortie standard
\Code{stdout}.
\medskip

\uncover<2->{
{\bf Rappel~:} cette fonction renvoie le nombre de caractères affichés.
Si une erreur a lieu, elle renvoie un entier négatif.
\medskip}

\begin{multicols}{2}
\begin{semiverbatim}\uncover<3->{
int num;
num = printf("%s+%d\\n",
    "test", 200);
printf("%d\\n", num);
\end{semiverbatim}}
\uncover<3->{
Ces instructions affichent \\
\Sortie{test+200 \\
9}}
\end{multicols}
\end{frame}

%%%%%%%%%%%%%%%%%%%%%%%%%%%%%%%%%%%%%%%%%%%%%%%%%%%%%%%%%%%%%%%%%%%%%%%%
\begin{frame} \frametitle{Indicateurs de conversion}
On utilise les \alert{indicateurs de conversion} suivants pour interpréter
chaque valeur à écrire (ou à lire, voir la partie suivante) de manière
adéquate~:
\begin{center}
    \begin{tabular}{c|c}
        Indicateur de conversion & Affichage \\ \hline
        \Code{d}, \Code{i} & Entier en base dix \\
        \Code{u} & Entier non signé en base dix \\
        \Code{x}, \Code{X} & Entier en hexadécimal \\
        \Code{c} & Caractère \\
        \Code{e}, \Code{E} & Flottant en notation scientifique \\
        \Code{f}, \Code{g} & Flottant \\
        \Code{s} & Chaîne de caractères \\
        \Code{p} & Pointeur
    \end{tabular}
\end{center}
\end{frame}

%%%%%%%%%%%%%%%%%%%%%%%%%%%%%%%%%%%%%%%%%%%%%%%%%%%%%%%%%%%%%%%%%%%%%%%%
\begin{frame} \frametitle{Caractères spéciaux}
Certains caractères ne s'affichent pas mais produisent un effet sur
la sortie. Ce sont des \alert{caractères spéciaux}.
\begin{center}
    \begin{tabular}{c|c}
        Caractère spécial & Rôle \\ \hline
        \Code{\textbackslash n} & Passage à la ligne suivante \\
        \Code{\textbackslash b} & Retour en arrière d'un caractère \\
        \Code{\textbackslash f} & Passage à la ligne suivante avec alinéa \\
        \Code{\textbackslash r} & Retour chariot \\
        \Code{\textbackslash t} & Tabulation horizontale \\
        \Code{\textbackslash v} & Tabulation verticale \\
    \end{tabular}
\end{center}
\end{frame}

%%%%%%%%%%%%%%%%%%%%%%%%%%%%%%%%%%%%%%%%%%%%%%%%%%%%%%%%%%%%%%%%%%%%%%%%
\begin{frame}[fragile] \frametitle{Caractères d'attribut}
Il est possible de faire suivre le \Code{\%} d'un indicateur de conversion
de \alert{caractères d'attribut} pour réaliser un formatage avancé.
\begin{center}\small
    \begin{tabular}{c|c}
        Caractère d'attribut & Rôle \\ \hline
        \Code{0N} & Affichage du nombre sur \Code{N} chiffres (ajout de \Code{0}) \\
        \Code{N} & Affichage du nombre sur \Code{N} chiffres (ajout d'espaces) \\
        \Code{+} & Force l'affichage du signe d'un nombre \\
        \Code{-} & Justifie à gauche un nombre (à droite par défaut) \\
    \end{tabular}
\end{center}
\bigskip

\uncover<2->{P.ex.,} \vspace{-.5em}
\begin{multicols}{2}
\begin{semiverbatim}\uncover<2->{
printf("%+5d\\n", 23);}
\end{semiverbatim}
    \uncover<2->{affiche
    \Sortie{\textvisiblespace \textvisiblespace +23}}
    \uncover<3->{et}
\begin{semiverbatim}\uncover<3->{
printf("%+-5d %d\\n", 23, 5);}
\end{semiverbatim}
    \uncover<3->{affiche
    \Sortie{+23\textvisiblespace \textvisiblespace \textvisiblespace 5}.}
\end{multicols}
\end{frame}

%%%%%%%%%%%%%%%%%%%%%%%%%%%%%%%%%%%%%%%%%%%%%%%%%%%%%%%%%%%%%%%%%%%%%%%%
\begin{frame}[fragile] \frametitle{Écriture caractère par caractère}
La fonction
\begin{center}
    \Code{int putchar(int c);}
\end{center}
permet d'\alert{afficher un caractère} sur la sortie standard.
\medskip

\uncover<2->{
Cette fonction renvoie le caractère écrit (converti en un \Code{int}).
Elle renvoie la constante \Code{EOF} (end-of-file) si une erreur a lieu.
\medskip}

\begin{multicols}{2}
\begin{semiverbatim}\small\uncover<3->{
int ret;
ret = putchar('a');
if (ret == EOF)
    exit(EXIT_FAILURE);}
\end{semiverbatim}
\uncover<3->{
Ces instructions affichent \Sortie{a}. Un test est réalisé pour
détecter une erreur éventuelle.}
\end{multicols}
\end{frame}

%%%%%%%%%%%%%%%%%%%%%%%%%%%%%%%%%%%%%%%%%%%%%%%%%%%%%%%%%%%%%%%%%%%%%%%%
%%%%%%%%%%%%%%%%%%%%%%%%%%%%%%%%%%%%%%%%%%%%%%%%%%%%%%%%%%%%%%%%%%%%%%%%
\subsection{Entrée}

%%%%%%%%%%%%%%%%%%%%%%%%%%%%%%%%%%%%%%%%%%%%%%%%%%%%%%%%%%%%%%%%%%%%%%%%
\begin{frame}[fragile] \frametitle{Lecture formatée}
La fonction
\begin{center}
    \Code{int scanf (char *format, PTR\_1, ..., PTR\_N);}
\end{center}
permet de réaliser une \alert{lecture formatée} sur l'entrée standard
\Code{stdin}.
\bigskip

\uncover<2->{
Cette fonction renvoie le nombre d'éléments lus correctement assignés.
\bigskip}

\begin{multicols}{2}
\begin{semiverbatim}\uncover<3->{
int num, ret;
char chaine[128];
chaine[0] = '\\0';
ret = scanf("%d W %s",
  &num, chaine);
printf("%d %d %s\\n",
  ret, num, chaine);}
\end{semiverbatim}
\small
\uncover<3->{
Ces instructions lisent sur l'entrée standard un entier, un caractère
\Code{'W'}, puis une chaîne de caractères.
\medskip}

\uncover<4->{
\begin{multicols}{2}
\begin{enumerate} \footnotesize
    \item \Code{25\textvisiblespace W\textvisiblespace abc\textbackslash n} \\
    $\to$ \Sortie{2 25 abc}}
    \uncover<5->{
    \item \Code{25\textvisiblespace W%
        \textvisiblespace\textvisiblespace abc\textbackslash n} \\
    $\to$ \Sortie{2 25 abc}}
    \uncover<6->{
    \item \Code{25Wabc\textbackslash n} \\
    $\to$ \Sortie{2 25 abc}}
    \bigskip

    \uncover<7->{
    \item \Code{25\textvisiblespace abc\textbackslash n} \\
    $\to$ \Sortie{1 25}}
    \uncover<8->{
    \item \Code{xy\textvisiblespace W\textvisiblespace abc\textbackslash n} \\
    $\to$ \Sortie{0 ?}}
    \uncover<9->{
    \item \Code{25\textvisiblespace w\textvisiblespace abc\textbackslash n} \\
    $\to$ \Sortie{1 25}}
\end{enumerate}
\end{multicols}
\end{multicols}
\end{frame}

%%%%%%%%%%%%%%%%%%%%%%%%%%%%%%%%%%%%%%%%%%%%%%%%%%%%%%%%%%%%%%%%%%%%%%%%
\begin{frame}[fragile] \frametitle{Exemple d'utilisation de {\tt scanf}}
Ce programme lit sur l'entrée standard une chaîne d'au plus sept
caractères (format \Code{\%Ns}), une espace, puis un entier.

\begin{semiverbatim}
\uncover<4->{#include <stdio.h>
#include <string.h>

int main() \{
    char prenom[8];
    int age, ret;}

    \uncover<2->{ret = scanf("%7s %d", prenom, &age);}
    \uncover<3->{while (ret != 2) \{
        printf("%lu\\n", strlen(prenom));
        ret = scanf("%7s %d", prenom, &age);
    \}}
    \uncover<4->{return 0;
\}}
\end{semiverbatim}
\end{frame}

%%%%%%%%%%%%%%%%%%%%%%%%%%%%%%%%%%%%%%%%%%%%%%%%%%%%%%%%%%%%%%%%%%%%%%%%
\begin{frame}[fragile] \frametitle{Lecture caractère par caractère}
La fonction
\begin{center}
    \Code{int getchar();}
\end{center}
permet de \alert{lire un caractère} sur l'entrée standard.
\medskip

\uncover<2->{
Cette fonction renvoie le caractère lu (converti en un \Code{int}).
Elle renvoie la constante \Code{EOF} (end-of-file) lorsque la fin
de fichier est détectée (touche {\tt Ctrl + D}).
\medskip}

\begin{multicols}{2}
\begin{semiverbatim}\small\uncover<3->{
char c;
while ((c = getchar()) != EOF)
    printf("%c", c);}
\end{semiverbatim}
\uncover<3->{
Ces instructions affichent chaque caractère lu en entrée, tant que
\Code{EOF} n'est pas rencontré.}
\end{multicols}
\end{frame}

%%%%%%%%%%%%%%%%%%%%%%%%%%%%%%%%%%%%%%%%%%%%%%%%%%%%%%%%%%%%%%%%%%%%%%%%
\begin{frame}[fragile] \frametitle{Le tampon d'entrée}
Considérons les instructions
\begin{semiverbatim}
char c;
while ((c = getchar()) != EOF)
    printf("%c", c);
\end{semiverbatim}
\bigskip

\uncover<2->{
Il est possible de saisir plusieurs caractères avant le
\Code{'\textbackslash n'} (touche {\tt Entrée}).
\bigskip}

\uncover<3->{
Avant d'être lus par le \Code{getchar} de la boucle, ils sont situés
temporairement dans le \alert{tampon d'entrée}.
\bigskip}

\uncover<4->{Le tampon d'entrée se comporte comme un tableau
de caractères.
\bigskip}

\uncover<5->{Chaque appel à \Code{getchar} considère le tampon d'entrée et~:
\begin{itemize}
    \item s'il est vide, on attend la saisie d'un caractère de l'entrée
    standard~;
    \smallskip

    \item s'il contient au moins un élément, il le considère et le
    supprime du tampon d'entrée.
\end{itemize}}
\end{frame}

%%%%%%%%%%%%%%%%%%%%%%%%%%%%%%%%%%%%%%%%%%%%%%%%%%%%%%%%%%%%%%%%%%%%%%%%
%%%%%%%%%%%%%%%%%%%%%%%%%%%%%%%%%%%%%%%%%%%%%%%%%%%%%%%%%%%%%%%%%%%%%%%%
\subsection{Fichiers}

%%%%%%%%%%%%%%%%%%%%%%%%%%%%%%%%%%%%%%%%%%%%%%%%%%%%%%%%%%%%%%%%%%%%%%%%
\begin{frame} \frametitle{Descripteurs de fichiers et tête de lecture/écriture}
Tout fichier est manipulé par un pointeur sur une variable de type
\Code{FILE}.
\medskip

C'est un type structuré déclaré dans \Code{stdio.h} qui contient diverses
informations nécessaires et relatives au fichier qu'il adresse.
\medskip

Toute variable de type \Code{FILE *} est un \alert{descripteur de fichier}.
\bigskip

\uncover<2->{%
La lecture écriture dans un fichier se fait par l'intermédiaire d'une
\alert{tête de lecture}.
\medskip

Celle-ci désigne un caractère du fichier considéré comme le
{\bf caractère observé} à un instant donné.
\medskip

Les fonctions de lecture/écriture ont pour effet (en particulier) de
mettre à jour cette tête de lecture en avançant dans leur positionnement
dans le fichier.}
\end{frame}

%%%%%%%%%%%%%%%%%%%%%%%%%%%%%%%%%%%%%%%%%%%%%%%%%%%%%%%%%%%%%%%%%%%%%%%%
\begin{frame} \frametitle{Tête de lecture/écriture}
Il existe trois fonctions de \Code{stdio.h} pour
\alert{déplacer manuellement la tête de lecture}~:

\begin{itemize}
    \item \Code{int ftell(FILE *f);} qui renvoie l'indice de la tête
    de lecture du fichier pointé par \Code{f}.
    \smallskip

    Cette fonction est sans effet de bord~;
    \medskip

    \uncover<2->{%
    \item \Code{int fseek(FILE *f, int decalage, int mode);} qui décale
    la tête de lecture du fichier pointé par \Code{f} de \Code{decalage}
    caractères selon \Code{mode}.
    \smallskip

    Ce paramètre explique à quoi le décalage est relatif
    (\Code{SEEK\_SET}~: début du fichier, \Code{SEEK\_CUR}~: position
    courante, \Code{SEEK\_END}~: fin du fichier).
    \smallskip

    Cette fonction renvoie \Code{0} si elle s'est bien exécutée et
    \Code{-1} sinon~;
    \medskip}

    \uncover<3->{%
    \item \Code{void rewind(FILE *f)}; place la tête de lecture au début
    du fichier. L'appel \Code{void rewind(f)}; est ainsi équivalent à
    \Code{int fseek(f, O, SEEK\_SET);}.}
\end{itemize}
\end{frame}

%%%%%%%%%%%%%%%%%%%%%%%%%%%%%%%%%%%%%%%%%%%%%%%%%%%%%%%%%%%%%%%%%%%%%%%%
\begin{frame} \frametitle{Ouverture de fichiers}
La fonction
\begin{center}
    \Code{FILE *fopen(const char *chemin, const char *mode);}
\end{center}
permet d'\alert{ouvrir un fichier}.
\medskip

\uncover<2->{%
Cette fonction renvoie un descripteur de fichier sur le fichier de chemin
\Code{chemin} (relatif par rapport à l'exécutable).
\medskip}

\uncover<3->{%
La tête de lecture est positionnée sur son 1\ier{} caractère.
\medskip}

\uncover<4->{%
Le paramètre \Code{mode} désigne le mode d'ouverture désiré.
\bigskip}

\uncover<5->{
{\bf Attention}~: \Code{fopen} renvoie \Code{NULL} si l'ouverture s'est
mal passée. Il faut donc toujours tester la valeur de retour de \Code{fopen}.
\medskip}
\end{frame}

%%%%%%%%%%%%%%%%%%%%%%%%%%%%%%%%%%%%%%%%%%%%%%%%%%%%%%%%%%%%%%%%%%%%%%%%
\begin{frame} \frametitle{Modes d'ouverture}
Il existe plusieurs \alert{modes d'ouverture}. Chacun répond à un besoin
particulier~:
\smallskip

\begin{itemize}
    \item \Code{"r"}~: lecture seule.
    \medskip

    \item \Code{"w"}~: écriture seule. Si le fichier n'existe pas, il
    est créé.
    \medskip

    \item \Code{"a"}~: écriture en ajout. Permet d'écrire dans le fichier
    en partant de la fin. Si le fichier n'existe pas, il est créé.
    \medskip

    \item \Code{"r+"}~: lecture et écriture.
    \medskip

    \item \Code{"w+"}~: lecture et écriture avec suppression préalable
    du contenu du fichier. Si le fichier n'existe pas, il est créé.
    \medskip

    \item \Code{"a+"}~: lecture et écriture en ajout. Permet de lire et
    d'écrire dans le fichier en partant de la fin. Si le fichier
    n'existe pas, il est créé.
\end{itemize}
\end{frame}

%%%%%%%%%%%%%%%%%%%%%%%%%%%%%%%%%%%%%%%%%%%%%%%%%%%%%%%%%%%%%%%%%%%%%%%%
\begin{frame} \frametitle{Fermeture de fichiers}
La fonction
\begin{center}
    \Code{int fclose(FILE *f);}
\end{center}
permet de \alert{fermer un fichier}.
\bigskip

\uncover<2->{
Cette fonction permet de mettre à jour le fichier pointé par \Code{f}
de sorte que toutes les modifications effectuées soient prisent en compte.
Le pointeur \Code{f} n'est plus alors utilisable pour accéder au fichier.
\bigskip}

\uncover<3->{
Cette fonction renvoie \Code{0} si la fermeture s'est bien déroulée et
\Code{EOF} dans le cas contraire.
\bigskip}

\uncover<4->{{\bf Attention}~: toute ouverture d'un fichier lors de
l'exécution d'un programme s'accompagne de la fermeture future du
fichier en question.}
\end{frame}

%%%%%%%%%%%%%%%%%%%%%%%%%%%%%%%%%%%%%%%%%%%%%%%%%%%%%%%%%%%%%%%%%%%%%%%%
\begin{frame}[fragile] \frametitle{Écriture dans un fichier}
La fonction
\begin{center}
    \Code{int fprintf(FILE *f, char *format, V\_1, ..., V\_N);}
\end{center}
permet de réaliser une \alert{écriture formatée} dans le fichier pointé
par \Code{f}.
\medskip

\uncover<2->{
Cette fonction se comporte comme \Code{printf}, à la différence que
cette dernière travaille sur le fichier \Code{stdout}.}

\begin{multicols}{2}
\begin{semiverbatim}\small\uncover<3->{
FILE *f;
int ret;
f = fopen("fic.txt", "w");
if (f == NULL)
    exit(EXIT_FAILURE);
fprintf(f, "abc\\n");
ret = fclose(f);
if (ret == EOF)
    exit(EXIT_FAILURE);}
\end{semiverbatim}
\uncover<3->{
Ces instructions déclarent un pointeur sur le fichier \Code{fic.txt}
et écrivent \Code{"abc"} dans \Code{fic.txt}.}
\bigskip
\bigskip
\bigskip
\bigskip
\bigskip
\bigskip
\end{multicols}
\end{frame}

%%%%%%%%%%%%%%%%%%%%%%%%%%%%%%%%%%%%%%%%%%%%%%%%%%%%%%%%%%%%%%%%%%%%%%%%
\begin{frame} \frametitle{Lecture dans un fichier}
La fonction
\begin{center}
    \Code{int fscanf(FILE *f, char *format, V\_1, ..., V\_N);}
\end{center}
permet de réaliser une \alert{lecture formatée} depuis le fichier pointé
par \Code{f}.
\bigskip

\uncover<2->{
Cette fonction se comporte comme \Code{scanf}, à la différence que
cette dernière travaille sur le fichier \Code{stdin}.}
\end{frame}

%%%%%%%%%%%%%%%%%%%%%%%%%%%%%%%%%%%%%%%%%%%%%%%%%%%%%%%%%%%%%%%%%%%%%%%%
%%%%%%%%%%%%%%%%%%%%%%%%%%%%%%%%%%%%%%%%%%%%%%%%%%%%%%%%%%%%%%%%%%%%%%%%
\subsection{Fichiers binaires}

%%%%%%%%%%%%%%%%%%%%%%%%%%%%%%%%%%%%%%%%%%%%%%%%%%%%%%%%%%%%%%%%%%%%%%%%
\begin{frame} \frametitle{Ouverture de fichiers en mode binaire}
On utilise les modes d'ouverture habituels avec un \Code{b} en plus
pour signaler l'ouverture en binaire.
\smallskip

\begin{itemize}
    \item \Code{"rb"}~: lecture binaire seule.
    \medskip

    \item \Code{"wb"}~: écriture binaire seule. Si le fichier n'existe
    pas, il est créé.
    \medskip

    \item \Code{"ab"}~: écriture binaire en ajout. Permet d'écrire dans le fichier
    en partant de la fin. Si le fichier n'existe pas, il est créé.
    \medskip

    \item \Code{"rb+"}~: lecture binaire  et écriture binaire .
    \medskip

    \item \Code{"wb+"}~: lecture binaire et écriture binaire
    avec suppression préalable
    du contenu du fichier. Si le fichier n'existe pas, il est créé.
    \medskip

    \item \Code{"ab+"}~: lecture binaire et écriture binaire en ajout. Permet de lire et
    d'écrire dans le fichier en partant de la fin. Si le fichier
    n'existe pas, il est créé.
\end{itemize}
\end{frame}

%%%%%%%%%%%%%%%%%%%%%%%%%%%%%%%%%%%%%%%%%%%%%%%%%%%%%%%%%%%%%%%%%%%%%%%%
\begin{frame} \frametitle{Écriture dans un fichier binaire}
On utilise la fonction
\begin{center}
    \Code{int fwrite(void *ptr, int taille, int nb, FILE *f);}
\end{center}
pour \alert{écrire} dans un fichier pointé par \Code{f} ouvert
en \alert{mode binaire}.
\bigskip

\uncover<2->{
Cette fonction écrit les \Code{nb} valeurs de taille \Code{taille} octets
pointées par le pointeur \Code{ptr} dans le fichier pointé
par \Code{f}.
\bigskip}

\uncover<3->{
Cette fonction renvoie le nombre de valeurs écrites.}
\end{frame}

%%%%%%%%%%%%%%%%%%%%%%%%%%%%%%%%%%%%%%%%%%%%%%%%%%%%%%%%%%%%%%%%%%%%%%%%
\begin{frame}[fragile] \frametitle{Exemple d'utilisation de {\tt fwrite}}
Ce programme écrit en mode binaire dans le fichier \Code{fic} la valeur
d'une variable d'un type structuré.
\medskip

\begin{semiverbatim}\footnotesize
\uncover<3->{#include <stdio.h>}
\uncover<2->{
typedef struct \{
    unsigned char rouge;
    unsigned char bleu;
    unsigned char vert;
\} Couleur;}

\uncover<3->{int main() \{
    FILE *f;
    Couleur coul;
    coul.rouge = 120; coul.bleu = 200; coul.vert = 12;}

    \uncover<4->{f = fopen("fic", "wb");}
    \uncover<5->{fwrite(&coul, sizeof(Couleur), 1, f);}
    \uncover<6->{fclose(f);}

    \uncover<3->{return 0;
\}}
\end{semiverbatim}
\end{frame}

%%%%%%%%%%%%%%%%%%%%%%%%%%%%%%%%%%%%%%%%%%%%%%%%%%%%%%%%%%%%%%%%%%%%%%%%
\begin{frame} \frametitle{Lecture dans un fichier binaire}
On utilise la fonction
\begin{center}
    \Code{int fread(void *ptr, int taille, int nb, FILE *f);}
\end{center}
pour \alert{lire} dans un fichier pointé par \Code{f} ouvert
en \alert{mode binaire}.
\bigskip

\uncover<2->{
Cette fonction lit \Code{nb} valeurs de taille \Code{taille} octets
dans le fichier pointé par \Code{f} et les place à l'adresse pointée
par le pointeur \Code{ptr}.
\bigskip}

\uncover<3->{
Cette fonction renvoie le nombre de valeurs lues.}
\end{frame}

%%%%%%%%%%%%%%%%%%%%%%%%%%%%%%%%%%%%%%%%%%%%%%%%%%%%%%%%%%%%%%%%%%%%%%%%
\begin{frame}[fragile] \frametitle{Exemple d'utilisation de {\tt fread}}
Ce programme écrit en mode binaire dans le fichier \Code{fic}
un tableau d'entiers et le lit.
\medskip

\begin{semiverbatim}\footnotesize
\uncover<2->{#include <stdio.h>}
\uncover<2->{
int main() \{
    FILE *f;
    int i, tab_1[12], tab_2[12];

    for (i = 0 ; i < 12 ; ++i)
        tab_1[i] = i;}

    \uncover<3->{f = fopen("fic", "wb");}
    \uncover<4->{fwrite(tab_1, sizeof(int), 12, f);}
    \uncover<5->{fclose(f);}

    \uncover<6->{f = fopen("fic", "rb");}
    \uncover<7->{fread(tab_2, sizeof(int), 12, f);}
    \uncover<8->{fclose(f);}

    \uncover<2->{return 0;
\}}
\end{semiverbatim}
\end{frame}
