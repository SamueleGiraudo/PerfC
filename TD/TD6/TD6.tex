% Auteur : Samuele Giraudo
% Création : 25-04-16
% Modifications : 25-04-16, 30-04-16

\documentclass[12pt]{article}

\usepackage[utf8]{inputenc}
\usepackage[french]{babel}
\usepackage{amsmath,amsthm,amsfonts,amssymb}
\usepackage{lmodern}
\usepackage[top=2.4cm,bottom=2.4cm,left=2.4cm,right=2.4cm]{geometry}
\usepackage{hyperref}
\usepackage{multicol}
\usepackage{enumitem}
\usepackage{listings}
\usepackage[dvipsnames]{xcolor}
\usepackage{tikz}

%\author{Samuele Giraudo}
\date{}
\title{{\bf Perfectionnement à la programmation en {\sf C}} \\
    Fiche de TD 6 \\
    {\small L2 Informatique 2015-2016} \\
    {\it \small Généricité}}

\theoremstyle{definition}
\newtheorem{Exercice}{Exercice}

\newcommand{\EnsNat}{\mathbb{N}}
\newcommand{\La}{{\tt a}}
\newcommand{\Lb}{{\tt b}}
\newcommand{\Lc}{{\tt c}}
\newcommand{\Ld}{{\tt d}}

% Configuration de listings.
\lstset{
    language=c,
    basicstyle=\ttfamily\footnotesize,
    identifierstyle=\color{Mahogany},
    keywordstyle=\color{NavyBlue},
    stringstyle=\color{Emerald},
    commentstyle=\it\color{Gray},
    columns=flexible,
    tabsize=4,
    extendedchars=true,
    showspaces=false,
    numbers=left,
    numberstyle=\tiny,
    breaklines=true,
    breakautoindent=true,
    captionpos=b,
    showstringspaces=true
}

\begin{document}

\maketitle

%%%%%%%%%%%%%%%%%%%%%%%%%%%%%%%%%%%%%%%%%%%%%%%%%%%%%%%%%%%%%%%%%%%%%%%%
%%%%%%%%%%%%%%%%%%%%%%%%%%%%%%%%%%%%%%%%%%%%%%%%%%%%%%%%%%%%%%%%%%%%%%%%
%%%%%%%%%%%%%%%%%%%%%%%%%%%%%%%%%%%%%%%%%%%%%%%%%%%%%%%%%%%%%%%%%%%%%%%%
\begin{Exercice} {\bf (Déclarations de pointeurs de fonctions)}\smallskip
\begin{enumerate}
    \item Déclarer un pointeur {\tt f\_1} sur une fonction paramétrée
    par deux entiers et qui renvoie un caractère.
    \smallskip

    \item Déclarer un pointeur {\tt f\_2} sur une fonction paramétrée
    par une chaîne de caractères et un flottant et qui renvoie un
    pointeur sur un entier.
    \smallskip

    \item Déclarer un pointeur {\tt f\_3} sur une fonction paramétrée
    par un caractère et un pointeur sur une fonction de même signature
    que celle de {\tt f\_1} et qui renvoie une chaîne de caractères.
    \smallskip

    \item Déclarer un pointeur {\tt f\_4} sur une fonction paramétrée
    par deux caractères et qui renvoie un pointeur sur une fonction
    de même signature que celle de {\tt f\_2}.
    \smallskip

    \item Déclarer un pointeur {\tt f\_5} sur une fonction paramétrée
    par une fonction de même signature que celle de {\tt f\_2} et qui
    renvoie un pointeur sur une fonction de même signature que celle
    de~{\tt f\_1}.
    \smallskip

    \item Déclarer un tableau statique {\tt tab\_f\_1} de {\tt 32}
    pointeurs de fonctions de mêmes signatures que celle de {\tt f\_1}.
\end{enumerate}
\end{Exercice}
\bigskip

%%%%%%%%%%%%%%%%%%%%%%%%%%%%%%%%%%%%%%%%%%%%%%%%%%%%%%%%%%%%%%%%%%%%%%%%
%%%%%%%%%%%%%%%%%%%%%%%%%%%%%%%%%%%%%%%%%%%%%%%%%%%%%%%%%%%%%%%%%%%%%%%%
%%%%%%%%%%%%%%%%%%%%%%%%%%%%%%%%%%%%%%%%%%%%%%%%%%%%%%%%%%%%%%%%%%%%%%%%
\begin{Exercice} {\bf (Pointeurs de fonction et tableaux)}\smallskip
\begin{enumerate}
    \item Écrire une fonction {\tt fois\_deux} qui renvoie le double
    de son argument entier.
    \smallskip

    \item Écrire une fonction {\tt fact} qui renvoie la factorielle
    de son argument entier.
    \smallskip

    \item Écrire une fonction
\begin{lstlisting}
void appliquer_tableau(int (*f)(int), int *tab, int n);
\end{lstlisting}
    qui modifie chaque élément du tableau {\tt tab} de taille {\tt n}
    en son image par la fonction pointée par {\tt f}.
    \smallskip

    \item En supposant que {\tt tab} est un tableau d'entiers de taille
    {\tt 128}, écrire une suite d'instructions qui modifie chaque
    élément du tableau en le double de sa factorielle. On suppose que
    les valeurs du tableau sont telles que chaque calcul peut se faire
    sans dépassement de capacité.
\end{enumerate}
\end{Exercice}
\bigskip

%%%%%%%%%%%%%%%%%%%%%%%%%%%%%%%%%%%%%%%%%%%%%%%%%%%%%%%%%%%%%%%%%%%%%%%%
%%%%%%%%%%%%%%%%%%%%%%%%%%%%%%%%%%%%%%%%%%%%%%%%%%%%%%%%%%%%%%%%%%%%%%%%
%%%%%%%%%%%%%%%%%%%%%%%%%%%%%%%%%%%%%%%%%%%%%%%%%%%%%%%%%%%%%%%%%%%%%%%%
\begin{Exercice} {\bf (Pointeurs de fonction et tris)}\smallskip
\label{ex:pointeurs_de_fonction_et_tris}

Considérons les deux fonctions suivantes~:
\begin{multicols}{2}
\begin{lstlisting}
void tri_croissant(int *tab, int n) {
    int i, i_min, j, tmp;
    for (i = 0 ; i < n - 1 ; i++) {
        i_min = i;
        for (j = i ; j < n ; j++) {
            if (tab[j] < tab[i_min])
                i_min = j;
        }
        tmp = tab[i_min];
        tab[i_min] = tab[i];
        tab[i] = tmp;
    }
}
\end{lstlisting}
\begin{lstlisting}
void tri_decroissant(int *tab, int n) {
    int i, i_max, j, tmp;
    for (i = 0 ; i < n - 1 ; i++) {
        i_max = i;
        for (j = i ; j < n ; j++) {
            if (tab[j] > tab[i_max])
                i_max = j;
        }
        tmp = tab[i_max];
        tab[i_max] = tab[i];
        tab[i] = tmp;
    }
}
\end{lstlisting}
\end{multicols}
L'une trie les éléments d'un tableau {\tt tab} de taille {\tt n} dans
l'ordre croissant et l'autre, dans l'ordre décroissant.
\smallskip

Ces deux fonctions sont identiques, à l'exception de l'opérateur de
comparaison en ligne 6 et de certains noms pour les variables locales.
L'utilisation de pointeurs de fonctions permet d'éviter cette redondance
de code et d'avoir ainsi une unique fonction qui réaliser, selon la
manière dont elle est appelée, l'un ou l'autre tri.
\begin{enumerate}
    \item Écrire une fonction
\begin{lstlisting}
int superieur(int a, int b);
\end{lstlisting}
    qui renvoie {\tt 1} si {\tt a} est strictement supérieur à {\tt b},
    {\tt 0} s'ils sont égaux et {\tt -1} sinon.
    \smallskip

    \item Écrire une fonction
\begin{lstlisting}
int inferieur(int a, int b);
\end{lstlisting}
    qui renvoie {\tt 1} si {\tt a} est strictement inférieur à {\tt b},
    {\tt 0} s'ils sont égaux et {\tt -1} sinon.
    \smallskip

    \item Écrire une fonction {\tt tri}, paramétrée par un tableau
    d'entiers {\tt tab}, sa taille {\tt n} et une fonction {\tt comparer}.
    Cette fonction modifie {\tt tab} de sorte à le trier selon la comparaison
    dictée par la fonction {\tt comparer}. Plus précisément, si {\tt a}
    et {\tt b} sont des éléments de {\tt tab} et que {\tt a} apparaît
    dans {\tt tab} à gauche de {\tt b}, il faut que la fonction
    { \tt comparer} appelée avec les arguments {\tt a} et {\tt b} renvoie
    {\tt -1} (ou {\tt 0} pour les répétitions d'éléments).
    \smallskip

    \item En supposant que {\tt tab} est un tableau d'entiers de taille
    {\tt 2047}, écrire une suite d'instructions qui trie {\tt tab} dans
    l'ordre croissant, affiche ses valeurs, trie {\tt tab} dans
    l'ordre décroissant et affiche ses valeurs.
\end{enumerate}
\end{Exercice}
\bigskip

%%%%%%%%%%%%%%%%%%%%%%%%%%%%%%%%%%%%%%%%%%%%%%%%%%%%%%%%%%%%%%%%%%%%%%%%
%%%%%%%%%%%%%%%%%%%%%%%%%%%%%%%%%%%%%%%%%%%%%%%%%%%%%%%%%%%%%%%%%%%%%%%%
%%%%%%%%%%%%%%%%%%%%%%%%%%%%%%%%%%%%%%%%%%%%%%%%%%%%%%%%%%%%%%%%%%%%%%%%
\begin{Exercice} {\bf (Pointeurs et données génériques)}\smallskip

Un {\em pointeur générique} est un pointeur de type {\tt void *}. Une
{\em donnée générique} est une variable adressée par un pointeur
générique.
\begin{enumerate}
    \item On suppose que {\tt a} est une variable de type {\tt int}.
    Définir un pointeur générique {\tt ptr} qui adresse~{\tt a}.
    \smallskip

    \item Multiplier par trois la valeur de {\tt a} en opérant uniquement
    sur {\tt ptr}.
    \smallskip

    \item On suppose maintenant que {\tt ptr} est un pointeur générique
    et que l'on dispose d'un type
\begin{lstlisting}
typedef struct {
    int x;
    int y;
} Couple;
\end{lstlisting}
    et d'une variable {\tt b} de ce type. Faire pointer {\tt ptr} vers
    {\tt b}.
    \smallskip

    \item Incrémenter le champ {\tt y} de {\tt b} en opérant uniquement
    sur {\tt ptr}.
    \smallskip

    \item Écrire une fonction
\begin{lstlisting}
void *twist(void *couple);
\end{lstlisting}
    qui travaille avec des pointeurs génériques voués à être des données
    de type {\tt Couple}. Cette fonction renvoie un nouveau couple
    obtenu en permutant les champs {\tt x} et {\tt y} du couple en
    argument.
    \smallskip

    \item En supposant que {\tt c} est une variable de type {\tt Couple},
    écrire une suite d'instructions appelant la fonction précédente
    sur {\tt c}.
\end{enumerate}
\end{Exercice}
\bigskip

%%%%%%%%%%%%%%%%%%%%%%%%%%%%%%%%%%%%%%%%%%%%%%%%%%%%%%%%%%%%%%%%%%%%%%%%
%%%%%%%%%%%%%%%%%%%%%%%%%%%%%%%%%%%%%%%%%%%%%%%%%%%%%%%%%%%%%%%%%%%%%%%%
%%%%%%%%%%%%%%%%%%%%%%%%%%%%%%%%%%%%%%%%%%%%%%%%%%%%%%%%%%%%%%%%%%%%%%%%
\begin{Exercice} {\bf (Tableaux génériques)}\smallskip

Un {\em tableau générique} est une variable de type {\tt void *},
interprétée comme une concaténation d'octets formant, bloc par bloc,
les cases du tableau. Pour manipuler une telle donnée, il est nécessaire
de connaître la taille du tableau (nombre de cases) ainsi que le nombre
d'octets nécessaires pour représenter un élément du tableau (nombre
d'octets par bloc).
\begin{enumerate}
    \item Calculer le nombre d'octets nécessaires pour représenter
    un tableau générique de taille $24$ voué à contenir des valeurs
    occupant chacune $8$ octets.
    \smallskip

    \item Représenter graphiquement un tableau générique de taille $4$
    voué à contenir des valeurs occupant chacune $2$ octets.
    \smallskip

    \item Représenter graphiquement un tableau générique de taille $2$
    voué à contenir des valeurs occupant chacune $4$ octets.
    \smallskip

    \item Supposons que {\tt tab} est un tableau générique. Supposons de
    plus que {\tt tab} est utilisé pour contenir des données de type
    {\tt short}. Donner deux manières d'accéder à la $4\ieme$ donnée du
    tableau.
    \smallskip

    \item Répondre à la même question que la précédente dans le cas où
    {\tt tab} est utilisé pour contenir des {\tt int}.
\end{enumerate}
\end{Exercice}
\bigskip


%%%%%%%%%%%%%%%%%%%%%%%%%%%%%%%%%%%%%%%%%%%%%%%%%%%%%%%%%%%%%%%%%%%%%%%%
%%%%%%%%%%%%%%%%%%%%%%%%%%%%%%%%%%%%%%%%%%%%%%%%%%%%%%%%%%%%%%%%%%%%%%%%
%%%%%%%%%%%%%%%%%%%%%%%%%%%%%%%%%%%%%%%%%%%%%%%%%%%%%%%%%%%%%%%%%%%%%%%%
\begin{Exercice} {\bf (Minimum générique dans un tableau)}\smallskip

L'objectif de cette exercice est d'écrire une fonction générique qui
calcule la plus petite valeur contenue dans un tableau générique.
\begin{enumerate}
    \item Déterminer la signature d'une fonction qui prend deux données
    génériques en argument et qui renvoie la plus petite des deux. En
    faire un type.
    \smallskip

    \item Écrire une telle fonction générique qui calcule la plus petite
    valeur entre deux entiers.
    \smallskip

    \item Écrire une telle fonction générique qui calcule la plus petite
    valeur entre deux chaînes de caractères (il est conseillé d'utiliser
    la fonction {\tt strcmp} de {\tt string.h}.
    \smallskip

    \item Déterminer la signature de la fonction {\tt min\_tab} qui
    répond à l'objectif de l'exercice. Elle est paramétrée, entre autres,
    par un tableau générique et une fonction de calcul de la plus petite
    valeur entre deux valeurs données.
    \smallskip

    \item Donner le corps de la fonction {\tt min\_tab}.
    \smallskip

    \item En supposant que {\tt tab} est un tableau d'entiers de taille
    {\tt 64}, écrire une suite d'instructions qui calcule et affiche
    sa plus petite valeur, en utilisant {\tt min\_tab}.
    \smallskip

    \item En supposant que {\tt tab} est un tableau de {\tt 64} chaînes
    de caractères et toutes de taille {\tt 96}, écrire une suite
    d'instructions qui calcule et affiche sa plus petite valeur, en
    utilisant {\tt min\_tab}.
\end{enumerate}
\end{Exercice}
\bigskip

\end{document}
