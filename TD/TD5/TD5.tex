% Auteur : Samuele Giraudo
% Création : 20-03-16
% Modifications : 20-03-16

\documentclass[12pt]{article}

\usepackage[utf8]{inputenc}
\usepackage[french]{babel}
\usepackage{amsmath,amsthm,amsfonts,amssymb}
\usepackage{lmodern}
\usepackage[top=2.4cm,bottom=2.4cm,left=2cm,right=2cm]{geometry}
\usepackage{hyperref}
\usepackage{multicol}
\usepackage{enumitem}
\usepackage{listings}
\usepackage[dvipsnames]{xcolor}
\usepackage{tikz}

%\author{Samuele Giraudo}
\date{}
\title{{\bf Perfectionnement à la programmation en {\sf C}} \\
    Fiche de TD 5 \\
    {\small L2 Informatique 2015-2016} \\
    {\it \small Pointeurs génériques et allocation dynamique}}

\theoremstyle{definition}
\newtheorem{Exercice}{Exercice}

\newcommand{\EnsNat}{\mathbb{N}}
\newcommand{\La}{{\tt a}}
\newcommand{\Lb}{{\tt b}}
\newcommand{\Lc}{{\tt c}}
\newcommand{\Ld}{{\tt d}}

% Configuration de listings.
\lstset{
    language=c,
    basicstyle=\ttfamily\footnotesize,
    identifierstyle=\color{Mahogany},
    keywordstyle=\color{NavyBlue},
    stringstyle=\color{Emerald},
    commentstyle=\it\color{Gray},
    columns=flexible,
    tabsize=4,
    extendedchars=true,
    showspaces=false,
    numbers=left,
    numberstyle=\tiny,
    breaklines=true,
    breakautoindent=true,
    captionpos=b,
    showstringspaces=true
}

\begin{document}

\maketitle

%%%%%%%%%%%%%%%%%%%%%%%%%%%%%%%%%%%%%%%%%%%%%%%%%%%%%%%%%%%%%%%%%%%%%%%%
%%%%%%%%%%%%%%%%%%%%%%%%%%%%%%%%%%%%%%%%%%%%%%%%%%%%%%%%%%%%%%%%%%%%%%%%
%%%%%%%%%%%%%%%%%%%%%%%%%%%%%%%%%%%%%%%%%%%%%%%%%%%%%%%%%%%%%%%%%%%%%%%%
\begin{Exercice} {\bf (Test d'égalité de zones de la mémoire)}\smallskip
\begin{enumerate}
    \item Écrire une fonction qui permet de tester l'égalité entre deux
    variables d'un type quelconque. La fonction admet le prototype
\begin{lstlisting}
int sont_egales(int nb_octets, void *var1, void *var2);
\end{lstlisting}
    et elle renvoie {\tt 1} si les {\tt nb\_octets} lus à partir des adresses
    {\tt var1} et {\tt var2} sont égaux deux à deux et {\tt 0} sinon.
    \smallskip

    \item On suppose que {\tt num1} et {\tt num2} sont deux variables
    de type {\tt short}. Écrire l'appel à la fonction {\tt sont\_egales}
    pour comparer les valeurs de ces variables.
    \smallskip

    \item On suppose que {\tt res} est une variable de type {\tt int}.
    Pour chacune des suites d'instructions suivantes, expliquer la
    valeur de {\tt res} à la fin de leur exécution.
    \begin{multicols}{2}
        \begin{enumerate}[label = ({\alph*})]
            \item ~
\begin{lstlisting}
int a;
char b;

a = 3;
b = 3;
res = sont_egales(1, &a, &b);
\end{lstlisting}

            \item ~
\begin{lstlisting}
int a;
int b;

a = (1 << 8) + 32;
b = 32;
res = sont_egales(1, &a, &b);
\end{lstlisting}
        \end{enumerate}
    \end{multicols}
\end{enumerate}
\end{Exercice}
\bigskip

%%%%%%%%%%%%%%%%%%%%%%%%%%%%%%%%%%%%%%%%%%%%%%%%%%%%%%%%%%%%%%%%%%%%%%%%
%%%%%%%%%%%%%%%%%%%%%%%%%%%%%%%%%%%%%%%%%%%%%%%%%%%%%%%%%%%%%%%%%%%%%%%%
%%%%%%%%%%%%%%%%%%%%%%%%%%%%%%%%%%%%%%%%%%%%%%%%%%%%%%%%%%%%%%%%%%%%%%%%
\begin{Exercice} {\bf (Tableaux statiques)}\smallskip

On souhaite manipuler une fonction paramétrée par un entier {\tt n} qui
renvoie un pointeur sur un tableau de {\tt n} entiers initialisés à {\tt 0}.
Expliquer pourquoi la fonction suivante ne répond pas à cette spécification.
Expliquer en particulier ce qu'il se passe lorsque cette fonction est
appelée (faire des dessins de mémoire).
\begin{lstlisting}
int *creer_tableau(int n) {
    int tab[n];
    int i;

    for (i = 0 ; i < n ; ++i)
        tab[i] = 0;
    return tab;
}
\end{lstlisting}
\end{Exercice}
\bigskip

%%%%%%%%%%%%%%%%%%%%%%%%%%%%%%%%%%%%%%%%%%%%%%%%%%%%%%%%%%%%%%%%%%%%%%%%
%%%%%%%%%%%%%%%%%%%%%%%%%%%%%%%%%%%%%%%%%%%%%%%%%%%%%%%%%%%%%%%%%%%%%%%%
%%%%%%%%%%%%%%%%%%%%%%%%%%%%%%%%%%%%%%%%%%%%%%%%%%%%%%%%%%%%%%%%%%%%%%%%
\begin{Exercice} {\bf (Tableaux dynamiques à une dimension)}\smallskip

Dans toutes les fonctions suivantes, on prendra soin de vérifier que
les allocations dynamiques (et donc les appels à la fonction {\tt malloc})
ne produisent pas d'erreur.
\begin{enumerate}
    \item Écrire une fonction {\tt creer\_tab} paramétrée par un entier
    {\tt n} et qui renvoie un pointeur sur un tableau de {\tt n} entiers
    initialisés à {\tt 0}.
    \smallskip

    \item Écrire une fonction {\tt detruire\_tab} paramétrée par un
    pointeur {\tt tab} sur un tableau d'entiers et qui libère la place
    mémoire occupée par {\tt tab}.
\end{enumerate}
\end{Exercice}
\bigskip

%%%%%%%%%%%%%%%%%%%%%%%%%%%%%%%%%%%%%%%%%%%%%%%%%%%%%%%%%%%%%%%%%%%%%%%%
%%%%%%%%%%%%%%%%%%%%%%%%%%%%%%%%%%%%%%%%%%%%%%%%%%%%%%%%%%%%%%%%%%%%%%%%
%%%%%%%%%%%%%%%%%%%%%%%%%%%%%%%%%%%%%%%%%%%%%%%%%%%%%%%%%%%%%%%%%%%%%%%%
\begin{Exercice} {\bf (Tableaux dynamiques à deux dimensions)}\smallskip

Dans toutes les fonctions suivantes, on prendra soin de vérifier que
les allocations dynamiques (et donc les appels à la fonction {\tt malloc})
ne produisent pas d'erreur.
\begin{enumerate}
    \item Écrire une fonction {\tt creer\_tab\_2d} paramétrée par des
    entiers {\tt n} et {\tt m} et qui renvoie un pointeur sur un tableau
    à deux dimensions de {\tt n} $\times$ {\tt m} entiers initialisés
    à {\tt 0}.
    \smallskip

    \item Écrire une fonction {\tt detruire\_tab\_2d} paramétrée par un
    entier {\tt n} et un pointeur {\tt tab} sur un tableau à deux
    dimensions de {\tt n} $\times$ {\tt m} entiers (il n'est pas
    nécessaire de connaître {\tt m} ici). Cette fonction doit libérer la
    place mémoire occupée par {\tt tab}.
    \smallskip

    \item Dessiner l'état de la mémoire étape par étape et de manière
    très précise lors de l'exécution des instructions
\begin{lstlisting}
int **tab = creer_tab_2d(2, 3);
detruire_tab_2d(tab, 2);
\end{lstlisting}
    \smallskip

    \item Expliquer comment représenter et gérer un tableau à deux
    dimensions par un tableau à une seule dimension.
\end{enumerate}
\end{Exercice}
\bigskip

%%%%%%%%%%%%%%%%%%%%%%%%%%%%%%%%%%%%%%%%%%%%%%%%%%%%%%%%%%%%%%%%%%%%%%%%
%%%%%%%%%%%%%%%%%%%%%%%%%%%%%%%%%%%%%%%%%%%%%%%%%%%%%%%%%%%%%%%%%%%%%%%%
%%%%%%%%%%%%%%%%%%%%%%%%%%%%%%%%%%%%%%%%%%%%%%%%%%%%%%%%%%%%%%%%%%%%%%%%
\begin{Exercice} {\bf (Tableaux dynamiques en dents de scie)}\smallskip

Dans toutes les fonctions suivantes, on prendra soin de vérifier que
les allocations dynamiques (et donc les appels à la fonction {\tt malloc})
ne produisent pas d'erreur.

\begin{enumerate}
    \item Écrire une fonction {\tt creer\_scie} paramétrée par un entier
    {\tt n} et qui renvoie un pointeur sur un tableau à deux dimensions
    de caractères. Pour tout $0 \leq i \leq {\tt n} - 1$, la $i\ieme$
    case du tableau à construire contient un tableau à une dimension de
    $i$ modulo $5$ plus un caractères {\tt '*'}.
    \smallskip

    \item Écrire une fonction {\tt detruire\_scie} paramétrée par un
    entier {\tt n} et un pointeur {\tt scie} sur un tableau à deux
    dimensions de caractères (pouvant être construit par un appel à
    {\tt creer\_scie}). Cette fonction doit libérer la place mémoire
    occupée par {\tt scie}.
\end{enumerate}
\end{Exercice}
\bigskip

\end{document}
