% Auteur : Samuele Giraudo
% Création : 11-12-15
% Modifications : 11-12-15, déc. 2015

\documentclass[12pt]{article}

\usepackage[utf8]{inputenc}
\usepackage[french]{babel}
\usepackage{amsmath,amsthm,amsfonts,amssymb}
\usepackage{lmodern}
\usepackage[top=2.4cm,bottom=2.4cm,left=2cm,right=2cm]{geometry}
\usepackage{hyperref}
\usepackage{multicol}
\usepackage{enumitem}
\usepackage{listings}
\usepackage[dvipsnames]{xcolor}
\usepackage{tikz}

%\author{Samuele Giraudo}
\date{}
\title{{\bf Perfectionnement à la programmation en {\sf C}} \\
    Fiche de TD 1 \\
    {\small L2 Informatique 2015-2016} \\
    {\it \small Effets de bords, structures de contrôle, fonctions}}

% Couleurs.
\definecolor{Noir}{RGB}{0,0,0}
\definecolor{Rouge}{RGB}{205,35,38}
\definecolor{Bleu}{RGB}{2,60,195}
\definecolor{Bleu1}{RGB}{121,176,197}
\definecolor{Vert}{RGB}{23,103,1}
\definecolor{Orange}{RGB}{245,113,15}
\definecolor{Blanc}{RGB}{255,255,255}
\definecolor{Marron}{RGB}{193,88,50}
\definecolor{Jaune}{RGB}{255,180,30}
\definecolor{Violet}{RGB}{181,18,225}

\theoremstyle{definition}
\newtheorem{Exercice}{Exercice}

% Configuration de listings.
\lstset{
    language=c,
    basicstyle=\ttfamily\footnotesize,
    identifierstyle=\color{Mahogany},
    keywordstyle=\color{NavyBlue},
    stringstyle=\color{Emerald},
    commentstyle=\it\color{Gray},
    columns=flexible,
    tabsize=4,
    extendedchars=true,
    showspaces=false,
    numbers=left,
    numberstyle=\tiny,
    breaklines=true,
    breakautoindent=true,
    captionpos=b,
    showstringspaces=true
}

\begin{document}

\maketitle

%%%%%%%%%%%%%%%%%%%%%%%%%%%%%%%%%%%%%%%%%%%%%%%%%%%%%%%%%%%%%%%%%%%%%%%%
%%%%%%%%%%%%%%%%%%%%%%%%%%%%%%%%%%%%%%%%%%%%%%%%%%%%%%%%%%%%%%%%%%%%%%%%
%%%%%%%%%%%%%%%%%%%%%%%%%%%%%%%%%%%%%%%%%%%%%%%%%%%%%%%%%%%%%%%%%%%%%%%%
\begin{Exercice} {\bf (Instructions à effets de bord)}\smallskip
\begin{enumerate}
    \item Rappeler ce qu'est une {\em expression à effet de bord}.
    \smallskip

    \item Déterminer si les instructions suivantes sont à effet de bord~:
    \begin{multicols}{2}
        \begin{enumerate}
            \item ~
\begin{lstlisting}
2 + (8 * 2);
\end{lstlisting}
            \medskip

            \item ~
\begin{lstlisting}
printf("Bonjour\n");
\end{lstlisting}
            \medskip

            \item ~
\begin{lstlisting}
int a;
\end{lstlisting}
            \medskip
            
            \item ~
\begin{lstlisting}
int a = 2;
\end{lstlisting}
            \medskip

            \item ~
\begin{lstlisting}
a = 2 + (8 * 2);
\end{lstlisting}
            \medskip

            \item ~
\begin{lstlisting}
a == 2 + (8 * 2);
\end{lstlisting}
            \medskip

            \item ~
\begin{lstlisting}
a * 2 + (8 * 2);
\end{lstlisting}
            \medskip

            \item ~
\begin{lstlisting}
if (a == 17) {a;}
\end{lstlisting}
            \medskip

            \item ~
\begin{lstlisting}
if (--a == 16) {a;}
\end{lstlisting}
            \medskip

            \item ~
\begin{lstlisting}
a++;
\end{lstlisting}
            \medskip

            \item ~
\begin{lstlisting}
a + 1;
\end{lstlisting}
            \medskip

        \item ~
\begin{lstlisting}
while (1) a + 1;
\end{lstlisting}
            \medskip

        \item ~
\begin{lstlisting}
while (1) a += 1;
\end{lstlisting}
        \medskip
        
        \item ~
\begin{lstlisting}
return 1;
\end{lstlisting}
        \medskip

        \item ~
\begin{lstlisting}
return a;
\end{lstlisting}
        \medskip
        
        \item ~
\begin{lstlisting}
return a + 1;
\end{lstlisting}
        \medskip
        
        
    
        \end{enumerate}
    \end{multicols}
\end{enumerate}
\end{Exercice}
\bigskip

%%%%%%%%%%%%%%%%%%%%%%%%%%%%%%%%%%%%%%%%%%%%%%%%%%%%%%%%%%%%%%%%%%%%%%%%
%%%%%%%%%%%%%%%%%%%%%%%%%%%%%%%%%%%%%%%%%%%%%%%%%%%%%%%%%%%%%%%%%%%%%%%%
%%%%%%%%%%%%%%%%%%%%%%%%%%%%%%%%%%%%%%%%%%%%%%%%%%%%%%%%%%%%%%%%%%%%%%%%
\begin{Exercice} {\bf (Fonctions à effets de bord)}\smallskip

\begin{enumerate}
    \item Rappeler ce qu'est une {\em fonction à effet de bord}.
    \smallskip

    \item Déterminer si les fonctions suivantes sont à effet de bord~:
    \begin{multicols}{2}
        \begin{enumerate}
            \item ~
\begin{lstlisting}
int somme(int *tab, int n) {
    int i, res;

    res = 0;
    for (i = 0; i < n; i++)
        res += tab[i];
    return res
}
\end{lstlisting}
            \bigskip
            \bigskip
            \bigskip

            \item ~
\begin{lstlisting}
void afficher(int *tab, int n) {
    int i;

    for (i = 0; i < n; i++)
        printf("%d ", tab[i]);
}
\end{lstlisting}
            \bigskip
            \bigskip
            \bigskip

            \item ~
\begin{lstlisting}
void echanger(int *x, int *y) {
    int tmp;

    tmp = *x;
    *x = *y;
    *y = tmp;
}
\end{lstlisting}
            \bigskip

            \item ~
\begin{lstlisting}
int nb_appels = 0;

int fct_1(int n) {
    nb_appels++;
    return n + 1;
}
\end{lstlisting}
            \medskip

            \item ~
\begin{lstlisting}
int nb_appels = 0;

int fct_2(int n) {
    if (nb_appels == 0)
        return 0;
    else
        return n + 1;
}
\end{lstlisting}

            \item ~
\begin{lstlisting}
int remplacer(char *tab, char a,
        char c) {
    int i, nb;
    nb = 0;
    i = 0;
    while (tab[i] != '\0') {
        if (tab[i] == a) {
            tab[i] = c;
            nb += 1;
        }
        i += 1;
    }
    return nb;
}
\end{lstlisting}
        \end{enumerate}
    \end{multicols}
\end{enumerate}
\end{Exercice}
\bigskip

%%%%%%%%%%%%%%%%%%%%%%%%%%%%%%%%%%%%%%%%%%%%%%%%%%%%%%%%%%%%%%%%%%%%%%%%
%%%%%%%%%%%%%%%%%%%%%%%%%%%%%%%%%%%%%%%%%%%%%%%%%%%%%%%%%%%%%%%%%%%%%%%%
%%%%%%%%%%%%%%%%%%%%%%%%%%%%%%%%%%%%%%%%%%%%%%%%%%%%%%%%%%%%%%%%%%%%%%%%
\begin{Exercice} {\bf (Programme douteux)}\smallskip

Expliquer ce qu'affiche le programme suivant et en quoi il n'est pas
recommandable.
\begin{lstlisting}
#include <stdio.h>

int etrange(int *n, int m) {
    *n += m;
    return *n + 1;
}

int main() {
    int n;

    n = 0;
    n = etrange(&n, etrange(&n, 10));
    printf("%d\n", n);

    return 0;
}
\end{lstlisting}
\end{Exercice}
\bigskip

%%%%%%%%%%%%%%%%%%%%%%%%%%%%%%%%%%%%%%%%%%%%%%%%%%%%%%%%%%%%%%%%%%%%%%%%
%%%%%%%%%%%%%%%%%%%%%%%%%%%%%%%%%%%%%%%%%%%%%%%%%%%%%%%%%%%%%%%%%%%%%%%%
%%%%%%%%%%%%%%%%%%%%%%%%%%%%%%%%%%%%%%%%%%%%%%%%%%%%%%%%%%%%%%%%%%%%%%%%
\begin{Exercice} {\bf (Portée lexicale)}\smallskip
\begin{enumerate}
    \item Rappeler ce qu'est un {\em bloc}.
    \smallskip

    \item Rappeler ce qu'est la {\em portée lexicale} d'une variable.
    \smallskip
    
    \item Déterminer la portée lexicale de chaque variable déclarée dans 
    les suite d'instructions ci-dessous.
\begin{multicols}{3}
\begin{enumerate}
\item~
\begin{lstlisting}
{
    int a, b, c;
    a = 10;
    c = 8;
    {
        char c;
        c = 'a';
    }
    {
        int c;
        c = 16;
    }
    a += 1;
    b = 5;
}
\end{lstlisting}

\item~
\begin{lstlisting}
int x;
{
    int y;
    float z;
    printf("%d", x);
    {
        float x;
        x = 3.4;
    }
    {
        char y;
        y = 'F';
    }
}
\end{lstlisting}
\bigskip

\item~
\begin{lstlisting}
    int a, b;
    a = 0;
    {
        int a;
        float b;
        a = 3;
        {
            b = 10;
            {
                int b;
                b = 3;
                a += 1;
            }
        }
    }
\end{lstlisting}
\end{enumerate}
\end{multicols}
\end{enumerate}
\end{Exercice}
\bigskip

%%%%%%%%%%%%%%%%%%%%%%%%%%%%%%%%%%%%%%%%%%%%%%%%%%%%%%%%%%%%%%%%%%%%%%%%
%%%%%%%%%%%%%%%%%%%%%%%%%%%%%%%%%%%%%%%%%%%%%%%%%%%%%%%%%%%%%%%%%%%%%%%%
%%%%%%%%%%%%%%%%%%%%%%%%%%%%%%%%%%%%%%%%%%%%%%%%%%%%%%%%%%%%%%%%%%%%%%%%
\begin{Exercice} {\bf (Structures de boucle)}\smallskip
\begin{enumerate}
    \item Rappeler quelles sont les structures de boucle et
    expliquer leur utilité.
    \smallskip

    \item Expliquer dans quelles circonstances il est préférable
    d'utiliser une boucle {\tt for} plutôt qu'une boucle {\tt while}
    et réciproquement. Donner des exemples pour illustrer le propos.
\end{enumerate}
\end{Exercice}
\bigskip

%%%%%%%%%%%%%%%%%%%%%%%%%%%%%%%%%%%%%%%%%%%%%%%%%%%%%%%%%%%%%%%%%%%%%%%%
%%%%%%%%%%%%%%%%%%%%%%%%%%%%%%%%%%%%%%%%%%%%%%%%%%%%%%%%%%%%%%%%%%%%%%%%
%%%%%%%%%%%%%%%%%%%%%%%%%%%%%%%%%%%%%%%%%%%%%%%%%%%%%%%%%%%%%%%%%%%%%%%%
\begin{Exercice} {\bf (Boucles et affichage de motifs)}\smallskip

Écrire, en utilisant judicieusement des boucles {\tt for},
{\tt while} ou encore {\tt do while}, les fonctions suivantes.
\begin{enumerate}
    \item ~
\begin{lstlisting}
void afficher_drapeau(int n);
\end{lstlisting}
    qui produit la sortie suivante
    (donnée ici dans le cas ${\tt n} = 4$)~:
    \begin{center}
        \begin{tabular}{cccc}
            - & - & - & - \\
            * & - & - & - \\
            * & * & - & - \\
            * & * & * & - \\
            * & * & * & *
        \end{tabular}
    \end{center}

    \item ~
\begin{lstlisting}
void afficher_damier(int n);
\end{lstlisting}
    qui produit la sortie suivante
    (donnée ici dans le cas ${\tt n} = 4$)~:
    \begin{center}
        \begin{tabular}{cccc}
            * & - & * & - \\
            - & * & - & * \\
            * & - & * & - \\
            - & * & - & *
        \end{tabular}
    \end{center}

    \item ~
\begin{lstlisting}
void afficher_triangle(int n);
\end{lstlisting}
    qui produit la sortie suivante
    (donnée ici dans le cas ${\tt n} = 6$)~:
    \begin{center}
        \begin{tabular}{cccccccccccccccc}
            * \\
            * & * \\
            * & * & * & * \\
            * & * & * & * & * & * & * \\
            * & * & * & * & * & * & * & * & * & * & * \\
            * & * & * & * & * & * & * & * & * & * & * & * & * & * & * & * \\
        \end{tabular}
    \end{center}
\end{enumerate}
\end{Exercice}
\bigskip

%%%%%%%%%%%%%%%%%%%%%%%%%%%%%%%%%%%%%%%%%%%%%%%%%%%%%%%%%%%%%%%%%%%%%%%%
%%%%%%%%%%%%%%%%%%%%%%%%%%%%%%%%%%%%%%%%%%%%%%%%%%%%%%%%%%%%%%%%%%%%%%%%
%%%%%%%%%%%%%%%%%%%%%%%%%%%%%%%%%%%%%%%%%%%%%%%%%%%%%%%%%%%%%%%%%%%%%%%%
\begin{Exercice} {\bf (Suite de Syracuse et récursivité)}\smallskip

La {\em suite de Syracuse} est une suite
$\left(S^{(n)}_i\right)_{i \geq 0}$ d'entiers dépendant d'un paramètre~$n$
définie de la manière suivante~:
\begin{equation*}
    S^{(n)}_i :=
    \begin{cases}
        n & \mbox{si $i = 0$}, \\[.75em]
        \dfrac{1}{2}\, S^{(n)}_{i - 1}
            & \mbox{si $S^{(n)}_{i - 1}$ est pair}, \\[.75em]
        3\, S^{(n)}_{i - 1} + 1 & \mbox{sinon}.
    \end{cases}
\end{equation*}
\smallskip

Une conjecture célèbre énonce que pour tout entier $n \geq 1$, il existe
un entier $i \geq 0$ tel que $S_i = 1$. Le statut de cette assertion 
demeure encore inconnu aujourd'hui (2016).
\smallskip

\begin{enumerate}
    \item Écrire un programme qui demande à l'utilisateur d'entrer au
    clavier un entier {\tt n} et qui affiche les éléments de la suite
    $\left(S^{(n)}_i\right)_{i \geq 0}$ et s'arrête dès qu'un
    terme est égal à~$1$.
    \item Expliquer si le programme précédent est un algorithme.
\end{enumerate}
\end{Exercice}
\bigskip

%%%%%%%%%%%%%%%%%%%%%%%%%%%%%%%%%%%%%%%%%%%%%%%%%%%%%%%%%%%%%%%%%%%%%%%%
%%%%%%%%%%%%%%%%%%%%%%%%%%%%%%%%%%%%%%%%%%%%%%%%%%%%%%%%%%%%%%%%%%%%%%%%
%%%%%%%%%%%%%%%%%%%%%%%%%%%%%%%%%%%%%%%%%%%%%%%%%%%%%%%%%%%%%%%%%%%%%%%%
\begin{Exercice} {\bf (Déclaration de fonctions)}\smallskip

\begin{enumerate}
    \item Identifier et donner les différentes parties
    (identificateur, signature, type de retour, instructions) de la
    fonction suivante~:
    \begin{lstlisting}
float aire(int a, int b) {
    return (.0 + a * b) / 2;
}
    \end{lstlisting}
    \smallskip

    \item Déclarer une fonction testant la primalité d'un entier.
    \smallskip

    \item Déclarer une fonction paramétrée par une chaîne de caractères
    et deux caractères. Cette fonction remplace les occurrences du
    1\ier{} caractère par le 2\ieme{} dans la chaîne de caractères.
    \smallskip

    \item Déclarer une fonction qui joue une note dans le terminal.
    La fonction accepte comme arguments la fréquence en Hz de
    la note à jouer ainsi que sa durée en ms.
\end{enumerate}
\end{Exercice}
\bigskip

%%%%%%%%%%%%%%%%%%%%%%%%%%%%%%%%%%%%%%%%%%%%%%%%%%%%%%%%%%%%%%%%%%%%%%%%
%%%%%%%%%%%%%%%%%%%%%%%%%%%%%%%%%%%%%%%%%%%%%%%%%%%%%%%%%%%%%%%%%%%%%%%%
%%%%%%%%%%%%%%%%%%%%%%%%%%%%%%%%%%%%%%%%%%%%%%%%%%%%%%%%%%%%%%%%%%%%%%%%
\begin{Exercice} {\bf (Pile et fonctions)}\smallskip

\begin{enumerate}
    \item Schématiser l'état de la pile à chaque instant de l'exécution
    de l'instruction
    \begin{lstlisting}
afficher(7, 2);
    \end{lstlisting}
    avec les définitions suivantes~:
    \begin{multicols}{2}
\begin{lstlisting}
void afficher(int larg, int haut) {
    int i;
    for (i = 1 ; i <= haut ; ++i) {
        afficher_ligne(larg);
        printf("\n");
    }
}
\end{lstlisting}
\begin{lstlisting}
void afficher_ligne(int larg) {
    int i;
    for (i = 1 ; i <= larg ; ++i)
        printf("*");
}
\end{lstlisting}
    \end{multicols}
    \smallskip

    \item Dessiner l'arbre des appels récursifs puis schématiser l'état 
    de la pile à chaque instant de l'exécution de l'instruction
    \begin{lstlisting}
tribo(5);
    \end{lstlisting}
    avec la définitions suivante~:
\begin{lstlisting}
int tribo(int n) {
    if (n <= 2)
        return n;
    return tribo(n - 1) + tribo(n - 2) + tribo(n - 3);
}
\end{lstlisting}
\end{enumerate}
\end{Exercice}
\bigskip

\end{document}
