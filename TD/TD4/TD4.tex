% Auteur : Samuele Giraudo
% Création : 27-01-16
% Modifications : 27-01-16, 20-03-16

\documentclass[12pt]{article}

\usepackage[utf8]{inputenc}
\usepackage[french]{babel}
\usepackage{amsmath,amsthm,amsfonts,amssymb}
\usepackage{lmodern}
\usepackage[top=2.4cm,bottom=2.4cm,left=2cm,right=2cm]{geometry}
\usepackage{hyperref}
\usepackage{multicol}
\usepackage{enumitem}
\usepackage{listings}
\usepackage[dvipsnames]{xcolor}
\usepackage{tikz}

%\author{Samuele Giraudo}
\date{}
\title{{\bf Perfectionnement à la programmation en {\sf C}} \\
    Fiche de TD 4 \\
    {\small L2 Informatique 2015-2016} \\
    {\it \small Données et arithmétique des pointeurs}}

\theoremstyle{definition}
\newtheorem{Exercice}{Exercice}

\newcommand{\EnsNat}{\mathbb{N}}
\newcommand{\La}{{\tt a}}
\newcommand{\Lb}{{\tt b}}
\newcommand{\Lc}{{\tt c}}
\newcommand{\Ld}{{\tt d}}

% Configuration de listings.
\lstset{
    language=c,
    basicstyle=\ttfamily\footnotesize,
    identifierstyle=\color{Mahogany},
    keywordstyle=\color{NavyBlue},
    stringstyle=\color{Emerald},
    commentstyle=\it\color{Gray},
    columns=flexible,
    tabsize=4,
    extendedchars=true,
    showspaces=false,
    numbers=left,
    numberstyle=\tiny,
    breaklines=true,
    breakautoindent=true,
    captionpos=b,
    showstringspaces=true
}

\begin{document}

\maketitle

%%%%%%%%%%%%%%%%%%%%%%%%%%%%%%%%%%%%%%%%%%%%%%%%%%%%%%%%%%%%%%%%%%%%%%%%
%%%%%%%%%%%%%%%%%%%%%%%%%%%%%%%%%%%%%%%%%%%%%%%%%%%%%%%%%%%%%%%%%%%%%%%%
%%%%%%%%%%%%%%%%%%%%%%%%%%%%%%%%%%%%%%%%%%%%%%%%%%%%%%%%%%%%%%%%%%%%%%%%
\begin{Exercice} {\bf (Conversions)}\smallskip

\begin{enumerate}
    \item Convertir en binaire les valeurs suivantes exprimées en base
    dix. Les résultats sont à donner sur $16$ bits.
    \begin{multicols}{4}
    \begin{enumerate}
        \item $0$~;
        \item $1$~;
        \item $10$~;
        \item $2224$~;
        \item $3200$~;
        \item $-0$~;
        \item $-1$~;
        \item $-10$~;
        \item $-16$~;
        \item $-32$~;
        \item $-2224$~;
        \item $-3200$.
    \end{enumerate}
    \end{multicols}
    \smallskip

    \item Convertir en hexadécimal les valeurs suivantes exprimées en
    base deux. Les résultats sont à donner sur huit chiffres hexadécimaux.
    \begin{multicols}{2}
    \begin{enumerate}
        \item {\tt 0011101010000001}~;
        \item {\tt 1}~;
        \item {\tt 11111111001001011111}~;
        \item {\tt 00010000}.
    \end{enumerate}
    \end{multicols}
    \smallskip

    \item Convertir en binaire les valeurs suivantes exprimées en
    hexadécimal. Les résultats sont à donner sur $16$ bits.
    \begin{multicols}{2}
    \begin{enumerate}
        \item {\tt 1010}~;
        \item {\tt ABC}~;
        \item {\tt F0B}~;
        \item {\tt 123A}.
    \end{enumerate}
    \end{multicols}

    \item Pour chaque déclaration et affectation de variable suivante,
    dessiner la zone de la mémoire impliquée ainsi que son contenu octet
    par octet.
    \begin{multicols}{3}
    \begin{enumerate}
        \item ~
\begin{lstlisting}
unsigned int x;
x = 0;
\end{lstlisting}
        \smallskip

        \item ~
\begin{lstlisting}
char x;
x = 'a';
\end{lstlisting}
        \smallskip

        \item ~
\begin{lstlisting}
int x;
x = 2224;
\end{lstlisting}
        \smallskip

        \item ~
\begin{lstlisting}
int x;
x = -2224;
\end{lstlisting}
        \smallskip

        \item ~
\begin{lstlisting}
short x;
x = -10;
\end{lstlisting}
        \smallskip

        \item ~
\begin{lstlisting}
unsigned short x;
x = -10;
\end{lstlisting}
        \smallskip

    \end{enumerate}
    \end{multicols}
\end{enumerate}
\end{Exercice}
\bigskip

%%%%%%%%%%%%%%%%%%%%%%%%%%%%%%%%%%%%%%%%%%%%%%%%%%%%%%%%%%%%%%%%%%%%%%%%
%%%%%%%%%%%%%%%%%%%%%%%%%%%%%%%%%%%%%%%%%%%%%%%%%%%%%%%%%%%%%%%%%%%%%%%%
%%%%%%%%%%%%%%%%%%%%%%%%%%%%%%%%%%%%%%%%%%%%%%%%%%%%%%%%%%%%%%%%%%%%%%%%
\begin{Exercice} {\bf (Lecture de zones mémoire)}\smallskip

Soit la suite d'instructions suivante~:
\begin{lstlisting}
unsigned int *ptr_int;
unsigned short *ptr_short;
unsigned char *ptr_char;
unsigned int a;
a = 0xBEA0201F;
\end{lstlisting}
\begin{enumerate}
    \item Expliquer ce qu'affichent les instructions
    \begin{multicols}{2}
    \begin{enumerate}
        \item ~
\begin{lstlisting}
ptr_int = &a;
printf("%x\n", *ptr_int);
\end{lstlisting}
    \smallskip

    \item ~
\begin{lstlisting}
ptr_short = (unsigned short *) &a;
printf("%x\n", *ptr_short);
\end{lstlisting}
    \smallskip

    \item ~
\begin{lstlisting}
ptr_char = (unsigned char *) &a;
printf("%x\n", *ptr_char);
\end{lstlisting}
    \smallskip

    \item ~
\begin{lstlisting}
ptr_short = (unsigned short *) &a;
printf("%x\n", *(ptr_short + 1));
\end{lstlisting}
    \smallskip

    \item ~
\begin{lstlisting}
ptr_char = (unsigned char *) &a;
printf("%x\n", *(ptr_char + 1));
\end{lstlisting}
    \smallskip

    \item ~
\begin{lstlisting}
ptr_char = (unsigned char *) &a;
printf("%x\n", *(ptr_char + 2));
\end{lstlisting}
    \end{enumerate}
    \end{multicols}

    \item {\em (Question facultative.)} Reprendre la question
    précédente dans le cas où l'on supprime les quatre occurrences
    de {\tt unsigned} dans la suite d'instructions.
\end{enumerate}
\end{Exercice}
\bigskip

%%%%%%%%%%%%%%%%%%%%%%%%%%%%%%%%%%%%%%%%%%%%%%%%%%%%%%%%%%%%%%%%%%%%%%%%
%%%%%%%%%%%%%%%%%%%%%%%%%%%%%%%%%%%%%%%%%%%%%%%%%%%%%%%%%%%%%%%%%%%%%%%%
%%%%%%%%%%%%%%%%%%%%%%%%%%%%%%%%%%%%%%%%%%%%%%%%%%%%%%%%%%%%%%%%%%%%%%%%
\begin{Exercice} {\bf (Lecture de zones mémoire et tableaux)}\smallskip

Soit la suite d'instructions suivante~:
\begin{lstlisting}
    unsigned char tab[8] = {0xAA, 0x10, 0x3B, 0x44, 0x21, 0x45, 0x00, 0x7C};
    unsigned char *ptr_char;
    unsigned short *ptr_short;
    unsigned int *ptr_int;
\end{lstlisting}
\begin{enumerate}
    \item Expliquer ce qu'affichent les instructions
    \begin{multicols}{2}
    \begin{enumerate}
        \item ~
\begin{lstlisting}
ptr_char = tab;
printf("%x\n", *ptr_char);
\end{lstlisting}
    \smallskip

    \item ~
\begin{lstlisting}
ptr_short = (unsigned short *) tab;
printf("%x\n", *ptr_short);
\end{lstlisting}
    \smallskip

    \item ~
\begin{lstlisting}
ptr_int = (unsigned int *) tab;
printf("%x\n", *ptr_int);
\end{lstlisting}
    \smallskip

    \item ~
\begin{lstlisting}
ptr_short = (unsigned short *) tab;
printf("%x\n", *(ptr_short + 1));
\end{lstlisting}
    \smallskip

    \item ~
\begin{lstlisting}
ptr_char = tab;
printf("%x\n", *(ptr_char + 1));
\end{lstlisting}
    \smallskip

    \item ~
\begin{lstlisting}
ptr_char = tab;
printf("%x\n", *(ptr_char + 2));
\end{lstlisting}
    \end{enumerate}
    \end{multicols}

    \item {\em (Question facultative.)} Reprendre la question
    précédente dans le cas où l'on supprime les quatre occurrences
    de {\tt unsigned} dans la suite d'instructions.
\end{enumerate}
\end{Exercice}
\bigskip

%%%%%%%%%%%%%%%%%%%%%%%%%%%%%%%%%%%%%%%%%%%%%%%%%%%%%%%%%%%%%%%%%%%%%%%%
%%%%%%%%%%%%%%%%%%%%%%%%%%%%%%%%%%%%%%%%%%%%%%%%%%%%%%%%%%%%%%%%%%%%%%%%
%%%%%%%%%%%%%%%%%%%%%%%%%%%%%%%%%%%%%%%%%%%%%%%%%%%%%%%%%%%%%%%%%%%%%%%%
\begin{Exercice} {\bf (Opérations sur les pointeurs)}\smallskip

Pour chacune des suites d'instructions suivantes, donner ligne par ligne
l'état des variables déclarées. Présenter la solution sous forme de
tableau et dessiner, étape par étape, l'état de la mémoire.
\begin{multicols}{4}
\begin{enumerate}[label = ({\alph*})]
\item ~
\begin{lstlisting}
int a, b;
int *p;

a = 10;
p = &a;
b = *p + 2;
*p = *p + 4;
a = *p;
\end{lstlisting}
\bigskip

\item ~
\begin{lstlisting}
int a, b;
int *p1, *p2;

a = 3;
p2 = &b;
*p2 = a + 1;
p1 = p2;
*p1 = 5;
\end{lstlisting}
\bigskip

\item ~
\begin{lstlisting}
int tab[10];
int *p1, *p2;

tab[0] = 4;
tab[3] = 2;
p1 = &tab[0];
tab[1] = *p1;
p2 = p1 + 3;
tab[2] = *p2;
\end{lstlisting}

\item ~
\begin{lstlisting}
int a, b;
int *p1;
int **p2;

p1 = &a;
p2 = &p1;
**p2 = 3;
*p2 = &b;
*p1 = 4;
\end{lstlisting}
\end{enumerate}
\end{multicols}
\end{Exercice}
\bigskip

%%%%%%%%%%%%%%%%%%%%%%%%%%%%%%%%%%%%%%%%%%%%%%%%%%%%%%%%%%%%%%%%%%%%%%%%
%%%%%%%%%%%%%%%%%%%%%%%%%%%%%%%%%%%%%%%%%%%%%%%%%%%%%%%%%%%%%%%%%%%%%%%%
%%%%%%%%%%%%%%%%%%%%%%%%%%%%%%%%%%%%%%%%%%%%%%%%%%%%%%%%%%%%%%%%%%%%%%%%
\begin{Exercice} {\bf (R-values et L-values)}\smallskip
\begin{enumerate}
    \item Rappeler ce qu'est une {\em R-value}.
    \smallskip

    \item Rappeler ce qu'est une {\em L-value}.
    \smallskip

    \item On suppose que {\tt num} est une variable de {\tt int},
    {\tt ptr} est un pointeur sur une variable de type {\tt int},
    {\tt fct\_1} est une fonction paramétrée par un entier et de type
    de retour {\tt int} et {\tt fct\_2} est une fonction paramétrée par
    un pointeur sur un entier et de type de retour {\tt int}. Dans les
    expressions suivantes, pour chaque occurrence de ces identificateurs,
    déterminer s'il s'agit de R-values et/ou de L-values~:
    \begin{enumerate}[label = ({\alph*})]
        \begin{multicols}{3}
            \item {\tt 23}
            \item {\tt num}
            \item {\tt ptr}
            \item {\tt *ptr}
            \item {\tt num = num + 1;}
            \item {\tt *ptr = num;}
            \item {\tt ptr = ptr + 2;}
            \item {\tt *(ptr + 2)}
            \item {\tt *(ptr + 2) = 3;}
            \item {\tt \&num}
            \item {\tt ptr = \&num;}
            \item {\tt *\&ptr}
            \item {\tt fct\_1(num)}
            \item {\tt fct\_2(\&num)}
            \item {\tt fct\_2(ptr)}
        \end{multicols}
    \end{enumerate}
\end{enumerate}
\end{Exercice}
\bigskip

\end{document}
