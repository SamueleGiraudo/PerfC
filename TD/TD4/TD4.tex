% Auteur : Samuele Giraudo
% Création : 27-01-16
% Modifications : 27-01-16

\documentclass[12pt]{article}

\usepackage[utf8]{inputenc}
\usepackage[french]{babel}
\usepackage{amsmath,amsthm,amsfonts,amssymb}
\usepackage{lmodern}
\usepackage[top=2.4cm,bottom=2.4cm,left=2cm,right=2cm]{geometry}
\usepackage{hyperref}
\usepackage{multicol}
\usepackage{enumitem}
\usepackage{listings}
\usepackage[dvipsnames]{xcolor}
\usepackage{tikz}

%\author{Samuele Giraudo}
\date{}
\title{{\bf Perfectionnement à la programmation en {\sf C}} \\
    Fiche de TD 4 \\
    {\small L2 Informatique 2015-2016} \\
    {\it \small Pointeurs et allocation dynamique}}

\theoremstyle{definition}
\newtheorem{Exercice}{Exercice}

\newcommand{\EnsNat}{\mathbb{N}}
\newcommand{\La}{{\tt a}}
\newcommand{\Lb}{{\tt b}}
\newcommand{\Lc}{{\tt c}}
\newcommand{\Ld}{{\tt d}}

% Configuration de listings.
\lstset{
    language=c,
    basicstyle=\ttfamily\footnotesize,
    identifierstyle=\color{Mahogany},
    keywordstyle=\color{NavyBlue},
    stringstyle=\color{Emerald},
    commentstyle=\it\color{Gray},
    columns=flexible,
    tabsize=4,
    extendedchars=true,
    showspaces=false,
    numbers=left,
    numberstyle=\tiny,
    breaklines=true,
    breakautoindent=true,
    captionpos=b,
    showstringspaces=true
}

\begin{document}

\maketitle

%%%%%%%%%%%%%%%%%%%%%%%%%%%%%%%%%%%%%%%%%%%%%%%%%%%%%%%%%%%%%%%%%%%%%%%%
%%%%%%%%%%%%%%%%%%%%%%%%%%%%%%%%%%%%%%%%%%%%%%%%%%%%%%%%%%%%%%%%%%%%%%%%
%%%%%%%%%%%%%%%%%%%%%%%%%%%%%%%%%%%%%%%%%%%%%%%%%%%%%%%%%%%%%%%%%%%%%%%%
\begin{Exercice} {\bf (Conversions)}\smallskip

\begin{enumerate}
    \item Convertir en binaire les valeurs suivantes exprimées en base 
    dix. Les résultats sont a donner sur $16$ bits.
    \begin{multicols}{3}
    \begin{enumerate}
        \item $0$~;
        \item $1$~;
        \item $10$~;
        \item $2224$~;
        \item $3200$~;
        \item $-0$~;
        \item $-1$~;
        \item $-10$~;
        \item $-16$~;
        \item $-32$~;
        \item $-2224$~;
        \item $-3200$.
    \end{enumerate}
    \end{multicols}
    \smallskip
    
    \item Convertir en hexadécimal les valeurs suivantes exprimées en
    base deux. Les résultats sont à donner sur huit chiffres hexadécimaux.
    \begin{multicols}{2}
    \begin{enumerate}
        \item {\tt 0011101010000001}~;
        \item {\tt 1}~;
        \item {\tt 11111111001001011111}~;
        \item {\tt 00010000}.
    \end{enumerate}
    \end{multicols}
    \smallskip
    
    \item Convertir en binaire les valeurs suivantes exprimées en 
    hexadécimal. Les résultats sont à donner sur $16$ bits.
    \begin{multicols}{2}
    \begin{enumerate}
        \item {\tt 1010}~;
        \item {\tt ABC}~;
        \item {\tt F0B}~;
        \item {\tt 123A}.
    \end{enumerate}
    \end{multicols}
\end{enumerate}
\end{Exercice}
\bigskip

%%%%%%%%%%%%%%%%%%%%%%%%%%%%%%%%%%%%%%%%%%%%%%%%%%%%%%%%%%%%%%%%%%%%%%%%
%%%%%%%%%%%%%%%%%%%%%%%%%%%%%%%%%%%%%%%%%%%%%%%%%%%%%%%%%%%%%%%%%%%%%%%%
%%%%%%%%%%%%%%%%%%%%%%%%%%%%%%%%%%%%%%%%%%%%%%%%%%%%%%%%%%%%%%%%%%%%%%%%
\begin{Exercice} {\bf (Lecture de zones mémoire)}\smallskip

Soient la suite d'instructions suivante~:
\begin{lstlisting}
unsigned int *ptr_int;
unsigned short *ptr_short;
unsigned char *ptr_char;
unsigned int a;
a = 0xBEA0201F;
\end{lstlisting}
\begin{enumerate}
    \item Expliquer ce qu'affichent les instructions
    \begin{multicols}{2}
    \begin{enumerate}
        \item ~
\begin{lstlisting}
ptr_int = &a;
printf("%x\n", *ptr_int);
\end{lstlisting}
    \smallskip
    
    \item ~
\begin{lstlisting}
ptr_short = (short *) &a;
printf("%x\n", *ptr_short);
\end{lstlisting}
    \smallskip
    
    \item ~
\begin{lstlisting}
ptr_char = (char *) &a;
printf("%x\n", *ptr_char);
\end{lstlisting}
    \smallskip
    
    \item ~
\begin{lstlisting}
ptr_short = (short *) &a;
printf("%x\n", *(ptr_short + 1));
\end{lstlisting}
    \smallskip
    
    \item ~
\begin{lstlisting}
ptr_char = (char *) &a;
printf("%x\n", *(ptr_char + 1));
\end{lstlisting}
    \smallskip
    
    \item ~
\begin{lstlisting}
ptr_char = (char *) &a;
printf("%x\n", *(ptr_char + 2));
\end{lstlisting}
    \end{enumerate}
    \end{multicols}
    
    \item {\em (Question facultative.)} Reprendre la question 
    précédente dans le cas où l'on supprime les quatre occurrences 
    de {\tt unsigned} dans la suite d'instructions.
\end{enumerate}
\end{Exercice}
\bigskip

%%%%%%%%%%%%%%%%%%%%%%%%%%%%%%%%%%%%%%%%%%%%%%%%%%%%%%%%%%%%%%%%%%%%%%%%
%%%%%%%%%%%%%%%%%%%%%%%%%%%%%%%%%%%%%%%%%%%%%%%%%%%%%%%%%%%%%%%%%%%%%%%%
%%%%%%%%%%%%%%%%%%%%%%%%%%%%%%%%%%%%%%%%%%%%%%%%%%%%%%%%%%%%%%%%%%%%%%%%
\begin{Exercice} {\bf (Lecture de zones mémoire et tableaux)}\smallskip

Soient la suite d'instructions suivante~:
\begin{lstlisting}
    unsigned char tab[8] = {0xAA, 0x10, 0x3B, 0x44, 0x21, 0x45, 0x00, 0x7C};
    unsigned char *ptr_char;
    unsigned short *ptr_short;
    unsigned int *ptr_int;
\end{lstlisting}
\begin{enumerate}
    \item Expliquer ce qu'affichent les instructions
    \begin{multicols}{2}
    \begin{enumerate}
        \item ~
\begin{lstlisting}
ptr_char = tab;
printf("%x\n", *ptr_char);
\end{lstlisting}
    \smallskip
    
    \item ~
\begin{lstlisting}
ptr_short = (short *) tab;
printf("%x\n", *ptr_short);
\end{lstlisting}
    \smallskip
    
    \item ~
\begin{lstlisting}
ptr_int = (int *) tab;
printf("%x\n", *ptr_int);
\end{lstlisting}
    \smallskip
    
    \item ~
\begin{lstlisting}
ptr_short = (short *) tab;
printf("%x\n", *(ptr_short + 1));
\end{lstlisting}
    \smallskip
    
    \item ~
\begin{lstlisting}
ptr_char = tab;
printf("%x\n", *(ptr_char + 1));
\end{lstlisting}
    \smallskip
    
    \item ~
\begin{lstlisting}
ptr_char = tab;
printf("%x\n", *(ptr_char + 2));
\end{lstlisting}
    \end{enumerate}
    \end{multicols}
    
    \item {\em (Question facultative.)} Reprendre la question 
    précédente dans le cas où l'on supprime les quatre occurrences 
    de {\tt unsigned} dans la suite d'instructions.
\end{enumerate}
\end{Exercice}
\bigskip

%%%%%%%%%%%%%%%%%%%%%%%%%%%%%%%%%%%%%%%%%%%%%%%%%%%%%%%%%%%%%%%%%%%%%%%%
%%%%%%%%%%%%%%%%%%%%%%%%%%%%%%%%%%%%%%%%%%%%%%%%%%%%%%%%%%%%%%%%%%%%%%%%
%%%%%%%%%%%%%%%%%%%%%%%%%%%%%%%%%%%%%%%%%%%%%%%%%%%%%%%%%%%%%%%%%%%%%%%%
\begin{Exercice} {\bf (Opérations sur les pointeurs)}\smallskip

Pour chacune des suites d'instructions suivantes, donner ligne par ligne
l'état des variables déclarées. Présenter la solution sous forme de
tableau et dessiner, étape par étape, l'état de la mémoire.
\begin{multicols}{4}
\begin{enumerate}[label = ({\alph*})]
\item ~
\begin{lstlisting}
int a, b;
int *p;

a = 10;
p = &a;
b = *p + 2;
*p = *p + 4;
a = *p;
\end{lstlisting}
\bigskip

\item ~
\begin{lstlisting}
int a, b;
int *p1, *p2;

a = 3;
p2 = &b;
*p2 = a + 1;
p1 = p2;
*p1 = 5;
\end{lstlisting}
\bigskip

\item ~
\begin{lstlisting}
int tab[10];
int *p1, *p2;

tab[0] = 4;
tab[3] = 2;
p1 = &tab[0];
tab[1] = *p1;
p2 = p1 + 3;
tab[2] = *p2;
\end{lstlisting}

\item ~
\begin{lstlisting}
int a, b;
int *p1;
int **p2;

p1 = &a;
p2 = &p1;
**p2 = 3;
*p2 = &b;
*p1 = 4;
\end{lstlisting}
\end{enumerate}
\end{multicols}
\end{Exercice}
\bigskip

%%%%%%%%%%%%%%%%%%%%%%%%%%%%%%%%%%%%%%%%%%%%%%%%%%%%%%%%%%%%%%%%%%%%%%%%
%%%%%%%%%%%%%%%%%%%%%%%%%%%%%%%%%%%%%%%%%%%%%%%%%%%%%%%%%%%%%%%%%%%%%%%%
%%%%%%%%%%%%%%%%%%%%%%%%%%%%%%%%%%%%%%%%%%%%%%%%%%%%%%%%%%%%%%%%%%%%%%%%
\begin{Exercice} {\bf (R-values et L-values)}\smallskip
\begin{enumerate}
    \item Rappeler ce qu'est une {\em R-value}.
    \smallskip

    \item Rappeler ce qu'est une {\em L-value}.
    \smallskip

    \item On suppose que {\tt num} est une variable de {\tt int},
    {\tt ptr} est un pointeur sur une variable de type {\tt int},
    {\tt fct\_1} est une fonction paramétrée par un entier et de type 
    de retour {\tt int} et {\tt fct\_2} est une fonction paramétrée par 
    un pointeur sur un entier et de type de retour {\tt int}. Dans les 
    expressions suivantes, pour chaque occurrence de ces identificateurs,
    déterminer s'il s'agit de R-values et/ou de L-values~:
    \begin{enumerate}[label = ({\alph*})]
        \begin{multicols}{3}
            \item {\tt 23}
            \item {\tt num}
            \item {\tt ptr}
            \item {\tt *ptr}
            \item {\tt num = num + 1;}
            \item {\tt *ptr = num;}
            \item {\tt ptr = ptr + 2;}
            \item {\tt *(ptr + 2)}
            \item {\tt *(ptr + 2) = 3;}
            \item {\tt \&num}
            \item {\tt ptr = \&num;}
            \item {\tt *\&ptr}
            \item {\tt fct\_1(num)}
            \item {\tt fct\_2(\&num)}
            \item {\tt fct\_2(ptr)}
        \end{multicols}
    \end{enumerate}
\end{enumerate}
\end{Exercice}
\bigskip

%%%%%%%%%%%%%%%%%%%%%%%%%%%%%%%%%%%%%%%%%%%%%%%%%%%%%%%%%%%%%%%%%%%%%%%%
%%%%%%%%%%%%%%%%%%%%%%%%%%%%%%%%%%%%%%%%%%%%%%%%%%%%%%%%%%%%%%%%%%%%%%%%
%%%%%%%%%%%%%%%%%%%%%%%%%%%%%%%%%%%%%%%%%%%%%%%%%%%%%%%%%%%%%%%%%%%%%%%%
\begin{Exercice} {\bf (Test d'égalité de zones de la mémoire)}\smallskip
\begin{enumerate}
    \item Écrire une fonction qui permet de tester l'égalité entre deux
    variables d'un type quelconque. La fonction admet le prototype
\begin{lstlisting}
int sont_egales(int nb_octets, void *var1, void *var2);
\end{lstlisting}
    et elle renvoie {\tt 1} si les {\tt nb\_octets} lus à partir des adresses
    {\tt var1} et {\tt var2} sont égaux deux à deux et {\tt 0} sinon.
    \smallskip

    \item On suppose que {\tt num1} et {\tt num2} sont deux variables
    de type {\tt short}. Écrire l'appel à la fonction {\tt sont\_egales}
    pour comparer les valeurs de ces variables.
    \smallskip

    \item On suppose que {\tt res} est une variable de type {\tt int}.
    Pour chacune des suites d'instructions suivantes, expliquer la
    valeur de {\tt res} à la fin de leur exécution.
    \begin{multicols}{2}
        \begin{enumerate}[label = ({\alph*})]
            \item ~
\begin{lstlisting}
int a;
char b;

a = 3;
b = 3;
res = sont_egales(1, &a, &b);
\end{lstlisting}

            \item ~
\begin{lstlisting}
int a;
int b;

a = (1 << 8) + 32;
b = 32;
res = sont_egales(1, &a, &b);
\end{lstlisting}
        \end{enumerate}
    \end{multicols}
\end{enumerate}
\end{Exercice}
\bigskip

%%%%%%%%%%%%%%%%%%%%%%%%%%%%%%%%%%%%%%%%%%%%%%%%%%%%%%%%%%%%%%%%%%%%%%%%
%%%%%%%%%%%%%%%%%%%%%%%%%%%%%%%%%%%%%%%%%%%%%%%%%%%%%%%%%%%%%%%%%%%%%%%%
%%%%%%%%%%%%%%%%%%%%%%%%%%%%%%%%%%%%%%%%%%%%%%%%%%%%%%%%%%%%%%%%%%%%%%%%
\begin{Exercice} {\bf (Tableaux statiques)}\smallskip

On souhaite manipuler une fonction paramétrée par un entier {\tt n} qui
renvoie un pointeur sur un tableau de {\tt n} entiers initialisés à {\tt 0}.
Expliquer pourquoi la fonction suivante ne répond pas à cette spécification.
Expliquer en particulier ce qu'il se passe lorsque cette fonction est
appelée (faire des dessins de mémoire).
\begin{lstlisting}
int *creer_tableau(int n) {
    int tab[n];
    int i;

    for (i = 0 ; i < n ; ++i)
        tab[i] = 0;
    return tab;
}
\end{lstlisting}
\end{Exercice}
\bigskip

%%%%%%%%%%%%%%%%%%%%%%%%%%%%%%%%%%%%%%%%%%%%%%%%%%%%%%%%%%%%%%%%%%%%%%%%
%%%%%%%%%%%%%%%%%%%%%%%%%%%%%%%%%%%%%%%%%%%%%%%%%%%%%%%%%%%%%%%%%%%%%%%%
%%%%%%%%%%%%%%%%%%%%%%%%%%%%%%%%%%%%%%%%%%%%%%%%%%%%%%%%%%%%%%%%%%%%%%%%
\begin{Exercice} {\bf (Tableaux dynamiques à une dimension)}\smallskip

Dans toutes les fonctions suivantes, on prendra soin de vérifier que
les allocations dynamiques (et donc les appels à la fonction {\tt malloc})
ne produisent pas d'erreur.
\begin{enumerate}
    \item Écrire une fonction {\tt creer\_tab} paramétrée par un entier
    {\tt n} et qui renvoie un pointeur sur un tableau de {\tt n} entiers
    initialisés à {\tt 0}.
    \smallskip

    \item Écrire une fonction {\tt detruire\_tab} paramétrée par un
    pointeur {\tt tab} sur un tableau d'entiers et qui libère la place
    mémoire occupée par {\tt tab}.
\end{enumerate}
\end{Exercice}
\bigskip

%%%%%%%%%%%%%%%%%%%%%%%%%%%%%%%%%%%%%%%%%%%%%%%%%%%%%%%%%%%%%%%%%%%%%%%%
%%%%%%%%%%%%%%%%%%%%%%%%%%%%%%%%%%%%%%%%%%%%%%%%%%%%%%%%%%%%%%%%%%%%%%%%
%%%%%%%%%%%%%%%%%%%%%%%%%%%%%%%%%%%%%%%%%%%%%%%%%%%%%%%%%%%%%%%%%%%%%%%%
\begin{Exercice} {\bf (Tableaux dynamiques à deux dimensions)}\smallskip

Dans toutes les fonctions suivantes, on prendra soin de vérifier que
les allocations dynamiques (et donc les appels à la fonction {\tt malloc})
ne produisent pas d'erreur.
\begin{enumerate}
    \item Écrire une fonction {\tt creer\_tab\_2d} paramétrée par des
    entiers {\tt n} et {\tt m} et qui renvoie un pointeur sur un tableau
    à deux dimensions de {\tt n} $\times$ {\tt m} entiers initialisés
    à {\tt 0}.
    \smallskip

    \item Écrire une fonction {\tt detruire\_tab\_2d} paramétrée par un
    entier {\tt n} et un pointeur {\tt tab} sur un tableau à deux
    dimensions de {\tt n} $\times$ {\tt m} entiers (il n'est pas
    nécessaire de connaître {\tt m} ici). Cette fonction doit libérer la
    place mémoire occupée par {\tt tab}.
    \smallskip

    \item Dessiner l'état de la mémoire étape par étape et de manière
    très précise lors de l'exécution des instructions
\begin{lstlisting}
int **tab = creer_tab_2d(2, 3);
detruire_tab_2d(tab, 2);
\end{lstlisting}
    \smallskip

    \item Expliquer comment représenter et gérer un tableau à deux
    dimensions par un tableau à une seule dimension.
\end{enumerate}
\end{Exercice}
\bigskip

%%%%%%%%%%%%%%%%%%%%%%%%%%%%%%%%%%%%%%%%%%%%%%%%%%%%%%%%%%%%%%%%%%%%%%%%
%%%%%%%%%%%%%%%%%%%%%%%%%%%%%%%%%%%%%%%%%%%%%%%%%%%%%%%%%%%%%%%%%%%%%%%%
%%%%%%%%%%%%%%%%%%%%%%%%%%%%%%%%%%%%%%%%%%%%%%%%%%%%%%%%%%%%%%%%%%%%%%%%
\begin{Exercice} {\bf (Tableaux dynamiques en dents de scie)}\smallskip

Dans toutes les fonctions suivantes, on prendra soin de vérifier que
les allocations dynamiques (et donc les appels à la fonction {\tt malloc})
ne produisent pas d'erreur.

\begin{enumerate}
    \item Écrire une fonction {\tt creer\_scie} paramétrée par un entier
    {\tt n} et qui renvoie un pointeur sur un tableau à deux dimensions
    de caractères. Pour tout $0 \leq i \leq {\tt n} - 1$, la $i\ieme$
    case du tableau à construire contient un tableau à une dimension de
    $i$ modulo $5$ plus un caractères {\tt '*'}.
    \smallskip

    \item Écrire une fonction {\tt detruire\_scie} paramétrée par un
    entier {\tt n} et un pointeur {\tt scie} sur un tableau à deux
    dimensions de caractères (pouvant être construit par un appel à
    {\tt creer\_scie}). Cette fonction doit libérer la place mémoire
    occupée par {\tt scie}.
\end{enumerate}
\end{Exercice}
\bigskip

\end{document}
