% Auteur : Samuele Giraudo
% Création : 11-12-15
% Modifications : 11-12-15, déc. 2015

\documentclass[12pt]{article}

\usepackage[utf8]{inputenc}
\usepackage[french]{babel}
\usepackage{amsmath,amsthm,amsfonts,amssymb}
\usepackage{lmodern}
\usepackage[top=2.4cm,bottom=2.4cm,left=2cm,right=2cm]{geometry}
\usepackage{hyperref}
\usepackage{multicol}
\usepackage{enumitem}
\usepackage{listings}
\usepackage[dvipsnames]{xcolor}
\usepackage{tikz}

%\author{Samuele Giraudo}
\date{}
\title{{\bf Perfectionnement à la programmation en {\sf C}} \\
    Fiche de TD 2 \\
    {\small L2 Informatique 2015-2016} \\
    {\it \small Pré-assertions, gestion d'erreurs}}

\theoremstyle{definition}
\newtheorem{Exercice}{Exercice}

\newcommand{\EnsNat}{\mathbb{N}}
\newcommand{\La}{{\tt a}}
\newcommand{\Lb}{{\tt b}}
\newcommand{\Lc}{{\tt c}}
\newcommand{\Ld}{{\tt d}}

% Configuration de listings.
\lstset{
    language=c,
    basicstyle=\ttfamily\footnotesize,
    identifierstyle=\color{Mahogany},
    keywordstyle=\color{NavyBlue},
    stringstyle=\color{Emerald},
    commentstyle=\it\color{Gray},
    columns=flexible,
    tabsize=4,
    extendedchars=true,
    showspaces=false,
    numbers=left,
    numberstyle=\tiny,
    breaklines=true,
    breakautoindent=true,
    captionpos=b,
    showstringspaces=true
}

%%%%%%%%%%%%%%%%%%%%%%%%%%%%%%%%%%%%%%%%%%%%%%%%%%%%%%%%%%%%%%%%%%%%%%%%
%%%%%%%%%%%%%%%%%%%%%%%%%%%%%%%%%%%%%%%%%%%%%%%%%%%%%%%%%%%%%%%%%%%%%%%%
%%%%%%%%%%%%%%%%%%%%%%%%%%%%%%%%%%%%%%%%%%%%%%%%%%%%%%%%%%%%%%%%%%%%%%%%
\begin{document}

\maketitle

%%%%%%%%%%%%%%%%%%%%%%%%%%%%%%%%%%%%%%%%%%%%%%%%%%%%%%%%%%%%%%%%%%%%%%%%
%%%%%%%%%%%%%%%%%%%%%%%%%%%%%%%%%%%%%%%%%%%%%%%%%%%%%%%%%%%%%%%%%%%%%%%%
%%%%%%%%%%%%%%%%%%%%%%%%%%%%%%%%%%%%%%%%%%%%%%%%%%%%%%%%%%%%%%%%%%%%%%%%
\begin{Exercice} {\bf (Déclarations et pré-assertions)}\smallskip

Pour chaque tâche décrite, proposer une déclaration de fonction y
répondant, ainsi qu'une liste de pré-assertions adéquate. Il est 
important de s'interroger sur les entrées de chaque problème ainsi que 
sur sa (ou ses) sortie(s). Pour obtenir les pré-assertions adéquates, il 
est nécessaire de s'interroger sur les entrées qui posent problème.

\begin{enumerate}
    \item Le calcul du quotient de deux entiers~;
    \smallskip
    
    \item le calcul du reste de la division euclidienne de deux entiers 
    $a$ et $b$, compris entre $0$ et $b - 1$ ({\it Attention~: l'opérateur 
    modulo {\tt \%} du {\sf C} peut renvoyer un reste négatif.})~;
    \smallskip
    
    \item le calcul du $n\ieme$ nombre de Fibonacci~;
    \smallskip
    
    \item le calcul de l'image en $x$ d'un polynôme du second degré
    $a_0 + a_1 x + a_2 x^2$~;
    \smallskip
    
    \item l'affichage d'une lettre $a$ minuscule suivie de sa version 
    en majuscule~;
    \smallskip
    
    \item l'affichage d'un rectangle composé de $n$ lignes et de $m$
    colonnes d'étoiles~;
    \smallskip
    
    \item l'échange des valeurs de deux entiers~;
    \smallskip
    
    \item le calcul du nombre de cases contenant des entiers positifs 
    dans un tableau~;
    \smallskip
    
    \item le calcul de la moyenne des valeurs contenues dans un tableau 
    de flottants~;
    \smallskip
    
    \item le calcul du maximum des valeurs contenues dans un tableau 
    d'entiers~;
    \smallskip
    
    \item le calcul de la distance d'un point du plan par rapport à 
    l'origine~;
    \smallskip
    
    \item l'initialisation de chaque case d'un tableau à deux dimensions 
    par une valeur $a$~;
    \smallskip
    
    \item le calcul du nombre de voyelles dans une chaîne de caractères 
    contenant des chiffres, des espaces ou des lettres~;
    \smallskip
    
    \item la création d'une chaîne de caractères obtenue en remplaçant
    toutes ses majuscules par des minuscules~;
    \smallskip
    
    \item la création d'une chaîne de caractères obtenue en remplaçant 
    tout caractère alphabétique $c$ par un caractère alphabétique~$c'$.
\end{enumerate}
\end{Exercice}
\bigskip

%%%%%%%%%%%%%%%%%%%%%%%%%%%%%%%%%%%%%%%%%%%%%%%%%%%%%%%%%%%%%%%%%%%%%%%%
%%%%%%%%%%%%%%%%%%%%%%%%%%%%%%%%%%%%%%%%%%%%%%%%%%%%%%%%%%%%%%%%%%%%%%%%
%%%%%%%%%%%%%%%%%%%%%%%%%%%%%%%%%%%%%%%%%%%%%%%%%%%%%%%%%%%%%%%%%%%%%%%%
\begin{Exercice} {\bf (Gestion d'erreurs 1)}\smallskip
    
Réécrire les fonctions suivantes de sorte à les munir d'un mécanisme 
de gestion d'erreurs. Pour chaque nouvelle fonction ainsi écrite, 
rédiger une documentation appropriée.
\begin{multicols}{2}
\begin{enumerate}
\item~
\begin{lstlisting}
int *allouer_tab_int(int taille) {
    assert(taille >= 1);

    return (int *) 
        malloc(sizeof(int) * taille);
}
\end{lstlisting}

\item~
\begin{lstlisting}
void copier_pointeur(char *source, 
        char *cible) {
    assert(source != NULL);
    
    cible = (char *) 
        malloc(sizeof(char));
    *cible = *source;
}
\end{lstlisting}

%\bigskip
%\bigskip

\item~
\begin{lstlisting}
void repeter_affichage(char *chaine, 
        int nombre) {
    int i;
    
    assert(chaine != NULL);
    assert(nombre >= 0);
    
    for (i = 0 ; i < nombre ; ++i) 
        printf("%s\n", chaine);
}
\end{lstlisting}
%\bigskip
%\bigskip
%\bigskip

\item~
\begin{lstlisting}
void doubler() {
    int entree;
    
    scanf("%d", &entree);
    printf("%d\n", 2 * entree);
}
\end{lstlisting}
\bigskip
\bigskip
\bigskip

\item~
\begin{lstlisting}
int premiere_position(char *chaine, 
        char lettre) {
    int i;

    assert(chaine != NULL);
    
    i = 0;
    while (chaine[i] != '\0') {
        if (chaine[i] == lettre)
            return i;
        i += 1;
    }
    return -1;
}
\end{lstlisting}
\end{enumerate}
\end{multicols}
\end{Exercice}
\bigskip

%%%%%%%%%%%%%%%%%%%%%%%%%%%%%%%%%%%%%%%%%%%%%%%%%%%%%%%%%%%%%%%%%%%%%%%%
%%%%%%%%%%%%%%%%%%%%%%%%%%%%%%%%%%%%%%%%%%%%%%%%%%%%%%%%%%%%%%%%%%%%%%%%
%%%%%%%%%%%%%%%%%%%%%%%%%%%%%%%%%%%%%%%%%%%%%%%%%%%%%%%%%%%%%%%%%%%%%%%%
\begin{Exercice} {\bf (Gestion d'erreurs 2)}\smallskip

Pour chaque fonction ci-dessous, la définir, la munir de pré-assertions,
imaginer les cas d'erreur possibles et mettre en place un mécanisme de
gestion d'erreurs.

\begin{enumerate}
    \item {\tt calculer\_moyenne} paramétrée par 
    un tableau de notes comprises entre {\tt 0} et {\tt 20}, calculant 
    la moyenne des notes~;
    \smallskip
    
    \item {\tt lettre\_frequente} paramétrée par une chaîne de caractères,
    calculant sa lettre strictement plus fréquente~;
    \smallskip
    
    \item {\tt dessiner\_etoiles} paramétrée par un tableau d'entiers 
    et dessinant, pour chaque entier $a$ du tableau lu de la droite vers 
    la gauche, une nouvelle ligne de $a$ occurrences de {\tt '*'} sur la 
    sortie standard~;
    \smallskip
    
    \item {\tt lire\_points\_isobarycentre} qui lit sur l'entrée standard 
    deux points du plan cartésien au format {\tt X Y} et affiche sur 
    la sortie standard leur isobarycentre.
\end{enumerate}
\end{Exercice}
\bigskip

%%%%%%%%%%%%%%%%%%%%%%%%%%%%%%%%%%%%%%%%%%%%%%%%%%%%%%%%%%%%%%%%%%%%%%%%
%%%%%%%%%%%%%%%%%%%%%%%%%%%%%%%%%%%%%%%%%%%%%%%%%%%%%%%%%%%%%%%%%%%%%%%%
%%%%%%%%%%%%%%%%%%%%%%%%%%%%%%%%%%%%%%%%%%%%%%%%%%%%%%%%%%%%%%%%%%%%%%%%
\begin{Exercice} {\bf (Gestion d'erreurs 3)}\smallskip
\begin{enumerate}
    \item Écrire une fonction {\tt est\_nom} paramétrée par une 
    chaîne de caractères qui teste si celle-ci est constituée uniquement 
    de caractères alphabétiques.
    \smallskip
    
    \item Écrire une fonction {\tt demander\_nom} qui lit une chaîne 
    de caractères sur l'entrée standard. Celle-ci doit gérer les erreurs
    qui peuvent survenir (une chaîne de caractère qui n'est pas un nom 
    est entrée ou bien une erreur provient de la fonction de lecture sur 
    l'entrée standard).
    \smallskip
    
    \item Utiliser la fonction {\tt demander\_nom} dans un programme 
    complet pour demander un nom à un utilisateur et l'afficher sur 
    la sortie standard. Lors de l'exécution, le programme demande un nom
    tant que l'utilisateur ne rentre pas un nom valide.
\end{enumerate}
\end{Exercice}
\bigskip

%%%%%%%%%%%%%%%%%%%%%%%%%%%%%%%%%%%%%%%%%%%%%%%%%%%%%%%%%%%%%%%%%%%%%%%%
%%%%%%%%%%%%%%%%%%%%%%%%%%%%%%%%%%%%%%%%%%%%%%%%%%%%%%%%%%%%%%%%%%%%%%%%
%%%%%%%%%%%%%%%%%%%%%%%%%%%%%%%%%%%%%%%%%%%%%%%%%%%%%%%%%%%%%%%%%%%%%%%%
\begin{Exercice} {\bf (Gestion d'erreurs 4)}\smallskip

Reprendre toutes les fonctions écrites dans la fiche de TD 1 et les 
munir d'un mécanisme de gestion d'erreurs lorsque cela est approprié.
On n'oubliera pas de les munir également de pré-assertions adéquates.
\end{Exercice}
\bigskip

\end{document}

