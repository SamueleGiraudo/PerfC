% Auteur : Samuele Giraudo
% Création : 07/02/13
% Modifications : jan. 2015, déc. 2015, mars 2016

\documentclass[12pt]{article}

\usepackage[utf8]{inputenc}
\usepackage[french]{babel}
\usepackage{amsmath,amsthm,amsfonts,amssymb}
\usepackage{lmodern}
\usepackage[top=2.4cm,bottom=2.4cm,left=2cm,right=2cm]{geometry}
\usepackage{hyperref}
\usepackage{multicol}
\usepackage{enumitem}
\usepackage{listings}
\usepackage{tikz}
\usetikzlibrary{fit}

%\author{Samuele Giraudo}
\date{}
\title{{\bf Perfectionnement à la programmation en {\sf C}} \\
    Fiche de TD 3 \\
    {\small L2 Informatique 2015-2016} \\
    {\it \small Programmation modulaire}}

\theoremstyle{definition}
\newtheorem{Exercice}{Exercice}

\newcommand{\EnsNat}{\mathbb{N}}
\newcommand{\La}{{\tt a}}
\newcommand{\Lb}{{\tt b}}
\newcommand{\Lc}{{\tt c}}
\newcommand{\Ld}{{\tt d}}

%\newcommand{\Sol}[1]{{\bf Solution~:} {\it #1}}
\newcommand{\Sol}[1]{}

\begin{document}

\maketitle

%%%%%%%%%%%%%%%%%%%%%%%%%%%%%%%%%%%%%%%%%%%%%%%%%%%%%%%%%%%%%%%%%%%%%%%%
%%%%%%%%%%%%%%%%%%%%%%%%%%%%%%%%%%%%%%%%%%%%%%%%%%%%%%%%%%%%%%%%%%%%%%%%
%%%%%%%%%%%%%%%%%%%%%%%%%%%%%%%%%%%%%%%%%%%%%%%%%%%%%%%%%%%%%%%%%%%%%%%%
\begin{Exercice} {\bf (Projets et graphes d'inclusions)} \smallskip

    On considère un projet constitué de cinq modules {\tt A}, {\tt B},
    {\tt C}, {\tt D} et {\tt E}. Ces modules sont utilisés dans un
    fichier {\tt Main.c} qui contient la fonction {\tt main}. Dans ce
    projet figurent les inclusions suivantes~:
    \begin{multicols}{2}
    \begin{itemize}
        \item {\tt A.h} inclut {\tt B.h}, {\tt C.h} et {\tt E.h}~;
        \item {\tt A.c} inclut {\tt D.h}~;
        \item {\tt B.h} inclut {\tt C.h}~;
        \item {\tt D.h} inclut {\tt A.h}~;
        \item {\tt E.c} inclut {\tt D.h}~;
        \item {\tt Main.c} inclut {\tt A.h} et {\tt E.h}.
    \end{itemize}
    \end{multicols}

    \begin{enumerate}
        \item Donner le nombre de fichiers qui constituent le projet.
        \smallskip

        \item Donner la liste des commandes qui permettent de compiler
        le projet.
        \smallskip

        \item Dire si l'ordre des commandes de compilation est important
        et justifier pourquoi.
        \smallskip

        \item Tracer le graphe d'inclusions (étendues) du projet.
        \smallskip

        \item Expliquer, graphe d'inclusions (étendues) à l'appui, si le
        projet est bien structuré. Mettre en évidence les éventuels
        problèmes qu'il contient.
        \smallskip

        \item Supposons (uniquement pour cette question) que {\tt B.h}
        inclut {\tt D.h}. Expliquer, graphe d'inclusions (étendues) à
        l'appui, si le projet est bien structuré. Mettre en évidence les
        éventuels problèmes qu'il contient.
        \smallskip

        \item Supposons (uniquement pour cette question) que {\tt B.c}
        inclut {\tt D.h}. Expliquer, graphe d'inclusions (étendues) à
        l'appui, si le projet est bien structuré. Mettre en évidence les
        éventuels problèmes qu'il contient.
        \smallskip

        \item Supposons (uniquement pour cette question) que {\tt B.h}
        inclut {\tt D.h}. On suppose également que dans l'en-tête du
        module {\tt B} est déclaré un type {\tt B\_type} et que dans
        l'en-tête du module {\tt A} est déclarée une fonction de prototype
        {\tt void a\_fct(B\_type x)}. Expliquer pourquoi la commande
        {\tt gcc -c B.c} provoque une erreur. Expliquer pourquoi la
        commande {\tt gcc -c A.c} produit bien un fichier objet.
        \smallskip

        \item Écrire un {\tt Makefile} complet pour compiler le projet.
    \end{enumerate}
\end{Exercice}
\bigskip

%%%%%%%%%%%%%%%%%%%%%%%%%%%%%%%%%%%%%%%%%%%%%%%%%%%%%%%%%%%%%%%%%%%%%%%%
%%%%%%%%%%%%%%%%%%%%%%%%%%%%%%%%%%%%%%%%%%%%%%%%%%%%%%%%%%%%%%%%%%%%%%%%
%%%%%%%%%%%%%%%%%%%%%%%%%%%%%%%%%%%%%%%%%%%%%%%%%%%%%%%%%%%%%%%%%%%%%%%%
\begin{Exercice} {\bf (Découpage d'un projet en modules~: le Gomoku)}\smallskip

Le but de cet exercice est de réaliser l'analyse d'un petit projet en le
découpant en modules et en écrivant ses fichiers d'en-tête. On ne donnera
pas l'implantation des modules.
\smallskip

Le projet consiste  à réaliser un jeu de {\em Gomoku}\footnote{En réalité,
il  s'agit ici d'une variante du Gomoku.}. Une partie se joue à deux joueurs
sur un plateau de $19 \times 19$ cases. À chaque tour de jeu, le joueur
qui a le trait (Blanc ou Noir) choisit une case du plateau de jeu et y
pose un pion de sa couleur. Le premier joueur qui réussit à aligner $5$
pions de sa couleur gagne la partie. Si le plateau est rempli de pions
sans aucune configuration de gain, la partie est déclarée nulle. Le joueur
Noir commence toujours la partie.
\smallskip

Au lancement du programme, il est de possible de~:
\begin{itemize}
    \item commencer une nouvelle partie~;
    \smallskip

    \item charger une partie depuis un fichier~;
    \smallskip

    \item quitter le programme.
\end{itemize}
Au cours de la partie, il est possible de~:
\begin{itemize}
    \item jouer un coup (par le joueur qui possède le trait)~;
    \smallskip

    \item abandonner la partie et céder ainsi la victoire à son
    adversaire~;
    \smallskip

    \item sauvegarder la partie en cours (sans quitter le programme, la
    partie continue)~;
    \smallskip

    \item quitter la partie en cours et revenir au menu de lancement du
    programme.
\end{itemize}
Lorsque la partie s'arrête (gain d'un joueur ou partie nulle), le résultat
est affiché, suivi du menu de lancement du programme.
\smallskip

Les parties sont sauvegardées dans un fichier selon un format choisi par
le programmeur. Ce format doit permettre de représenter toutes les
informations d'une partie pour que, à partir d'un fichier de sauvegarde,
le programme puisse entièrement restaurer la partie qu'il représente. De
plus, l'interaction avec les utilisateurs se fait par le biais de l'entrée
et de la sortie standard.
\medskip

\begin{enumerate}
    \item Découper ce projet en modules. Ce découpage doit permettre
    de le faire facilement évoluer. Expliquer comment prendre en compte
    les variantes suivantes~:
    \begin{itemize}
        \item changement du format des fichiers de sauvegarde~;
        \smallskip

        \item changement des règles ($6$ pions alignés pour le gain au
        lieu de $5$, taille du plateau $21 \times 21$ au lieu
        de $19 \times 19$, trois joueurs au lieu de deux)~;
        \smallskip

        \item affichage graphique à la place d'un affichage sur la sortie
        standard~;
        \smallskip

        \item ajout d'un mode de jeu humain {\em vs} machine.
    \end{itemize}
    Pour chacune des ces quatre évolutions, expliquer leur impact sur
    l'ensemble des modules du projet.
    \smallskip

    \item Déterminer les types et les prototypes des fonctions que chaque
    module doit contenir. En déduire les en-têtes des modules du projet.
\end{enumerate}
\end{Exercice}
\bigskip

%%%%%%%%%%%%%%%%%%%%%%%%%%%%%%%%%%%%%%%%%%%%%%%%%%%%%%%%%%%%%%%%%%%%%%%%
%%%%%%%%%%%%%%%%%%%%%%%%%%%%%%%%%%%%%%%%%%%%%%%%%%%%%%%%%%%%%%%%%%%%%%%%
%%%%%%%%%%%%%%%%%%%%%%%%%%%%%%%%%%%%%%%%%%%%%%%%%%%%%%%%%%%%%%%%%%%%%%%%
\begin{Exercice} {\bf (Modélisation d'un projet~: polynômes)} \smallskip

    On souhaite réaliser un projet qui permet de manipuler des polynômes
    sur une variable~$x$ et à coefficients flottants ({\tt double}).
    Plus précisément, le programme doit pouvoir~:
    \begin{itemize}
        \item lire un polynôme à partir d'un fichier, calculer sa
        dérivée par rapport à~$x$ et afficher le résultat sur la sortie standard~;
        \smallskip

        \item lire un polynôme à partir d'un fichier, calculer sa $n\ieme$
        puissance et afficher le résultat sur la sortie standard~;
        \smallskip

        \item lire deux polynômes à partir d'un fichier, calculer
        leur somme et afficher le résultat sur la sortie standard~;
        \smallskip

        \item lire deux polynômes à partir d'un fichier, calculer
        leur produit et afficher le résultat sur la sortie standard~;
        \smallskip

        \item lire un polynôme à partir d'un fichier et afficher dans
        une fenêtre graphique le graphe du polynôme dans un intervalle
        d'abscisse et d'ordonnée spécifié.
        \smallskip
    \end{itemize}
    L'utilisateur peut choisir l'une ou l'autre de ces fonctionnalités
    par le biais d'une option (respectivement {\tt -d}, {\tt -p N},
    {\tt -s}, {\tt -m}, {\tt -g}). Le dernier paramètre de l'exécutable
    est le nom du fichier contenant le(s) polynôme(s) à traiter.
    Dans le cas où l'option {\tt -g} est choisie, après le nom du fichier
    figurent les paramètres {\tt X\_MIN}, {\tt X\_MAX}, {\tt Y\_MIN} et
    {\tt Y\_MAX} qui définissent les dimensions du repère du graphique.
    \smallskip

    Le format des fichiers contenant les polynômes est spécifié par
    l'exemple suivant~: le polynôme $-3 + 2.5x^5 - 4.7x^{11}$ est
    codé par
    \begin{center}
        {\tt -3 + 2.5x\^{}5 - 4.7x\^{}11}.
    \end{center}
    \smallskip

    \begin{enumerate}
        \item Découper ce projet en modules.
        \smallskip

        \item Pour chacun des modules, donner les fichiers d'en-tête
        au complet (directives du pré-processeur, définitions de types,
        déclarations de fonctions).
        \smallskip

        \item Dessiner le graphe d'inclusions du projet.
        \smallskip

        \item Écrire le code de la fonction {\tt main} du projet.
        \smallskip

        \item Écrire un {\tt Makefile} complet pour compiler le projet.
    \end{enumerate}

    \Sol{
    Rien n'est imposé mais je pense que le découpage suivant est adéquat~:
    \begin{itemize}
        \item un module pour gérer des listes de {\tt double} (pour
        représenter les polynômes)~;
        \item un module pour gérer les polynômes~;
        \item un module pour gérer l'affichage du graphe~;
        \item un module pour l'analyseur syntaxique qui permet de lire
        un polynôme depuis un fichier.
    \end{itemize}
    Ensuite, le reste est sans pièges.
    }
\end{Exercice}
\bigskip

\end{document}
